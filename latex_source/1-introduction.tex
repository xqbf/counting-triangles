\section{Introduction}
\label{sec:intro}

In many real-world applications (e.g., email networks,  social networks, and phone call networks), the relationships between entities can be modeled as a temporal graph \cite{zhao2019t,han2014chronos,erlebach2021temporal,kovanen2013temporal}, in which each edge is associated with a timestamp representing the interaction time.
%
Figure \ref{fig:temporal_graph}(a) depicts an example temporal graph, where the numbers on the edges represent their occurring timestamps.
%
For example, the two temporal edges between vertices $v_1$ and $v_4$ indicate that they have two interactions at timestamps $0$ and $2$.

\begin{figure}[ht]
    \small
    \centering
    \subfigure[An example temporal graph $G$]{
    \includegraphics[width = .27\linewidth]{figure/temporal graph example.pdf}
    }
    \text{  }
    \subfigure[Two 1-temporal  triangles]{
    \includegraphics[width = .3\linewidth]{figure/weighted triangle example.pdf}
    }
    \caption{An example of counting $\delta$-temporal triangles.}
    \label{fig:temporal_graph}
\end{figure}
\begin{comment}
    Triangle counting is an essential task in network analysis and has been extensively researched, as indicated by many studies \cite{tsourakakis2011spectral}. This problem has a wide range of applications including spam detection\cite{becchetti2010efficient}, online link recommendation\cite{tsourakakis2011spectral}, and so on. 

There are several works aiming for temporal triangle counting. 
\end{comment}

As a fundamental task in temporal graph analysis, triangle counting has received a great deal of attention \cite{lee2020temporal,paranjape2017motifs,pashanasangi2021faster}. Many analytical parameters contain the result of triangle counting, for instance, transitivity \cite{chu2011triangle} and clustering coefficient \cite{soffer2005network}. Besides, triangle counting plays an important role in many applications, such as spam detection \cite{becchetti2008efficient}, social network analysis \cite{pfeiffer2012fast},  community detection \cite{klymko2014using,gleich2012vertex}, and so on \cite{huang2018triangle, khan2011neighborhood}.
%
Different from triangles in static graphs, the triangles in temporal graphs often take the timestamps into consideration and several temporal triangle models have been developed, {\color{black}including $\delta$-temporal triangle \cite{paranjape2017motifs}, sliding-window triangle \cite{gou2021sliding}, $(\delta_{1,3}, \delta_{1,2}, \delta_{2,3})$-temporal triangle \cite{pashanasangi2021faster}, 
and temporal-triangle subgraph \cite{zhu2019scalable}}. 
%
Among these models, the $\delta$-temporal triangle, proposed by Paranjape, Benson, and Leskovec \cite{paranjape2017motifs}, has been demonstrated effective in many real applications.
%
Conceptually, a $\delta$-temporal triangle is a triangle formed by three temporal edges, such that the gap between the timestamps of any two temporal edges is bounded by a threshold $\delta$.
%
Consider the temporal graph in Figure \ref{fig:temporal_graph}(a) and let $\delta$=1. Then there are two 1-temporal triangles as shown in Figure \ref{fig:temporal_graph}(b).

In this paper, we study the problem of efficiently counting $\delta$-temporal triangles in large temporal graphs.
%
Given a temporal graph $G$, a duration $\delta$, and a time window $[t_s, t_e]$, the goal is to count all the $\delta$-temporal triangles within $[t_s, t_e]$.
%
{\color{black}For example, in Figure \ref{fig:temporal_graph}(a), assuming that $\delta$=1 and $[t_s,t_e]$=[0,3], then there are two 1-temporal triangles within the time window $[0,3]$ as depicted in Figure \ref{fig:temporal_graph}(b).}
%
As shown in the literature \cite{paranjape2017motifs,pashanasangi2021faster,bouritsas2022improving}, counting $\delta$-temporal triangles in temporal graphs has been demonstrated effective in many real applications, to name a few:
%
\begin{itemize}
    \item \textbf{Graph cohesiveness analysis.} A triangle represents the strong and stable relationship among three vertices \cite{huang2014querying}. As a result, $\delta$-temporal triangle counting can be used to measure the cohesiveness of temporal graphs.
    %
    For example, the triangle density of a graph is defined as the total number of triangles over the number of vertices \cite{tsourakakis2015k,samusevich2016local}, and it can be used to identify graph communities \cite{fang2019efficient}.

    \item \textbf{Graph transitivity.} Given a graph, its transitivity \cite{chu2011triangle} evaluates how strong the vertices are aggregated, and is defined as three times the number of triangles over the number of wedges, where a wedge is a path of two connected edges. The transitivity can be easily extended for temporal graphs by using $\delta$-temporal triangles.

    \item \textbf{Higher-order network analysis.} As a typical higher-order structure, triangles are one building block of many networks \cite{benson2016higher}, and $\delta$-temporal triangle counting is useful in the analysis of users' behaviors (e.g., blocking communication and bitcoin transfer) in temporal networks \cite{paranjape2017motifs}.
\end{itemize}

{\color{black} %(R3.A1-A2) 
Recently, we have also found that the $\delta$-temporal triangle counting can be used for analyzing the evolution of research topics in academic areas.
%
%To further demonstrate the effectiveness of $\delta$-temporal triangle counting, we present a case study in our experiments (Section \ref{sec:experiment-case-study}), analyzing the evolution of research topics within the KDD community from 2000 to 2023.
%
Specifically, we have tried to analyze the evolution of the research community of the KDD conference from 2000 to 2023.
%
We first construct two temporal co-authorship networks for the DB and AI communities respectively, and divide the entire time window [2000, 2023] into six disjoint intervals, each spanning four years.
%
We then adapt the triangle density \cite{samusevich2016local,tsourakakis2015k}, which is defined as the number of $\delta$-temporal triangles over the number of vertices, to analyze collaboration trends in the DB and AI communities within KDD.
%
Our finding of triangle density, depicted in Figure \ref{fig:case-study-weighted-intro}, shows that after 2016, the triangle density of the AI community surpassed that of the DB community.
%
This implies that the AI community has become more active than the traditional DB community since 2016.
%
Besides, the number of $\delta$-temporal triangles with $\delta$=1 is significantly higher than that with $\delta$=0, while the difference between $\delta$=2 and $\delta$=3  is minimal, suggesting that authors tend to collaborate more frequently over short time intervals.}

\begin{figure}[t]
        \subfigure[$\delta \in \{0,1\}$]{
	\begin{tikzpicture}[scale = 0.5]
	   \begin{axis}[
                    width=0.7\textwidth,
                    height=0.42\textwidth,
				xtick={0,1,2,3,4,5,6},
				xticklabels={,2000,2004,2008,2012,2016,2020},
				xmin=0.5,xmax=6.5,
				ymin=0,ymax=20,
                    legend style = {
                        legend columns=2,
                        draw=none,
                        at={(0.9,1)/}
				},
				% ymode = log,
				mark size=5pt,
                    line width=2.5pt,
				% xlabel={\huge \bf $\delta$ length ratio},
    ylabel={\bf $\delta$-temporal triangle density},
				every axis plot/.append style={ultra thick},
				every axis/.append style={ultra thick},
				]
				
    \addplot [mark=x,color=c8,line width=2.5pt] table[x=time,y=AI0]{\weighteddensity};
    \addplot [mark=o,color=c2,line width=2.5pt] table[x=time,y=DB0]{\weighteddensity};
    \addplot [mark=x,color=c7,line width=2.5pt] table[x=time,y=AI1]{\weighteddensity};
    \addplot [mark=o,color=c4,line width=2.5pt] table[x=time,y=DB1]{\weighteddensity};
    \legend{{ { AI ($\delta = 0$)}}, {  DB ($\delta = 0$)}, { { AI ($\delta = 1$)}}, { { DB ($\delta = 1$)}}}
    \end{axis}
    \end{tikzpicture}
    }
        \subfigure[$\delta\in \{2,3\}$]{
	\begin{tikzpicture}[scale = 0.5]
	   \begin{axis}[
                    width=0.7\textwidth,
                    height=0.42\textwidth,
				xtick={0,1,2,3,4,5,6},
				xticklabels={,2000,2004,2008,2012,2016,2020},
				xmin=0.5,xmax=6.5,
				ymin=0.001,ymax=30,
                    legend style = {
                        legend columns=2,
                        draw=none,
                        at={(0.9,1)/}
				},
				mark size=5pt,
                    line width=2.5pt,
                    ylabel={\bf $\delta$-temporal triangle density},
				every axis plot/.append style={ultra thick},
				every axis/.append style={ultra thick},
				]
					
                \addplot [mark=x,color=c8,line width=2.5pt] table[x=time,y=AI2]{\weighteddensity};
			\addplot [mark=o,color=c2,line width=2.5pt] table[x=time,y=DB2]{\weighteddensity};
                \addplot [mark=x,color=c7,line width=2.5pt] table[x=time,y=AI3]{\weighteddensity};
			\addplot [mark=o,color=c4,line width=2.5pt] table[x=time,y=DB3]{\weighteddensity};
                \legend{{ { AI($\delta = 2$)}}, {  DB($\delta = 2$)}, { { AI($\delta = 3$)}}, { { DB($\delta = 3$)}}}
			\end{axis}
	\end{tikzpicture}
	}   
    \caption{{\color{black}$\delta$-temporal triangle density.}}
    \label{fig:case-study-weighted-intro}
    \vspace{-0.05in}
\end{figure}


%The detailed studies of the above two applications are provided in Section \ref{sec:experiment-case-study}.

%counting on static graphs, triangle counting on temporal graphs can take the temporal ordering of the edges into concern to get more information. Thus, Paranjape, Benson, and Leskovec introduced $\delta$-temporal triangles that all edges of the triangle have to occur within $\delta$ time stamps \cite{paranjape2017motifs}. Then Noujan and Seshadhri improved the time complexity of $\delta$-temporal triangle counting and generalized it into $(\delta_{1,3}, \delta_{1,2}, \delta_{2,3})$-temporal triangle counting \cite{pashanasangi2021faster}, which is the state-of-art algorithm for $\delta$-temporal triangle counting so far. However, the definition of $(\delta_{1,3}, \delta_{1,2}, \delta_{2,3})$-temporal triangle is too strict for applications where the specific relationships inside the triangle are irrelevant. Nowadays, many temporal graph applications analyze the graph problems in different time windows \cite{mertzios2019sliding,akrida2020temporal,gou2021sliding,xie2023querying}. Thus, we want to discuss counting $\delta$-temporal triangles in arbitrary time windows.

Despite its wide usefulness, counting $\delta$-temporal triangles in large temporal graphs is a very challenging task, due to  the following two reasons:
1) Different from counting triangles in static graphs, counting $\delta$-temporal triangles is more complicated since it needs to consider the timestamp of each edge;
and 2) the number of $\delta$-temporal triangles is huge, especially when $\delta$ is large and the query time window covers a long duration.

{\bf Prior works.} To tackle the above challenges, Paranjape et al.~\cite{paranjape2017motifs} developed a $\delta$-temporal triangle counting algorithm with $O(m\sqrt{\Gamma})$ time, where $m$ is the number of edges in the graph $G$ and $\Gamma\leq O(m^{1.5})$ is the number of static triangles in $G$, so it achieves a time complexity of $O(m^{1.75})$.
%
Noujan and Seshadhri proposed the state-of-the-art (SOTA) algorithm for $\delta$-temporal triangle counting, \DOTTT, with $O(m\kappa\log(m))$ time cost \cite{pashanasangi2021faster}, where $\kappa$ is the degeneracy of the graph and is up to $O(m^{0.5})$.
%
{\color{black} Additionally, Zhu et al. \cite{zhu2021leveraging} designed a temporal subgraph reporting method and it could be adapted for counting $\delta$-temporal triangles.}
%
Nevertheless, these algorithms are still inefficient for processing large temporal graphs. 
% %
% Besides, they are slower than the triangle counting algorithm for static graphs through enumeration, which can achieve $O(m\sqrt{m})$ \cite{al2018triangle} but cannot be applied for counting $\delta$-temporal triangles.
% %
% Note that although triangle counting through matrix multiplication can achieve a lower time complexity than $O(m\sqrt{m})$, the space complexity of matrix multiplication is up to $O(n^2)$, which is too large to be applied to large graphs, so $O(m\sqrt{m})$ is still the best bound for counting triangles on large graphs.
% %
% Hence, it is desirable to develop faster algorithms for counting $\delta$-temporal triangles.

\textbf{Online solutions.} 
To efficiently count $\delta$-temporal triangles, we first propose an efficient online algorithm \OTTC.
%
Specifically, given a duration $\delta$ and a query time window $[t_s,t_e]$, we first derive a projected graph $G_{[t_s,t_e]}$ by extracting edges appearing in $[t_s,t_e]$.
%
Then, we enumerate the temporal edges chronologically to count $\delta$-temporal triangles.
%
We theoretically show that \OTTC completes in $O(m\kappa)$ time, so it is faster than the SOTA algorithm \cite{pashanasangi2021faster} with $O(m\kappa \log(m))$ time cost, as shown in Table \ref{tab:overview}.

\textbf{Index-based solutions.}
Although the above online algorithm outperforms existing algorithms, it may not be efficient in some scenarios because in practice, to accomplish a specific task, users often have to frequently issue $\delta$-temporal triangle counting queries by varying the time window $[t_s,t_e]$ and duration $\delta$ multiple times.
%
In light of this, we propose an elegant index-based solution, which converts the triangle counting problem into a point counting problem.
%
Specifically, for each $\delta$-temporal triangle, we first convert it into a 2-dimensional point $(t_s,t_e)$, where $[t_s,t_e]$ is the minimal time interval that $G_{[t_s,t_e]}$ contains this $\delta$-temporal triangle.
%
Afterward, we employ the wavelet tree index to solve this 2-dimensional point counting problem.
%
{\color{black} 
Besides, in many real-world applications, the temporal graphs often evolve frequently with many new inserted temporal edges as time goes on. 
%
For example, Yahoo! processed over 500M new communications between 33M IP addresses in one day\footnote{\url{https://webscope.sandbox.yahoo.com/catalog.php?datatype=g&guccounter=1}\label{web}}.
%
Given the dynamic nature of temporal graphs, it is crucial to efficiently maintain the index for dynamic temporal graphs.
%
Therefore, we also design a fast wavelet tree index maintenance algorithm to avoid rebuilding the index from scratch for every update.}

Besides, in some real-world scenarios, we may only need to consider the existence of $\delta$-temporal triangles among three vertices, so we introduce the binary $\delta$-temporal triangle counting problem, which counts the number of vertex triplets that form one or more $\delta$-temporal triangles.
%
We also develop efficient online and index-based solutions to solve this variant.

In addition, we have performed extensive experimental evaluations on four real and one synthetic temporal graphs, and the results show that our algorithms are highly efficient and scalable.
%
In particular, our online algorithm is up to 70$\times$ faster than the SOTA online algorithms, and our index-based algorithm runs up to $10^8\times$ faster than the online algorithms.
%
Besides, our algorithms for binary $\delta$-temporal triangle counting are also highly efficient.

In summary, our principal contributions are as follows.
%
\begin{itemize}
    \item We have developed an efficient online algorithm for counting $\delta$-temporal triangles, whose time complexity is lower than that of the SOTA algorithm.
    
    \item We have designed a novel index-based solution by converting the $\delta$-temporal triangles into 2-dimensional points.

    \item We have introduced the binary $\delta$-temporal triangle counting problem and also developed efficient algorithms.

    \item We have performed comprehensive experiments on both real and synthetic large temporal graphs, and the results show that our algorithms are highly effective and efficient.
\end{itemize}

\textbf{Outline.} 
We present the $\delta$-temporal triangle counting problem in Section \ref{sec:preliminaries}.
%
We introduce our online and index-based solutions in Sections \ref{sec:online} and \ref{sec:weighted}, respectively.
%
In Section \ref{sec:binary}, we focus on binary $\delta$-temporal triangles counting.
%
Section \ref{sec:experiment} shows experimental results.
%
We review related works in Section \ref{sec:related} and conclude in Section \ref{sec:conclusion}.

% {\color{black} (R3.A4) 
% To better illustrate our solutions for $\delta$-temporal triangle counting, we provide a flow chart of our proposed solutions in Figure \ref{fig:flow-chart}. We have also provided the bird's-eye views in front of Section \ref{sec:edttc} and \ref{sec:weighted}.

% \begin{figure}[ht]
%     \centering
%     \includegraphics[width=\linewidth]{figure/workflow of δ-temporal triangle counting.pdf}
%     \caption{\color{black} Flow chart of our solutions.}
%     \label{fig:flow-chart}
% \end{figure}
% }

\begin{table}[t]
  \small
  \centering
  \caption{Overview of the complexities of solutions.}
  \label{tab:overview}
  \begin{threeparttable}
  \begin{tabular}{c|c|c|c|c}
    \hline
      \multirow{2}{*}{Algorithm} & \multicolumn{2}{c|}{Indexing complexity}  & \multicolumn{2}{c}{Counting complexity}\\
     
     \cline{2-5}
     {} & time & space & time & space \\
     \hline\hline
     
     \DOTTT \cite{pashanasangi2021faster} & -- & --  & $O(m \kappa \log(m))$ &$O(m)$\\
     \cline{1-5}
     
     \OTTC & -- & --  & $O(m \kappa)$ &$O(m)$\\
     \cline{1-5}
    \color{black} {\tt TSRjoin} \cite{zhu2021leveraging} & 
    \tabincell{c}{\color{black}$O(n^2+$ \color{black}$m^2\log(m))$}
    & \color{black} $O(m^2)$ 
    & \color{black} $O(\max(m, \Delta))$ &
    \color{black} $O(m^2)$ \\
    
    \cline{1-5}
     
     {\tt WT-Index} & 
     
     %$O(m\kappa+ \Delta \times \log(\Delta))$
     \tabincell{c}{\color{black}$O(m\kappa+$ \color{black}$\pi \log(m))$}
     
     &
     
     %$O(m+\Delta \log(\Bar{c}))$ 
     \tabincell{c}{\color{black}$O(\pi)$}
     
     &\color{black}$O(poly\log(m))$ & 
     \color{black}$O(\pi)$\\
     \hline
  \end{tabular}
  {\color{black} \raggedright Note: given a temporal graph $G$, $n$ and $m$ are the numbers of vertices and edges in $G$ respectively, $\kappa$ is the degeneracy of $G$, $\pi$ is the number of C-points of $G$, and $\Delta$ is the number of counted $\delta$-temporal triangles in a query.
  \par}
  \end{threeparttable}
\end{table}