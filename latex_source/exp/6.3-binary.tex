\subsection{Efficiency of Binary $\delta$-Temporal Triangle Counting}
\label{sec:experiment-binary}

$\bullet$ {\bf Overall results.}
%
For each dataset, we report the average response
time of each algorithm in Figure \ref{fig:overall-binary}. 
%
Our online algorithm \BTTC consistently outperforms {\tt B-DOTTT} across all datasets.
%
For example, on the CT dataset, \BTTC is 23$\times$ faster than {\tt B-DOTTT}. 
%
Moreover, {\tt KDT-Index-query} achieves the best performance, as it is up to four orders of magnitude faster than \BTTC.

% \begin{figure}[h]	
% 	\centering
% 	\begin{tikzpicture}[scale=0.65]
%     		\begin{axis}[
%     			ybar,
%     			bar width=0.35cm,
%     			width=.73\textwidth,
%                     height=0.26\textwidth,
%     			xtick={1,2,3,4,5},	
%                     xticklabels={GR,ST,WK,EM,CT},
%     			legend style = {
%                         legend columns=-1,
%             		font=\large,
%                         draw=none,
%                         at={(0.64,1)/}
%                     },
%     			legend entries={{\tt B-DOTTT}, {\tt BTTC},{\tt KDT-Index-query}},
%     			xmin=0.5,xmax = 5.5,
%     			ymin=0,ymax=20000000,
%     			ymode =log,
%                     log origin=infty,
%     			% ylabel style={yshift=-4pt},
%     			ylabel={\LARGE \bf Time (ms)},
%     			ticklabel style={font=\LARGE},
% 				every axis plot/.append style={ultra thick},
% 				every axis/.append style={ultra thick},
%                     x dir=reverse,
%     		]
%     		\addplot[pattern=north west lines, pattern color=c1] table[x=datasets,y=DOTTT]{\binarybaseline};
%     		\addplot[pattern = grid, pattern color=c2] table[x=datasets,y=BTTC]{\binarybaseline};
%     		% \addplot[pattern = crosshatch dots,pattern color=green] table[x=datasets,y=kd-tree]{\baseline};
%                 \addplot[pattern = crosshatch,pattern color=c3] table[x=datasets,y=KD]{\binarybaseline};
%         \end{axis}
%         \end{tikzpicture}
%         \caption{Efficiency of binary $\delta$-temporal triangle counting.}
% 	\label{fig:overall-binary}
% \end{figure}
 
$\bullet$ {\bf Effect of $(t_e-t_s)$.}
%
In this experiment, we test five different lengths: $20\%$, $40\%$, $60\%$, $80\%$, and $100\%$ of $t_{max}$, with $\delta$ set to the default value.
%
For each length, we execute 1,000 counting queries with randomly selected $t_s$ and report the average response time in Figure \ref{fig:length-binary}.
%
Again, the response time increases as the interval length increases since more binary $\delta$-temporal triangles are involved.

$\bullet$ {\bf Effect of $\delta$.}
%
In this experiment, we set $[t_s,t_e]= [0, t_{max}]$, and vary $\delta = t_{max}\cdot y$ with $y\in \{10\%, 30\%, 50\%, 70\%, 90\%\}$.
%
For each $\delta$, we conduct 1,000 queries and record the average response time.
%
The findings are summarized in Figure  \ref{fig:query-delta-binary}.
%
We observe that $\delta$ has little effect on the efficiency on \BTTC and {\tt B-DOTTT}. But it affects the efficiency of {\tt KDT-Index-query} a lot because when $\delta$ goes larger, the response time complexity of {\tt KDT-Index} approaches $O(1)$.


% $\bullet$ {\bf Time and space costs of index construction.}
%
% {\color{red}In this experiment, we detail the time and space costs of index construction for all graphs in Figure \ref{fig:cons}. 
% %
% The time and space requirements of the {\tt WT-Index} scale with the graph sizes. Notably, the space cost of ST is larger than GR, possibly because the tree height of {\tt WT-Index} of ST is larger than that of GR since the $t_{max}$ of ST is larger than that of GR.
% }
%
% In this experiment, we evaluate the time and space costs of index construction for all graphs. As shown in Figure \ref{fig:cons-binary}, the time and space requirements of the {\tt KDT-Index} scale with graph sizes.

