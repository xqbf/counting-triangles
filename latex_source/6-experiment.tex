\section{Experiments}
\label{sec:experiment}

We now present the experimental results.
%
We begin with the setup in Section \ref{sec:experiment-setup}, then show the efficiency results in Sections \ref{sec:experiment-weighted} and \ref{sec:experiment-binary}, and finally present case study results in Section \ref{sec:experiment-case-study}.


\subsection{Setup}
\label{sec:experiment-setup}

\begin{table}[h]
    \small
    \caption{Datasets used in our experiments.}
    \label{tab:datasets}
    \centering
    \setlength{\tabcolsep}{0.7mm}{
    \begin{tabular}{c|c|r|r|r|r|r}
    \hline
    Datasets & Abbr. & \multicolumn{1}{c|}{$n$} & \multicolumn{1}{c|}{$m$} & \multicolumn{1}{c|}{$t_{max}$} & \multicolumn{1}{c|}{$\pi$} & \multicolumn{1}{c}{$\kappa$} \\
    \hline\hline
    contact & CT & 274 & 28,244 & 15,661 & 15M & 39 \\
    \hline
    email-eu & EM & 986 & 332,334 & 207,879 & 211M & 34 \\
    \hline
    wiki-talk & WK & 1,140,149 & 7,833,140 & 5,799,205 & 651M & 119 \\
    \hline
    stackoverflow & ST & 2,601,977 & 63,497,050 & 41,484,768 & 2.7B &198 \\
    \hline
    graph500-23 & GR & 4,610,222 & 129,333,677 & 6,807,835 & 
 3.2B&1,222 \\
    \hline
    \end{tabular}
    }
\end{table}

{\bf Datasets.}
%
We use four real-world temporal graphs sourced from SNAP \cite{leskovec2016snap} and KONECT \cite{konect}, along with a synthetic temporal graph generated from LDBC \cite{erling2015ldbc}.
%
Table \ref{tab:datasets} summarizes the statistics of each graph, including the number of vertices ($n$) and edges ($m$), the maximum timestamp ($t_{max}$), the number of C-points ($\pi$), and the degeneracy ($\kappa$).
%
The last graph does not have timestamps, so we randomly generate timestamps for it.


{\bf Algorithms.} We mainly evaluate the following algorithms.

\noindent $\bullet$ {\tt DOTTT}~\cite{pashanasangi2021faster}: SOTA online $\delta$-temporal triangle counting algorithm;
    
\noindent $\bullet$ {\tt OTTC}: our online $\delta$-temporal triangle counting algorithm;
    
%\item {\tt WT-Index}: the {\tt WT-Index} construction algorithm as illustrated in Section \ref{sec:overview-wavelet};

{\color{black}
\noindent $\bullet$ {\tt TSRjoin}~\cite{zhu2021leveraging}: an index-based $\delta$-temporal triangle counting algorithm, originally designed for counting temporal-clique subgraphs;}

\noindent $\bullet$ {\tt WT-Index-query}: the {\tt WT-Index}-based $\delta$-temporal triangle counting algorithm;

\noindent $\bullet$ {\tt B-DOTTT}: adapted \DOTTT for binary $\delta$-temporal triangle counting;

\noindent $\bullet$ {\tt BTTC}: our online binary $\delta$-temporal triangle counting algorithm;

%\noindent $\bullet$ {\tt KDT-Index}: the {\tt KDT-Index} construction algorithm to construct a {\tt KDT-Index}~\cite{10.1145/361002.361007, 10.1007/BF00263763} for all BC-points;

\noindent $\bullet$ {\tt KDT-Index-query}: the {\tt KDT-Index}-based binary $\delta$-triangle counting algorithm.

{\color{black}
We have implemented all the algorithms above and placed the codes in a GitHub repository\footnote{\url{https://github.com/xqbf/counting-triangles}}.
%
Note that {\tt TSRjoin} was designed for counting temporal-clique subgraphs, but it can be adapted for counting $\delta$-temporal triangle as follows:
%
Given a temporal graph $G$ and a duration $\delta$, we first convert each temporal edge $(u,v,t)$ to an edge $(u,v)$ with a time interval $[t,t+\delta]$, then convert the query time interval $[t_s,t_e]$ to $[t_s+\delta, t_e]$, and finally apply the algorithm in \cite{zhu2021leveraging} to get the result.
%
Its time and space costs are analyzed as follows:
%
{\it (1) Index construction}: For a specific $\delta$, the index construction time for the {\tt TSRjoin} is $O(m \log(m) + n^2)$, with a space complexity $O(m)$.
%
To accommodate all possible values of $\delta$, the solution needs to convert one temporal edge into $O(m)$ edges with different time intervals and then build the index with total $O(m^2)$ converted edges.
%
Consequently, the overall time and space complexities become $O(m^2 \log(m) + n^2)$ and $O(m^2)$, respectively.
%
{\it (2) Query processing}: Given a specific $\delta$, {\tt TSRjoin} needs to enumerate the graph and all $\delta$-temporal triangles, costing $O(\max(m, \Delta))$ time.}

To evaluate the efficiency, for each dataset, we consider 5 different interval lengths $|t_e-t_s|=t_{max}\cdot x$ with $x\in\{20\%, 40\%, 60\%, 80\%,$ $100\%\}$, and then for each interval length, we randomly generate 1,000 queries by varying $\delta = |t_e-t_s| \cdot y$ with $y\in \{10\%, 30\%, 50\%, 70\%,$ $90\%\}$, where the default interval length and $\delta$ are set to $100\% \times t_{max}$ and $10\%|t_e-t_s|$ respectively.
%
%Subsequently, we execute these queries sequentially and compute the average query time.
%
All algorithms are implemented in C++ and compiled with the g++ compiler at the -O3 optimization level.
%
The experiments are conducted on a Linux machine equipped with an Intel Xeon 2.90GHz CPU and 512GB RAM.
%
%Memory requirements are measured by the maximum RES memory used during the process.



\pgfplotstableread[row sep=\\,col sep=&]{
	length & WT &OTTC&DOTTT & tsrjoin \\
	0.2 & 0.184 & 24513 & 39485& 29501\\
	0.4 & 0.1664 & 48495 & 94581& 40065\\
	0.6 & 0.1527 & 75381 & 192302& 46778\\
	0.8 & 0.1379 & 102008 & 258482& 52960\\
	1.0 & 0.0036 & 148810 & 466289& 65843\\
}\stackoverflow

\pgfplotstableread[row sep=\\,col sep=&]{
	length & WT &OTTC&DOTTT& tsrjoin \\
	0.2 & 0.197 & 83142 & 268044& 914462\\
	0.4 & 0.1934 & 218599 & 947287& 918252\\
	0.6 & 0.1611 & 423764 & 2179308& 1005424\\
	0.8 & 0.1283 & 620516 & 3825394& 1111297\\
	1.0 & 0.0034 & 878521 & 6784310& 1190203\\
}\graphfive

\pgfplotstableread[row sep=\\,col sep=&]{
	length & WT & OTTC & DOTTT & tsrjoin \\
	0.2 & 0.0302 & 1599 & 3339& 1795\\
	0.4 & 0.0287 & 2796 & 10779& 2583\\
	0.6 & 0.0306 & 4116 & 19781& 2862\\
	0.8 & 0.0259 & 5430 & 30516& 3755\\
	1.0 & 0.0032 & 7118 & 44509& 4842\\
}\wiki

\pgfplotstableread[row sep=\\,col sep=&]{
	length & WT & OTTC & DOTTT & tsrjoin \\
	0.2 & 0.0259 & 15 & 302& 53\\
	0.4 & 0.0265 & 34 & 854 & 123\\
	0.6 & 0.0256 & 57 & 1547 & 178\\
	0.8 & 0.0272 & 81 & 2319 & 253\\
	1.0 & 0.0042 & 106 & 3292 & 336\\
}\email

\pgfplotstableread[row sep=\\,col sep=&]{
	length & WT & OTTC & DOTTT &tsrjoin \\
	0.2 & 0.0087 & 2 & 73& 14\\
	0.4 & 0.0077 & 4 & 157& 26\\
	0.6 & 0.0077 & 6 & 269& 34\\
	0.8 & 0.007 & 8 & 408& 49\\
	1.0 & 0.0024 & 10 & 742& 73\\
}\tsv

\pgfplotstableread[row sep=\\,col sep=&]{
	length & KD &BTTC&DOTTT \\
	0.2 & 356 & 19510 & 39485\\
	0.4 & 770 & 41176 & 94581\\
	0.6 & 1218 & 67995 & 192302\\
	0.8 & 1651 & 103166 & 258482\\
	1.0 & 2196 & 135077 & 466289\\
}\stackoverflowbinary

\pgfplotstableread[row sep=\\,col sep=&]{
	length & KD &BTTC&DOTTT& kd-tree \\
	0.2 & 3961 & 138338 & 268044& 15\\
	0.4 & 10737 & 609170 & 947287& 42\\
	0.6 & 19264 & 1243640 & 2179308& 77\\
	0.8 & 28765 & 2361215 & 3825394& 114\\
	1.0 & 37011 & 4222026 & 6784310& 146\\
}\graphfivebinary

\pgfplotstableread[row sep=\\,col sep=&]{
	length & KD & BTTC & DOTTT & kd-tree \\
	0.2 & 27 & 1506 & 3339& 0.3561\\
	0.4 & 60 & 3490 & 10779& 0.8596\\
	0.6 & 96 & 5718 & 19781& 1.553\\
	0.8 & 134 & 8277 & 30516& 2.5349\\
	1.0 & 171 & 11314 & 44509& 3.4783\\
}\wikibinary

\pgfplotstableread[row sep=\\,col sep=&]{
	length & KD & BTTC & DOTTT & kd-tree \\
	0.2 & 3.5 & 69 & 302& 0.02\\
	0.4 & 6.9 & 175 & 854 & 0.04\\
	0.6 & 9.6 & 296 & 1547 & 0.06\\
	0.8 & 11.3 & 430 & 2319 & 0.07\\
	1.0 & 11.08 & 574 & 3292 & 0.06\\
}\emailbinary

\pgfplotstableread[row sep=\\,col sep=&]{
	length & KD & BTTC & DOTTT &kd-tree \\
	0.2 & 0.8 & 10 & 73& 0.008\\
	0.4 & 1.6 & 24 & 157& 0.02\\
	0.6 & 2.2 & 40 & 269& 0.02\\
	0.8 & 2.5 & 57 & 408& 0.02\\
	1.0 & 2.38 & 80 & 742& 0.03\\
}\tsvbinary




\pgfplotstableread[row sep=\\,col sep=&]{
	length & WT &OTTC& DOTTT& tsrjoin \\
	0.1 & 0.0036 & 148810 & 466289& 65843\\
	0.3 & 0.0033 & 154328 & 489479& 62627\\
	0.5 & 0.0015 & 151089 & 456316& 57711\\
	0.7 & 0.0034 & 124471 & 472154& 54273\\
	0.9 & 0.0036 & 125371 & 477172& 52533\\
}\stackoverflowd

\pgfplotstableread[row sep=\\,col sep=&]{
	length & WT &OTTC & DOTTT& tsrjoin \\
	0.1 & 0.0034 & 878521 & 6784310& 1190203\\
	0.3 & 0.0037 & 891262 & 6554200& 1182613\\
	0.5 & 0.0013 & 954383 & 6363630& 1108821\\
	0.7 & 0.0032 & 1039122 & 6531770&1074491\\
	0.9 & 0.0032 & 932256 & 6333730& 1050953\\
}\graphfived

\pgfplotstableread[row sep=\\,col sep=&]{
	length & WT &DOTTT & OTTC & tsrjoin \\
	0.1 & 0.0032 & 44509 & 7188 & 4842\\
	0.3 & 0.0033 & 43830 & 7412 & 4646\\
	0.5 & 0.0013 & 43723 & 7257 &  4325\\
	0.7 & 0.003 & 43584 & 6641 & 4051\\
	0.9 & 0.003 & 42510 & 6058 & 3887\\
}\wikid

\pgfplotstableread[row sep=\\,col sep=&]{
	length & WT & DOTTT & OTTC & tsrjoin \\
	0.1 & 0.0042 & 3292 & 106 & 336\\
	0.3 & 0.0052 & 3581 & 101 & 363\\
	0.5 & 0.0027 & 3615 & 88 & 339\\
	0.7 & 0.0043 & 3607 & 75 & 319\\
	0.9 & 0.0044 & 3259 & 67 & 292\\
}\emaild

\pgfplotstableread[row sep=\\,col sep=&]{
	length & WT & DOTTT& OTTC& tsrjoin \\
	0.1 & 0.0024 & 742& 10 & 73 \\
	0.3 & 0.0021 & 671& 8.6 & 77\\
	0.5 & 0.0011 & 656& 7.8 & 73\\
	0.7 & 0.0035 & 602& 6.9 & 66\\
	0.9 & 0.0021 & 554& 6.2 & 61\\
}\tsvd

\pgfplotstableread[row sep=\\,col sep=&]{
	length & KD &BTTC& DOTTT& kd-tree \\
	0.1 & 2196 & 135077 & 466289& 26\\
	0.3 & 1659 & 129966 & 489479& 24\\
	0.5 & 1070 & 131001 & 456316& 19\\
	0.7 & 520 & 143781 & 472154& 15\\
	0.9 & 97 & 144079 & 477172& 12\\
}\stackoverflowdbinary

\pgfplotstableread[row sep=\\,col sep=&]{
	length & KD &BTTC & DOTTT& kd-tree \\
	0.1 & 37011 & 4222026 & 6784310& 146\\
	0.3 & 47151 & 4035798 & 6554200& 193\\
	0.5 & 33729 & 4322033 & 6363630& 197\\
	0.7 & 23585 & 4079659 & 6531770&180\\
	0.9 & 5618 & 4091265 & 6333730& 161\\
}\graphfivedbinary

\pgfplotstableread[row sep=\\,col sep=&]{
	length & KD &DOTTT & BTTC & kd-tree \\
	0.1 & 171 & 44509 & 11314 & 1.2\\
	0.3 & 142 & 43830 & 11479 & 6\\
	0.5 & 95 & 43723 & 11713 &  14.4\\
	0.7 & 47 & 43584 & 11376 & 28.8\\
	0.9 & 8 & 42510 & 11391 & 44.4\\
}\wikidbinary

\pgfplotstableread[row sep=\\,col sep=&]{
	length & KD & DOTTT & BTTC & kd-tree \\
	0.1 & 11.08 & 3292 & 238 & 31\\
	0.3 & 5.0016 & 3581 & 211 & 20\\
	0.5 & 1.6591 & 3615 & 203 & 16\\
	0.7 & 0.577 & 3607 & 174 & 14.6\\
	0.9 & 0.0461 & 3259 & 158 & 13\\
}\emaildbinary

\pgfplotstableread[row sep=\\,col sep=&]{
	length & KD & DOTTT& BTTC& kd-tree \\
	0.1 & 2.38 & 742& 37 & 0.03 \\
	0.3 & 1.5 & 671& 34 & 0.018\\
	0.5 & 1.2 & 656& 33 & 0.016\\
	0.7 & 1.2 & 602& 29 & 0.014\\
	0.9 & 1.1 & 554& 24 & 0.014\\
}\tsvdbinary


\pgfplotstableread[row sep=\\,col sep=&]{
	length & LSC &wavelet tree \\
	0.01 & 577425 & 609146 \\
	0.05 & 798674 & 808090 \\
	0.1 & 918819 & 1154172 \\
	0.2 & 1288712 & 1476684 \\
	0.3 & 1973168 & 1983985 \\
}\stackoverflowf

\pgfplotstableread[row sep=\\,col sep=&]{
	length & LSC &wavelet tree \\
	0.01 & 2038564 & 2172252 \\
	0.05 & 2189398 &  2123597\\
	0.1 & 2099664 &  2288220 \\
	0.2 & 2208948 &  2259616\\
	0.3 & 2288616 & 2471462 \\
}\graphfivef

\pgfplotstableread[row sep=\\,col sep=&]{
	length & LSC &wavelet tree \\
	0.01 & 67333 & 74486 \\
	0.05 & 104348 & 129932 \\
	0.1 & 134255 & 163451 \\
	0.2 & 237004 & 261851 \\
	0.3 & 288011 & 353856 \\
}\wikif

\pgfplotstableread[row sep=\\,col sep=&]{
	length & LSC &wavelet tree \\
	0.01 & 8393 & 9800 \\
	0.05 & 18320 & 22861 \\
	0.1 & 30235 & 39094 \\
	0.2 & 50959 & 68203 \\
	0.3 & 66685 & 94540 \\
}\emailf

\pgfplotstableread[row sep=\\,col sep=&]{
	length & LSC &wavelet tree \\
	0.01 & 637 & 702 \\
	0.05 & 1109 & 1316 \\
	0.1 & 1687 & 2014 \\
	0.2 & 2775 & 3319 \\
	0.3 &  3676 & 4454 \\
}\tsvf



\pgfplotstableread[row sep=\\,col sep=&]{
	length & LSC &wavelet tree \\
	0.01 & 0.053 & 0.0134 \\
	0.05 & 0.0647 & 0.0154 \\
	0.1 & 0.0748 & 0.016 \\
	0.2 & 0.0856 & 0.0168 \\
	0.3 & 0.1066 & 0.0245 \\
}\stackoverflowfq

\pgfplotstableread[row sep=\\,col sep=&]{
	length & LSC &wavelet tree \\
	0.01 & 0.021 & 0.0113 \\
	0.05 & 0.0361 &  0.0118\\
	0.1 & 0.0314 &  0.0117 \\
	0.2 & 0.0385 & 0.0121 \\
	0.3 & 0.0404 & 0.0132 \\
}\graphfivefq

\pgfplotstableread[row sep=\\,col sep=&]{
	length & LSC &wavelet tree& kd-tree \\
	0.01 & 0.0224 & 0.0091& 1 \\
	0.05 & 0.0331 & 0.0124& 5\\
	0.1 & 0.0422 & 0.0125&  12 \\
	0.2 & 0.0485 & 0.0136& 24 \\
	0.3 & 0.051 & 0.0141&37 \\
}\wikifq

\pgfplotstableread[row sep=\\,col sep=&]{
	length & LSC &wavelet tree&kd-tree \\
	0.01 & 0.0109 & 0.006& 0.07 \\
	0.05 & 0.0157 & 0.0073& 0.4 \\
	0.1 & 0.018 & 0.0077& 0.7 \\
	0.2 & 0.021 & 0.0084&1.4 \\
	0.3 & 0.0218 & 0.0092& 2 \\
}\emailfq

\pgfplotstableread[row sep=\\,col sep=&]{
	length & LSC &wavelet tree \\
	0.01 & 0.0042 & 0.0038 \\
	0.05 & 0.0055 & 0.0045 \\
	0.1 & 0.0059 & 0.0046 \\
	0.2 & 0.0064 & 0.0048 \\
	0.3 &  0.0069 & 0.005 \\
}\tsvfq





\pgfplotstableread[row sep=\\, col sep = &]{
    length & LSC & TTC & kd-tree\\
    0.2 & 0.0398 & 14870& 1.2947\\
    0.4 &0.0395 & 33858& 2.8918\\
    0.6 &0.041 & 59278& 4.5552\\
    0.8 &0.0426 & 81808& 7.0367\\
    1 & 0.0181& 105739&10.5806\\
}\stackoverflowdirectcircle


\pgfplotstableread[row sep=\\, col sep = &]{
    length & LSC & TTC & kd-tree\\
    0.2 & 0.0903 & 966& 0.3561\\
    0.4 &0.0974 & 2089& 0.8596\\
    0.6 &0.0998 & 3131& 1.553\\
    0.8 &0.1085 & 4568& 2.5349\\
    1 & 0.0201& 6551& 3.4783\\
}\wikidirectcircle

\pgfplotstableread[row sep=\\, col sep = &]{
datasets & WT & OTTC & tsrjoin & DOTTT\\
1 & 0.0034 & 878521& 1190203 & 6784310\\
2 & 0.0036 & 148810 & 65843 &466289\\
3 & 0.0032 & 7188 & 4842 & 44509 \\
4 & 0.0042 & 106 & 336 & 3292\\
5 & 0.0018 & 10 & 73 & 742\\
}\baseline


\pgfplotstableread[row sep=\\, col sep = &]{
datasets & KD & BTTC  & DOTTT\\
1 & 37011 & 4222026 & 6784310\\
2 & 2196 & 135077  &466289\\
3 & 171 & 11915  & 44509 \\
4 & 11 & 238  & 3292\\
5 & 5 & 33  & 742\\
}\binarybaseline


\pgfplotstableread[row sep=\\,col sep=&]{
	datasets &WT & TSR \\
	5 &  9537886 & 7550215\\
	4 &  5986268 & 367765 \\
	3 &  1232820 & 36895 \\
	2 & 283948 & 106 \\
	1 & 12896 & 10\\
}\constructiontime

\pgfplotstableread[row sep=\\,col sep=&]{
	datasets & WT & TSR & origin \\
	5 & 387379& 72704 & 3002\\
	4 & 393830& 8397 & 1119\\
	3 & 82227& 1536 & 116\\
	2 & 21709& 31 & 5\\
	1 & 1946& 6 & 1 \\
}\constructionspace

\pgfplotstableread[row sep=\\,col sep=&]{
	length & CT & EM & WK & ST & GR \\
	1 & 704 & 5228 & 128050 & 926023 & 423414 \\
	2 & 2446 & 31705 & 305446 & 2100846 & 1692602\\
	3 & 5073 & 80703 & 527970 & 3330921 & 3160997\\
	4 & 9383 & 169046 & 764345 & 4910566 & 6949984\\
	5 & 12896 &283948 & 1232820 & 5986268 & 9537886\\
}\sctime

\pgfplotstableread[row sep=\\,col sep=&]{
	length & CT & EM & WK & ST & GR \\
	1 & 71 & 587 & 11059 & 68813 & 17306\\
	2 & 268 & 2662 & 24371 & 151347 & 63078\\
	3 & 550 & 6554 & 42701 & 234598 & 129946\\
	4 & 911 & 12390 & 60314 &315392 & 244326\\
	5 & 1946 &21709 & 82227& 393830 & 387379\\
}\scspace



\begin{figure*}
        \centering
	\ref{named1}\\
        \vspace{-20pt}
	\subfigure[CT]{
	   \begin{tikzpicture}[scale=0.49]
	   \begin{axis}[
                    width=.39\textwidth,
                    height=0.3\textwidth,
				xticklabels={,,$20\%$,$40\%$,$60\%$,$80\%$,100\%},
				xmin=0.1,xmax=1.1,
				ymin=0,ymax=100000,
				ymode = log,
				mark size=5pt,
                    line width=2.5pt,
				ylabel={\LARGE \bf Response time (ms)},
                    ylabel style={xshift=12pt,yshift=-4pt},
				ticklabel style={font=\Large},
				every axis plot/.append style={ultra thick},
				every axis/.append style={ultra thick},
			]
			\addplot [mark=x,color=c4,line width=2.5pt] table[x=length,y=WT]{\tsv};
			\addplot [mark=o,color=c6,line width=2.5pt] table[x=length,y=OTTC]{\tsv};
            \addplot [mark=star,color=c7,line width=2.5pt] table[x=length,y=DOTTT]{\tsv};
            \addplot [mark=o,color=c8,line width=2.5pt] table[x=length,y=tsrjoin]{\tsv};
		\end{axis}
	\end{tikzpicture}
	}
        \subfigure[EM]{
	   \begin{tikzpicture}[scale=0.49]
	   \begin{axis}[
                    width=.39\textwidth,
                    height=0.3\textwidth,
				xticklabels={,,$20\%$,$40\%$,$60\%$,$80\%$,$100\%$},
				xmin=0.1,xmax=1.1,
				ymin=0,ymax=100000,
				ymode = log,
				mark size=5pt,
                    line width=2.5pt,
				ticklabel style={font=\Large},
				every axis plot/.append style={ultra thick},
				every axis/.append style={ultra thick},
				]
                    \addplot [mark=x,color=c4,line width=2.5pt] table[x=length,y=WT]{\email};	
                    \addplot [mark=o,color=c6,line width=2.5pt] table[x=length,y=OTTC]{\email};
                    \addplot [mark=star,color=c7,line width=2.5pt] table[x=length,y=DOTTT]{\email};
                    \addplot [mark=o,color=c8,line width=2.5pt] table[x=length,y=tsrjoin]{\email};
			\end{axis}
	\end{tikzpicture}
	}
        \subfigure[WK]{
	   \begin{tikzpicture}[scale=0.49]
	   \begin{axis}[
                    width=.39\textwidth,
                    height=0.3\textwidth,
				xticklabels={,,$20\%$,$40\%$,$60\%$,$80\%$,$100\%$},
				xmin=0.1,xmax=1.1,
				ymin=0,ymax=100000,
				ymode = log,
				mark size=5pt,
                    line width=2.5pt,
				ticklabel style={font=\Large},
				every axis plot/.append style={ultra thick},
				every axis/.append style={ultra thick},
				]
				\addplot [mark=x,color=c4,line width=2.5pt] table[x=length,y=WT]{\wiki};
				\addplot [mark=o,color=c6,line width=2.5pt] table[x=length,y=OTTC]{\wiki};
                \addplot [mark=star,color=c7,line width=2.5pt] table[x=length,y=DOTTT]{\wiki};
                \addplot [mark=o,color=c8,line width=2.5pt] table[x=length,y=tsrjoin]{\wiki};
			\end{axis}
	\end{tikzpicture}
	}
        \subfigure[ST]{
	   \begin{tikzpicture}[scale=0.49]
	   \begin{axis}[
                    width=.39\textwidth,
                    height=0.3\textwidth,
				xticklabels={,,$20\%$,$40\%$,$60\%$,$80\%$,$100\%$},
				xmin=0.1,xmax=1.1,
				ymin=0,ymax=10000000,
				ymode = log,
				mark size=5pt,
                    line width=2.5pt,
				ticklabel style={font=\Large},
				every axis plot/.append style={ultra thick},
				every axis/.append style={ultra thick},
				]
                    \addplot [mark=x,color=c4,line width=2.5pt] table[x=length,y=WT]{\stackoverflow};
				\addplot [mark=o,color=c6,line width=2.5pt] table[x=length,y=OTTC]{\stackoverflow};
                \addplot [mark=star,color=c7,line width=2.5pt] table[x=length,y=DOTTT]{\stackoverflow};
                \addplot [mark=o,color=c8,line width=2.5pt] table[x=length,y=tsrjoin]{\stackoverflow};
		\end{axis}
	\end{tikzpicture}
        }
        \subfigure[GR]{
	   \begin{tikzpicture}[scale=0.49]
	   \begin{axis}[
			legend style = {
                    legend columns=-1,
                    draw=none,
			},
                legend to name=named1,
                legend image post style={scale=0.7, ultra thick},
                width=.39\textwidth,
                height=0.3\textwidth,
			xticklabels={,,$20\%$,$40\%$,$60\%$,$80\%$,$100\%$},
			xmin=0.1,xmax=1.1,
			% ymin=0,ymax=10000000,
			ymode = log,
			mark size=5pt,
                line width=2.5pt,
			ticklabel style={font=\Large},
			every axis plot/.append style={ultra thick},
			every axis/.append style={ultra thick},
			]
			\addplot [mark=x,color=c4,line width=2.5pt] table[x=length,y=WT]{\graphfive};
			\addplot [mark=o,color=c6,line width=2.5pt] table[x=length,y=OTTC]{\graphfive};
            \addplot [mark=star,color=c7,line width=2.5pt] table[x=length,y=DOTTT]{\graphfive};
            \addplot [mark=o,color=c8,line width=2.5pt] table[x=length,y=tsrjoin]{\graphfive};
			\legend{{\footnotesize {\tt WT-Index-query}}, {\footnotesize  \OTTC}, {\footnotesize \DOTTT}, {\footnotesize {\tt TSRjoin}}}
			\end{axis}
	\end{tikzpicture}
    }
    \setlength{\abovecaptionskip}{0.05cm}
    \caption{Effect of $(t_e-t_s)$ (which varies from
    20\% to 40\%, 60\%, 80\%, and 100\% of $t_{max}$).}
    \label{fig:length}
\end{figure*}

\begin{figure*}
        \centering
	\ref{edelta}\\
        \vspace{-20pt}
        \subfigure[CT]{
	\begin{tikzpicture}[scale = 0.49]
	   \begin{axis}[
                    width=0.39\textwidth,
                    height=0.3\textwidth,
				xtick={0,0.1,0.3,0.5,0.7,0.9},
				xticklabels={,$10\%$,$30\%$,$50\%$,$70\%$,$90\%$},
				xmin=0,xmax=1,
				ymin=0,ymax=10000,
				ymode = log,
				mark size=5pt,
                    line width=2.5pt,
				% xlabel={\huge \bf $\delta$ length ratio},
                    ylabel={\LARGE \bf Response time (ms)},
				ylabel style={xshift=12pt,yshift=-3pt},
				ticklabel style={font=\Large},
				every axis plot/.append style={ultra thick},
				every axis/.append style={ultra thick},
				]
				\addplot [mark=x,color=c4,line width=2.5pt] table[x=length,y=WT]{\tsvd};
				\addplot [mark=o,color=c6,line width=2.5pt] table[x=length,y=OTTC]{\tsvd};		
                    % \addplot [mark=star,color=c8,line width=2.5pt] table[x=length,y=kd-tree]{\tsvd};
                    \addplot [mark=star,color=c7,line width=2.5pt] table[x=length,y=DOTTT]{\tsvd};
                    \addplot [mark=o,color=c8,line width=2.5pt] table[x=length,y=tsrjoin]{\tsvd};
			\end{axis}
	\end{tikzpicture}
	}
        \subfigure[EM]{
	\begin{tikzpicture}[scale = 0.49]
	   \begin{axis}[
                    width=.39\textwidth,
                    height=0.3\textwidth,
				xtick={0,0.1,0.3,0.5,0.7,0.9},
				xticklabels={,$10\%$,$30\%$,$50\%$,$70\%$,$90\%$},
				xmin=0,xmax=1,
				ymin=0,ymax=100000,
				ymode = log,
				mark size=5pt,
                    line width=2.5pt,
				% ylabel={\huge \bf Running time (ms)},
				% ylabel style={yshift=-5pt},
				% xlabel={\huge \bf $\delta$ length ratio},
				ticklabel style={font=\Large},
				every axis plot/.append style={ultra thick},
				every axis/.append style={ultra thick},
				]
				\addplot [mark=x,color=c4,line width=2.5pt] table[x=length,y=WT]{\emaild};
				\addplot [mark=o,color=c6,line width=2.5pt] table[x=length,y=OTTC]{\emaild};		
                    % \addplot [mark=star,color=c8,line width=2.5pt] table[x=length,y=kd-tree]{\emaild};
                    \addplot [mark=star,color=c7,line width=2.5pt] table[x=length,y=DOTTT]{\emaild};
                    \addplot [mark=o,color=c8,line width=2.5pt] table[x=length,y=tsrjoin]{\emaild};
			\end{axis}
	\end{tikzpicture}
	}
        \subfigure[WK]{
	\begin{tikzpicture}[scale = 0.49]
	   \begin{axis}[
                    width=.39\textwidth,
                    height=0.3\textwidth,
				xtick={0,0.1,0.3,0.5,0.7,0.9},
				xticklabels={,$10\%$,$30\%$,$50\%$,$70\%$,$90\%$},
				xmin=0,xmax=1,
				ymin=0,ymax=200000,
				ymode = log,
				mark size=5pt,
                    line width=2.5pt,
				% ylabel={\huge \bf Running time (ms)},
				% ylabel style={yshift=-5pt},
				% xlabel={\huge \bf $\delta$ length ratio},
				ticklabel style={font=\Large},
				every axis plot/.append style={ultra thick},
				every axis/.append style={ultra thick},
				]
				\addplot [mark=x,color=c4,line width=2.5pt] table[x=length,y=WT]{\wikid};
				\addplot [mark=o,color=c6,line width=2.5pt] table[x=length,y=OTTC]{\wikid};		
                    \addplot [mark=star,color=c7,line width=2.5pt] table[x=length,y=DOTTT]{\wikid};
                    \addplot [mark=o,color=c8,line width=2.5pt] table[x=length,y=tsrjoin]{\wikid};
			\end{axis}
	\end{tikzpicture}
	}
        \subfigure[ST]{
	\begin{tikzpicture}[scale = 0.49]
	   \begin{axis}[
                    width=.39\textwidth,
                    height=0.3\textwidth,
				xtick={0,0.1,0.3,0.5,0.7,0.9},
				xticklabels={,$10\%$,$30\%$,$50\%$,$70\%$,$90\%$},
				xmin=0,xmax=1,
				ymin=0,ymax=10000000,
				ymode = log,
				mark size=5pt,
                    line width=2.5pt,
				% ylabel={\huge \bf Running time (ms)},
				% ylabel style={yshift=-5pt},
				% xlabel={\huge \bf $\delta$ length ratio},
				ticklabel style={font=\Large},
				every axis plot/.append style={ultra thick},
				every axis/.append style={ultra thick},
				]
				\addplot [mark=x,color=c4,line width=2.5pt] table[x=length,y=WT]{\stackoverflowd};
				\addplot [mark=o,color=c6,line width=2.5pt] table[x=length,y=OTTC]{\stackoverflowd};		
                    % \addplot [mark=star,color=c8] table[x=length,y=kd-tree]{\stackoverflowd};
                    \addplot [mark=star,color=c7,line width=2.5pt] table[x=length,y=DOTTT]{\stackoverflowd};
                    \addplot [mark=o,color=c8,line width=2.5pt] table[x=length,y=tsrjoin]{\stackoverflowd};
			\end{axis}
	\end{tikzpicture}
	}
        \subfigure[GR]{
	\begin{tikzpicture}[scale = 0.49]
	   \begin{axis}[
			    legend style = {
			          legend columns=-1,
                        draw=none,
				},
				legend to name=edelta,
                    legend image post style={scale=0.8, ultra thick},
                    width=.39\textwidth,
                    height=0.3\textwidth,
                    xtick={0,0.1,0.3,0.5,0.7,0.9},
				xticklabels={,$10\%$,$30\%$,$50\%$,$70\%$,$90\%$},
				xmin=0,xmax=1,
				ymin=0,ymax=100000000,
				ymode = log,
				mark size=5pt,
                    line width=2.5pt,
				ticklabel style={font=\Large},
				every axis plot/.append style={ultra thick},
				every axis/.append style={ultra thick},
				]
				\addplot [mark=x,color=c4,line width=2.5pt] table[x=length,y=WT]{\graphfived};
				\addplot [mark=o,color=c6,line width=2.5pt]     table[x=length,y=OTTC]{\graphfived};	
                    \addplot [mark=star,color=c7,line width=2.5pt] table[x=length,y=DOTTT]{\graphfived};
                    \addplot [mark=o,color=c8,line width=2.5pt] table[x=length,y=tsrjoin]{\graphfived};
				\legend{{\footnotesize {\tt WT-Index-query} },{\footnotesize  \OTTC}, {\footnotesize \DOTTT}, {\footnotesize {\tt TSRjoin}}}
			\end{axis}
	   \end{tikzpicture}
	}
   \setlength{\abovecaptionskip}{0.05cm}
    \caption{\color{black}Effect of $\delta$ (which varies from 10\% to 30\%, 50\%, 70\%, and 90\% of $t_{max}$).}
\label{fig:query-delta}
 \end{figure*}


 \begin{figure*}	
    \begin{minipage}[t]{0.47\textwidth}
    \vspace{0pt}
        \begin{tikzpicture}
            \begin{axis}[
        	ybar,
        	bar width=0.16cm,
        	width=\textwidth,
                height=0.5\textwidth,
        	xtick={1,2,3,4,5},	
                xticklabels={GR,ST,WK,EM,CT},
        	legend style = {
                    legend columns=-1,
                    font=\footnotesize,
                    draw=none,
                    at={(1.1,1.35)/}
                },
                ylabel style={yshift=-5pt},
        	legend entries={{\tt DOTTT}, {\tt OTTC},{\tt TSRjoin},{\tt WT-Index-query}},
                legend image post style={scale=0.5, ultra thick},
    		xmin=0.5,xmax = 5.5,
    		ymin=0,ymax=100000000,
    		ymode =log,
                log origin=infty,
    		ylabel={\scriptsize \bf Time (ms)},
    		ticklabel style={font=\scriptsize},
			every axis plot/.append style={ultra thick},
			every axis/.append style={ultra thick},
                x dir=reverse,
    	]
    	\addplot[pattern=north west lines, pattern color=c1] table[x=datasets,y=DOTTT]{\baseline};
    	\addplot[pattern = grid, pattern color=c2] table[x=datasets,y=OTTC]{\baseline};
    	\addplot[pattern = crosshatch dots,pattern color=green] table[x=datasets,y=tsrjoin]{\baseline};
            \addplot[pattern = crosshatch,pattern color=c3] table[x=datasets,y=WT]{\baseline};
        \end{axis}
        \end{tikzpicture}
        \caption{Efficiency of $\delta$-temporal triangle counting.}
	\label{fig:overall}
    \end{minipage}
    \begin{minipage}[t]{0.5\textwidth}
    \vspace{0pt}
        \centering
	\ref{tsindex}\\
        \vspace{-6pt}
        \subfigure[Time cost]{
    	\begin{tikzpicture}[scale=0.5]
        		\begin{axis}[
        			ybar,
        			bar width=0.3cm,
        			width=.935\textwidth,
                        height=0.65\textwidth,
        			xtick={1,2,3,4,5},	
                        xticklabels={CT,EM,WK,ST,GR},
        			xmin=0.5,xmax = 5.5,
        			ymax=2000000000,
        			ymode =log,
        			ylabel style={yshift=-4pt},
        			ylabel={\LARGE \bf Time (ms)},
        			ticklabel style={font=\LARGE},
    				every axis plot/.append style={line width=2pt},
			          every axis/.append style={line width=2pt},
        		]
          \addplot[pattern = crosshatch dots,pattern color=green] table[x=datasets,y=TSR]{\constructiontime};
                    \addplot[pattern = crosshatch dots,pattern color=blue] table[x=datasets,y=WT]{\constructiontime};
                    
                \end{axis}
            \end{tikzpicture}
        }
        \subfigure[{\color{black}Space cost}]{
            \begin{tikzpicture}[scale=0.5]
        		\begin{axis}[
        			ybar,
        			bar width=0.2cm,
        			width=.935\textwidth,
                        height=0.65\textwidth,
        			xtick={1,2,3,4,5},	
                        xticklabels={CT,EM,WK,ST,GR},
        			legend style = {
                            legend columns=-1,
                		font=\footnotesize,
                            draw=none,
                        },
        			legend entries={{\tt Graph size},{\tt TSRjoin}, {\tt WT-Index}},
                        legend to name = tsindex,
                        legend image post style={scale=0.6},
        			xmin=0.5,xmax = 5.5,
        			ymax=100000000,
        			ymode =log,
                        log origin=infty,
        			ylabel style={yshift=-4pt},
        			ylabel={\LARGE \bf Space (mb)},
        			ticklabel style={font=\LARGE},
    				every axis plot/.append style={line width=2pt},
    				every axis/.append style={line width=2pt},
        		]
                \addplot[pattern=north west lines, pattern color=c2] table[x=datasets,y=origin]{\constructionspace};
                \addplot[pattern = crosshatch dots,pattern color=green] table[x=datasets,y=TSR]{\constructionspace};
                    \addplot[pattern = crosshatch dots,pattern color=blue] table[x=datasets,y=WT]{\constructionspace};
                    
            \end{axis}
            \end{tikzpicture}
        }
        \setlength{\abovecaptionskip}{0.07cm}
        \caption{Time and space cost of {\tt WT-Index} construction.}
	\label{fig:cons}
    \end{minipage}
\end{figure*}


% \begin{figure}
% 	\centering
% 	\begin{tikzpicture}[scale=0.65]
%     		\begin{axis}[
%     			ybar,
%     			bar width=0.35cm,
%     			width=.7\textwidth,
%                     height=0.26\textwidth,
%     			xtick={1,2,3,4,5},	
%                     xticklabels={GR,ST,WK,EM,CT},
%     			legend style = {
%                         legend columns=-1,
%             		font=\large,
%                         draw=none,
%                         at={(0.81,1)/}
%                     },
%     			legend entries={{\tt DOTTT}, {\tt OTTC},{\tt TSRjoin},{\tt WT-Index-query}},
%     			xmin=0.5,xmax = 5.5,
%     			ymin=0,ymax=100000000,
%     			ymode =log,
%                     log origin=infty,
%     			% ylabel style={yshift=-4pt},
%     			ylabel={\LARGE \bf Time (ms)},
%     			ticklabel style={font=\LARGE},
% 				every axis plot/.append style={ultra thick},
% 				every axis/.append style={ultra thick},
%                     x dir=reverse,
%     		]
%     		\addplot[pattern=north west lines, pattern color=c1] table[x=datasets,y=DOTTT]{\baseline};
%     		\addplot[pattern = grid, pattern color=c2] table[x=datasets,y=OTTC]{\baseline};
%     		\addplot[pattern = crosshatch dots,pattern color=green] table[x=datasets,y=tsrjoin]{\baseline};
%                 \addplot[pattern = crosshatch,pattern color=c3] table[x=datasets,y=WT]{\baseline};
%         \end{axis}
%         \end{tikzpicture}
%         \caption{{\color{black}Efficiency of $\delta$-temporal triangle counting.}}
% 	\label{fig:overall}
% \end{figure}

% \begin{figure}[h]	
% 	\centering
% 	\ref{named2}\\
%         \vspace{-5pt}
%         \subfigure[{\color{black}Time cost}]{
%     	\begin{tikzpicture}[scale=0.5]
%         		\begin{axis}[
%         			ybar,
%         			bar width=0.4cm,
%         			width=0.45\textwidth,
%                         height=0.3\textwidth,
%         			xtick={1,2,3,4,5},	
%                         xticklabels={CT,EM,WK,ST,GR},
%         			% legend style = {
%            %                  legend columns=-1,
%            %      		font=\huge,
%            %                  draw=none,
%            %                  at={(1,1)/}
%            %              },
%         			% legend entries={{\tt TSRjoin},{\tt WT-Index}},
%         			xmin=0.5,xmax = 5.5,
%         			ymax=2000000000,
%         			ymode =log,
%         			ylabel style={yshift=-4pt},
%         			ylabel={\huge \bf Time (ms)},
%         			ticklabel style={font=\huge},
%     				every axis plot/.append style={line width=2pt},
%     				every axis/.append style={line width=2pt},
%         		]
%           \addplot[pattern = crosshatch dots,pattern color=green] table[x=datasets,y=TSR]{\constructiontime};
%                     \addplot[pattern = crosshatch dots,pattern color=blue] table[x=datasets,y=WT]{\constructiontime};
                    
%                 \end{axis}
%             \end{tikzpicture}
%         }
%         \subfigure[{\color{black}Space cost}]{
%             \begin{tikzpicture}[scale=0.5]
%         		\begin{axis}[
%         			ybar,
%         			bar width=0.25cm,
%         			width=0.45\textwidth,
%                         height=0.3\textwidth,
%         			xtick={1,2,3,4,5},	
%                         xticklabels={CT,EM,WK,ST,GR},
%         			legend style = {
%                             legend columns=-1,
%                 		font=\footnotesize,
%                             draw=none,
%                             %at={(1,1)/}
%                         },
%         			legend entries={{\tt graph size},{\tt TSRjoin}, {\tt WT-Index}},
%                         legend to name = named2,
%                         legend image post style={scale=0.8},
%         			xmin=0.5,xmax = 5.5,
%         			ymax=100000000,
%         			ymode =log,
%                         log origin=infty,
%         			ylabel style={yshift=-4pt},
%         			ylabel={\huge \bf Space (mb)},
%         			ticklabel style={font=\huge},
%     				every axis plot/.append style={line width=2pt},
%     				every axis/.append style={line width=2pt},
%         		]
%                 \addplot[pattern=north west lines, pattern color=c2] table[x=datasets,y=origin]{\constructionspace};
%                 \addplot[pattern = crosshatch dots,pattern color=green] table[x=datasets,y=TSR]{\constructionspace};
%                     \addplot[pattern = crosshatch dots,pattern color=blue] table[x=datasets,y=WT]{\constructionspace};
                    
%             \end{axis}
%             \end{tikzpicture}
%         }
%         \caption{Time and space cost of {\tt WT-Index} construction.}
% 	\label{fig:cons}
% \end{figure}

 \begin{figure*}	
    \begin{minipage}[t]{0.48\textwidth}
    \vspace{0pt}
    \centering
        \ref{sctest}\\
        \vspace{-10pt}
        \subfigure[Time cost]{
    	\begin{tikzpicture}[scale=0.54]
        		\begin{axis}[
        			width=.9\textwidth,
                        height=0.65\textwidth,
                        xtick={1,2,3,4,5},
                        xticklabels={20\%,40\%,60\%,80\%,100\%},
        			legend style = {
                            legend columns=-1,
                            draw=none,
                        },
                        mark size=5pt,
                        line width=2.5pt,
                        legend to name=sctest,
                        legend image post style={scale=0.8, ultra thick},
        			xmin=0.5,xmax = 5.5,
        			ymin=0,ymax=100000000,
        			ymode =log,
        			ylabel={\large \bf Time (ms)},
                        ylabel style={yshift=-4pt},
        			ticklabel style={font=\large},
    				every axis plot/.append style={ultra thick},
    				every axis/.append style={ultra thick},
        		]
                    \addplot [mark=x,color=c3,line width=2.5pt] table[x=length,y=CT]{\sctime};
				\addplot [mark=o,color=c4,line width=2.5pt] table[x=length,y=EM]{\sctime};
                    \addplot [mark=square,color=c5,line width=2.5pt] table[x=length,y=WK]{\sctime};
                    \addplot [mark=triangle,color=c6,line width=2.5pt] table[x=length,y=ST]{\sctime};
                    \addplot [mark=diamond,color=c7,line width=2.5pt] table[x=length,y=GR]{\sctime};
				\legend{{\footnotesize \tt CT},{\footnotesize \tt EM}, {\footnotesize \tt WK}, {\footnotesize \tt ST}, {\footnotesize \tt GR}}
                \end{axis}
            \end{tikzpicture}
        }
        \subfigure[Space cost]{
            \begin{tikzpicture}[scale=0.54]
        		\begin{axis}[
        			width=.9\textwidth,
                        height=0.65\textwidth,
    				every axis plot/.append style={ultra thick},
                        every axis/.append style={ultra thick},	
                        mark size=5pt,
                        line width=2.5pt,
                        xtick={1,2,3,4,5},
                        xticklabels={20\%,40\%,60\%,80\%,100\%},
        			xmin=0.5,xmax = 5.5,
        			ymin=0,ymax=1000000,
        			ymode =log,
        			ylabel style={yshift=-4pt},
        			ylabel={\large \bf Space (mb)},
        			ticklabel style={font=\large},
        		]
                    \addplot [mark=x,color=c3,line width=2.5pt] table[x=length,y=CT]{\scspace};
				\addplot [mark=o,color=c4,line width=2.5pt] table[x=length,y=EM]{\scspace};
                    \addplot [mark=square,color=c5,line width=2.5pt] table[x=length,y=WK]{\scspace};
                    \addplot [mark=triangle,color=c6,line width=2.5pt] table[x=length,y=ST]{\scspace};
                    \addplot [mark=diamond,color=c7,line width=2.5pt] table[x=length,y=GR]{\scspace};
            \end{axis}
            \end{tikzpicture}
        }
        \caption{Scalability test of indexing time and space.}
	\label{fig:scalable}
    \end{minipage}
    \begin{minipage}[t]{0.48\textwidth}
    \vspace{0pt}
    \centering
	\begin{tikzpicture}[scale=0.49]
    		\begin{axis}[
    			ybar,
    			bar width=0.35cm,
    			width=2\textwidth,
                    height=0.8\textwidth,
    			xtick={1,2,3,4,5},	
                    xticklabels={GR,ST,WK,EM,CT},
    			legend style = {
                        legend columns=-1,
            		font=\LARGE,
                        draw=none,
                        at={(0.9,1.2)/}
                    },
    			legend entries={{\tt B-DOTTT}, {\tt BTTC},{\tt KDT-Index-query}},
    			xmin=0.5,xmax = 5.5,
    			ymin=0,ymax=20000000,
    			ymode =log,
                    log origin=infty,
    			ylabel={\LARGE \bf Time (ms)},
    			ticklabel style={font=\LARGE},
				every axis plot/.append style={ultra thick},
				every axis/.append style={ultra thick},
                    x dir=reverse,
    		]
    		\addplot[pattern=north west lines, pattern color=c1] table[x=datasets,y=DOTTT]{\binarybaseline};
    		\addplot[pattern = grid, pattern color=c2] table[x=datasets,y=BTTC]{\binarybaseline};
                \addplot[pattern = crosshatch,pattern color=c3] table[x=datasets,y=KD]{\binarybaseline};
        \end{axis}
        \end{tikzpicture}
        \caption{Efficiency of binary $\delta$-temporal triangle counting.}
	\label{fig:overall-binary}
    \end{minipage}
\end{figure*}

% \begin{figure}[h]	
% 	\centering
%         \ref{sctest}\\
%         \vspace{-10pt}
%         \subfigure[Time cost]{
%     	\begin{tikzpicture}[scale=0.75]
%         		\begin{axis}[
%         			width=.5\textwidth,
%                         height=0.3\textwidth,	
%                         xticklabels={,,20\%,40\%,60\%,80\%,100\%},
%         			legend style = {
%                             legend columns=-1,
%                 		font=\Large,
%                             draw=none,
%                         },
%                         mark size=5pt,
%                         line width=2.5pt,
%                         legend to name=sctest,
%                         legend image post style={scale=0.8, ultra thick},
%         			xmin=0.5,xmax = 5.5,
%         			ymin=0,ymax=100000000,
%         			ymode =log,
%         			% ylabel style={yshift=-4pt},
%         			ylabel={\normalsize \bf Time (ms)},
%         			ticklabel style={font=\large},
%     				every axis plot/.append style={ultra thick},
%     				every axis/.append style={ultra thick},
%         		]
%                     \addplot [mark=x,color=c3,line width=2.5pt] table[x=length,y=CT]{\sctime};
% 				\addplot [mark=o,color=c4,line width=2.5pt] table[x=length,y=EM]{\sctime};
%                     \addplot [mark=square,color=c5,line width=2.5pt] table[x=length,y=WK]{\sctime};
%                     \addplot [mark=triangle,color=c6,line width=2.5pt] table[x=length,y=ST]{\sctime};
%                     \addplot [mark=diamond,color=c7,line width=2.5pt] table[x=length,y=GR]{\sctime};
% 				\legend{{\small \tt CT},{\small \tt EM}, {\small \tt WK}, {\small \tt ST}, {\small \tt GR}}
%                 \end{axis}
%             \end{tikzpicture}
%         }
%         \subfigure[Space cost]{
%             \begin{tikzpicture}[scale=0.75]
%         		\begin{axis}[
%         			width=.5\textwidth,
%                         height=0.3\textwidth,
%     				every axis plot/.append style={ultra thick},
%                         every axis/.append style={ultra thick},	
%                         mark size=5pt,
%                         line width=2.5pt,
%                         xticklabels={,,20\%,40\%,60\%,80\%,100\%},
%         			xmin=0.5,xmax = 5.5,
%         			ymin=0,ymax=1000000,
%         			ymode =log,
%         			% ylabel style={yshift=-4pt},
%         			ylabel={\normalsize \bf Space (mb)},
%         			ticklabel style={font=\large},
%         		]
%                     \addplot [mark=x,color=c3,line width=2.5pt] table[x=length,y=CT]{\scspace};
% 				\addplot [mark=o,color=c4,line width=2.5pt] table[x=length,y=EM]{\scspace};
%                     \addplot [mark=square,color=c5,line width=2.5pt] table[x=length,y=WK]{\scspace};
%                     \addplot [mark=triangle,color=c6,line width=2.5pt] table[x=length,y=ST]{\scspace};
%                     \addplot [mark=diamond,color=c7,line width=2.5pt] table[x=length,y=GR]{\scspace};
%             \end{axis}
%             \end{tikzpicture}
%         }
%         \caption{Scalability test of indexing time and space.}
% 	\label{fig:scalable}
% \end{figure}



%  

% \begin{figure}[h]	
% 	\centering
% 	\begin{tikzpicture}[scale=0.7]
%     		\begin{axis}[
%     			ybar,
%     			bar width=0.35cm,
%     			width=.65\textwidth,
%                     height=0.3\textwidth,
%     			xtick={1,2,3,4,5},	
%                     xticklabels={GR,ST,WK,EM,CT},
%     			legend style = {
%                         legend columns=-1,
%             		font=\large,
%                         draw=none,
%                         at={(0.8,1)/}
%                     },
%     			legend entries={{\tt DOTTT}, {\tt EDTTC},{\tt LSC-query}},
%     			xmin=0,xmax = 6,
%     			ymin=0,ymax=100000000,
%     			ymode =log,
%                     log origin=infty,
%     			ylabel style={yshift=-4pt},
%     			ylabel={\LARGE Time (ms)},
%     			ticklabel style={font=\LARGE},
% 				every axis plot/.append style={ultra thick},
% 				every axis/.append style={ultra thick},
%     		]
%     		\addplot[pattern=north west lines, pattern color=orange] table[x=datasets,y=DOTTT]{\baseline};
%     		\addplot[pattern = grid, pattern color=blue] table[x=datasets,y=TTC]{\baseline};
%     		% \addplot[pattern = crosshatch dots,pattern color=green] table[x=datasets,y=kd-tree]{\baseline};
%                 \addplot[pattern = crosshatch dots,pattern color=red] table[x=datasets,y=LSC]{\baseline};
%         \end{axis}
%         \end{tikzpicture}
%         \caption{Efficiency of $\delta$-triangle counting on directed graphs.}
% 	\label{fig:directed}
% \end{figure}


% \begin{figure}[h]	
% 	\centering
%         \subfigure[Time cost]{
%     	\begin{tikzpicture}[scale=0.5]
%         		\begin{axis}[
%         			ybar,
%         			bar width=0.5cm,
%         			width=.45\textwidth,
%                         height=0.33\textwidth,
%         			xtick={1,2,3,4,5},	
%                         xticklabels={CT,EM,WK,ST,GR},
%         			legend style = {
%                             legend columns=-1,
%                 		font=\huge,
%                             draw=none,
%                             at={(0.55,1)/}
%                         },
%         			legend entries={{\tt KDT-Index}},
%         			xmin=0,xmax = 6,
%         			ymax=100000000,
%         			ymode =log,
%         			ylabel style={yshift=-4pt},
%         			ylabel={\huge \bf Time (ms)},
%         			ticklabel style={font=\huge},
%     				every axis plot/.append style={line width=2pt},
%     				every axis/.append style={line width=2pt},
%         		]
%                     \addplot[pattern = crosshatch dots,pattern color=red] table[x=datasets,y=KD]{\constructiontime};
%                 \end{axis}
%             \end{tikzpicture}
%         }
%         \subfigure[Space cost]{
%             \begin{tikzpicture}[scale=0.5]
%         		\begin{axis}[
%         			ybar,
%         			bar width=0.5cm,
%         			width=.45\textwidth,
%                         height=0.33\textwidth,
%         			xtick={1,2,3,4,5},	
%                         xticklabels={CT,EM,WK,ST,GR},
%         			legend style = {
%                             legend columns=-1,
%                 		font=\huge,
%                             draw=none,
%                             at={(0.55,1)/}
%                         },
%         			legend entries={{\tt KDT-Index}},
%         			xmin=0,xmax = 6,
%         			ymax=1000000, ymin=10,
%         			ymode =log,
%         			ylabel style={yshift=-4pt},
%         			ylabel={\huge \bf Space (mb)},
%         			ticklabel style={font=\huge},
%     				every axis plot/.append style={line width=2pt},
%     				every axis/.append style={line width=2pt},
%         		]
%                     \addplot[pattern = crosshatch dots,pattern color=blue] table[x=datasets,y=KD]{\constructionspace};
%             \end{axis}
%             \end{tikzpicture}
%         }
%         \caption{Time and space cost of {\tt KDT-Index} construction.}
% 	\label{fig:cons-binary}
% \end{figure}

%==========case study=============

\pgfplotstableread[row sep=\\,col sep=&]{
time&AI0& AI1 & AI2 & AI3 &DB0 &DB1 &DB2&DB3 \\
6 & 9811 & 23753 & 31651 & 34581 & 4863 & 10249 &13235 & 13428 \\
5 & 2948& 4841 & 5870 & 6415 & 2081 & 3462 &4314& 4798\\
4 & 1336& 2130 & 2553 & 2734 & 2061 & 2877 & 3959 & 4075\\
3 & 434& 593 & 864 & 896& 590 & 1168 & 1353 & 1431\\
2 & 165& 239& 254 & 261 & 309 & 471 & 504 & 511\\
1 & 9& 11& 11 & 11& 23& 32 & 32 & 32\\
}\weighted


\pgfplotstableread[row sep=\\,col sep=&]{
time&AI0& AI1 & AI2 & AI3 &DB0 &DB1 &DB2&DB3 \\
6 & 0.592 & 0.458 & 0.415 & 0.398 & 0.745 & 0.592 &0.533 & 0.512 \\
5 & 0.614& 0.469 & 0.411 & 0.388 & 0.732 & 0.586 &0.514& 0.497\\
4 & 0.76& 0.63 & 0.563 & 0.548 & 0.779 & 0.629 & 0.557& 0.523\\
3 & 0.758& 0.587 & 0.526 & 0.5& 0.787 & 0.647 & 0.584& 0.552\\
2 & 0.784& 0.617& 0.542 & 0.505 & 0.718 & 0.585 & 0.544 & 0.501\\
1 & 1& 0.73& 0.711 & 0.711& 0.563& 0.467 & 0.429 & 0.414\\
}\binary

\pgfplotstableread[row sep=\\,col sep=&]{
time&AI0& AI1 & AI2 & AI3 &DB0 &DB1 &DB2&DB3 \\
6 & 5.451 & 13.196 & 17.584 & 19.212 & 2.702 & 5.694 &7.353 & 7.46\\
5 & 1.638& 2.689 & 3.261 & 3.564 & 1.156 & 1.923 &2.397& 2.666\\
4 & 0.742& 1.183 & 1.418 & 1.519 & 1.145 & 1.598 & 2.199 & 2.264\\
3 & 0.241& 0.329 & 0.48 & 0.498& 0.328 & 0.649 & 0.752 & 0.795\\
2 & 0.092& 0.133& 0.141 & 0.145 & 0.172 & 0.262 & 0.28 & 0.284\\
1 & 0.005& 0.006& 0.006 & 0.006& 0.013& 0.018 & 0.018 & 0.018\\
}\weighteddensity



\begin{figure*}[h]
        \centering
	\ref{tbc}\\
        \vspace{-20pt}
	\subfigure[CT]{
	   \begin{tikzpicture}[scale=0.49]
	   \begin{axis}[
                    width=.39\textwidth,
                    height=0.3\textwidth,
				xticklabels={,,$20\%$,$40\%$,$60\%$,$80\%$,100\%},
				xmin=0.1,xmax=1.1,
				ymin=0,ymax=100000,
				ymode = log,
				mark size=5pt,
                    line width=2.5pt,
				ylabel={\LARGE \bf Response time (ms)},
                    ylabel style={xshift=12pt,yshift=-2pt},
				ticklabel style={font=\Large},
				every axis plot/.append style={ultra thick},
				every axis/.append style={ultra thick},
			]
			\addplot [mark=x,color=c4,line width=2.5pt] table[x=length,y=KD]{\tsvbinary};
			\addplot [mark=o,color=c6,line width=2.5pt] table[x=length,y=BTTC]{\tsvbinary};
            \addplot [mark=star,color=c7,line width=2.5pt] table[x=length,y=DOTTT]{\tsvbinary};
		\end{axis}
	\end{tikzpicture}
	}
        \subfigure[EM]{
	   \begin{tikzpicture}[scale=0.49]
	   \begin{axis}[
                    width=.39\textwidth,
                    height=0.3\textwidth,
				xticklabels={,,$20\%$,$40\%$,$60\%$,$80\%$,$100\%$},
				xmin=0.1,xmax=1.1,
				ymin=0,ymax=100000,
				ymode = log,
				mark size=5pt,
                    line width=2.5pt,
				ticklabel style={font=\Large},
				every axis plot/.append style={ultra thick},
				every axis/.append style={ultra thick},
				]
                    \addplot [mark=x,color=c4,line width=2.5pt] table[x=length,y=KD]{\emailbinary};	
                    \addplot [mark=o,color=c6,line width=2.5pt] table[x=length,y=BTTC]{\emailbinary};
                    \addplot [mark=star,color=c7,line width=2.5pt] table[x=length,y=DOTTT]{\emailbinary};
			\end{axis}
	\end{tikzpicture}
	}
        \subfigure[WK]{
	   \begin{tikzpicture}[scale=0.49]
	   \begin{axis}[
                    width=.39\textwidth,
                    height=0.3\textwidth,
				xticklabels={,,$20\%$,$40\%$,$60\%$,$80\%$,$100\%$},
				xmin=0.1,xmax=1.1,
				ymin=0,ymax=100000,
				ymode = log,
				mark size=5pt,
                    line width=2.5pt,
				ticklabel style={font=\Large},
				every axis plot/.append style={ultra thick},
				every axis/.append style={ultra thick},
				]
				\addplot [mark=x,color=c4,line width=2.5pt] table[x=length,y=KD]{\wikibinary};
				\addplot [mark=o,color=c6,line width=2.5pt] table[x=length,y=BTTC]{\wikibinary};
                \addplot [mark=star,color=c7,line width=2.5pt] table[x=length,y=DOTTT]{\wikibinary};
			\end{axis}
	\end{tikzpicture}
	}
        \subfigure[ST]{
	   \begin{tikzpicture}[scale=0.49]
	   \begin{axis}[
                    width=.39\textwidth,
                    height=0.3\textwidth,
				xticklabels={,,$20\%$,$40\%$,$60\%$,$80\%$,$100\%$},
				xmin=0.1,xmax=1.1,
				ymin=0,ymax=10000000,
				ymode = log,
				mark size=5pt,
                    line width=2.5pt,
				ticklabel style={font=\Large},
				every axis plot/.append style={ultra thick},
				every axis/.append style={ultra thick},
				]
                    \addplot [mark=x,color=c4,line width=2.5pt] table[x=length,y=KD]{\stackoverflowbinary};
				\addplot [mark=o,color=c6,line width=2.5pt] table[x=length,y=BTTC]{\stackoverflowbinary};
                \addplot [mark=star,color=c7,line width=2.5pt] table[x=length,y=DOTTT]{\stackoverflowbinary};
		\end{axis}
	\end{tikzpicture}
        }
        \subfigure[GR]{
	   \begin{tikzpicture}[scale=0.49]
	   \begin{axis}[
			legend style = {
                    legend columns=-1,
                    draw=none,
			},
                legend to name=tbc,
                legend image post style={scale=0.8, ultra thick},
                width=.39\textwidth,
                height=0.3\textwidth,
			xticklabels={,,$20\%$,$40\%$,$60\%$,$80\%$,$100\%$},
			xmin=0.1,xmax=1.1,
			% ymin=0,ymax=10000000,
			ymode = log,
			mark size=5pt,
                line width=2.5pt,
			ticklabel style={font=\Large},
			every axis plot/.append style={ultra thick},
			every axis/.append style={ultra thick},
			]
			\addplot [mark=x,color=c4,line width=2.5pt] table[x=length,y=KD]{\graphfivebinary};
			\addplot [mark=o,color=c6,line width=2.5pt] table[x=length,y=BTTC]{\graphfivebinary};
            \addplot [mark=star,color=c7,line width=2.5pt] table[x=length,y=DOTTT]{\graphfivebinary};
			\legend{{\footnotesize {\tt KDT-Index-query}}, {\footnotesize  \BTTC}, {\footnotesize {\tt B-DOTTT}}}
			\end{axis}
	\end{tikzpicture}
    }
    \setlength{\abovecaptionskip}{0.05cm}
    \caption{Effect of $(t_e-t_s)$ (which varies from
    20\% to 40\%, 60\%, 80\%, and 100\% of $t_{max}$) for binary counting.}
    \label{fig:length-binary}
\end{figure*}



\begin{figure*}[h]
        \centering
	\ref{dbc}\\
        \vspace{-20pt}
        \subfigure[CT]{
	\begin{tikzpicture}[scale = 0.49]
	   \begin{axis}[
                    width=0.39\textwidth,
                    height=0.3\textwidth,
				xtick={0,0.1,0.3,0.5,0.7,0.9},
				xticklabels={,$10\%$,$30\%$,$50\%$,$70\%$,$90\%$},
				xmin=0,xmax=1,
				ymin=0,ymax=10000,
				ymode = log,
				mark size=5pt,
                    line width=2.5pt,
                    ylabel={\LARGE \bf Response time (ms)},
				ylabel style={xshift=12pt,yshift=-2pt},
				% xlabel={\huge \bf $\delta$ length ratio},
				ticklabel style={font=\Large},
				every axis plot/.append style={ultra thick},
				every axis/.append style={ultra thick},
				]
				\addplot [mark=x,color=c4,line width=2.5pt] table[x=length,y=KD]{\tsvdbinary};
				\addplot [mark=o,color=c6,line width=2.5pt] table[x=length,y=BTTC]{\tsvdbinary};		
                    % \addplot [mark=star,color=c8,line width=2.5pt] table[x=length,y=kd-tree]{\tsvd};
                    \addplot [mark=star,color=c7,line width=2.5pt] table[x=length,y=DOTTT]{\tsvdbinary};
			\end{axis}
	\end{tikzpicture}
	}
        \subfigure[EM]{
	\begin{tikzpicture}[scale = 0.49]
	   \begin{axis}[
                    width=.39\textwidth,
                    height=0.3\textwidth,
				xtick={0,0.1,0.3,0.5,0.7,0.9},
				xticklabels={,$10\%$,$30\%$,$50\%$,$70\%$,$90\%$},
				xmin=0,xmax=1,
				ymin=0,ymax=100000,
				ymode = log,
				mark size=5pt,
                    line width=2.5pt,
				% ylabel={\huge \bf Running time (ms)},
				% ylabel style={yshift=-5pt},
				% xlabel={\huge \bf $\delta$ length ratio},
				ticklabel style={font=\Large},
				every axis plot/.append style={ultra thick},
				every axis/.append style={ultra thick},
				]
				\addplot [mark=x,color=c4,line width=2.5pt] table[x=length,y=KD]{\emaildbinary};
				\addplot [mark=o,color=c6,line width=2.5pt] table[x=length,y=BTTC]{\emaildbinary};		
                    % \addplot [mark=star,color=c8,line width=2.5pt] table[x=length,y=kd-tree]{\emaild};
                    \addplot [mark=star,color=c7,line width=2.5pt] table[x=length,y=DOTTT]{\emaildbinary};
			\end{axis}
	\end{tikzpicture}
	}
        \subfigure[WK]{
	\begin{tikzpicture}[scale = 0.49]
	   \begin{axis}[
                    width=.39\textwidth,
                    height=0.3\textwidth,
				xtick={0,0.1,0.3,0.5,0.7,0.9},
				xticklabels={,$10\%$,$30\%$,$50\%$,$70\%$,$90\%$},
				xmin=0,xmax=1,
				ymin=0,ymax=200000,
				ymode = log,
				mark size=5pt,
                    line width=2.5pt,
				% ylabel={\huge \bf Running time (ms)},
				% ylabel style={yshift=-5pt},
				% xlabel={\huge \bf $\delta$ length ratio},
				ticklabel style={font=\Large},
				every axis plot/.append style={ultra thick},
				every axis/.append style={ultra thick},
				]
				\addplot [mark=x,color=c4,line width=2.5pt] table[x=length,y=KD]{\wikidbinary};
				\addplot [mark=o,color=c6,line width=2.5pt] table[x=length,y=BTTC]{\wikidbinary};		
                    % \addplot [mark=star,color=c8,line width=2.5pt] table[x=length,y=kd-tree]{\wikid};
                    \addplot [mark=star,color=c7,line width=2.5pt] table[x=length,y=DOTTT]{\wikidbinary};
			\end{axis}
	\end{tikzpicture}
	}
        \subfigure[ST]{
	\begin{tikzpicture}[scale = 0.49]
	   \begin{axis}[
                    width=.39\textwidth,
                    height=0.3\textwidth,
				xtick={0,0.1,0.3,0.5,0.7,0.9},
				xticklabels={,$10\%$,$30\%$,$50\%$,$70\%$,$90\%$},
				xmin=0,xmax=1,
				ymin=0,ymax=10000000,
				ymode = log,
				mark size=5pt,
                    line width=2.5pt,
				% ylabel={\huge \bf Running time (ms)},
				% ylabel style={yshift=-5pt},
				% xlabel={\huge \bf $\delta$ length ratio},
				ticklabel style={font=\Large},
				every axis plot/.append style={ultra thick},
				every axis/.append style={ultra thick},
				]
				\addplot [mark=x,color=c4,line width=2.5pt] table[x=length,y=KD]{\stackoverflowdbinary};
				\addplot [mark=o,color=c6,line width=2.5pt] table[x=length,y=BTTC]{\stackoverflowdbinary};		
                    % \addplot [mark=star,color=c8] table[x=length,y=kd-tree]{\stackoverflowd};
                    \addplot [mark=star,color=c7,line width=2.5pt] table[x=length,y=DOTTT]{\stackoverflowdbinary};
			\end{axis}
	\end{tikzpicture}
	}
        \subfigure[GR]{
	\begin{tikzpicture}[scale = 0.49]
	   \begin{axis}[
			    legend style = {
				      legend columns=-1,
                        draw=none,
				},
				legend to name=dbc,
                    legend image post style={scale=0.8, ultra thick},
                    width=.39\textwidth,
                    height=0.3\textwidth,
                    xtick={0,0.1,0.3,0.5,0.7,0.9},
				xticklabels={,$10\%$,$30\%$,$50\%$,$70\%$,$90\%$},
				xmin=0,xmax=1,
				ymin=0,ymax=100000000,
				ymode = log,
				mark size=5pt,
                    line width=2.5pt,
				ticklabel style={font=\Large},
				every axis plot/.append style={ultra thick},
				every axis/.append style={ultra thick},
				]
				\addplot [mark=x,color=c4,line width=2.5pt] table[x=length,y=KD]{\graphfivedbinary};
				\addplot [mark=o,color=c6,line width=2.5pt] table[x=length,y=BTTC]{\graphfivedbinary};		
                    \addplot [mark=star,color=c7,line width=2.5pt] table[x=length,y=DOTTT]{\graphfivedbinary};
				\legend{{\footnotesize {\tt KDT-Index-query}}, {\footnotesize  \BTTC}, {\footnotesize {\tt B-DOTTT}}}
			\end{axis}
	   \end{tikzpicture}
	}
 \setlength{\abovecaptionskip}{0.05cm}
\caption{Effect of $\delta$ (which varies from 10\% to 30\%, 50\%, 70\%, and 90\% of $t_{max}$) for binary counting.}
\label{fig:query-delta-binary}
% \vspace{-0.1in}
 \end{figure*}
\subsection{Efficiency of $\delta$-Temporal Triangle Counting}
\label{sec:experiment-weighted}

$\bullet$ {\bf Overall results.}
%
For each dataset, we report the average response time of each algorithm in Figure \ref{fig:overall}. 
%
Clearly, our online algorithm \OTTC consistently outperforms \DOTTT across all datasets.
%
For example, on the CT dataset, \OTTC is 70 $\times$ faster than \DOTTT.
%
{\color{black}
The baseline index solution, {\tt TSRjoin}, performs similarly to our \OTTC.
%
This is because {\tt TSRjoin} needs to enumerate the graph and all $\delta$-temporal triangles, which is time-consuming. }
%
{\color{black}Moreover, {\tt WT-Index-query} demonstrates superior performance since it is up to eight orders of magnitude faster than \OTTC, {\tt TSRjoin}, and \DOTTT}.

% Notably, our online algorithm \OTTC outperforms \DOTTT on all datasets. For instance, on the CT dataset, \OTTC is 70$\times$ faster than \DOTTT. Additionally, {\tt WT-Index-query} achieves the best performance among all algorithms.
% Specifically, {\tt WT-Index-query} is up to eight orders of magnitude faster than \OTTC and \DOTTT.
% demonstrating at most $10^8\times$ faster execution than \OTTC. 

$\bullet$ {\bf Effect of $(t_e-t_s)$.}
%
In this experiment, we consider five different interval lengths: 20\%, 40\%, 60\%, 80\%, and 100\% of $t_{max}$.
%
For each interval length, we conduct 1,000 counting queries with $t_s$ randomly selected, where $\delta$ is consistently set to 10\% of each interval length, and depict the efficiency results in Figure \ref{fig:length}.
%
As the interval length increases, \OTTC, { \color{black}{\tt TSRjoin}}, and \DOTTT experience longer response times, since the number of $\delta$-temporal triangles increases.
%
In contrast, the response time of {\tt WT-Index} remains relatively stable, exhibiting a slight decrease. 
%
This is because when the query interval spans the entire duration $[0, t_{max}]$, the {\tt WT-Index} responses with $O(1)$ time complexity.


% To evaluate the impact of query interval lengths, we explore five different lengths for query time intervals: $20\%$, $40\%$, $60\%$, $80\%$, and $100\%$ of $t_{max}$ respectively.
% %
% For each length, we execute 1000 queries with $t_s$ randomly selected. 
% %
% The query parameter $\delta$ is set to 10\% of each interval length.
% %
% The results, depicted in Figure \ref{fig:length}, illustrate the average query response times across all five graphs.
% %
% As the time interval widens, \OTTC and \DOTTT experience increased response times. Conversely, the response time of {\tt WT-Index} remains relatively stable, exhibiting a slight decrease. This stability arises from the fact that when the query interval spans the entire duration $[0,t_{max}]$, the {\tt WT-Index} entails only $O(1)$ time complexity for the response.

% To evaluate the effect of the length of query intervals, for each graph, we fix the query $\delta$ and sampling factor as defaulted and consider five query time interval lengths, i.e., $20\%$, $40\%$, $60\%$, $80\%$, $100\%$ of $t_{max}$ respectively. For each length, we generate 10000 queries where $t_s$ is selected randomly (For the \EETTC algorithm on ST and GR, we only run 1000 queries due to the large time cost). Figure \ref{fig:length} reports the average time cost of answering one query on all five graphs as efficiency results. Note that since \DOTTT is designed only for different duration $\delta$ and it is not as efficient as our \online algorithm, we only use \online as our online algorithm. When the time interval becomes larger, \online and kd-tree take more time to respond, 
% while \LSC's cost does not change much, even taking less time. Because when the query interval is $[0,t_{max}]$, the $t_s, t_e$ indexes only cost $O(1)$ time complexity to respond.

$\bullet$ {\bf Effect of $\delta$.}
%
In this experiment, we set $[t_s,t_e]= [0, t_{max}]$, and vary $\delta = t_{max}\cdot y$ with $y\in \{10\%, 30\%, 50\%, 70\%, 90\%\}$.
%
For each $\delta$, we conduct 1,000 queries and record the average response time. The findings are summarized in Figure \ref{fig:query-delta}.
%
Clearly, the choice of $\delta$ demonstrates little impact on counting time cost because the time complexity of all {\color{black} four} algorithms is independent of $\delta$. 
%
Again, the {\tt WT-Index-query} is much faster than all other algorithms.


% In this experiment, we investigate how varying the value of $\delta$ influences query efficiency. We maintain a constant time interval length $[0,t_{max}]$ and explore different ratios of $\delta$ to the length of time interval, selecting from the set $\{10\%, 30\%, 50\%, 70\%, 90\%\}$.
% %
% For each $\delta$ value, we conduct 1,000 queries and record the average response time. The findings are summarized in Figure \ref{fig:query-delta}.
% %
% Notably, the choice of $\delta$ demonstrates minimal impact on query time. Moreover, the index-based algorithm exhibits a significant performance advantage, with response speeds up to at most eight orders of magnitude faster compared to online approaches.



$\bullet$ {\bf Time and space costs of index construction.}
%
In this experiment, we report the time and space costs of index construction for all graphs in Figure \ref{fig:cons}.
%
{\color{black} %(R2.A1)
Since building the {\tt TSRjoin} index with all the converted $O(m^2)$ edges caused an out-of-memory error even for the second smallest dataset (i.e., EM), we only use the converted edges whose length equals one of the nine different $\delta$ values used in experiments for each dataset to build the index. 
%
As a result, the time and space costs of constructing the {\tt TSRjoin} index with only $O(m)$ edges are small.
%
Clearly, the time and space costs of the {\tt WT-Index} and {\tt TSRjoin} index increase as the graphs become larger.
%
Note that the space cost of {\tt WT-Index} on ST is larger than that of GR, mainly because the $t_{max}$ of ST is larger than that of GR, leading to a larger tree height for the {\tt WT-Index} of ST.
%
We also compare the space cost of our index with the graph size in Figure \ref{fig:cons}(b).
%
Note that in the CT dataset, the space cost of {\tt WT-Index} is 5000$\times$ larger than the graph size. This is due to the graph's high density; despite having only 28,244 edges and 274 nodes, it contains 15M C-points.
%
Since the number of C-points $\pi$ is bounded by $O(m^2)$, the space cost of our {\tt WT-Index} is larger than the graph size, but it is still affordable.}

In the above experiments, the efficiency of the index-based counting algorithms is measured without considering index construction time, so it may be unfair when the indexing time should be considered. 
%
To make a fair comparison, we amortize the index construction time across the index-based queries and compare it with the online query algorithm \OTTC. 
%
Table \ref{tab:amortize} reports the number of counting queries required for our index method to surpass the online algorithm for each dataset, where the queries span the full-length time interval with randomly selected $\delta$.
%
These values are remarkably low compared to the total number of timestamps, especially for larger graphs.
%
Hence, even with a modest number of queries, the index-based algorithm consistently outperforms the online algorithm.


% {\color{red} Why do we need to repeatedly query the number of triangles in the entire graph? The $\delta$ of each query will be different.}

\begin{table}[htbp]
  \small
  \setlength{\abovecaptionskip}{0.15cm}
  \caption{Number of counting queries to offset indexing time.}
  \label{tab:amortize}
  \centering
  \begin{tabular}{c|c|c|c|c|c}
    \hline
     Dataset & CT & EM & WK & ST & GR \\
    \hline\hline
    Number of counting queries & 1.6K & 2.8K & 172 &41 & 11\\
    \hline
  \end{tabular}
\end{table}




$\bullet$ \textbf{Scalability test.}
% 
To evaluate the scalability of our index construction algorithm, we randomly select 20\%, 40\%, 60\%, 80\%, and 100\% of the edges from each graph, thereby obtaining five induced subgraphs from these edges.
%
We then build indices on these subgraphs of all datasets.
%
As shown in Figure \ref{fig:scalable}, the indexing time and space costs of our index construction algorithm increase linearly with the graph size, thereby demonstrating good scalability.

$\bullet$ {\bf Index maintenance.}
%
In this experiment, we first build the {\tt WT-Index} by using edges in the range $[0,0.8t_{max}]$, and then update {\tt WT-Index} by sequentially considering the remaining edges from $[0.8t_{max}+1, t_{max}]$.
%
{\color{black} Table \ref{tab:update} presents the average time cost of updating {\tt WT-Index} for each new edge across various datasets.
%
Our results demonstrate that the proposed index maintenance algorithm is significantly faster than rebuilding {\tt WT-Index} from scratch.}

\begin{table}[ht]
  \small
  \centering
  \setlength{\abovecaptionskip}{0.15cm}
  \caption{\color{black}Average time cost of updating {\tt WT-Index} 
 for each edge.}
  \label{tab:update}
  % \small
  \begin{tabular}{c|c|c|c|c|c}
    \hline
     Dataset & CT & EM & WK & ST & GR \\
    \hline\hline
    time costs $(ms)$ & 2.03 & 6.39 & 2.13 & 0.62 & 0.43\\
    \hline
  \end{tabular}
\end{table}
\subsection{Efficiency of Binary $\delta$-Temporal Triangle Counting}
\label{sec:experiment-binary}

$\bullet$ {\bf Overall results.}
%
For each dataset, we report the average response
time of each algorithm in Figure \ref{fig:overall-binary}. 
%
Our online algorithm \BTTC consistently outperforms {\tt B-DOTTT} across all datasets.
%
For example, on the CT dataset, \BTTC is 23$\times$ faster than {\tt B-DOTTT}. 
%
Moreover, {\tt KDT-Index-query} achieves the best performance, as it is up to four orders of magnitude faster than \BTTC.

% \begin{figure}[h]	
% 	\centering
% 	\begin{tikzpicture}[scale=0.65]
%     		\begin{axis}[
%     			ybar,
%     			bar width=0.35cm,
%     			width=.73\textwidth,
%                     height=0.26\textwidth,
%     			xtick={1,2,3,4,5},	
%                     xticklabels={GR,ST,WK,EM,CT},
%     			legend style = {
%                         legend columns=-1,
%             		font=\large,
%                         draw=none,
%                         at={(0.64,1)/}
%                     },
%     			legend entries={{\tt B-DOTTT}, {\tt BTTC},{\tt KDT-Index-query}},
%     			xmin=0.5,xmax = 5.5,
%     			ymin=0,ymax=20000000,
%     			ymode =log,
%                     log origin=infty,
%     			% ylabel style={yshift=-4pt},
%     			ylabel={\LARGE \bf Time (ms)},
%     			ticklabel style={font=\LARGE},
% 				every axis plot/.append style={ultra thick},
% 				every axis/.append style={ultra thick},
%                     x dir=reverse,
%     		]
%     		\addplot[pattern=north west lines, pattern color=c1] table[x=datasets,y=DOTTT]{\binarybaseline};
%     		\addplot[pattern = grid, pattern color=c2] table[x=datasets,y=BTTC]{\binarybaseline};
%     		% \addplot[pattern = crosshatch dots,pattern color=green] table[x=datasets,y=kd-tree]{\baseline};
%                 \addplot[pattern = crosshatch,pattern color=c3] table[x=datasets,y=KD]{\binarybaseline};
%         \end{axis}
%         \end{tikzpicture}
%         \caption{Efficiency of binary $\delta$-temporal triangle counting.}
% 	\label{fig:overall-binary}
% \end{figure}
 
$\bullet$ {\bf Effect of $(t_e-t_s)$.}
%
In this experiment, we test five different lengths: $20\%$, $40\%$, $60\%$, $80\%$, and $100\%$ of $t_{max}$, with $\delta$ set to the default value.
%
For each length, we execute 1,000 counting queries with randomly selected $t_s$ and report the average response time in Figure \ref{fig:length-binary}.
%
Again, the response time increases as the interval length increases since more binary $\delta$-temporal triangles are involved.

$\bullet$ {\bf Effect of $\delta$.}
%
In this experiment, we set $[t_s,t_e]= [0, t_{max}]$, and vary $\delta = t_{max}\cdot y$ with $y\in \{10\%, 30\%, 50\%, 70\%, 90\%\}$.
%
For each $\delta$, we conduct 1,000 queries and record the average response time.
%
The findings are summarized in Figure  \ref{fig:query-delta-binary}.
%
We observe that $\delta$ has little effect on the efficiency on \BTTC and {\tt B-DOTTT}. But it affects the efficiency of {\tt KDT-Index-query} a lot because when $\delta$ goes larger, the response time complexity of {\tt KDT-Index} approaches $O(1)$.


% $\bullet$ {\bf Time and space costs of index construction.}
%
% {\color{red}In this experiment, we detail the time and space costs of index construction for all graphs in Figure \ref{fig:cons}. 
% %
% The time and space requirements of the {\tt WT-Index} scale with the graph sizes. Notably, the space cost of ST is larger than GR, possibly because the tree height of {\tt WT-Index} of ST is larger than that of GR since the $t_{max}$ of ST is larger than that of GR.
% }
%
% In this experiment, we evaluate the time and space costs of index construction for all graphs. As shown in Figure \ref{fig:cons-binary}, the time and space requirements of the {\tt KDT-Index} scale with graph sizes.


\subsection{Case Study}
\label{sec:experiment-case-study}

We consider two temporal co-authorship graphs of papers published in database (DB) and artificial intelligence (AI) areas from 2000 to 2023, respectively.
%
Specifically, we first identify the top-50 most frequent keywords in titles of papers in SIGMOD, VLDB, and ICDE as representative DB keywords, and the top-50 most frequent keywords in titles of papers in NIPS, ICML, and ICLR as representative AI keywords (stopwords are omitted).
%
Then, we classify each paper into DB, or AI, or none of them, if the corresponding area has more representative keywords in its title.
%
Afterward, we build two temporal graphs, $G_{AI} = (V, E_{AI})$ and $G_{DB} = (V, E_{DB})$, where $V$ consists of authors who have published at least three papers at KDD, an edge $(u,v,t) \in E_{AI}$ indicates that authors $u$ and $v$ collaborate on an AI paper published in year $t$, and an edge in $E_{DB}$ indicates similar collaborations on DB papers.
%
{\color{black}
We find that $|V| = 2024$, $|E_{AI}| = 9,049$, and $|E_{DB}| = 7,093$, indicating the number of collaborations in the AI community is more than that in the DB community.}
%
Finally, we divide the whole time interval $[2000,2023]$ into six disjoint intervals, each having a 4-year length, and analyze the AI and DB communities by counting $\delta$-temporal triangles.

\begin{figure}[h]
    \centering
    \centering %图片居中
    \subfigure[$\delta \in \{0,1\}$]{
        \begin{tikzpicture} %tikz图片
        \begin{axis}[
            xlabel= {\scriptsize \bf Number of $\delta$-temporal triangles}, %横坐标名
            ylabel= {\scriptsize \bf Number of authors}, %纵坐标名
            ylabel style={yshift=-5pt},
            xlabel style={yshift=5pt},
            ymode = normal,
            xmode = log,
            xmin=0,xmax=5000,
            ymin=0,ymax=250,
            mark size=0.0pt,
            width=0.4\textwidth,
            height=0.25\textwidth,
            ticklabel style={font=\footnotesize},
            every axis plot/.append style={line width= 1.1pt},
            ytick = {10, 50, 100,200},
            xtick = {1, 10, 1e2,1e3,1e4},
            every axis/.append style={line width= 0.8pt},
            legend style = {
                at={(0.97,1)},
    		legend columns=2,
                draw=none,
                font=\Huge,
                nodes={scale=0.35, transform shape}
    	},
            legend image post style={scale=0.45, ultra thick},
         ]
        \addplot[smooth,mark=*,color=c8] table {figure/trend/ai0.txt};
        \addplot[smooth,mark=*,color = c2] table {figure/trend/db0.txt};
        %\addlegendentry{AI ($\delta$=0)}
        \addplot[smooth,mark=*,color =c7] table {figure/trend/ai1.txt};
        %\addlegendentry{AI ($\delta$=1)}
        
        %\addlegendentry{DB ($\delta$=0)}
        \addplot[smooth,mark=*,color = c4] table {figure/trend/db1.txt};
        %\addlegendentry{DB ($\delta$=1)}
        \legend{{ {AI ($\delta$=0)}}, { DB ($\delta$=0)}, { { AI ($\delta$=1)}}, { { DB ($\delta$=1)}}}
        % \addplot[smooth,mark=*,cc5] table {pic/k_0/lj.dat};
        % \addlegendentry{LJ}
        % \addplot[smooth,mark=*,cc6] table {pic/k_0/ew.dat};
        % \addlegendentry{EW}
        % \addplot[smooth,mark=*,cc7] table {pic/k_0/hw.dat};
        % \addlegendentry{HW}
        % \addplot[smooth,mark=*,cc8] table {pic/k_0/wb.dat};
        % \addlegendentry{WB}
        % \addplot[smooth,mark=*,cc9] table {pic/k_0/it.dat};
        % \addlegendentry{IT}
        % \addplot[smooth,mark=*,cc10] table {pic/k_0/uk.dat};
        % \addlegendentry{UK}
     \end{axis}
     \end{tikzpicture}
 }
 \subfigure[$\delta \in \{2,3\}$]{
        \begin{tikzpicture} %tikz图片
        \begin{axis}[
            xlabel= {\scriptsize \bf Number of $\delta$-temporal triangles}, %横坐标名
            %ylabel= {\footnotesize \bf Number of authors}, %纵坐标名
            % ylabel style={yshift=-12pt},
            xlabel style={yshift=5pt},
            ymode = normal,
            xmode = log,
            xmin=0,xmax=5000,
            ymin=0,ymax=250,
            mark size=0.0pt,
            width=0.4\textwidth,
            height=0.25\textwidth,
            ticklabel style={font=\scriptsize},
            every axis plot/.append style={line width= 1.1pt},
            ytick = {10, 50, 100,200},
            xtick = {1, 10, 1e2,1e3,1e4},
            every axis/.append style={line width= 0.8pt},
            legend style = {
                at={(0.97,1)},
    		legend columns=2,
                %font=\footnotesize,
                draw=none,
                font=\Huge,
                nodes={scale=0.35, transform shape}
    	},
            legend image post style={scale=0.45, ultra thick},
         ]
        \addplot[smooth,mark=*,color=c8] table {figure/trend/ai2.txt};
        \addplot[smooth,mark=*,color = c2] table {figure/trend/db2.txt};
        %\addlegendentry{AI ($\delta$=0)}
        \addplot[smooth,mark=*,color =c7] table {figure/trend/ai3.txt};
        %\addlegendentry{AI ($\delta$=1)}
        
        %\addlegendentry{DB ($\delta$=0)}
        \addplot[smooth,mark=*,color = c4] table {figure/trend/db3.txt};
        %\addlegendentry{DB ($\delta$=1)}
        \legend{{ {AI ($\delta$=2)}}, { DB ($\delta$=2)}, { { AI ($\delta$=3)}}, { { DB ($\delta$=3)}}}
        % \addplot[smooth,mark=*,cc5] table {pic/k_0/lj.dat};
        % \addlegendentry{LJ}
        % \addplot[smooth,mark=*,cc6] table {pic/k_0/ew.dat};
        % \addlegendentry{EW}
        % \addplot[smooth,mark=*,cc7] table {pic/k_0/hw.dat};
        % \addlegendentry{HW}
        % \addplot[smooth,mark=*,cc8] table {pic/k_0/wb.dat};
        % \addlegendentry{WB}
        % \addplot[smooth,mark=*,cc9] table {pic/k_0/it.dat};
        % \addlegendentry{IT}
        % \addplot[smooth,mark=*,cc10] table {pic/k_0/uk.dat};
        % \addlegendentry{UK}
     \end{axis}
     \end{tikzpicture}
 }
 \setlength{\abovecaptionskip}{-0.1cm}
 \setlength{\belowcaptionskip}{-2pt}
\caption{\color{black}Distribution of $\delta$-temporal triangles.}
\label{fig:triangle-distribution}
\end{figure}

$\bullet$ {\bf Collaboration density trends of DB and AI communities.} 
%
As a well-known metric of measuring the subgraph cohesiveness \cite{samusevich2016local,tsourakakis2015k}, the triangle density of a graph is defined as the number of $\delta$-temporal triangles over the number of vertices.
%
Figure \ref{fig:case-study-weighted-intro} shows the $\delta$-temporal triangle densities for the DB and AI communities across all time intervals with varying $\delta$ values.
%
We observe that after 2016, the $\delta$-temporal triangle density of the AI community surpasses that of the DB community, indicating AI's rising prominence post-2016.
%
Besides, the number of $\delta$-temporal triangles with $\delta$=1 is significantly higher than that with $\delta$=0, while the difference between $\delta$=2 and $\delta$=3  is minimal.
%
{\color{black} For instance, during the time interval [2020, 2023], the numbers of $\delta$-temporal triangles in the AI community are 9,811, 23,753, 31,651, and 34,581, when $\delta$ is set to 0, 1, 2, and 3, respectively.}
%
This suggests that authors prefer to continue collaborating over short periods.

{\color{black}
Besides, we count the number of $\delta$-temporal triangles that each author is involved in, and report the distribution in Figure \ref{fig:triangle-distribution}, where each point $(x,y)$ means that there are $y$ authors with each participating $x$ $\delta$-temporal triangles.
%
The distribution roughly follows the long-tail distribution \cite{kordumova2016exploring}, indicating that most authors engage with only a few $\delta$-temporal triangles.
%
% Similar to the triangle density, the difference in distribution is more significant between $\delta$=0 and $\delta$=1 compared to $\delta$=2 and $\delta$=3.}

\begin{figure}[h]
        \subfigure[$\delta\in \{0,1\}$]{
	\begin{tikzpicture}[scale = 0.5]
	   \begin{axis}[
                    width=0.7\textwidth,
                    height=0.42\textwidth,
				xtick={0,1,2,3,4,5,6},
				xticklabels={,2000,2004,2008,2012,2016,2020},
				xmin=0.5,xmax=6.5,
				ymin=0.2,ymax=1.4,
                    legend style = {
                        legend columns=2,
                        draw=none,
                        at={(0.9,1)/}
				},
				mark size=5pt,
                    line width=2.5pt,
				% xlabel={\huge \bf $\delta$ length ratio},
                ylabel={\huge \bf $\delta$-transitivity},
				ticklabel style={font=\huge},
				every axis plot/.append style={ultra thick},
				every axis/.append style={ultra thick},
				]
				
            \addplot [mark=x,color=c8,line width=2.5pt] table[x=time,y=AI0]{\binary};
				\addplot [mark=o,color=c2,line width=2.5pt] table[x=time,y=DB0]{\binary};
                    \addplot [mark=x,color=c7,line width=2.5pt] table[x=time,y=AI1]{\binary};
				\addplot [mark=o,color=c4,line width=2.5pt] table[x=time,y=DB1]{\binary};			\legend{{ {\LARGE AI ($\delta$=0)}}, { \LARGE DB ($\delta$=0)}, { {\LARGE AI ($\delta$=1)}}, { {\LARGE DB ($\delta$=1)}}}
                    
			\end{axis}
	\end{tikzpicture}
	}
        \subfigure[$\delta\in \{2,3\}$]{
	\begin{tikzpicture}[scale = 0.5]
	   \begin{axis}[
                    width=0.7\textwidth,
                    height=0.42\textwidth,
				xtick={0,1,2,3,4,5,6},
				xticklabels={,2000,2004,2008,2012,2016,2020},
				xmin=0.5,xmax=6.5,
				ymin=0.2,ymax=1.4,
                    legend style = {
                        legend columns=2,
                        draw=none,
                        at={(0.9,1)/}
				},
				%ymode = log,
				mark size=5pt,
                    line width=2.5pt,
				ticklabel style={font=\huge},
				every axis plot/.append style={ultra thick},
				every axis/.append style={ultra thick},
				]
					
                    \addplot [mark=x,color=c8,line width=2.5pt] table[x=time,y=AI2]{\binary};
				\addplot [mark=o,color=c2,line width=2.5pt] table[x=time,y=DB2]{\binary};
                    \addplot [mark=x,color=c7,line width=2.5pt] table[x=time,y=AI3]{\binary};
				\addplot [mark=o,color=c4,line width=2.5pt] table[x=time,y=DB3]{\binary};	
                    \legend{{ {\LARGE AI ($\delta$=2)}}, { \LARGE DB ($\delta$=2)}, { {\LARGE AI ($\delta$=3)}}, { {\LARGE DB ($\delta$=3)}}}
			\end{axis}
	\end{tikzpicture}
	}
    \setlength{\abovecaptionskip}{-0.1cm}
 \setlength{\belowcaptionskip}{-2pt}
    \caption{$\delta$-transitivity.}
    \label{fig:case-study-binary}
\end{figure}

$\bullet$ {\bf Transitivity trends of DB and AI communities.}
% 
Transitivity is a widely used metric for measuring graph sparsity \cite{chu2011triangle}.
%
We extend the $\delta$-transitivity as three times the number of binary $\delta$-temporal triangles divided by the number of binary $\delta$-temporal wedges, where a binary $\delta$-temporal wedge is a path of three vertices $u$-$v$-$w$ with the timestamp gap of the two edges not exceeding $\delta$.
%
Figure \ref{fig:case-study-binary} shows the $\delta$-transitivity of the DB and AI communities in all the six time intervals with varying $\delta$ values.
%
We observe a continuous decline in the $\delta$-transitivity of the AI community, indicating that it has become sparser as more researchers join it.

% \pgfplotstableread[row sep=\\,col sep=&]{
% 	length & LSC &wavelet tree&online & kd-tree \\
% 	0.2 & 0.05 & 0.02 & 47393& 4\\
% 	0.4 & 0.07 & 0.02 & 134577& 8\\
% 	0.6 & 0.07 & 0.02 & 225871& 13\\
% 	0.8 & 0.05 & 0.01 & 376713& 19\\
% 	1.0 & 0.01 &0.002 & 491971& 26\\
% }\stackoverflow

% \pgfplotstableread[row sep=\\,col sep=&]{
% 	length & LSC &wavelet tree&online& kd-tree \\
% 	0.2 & 0.02 & 0.01 & 64280& 15\\
% 	0.4 & 0.03 & 0.01 & 166448& 42\\
% 	0.6 & 0.02 & 0.01 & 336011& 77\\
% 	0.8 & 0.02 & 0.01 & 676620& 114\\
% 	1.0 & 0.005 &0.002 & 1133938& 146\\
% }\graphfive

% \pgfplotstableread[row sep=\\,col sep=&]{
% 	length & LSC &wavelet tree&online& kd-tree \\
% 	0.2 & 0.021 & 0.01 & 1489& 0.2\\
% 	0.4 & 0.023 & 0.01 & 3490& 0.4\\
% 	0.6 & 0.024 & 0.012 & 5718& 0.7\\
% 	0.8 & 0.022 &0.009 & 8277& 1\\
% 	1.0 & 0.007 &0.002 & 11915& 1.2\\
% }\wiki

% \pgfplotstableread[row sep=\\,col sep=&]{
% 	length & LSC &wavelet tree&online& kd-tree \\
% 	0.2 & 0.011 & 0.007 & 30& 0.02\\
% 	0.4 & 0.012 & 0.007 & 73& 0.04\\
% 	0.6 & 0.012 & 0.008 & 124& 0.06\\
% 	0.8 & 0.011 &0.006 & 181& 0.07\\
% 	1.0 & 0.004 &0.002 & 238& 0.06\\
% }\email

% \pgfplotstableread[row sep=\\,col sep=&]{
% 	length & LSC &wavelet tree&online&kd-tree \\
% 	0.2 & 0.0052 & 0.0041 & 5& 0.008\\
% 	0.4 & 0.0046 & 0.004 & 12& 0.02\\
% 	0.6 & 0.005 & 0.004 & 18& 0.02\\
% 	0.8 & 0.004 &0.004 & 25& 0.02\\
% 	1.0 & 0.003 &0.002 & 33& 0.03\\
% }\tsv



% \pgfplotstableread[row sep=\\,col sep=&]{
% 	length & LSC &DOTTT&TTC& kd-tree \\
% 	0.1 & 0.0123 & 466289 & 135077& 26\\
% 	0.3 & 0.0145 & 489479 & 129966& 24\\
% 	0.5 & 0.0122 & 456316 & 131001& 19\\
% 	0.7 & 0.0112 & 472154 & 143781& 15\\
% 	0.9 & 0.0107 & 477172 & 144079& 12\\
% }\stackoverflowd

% \pgfplotstableread[row sep=\\,col sep=&]{
% 	length & LSC &DOTTT & TTC& kd-tree \\
% 	0.1 & 0.0056 & 2235810 & 885420& 146\\
% 	0.3 & 0.0057 & 2189250 & 998820& 193\\
% 	0.5 & 0.0063 & 2107530 & 887647& 197\\
% 	0.7 & 0.0073 & 2236160 & 902199&180\\
% 	0.9 & 0.0056 &3149010 & 990356& 161\\
% }\graphfived

% \pgfplotstableread[row sep=\\,col sep=&]{
% 	length & LSC &DOTTT & TTC & kd-tree \\
% 	0.1 & 0.0073 & 44509 & 11314 & 1.2\\
% 	0.3 & 0.0081 & 43830 & 11479 & 6\\
% 	0.5 & 0.0077 & 43723 & 11713 &  14.4\\
% 	0.7 & 0.0078 & 43584 & 11376 & 28.8\\
% 	0.9 & 0.0081 & 42510 & 11391 & 44.4\\
% }\wikid

% \pgfplotstableread[row sep=\\,col sep=&]{
% 	length & LSC &DOTTT & TTC & kd-tree \\
% 	0.1 & 0.0042 & 3898 & 238 & 31\\
% 	0.3 & 0.0049 & 3781 & 211 & 20\\
% 	0.5 & 0.0052 & 3615 & 203 & 16\\
% 	0.7 & 0.0056 & 3607 & 174 & 14.6\\
% 	0.9 & 0.0054 & 3259 & 158 & 13\\
% }\emaild

% \pgfplotstableread[row sep=\\,col sep=&]{
% 	length & LSC &DOTTT&TTC& kd-tree \\
% 	0.1 & 0.0026 & 742& 37 & 0.03 \\
% 	0.3 & 0.0029 & 671& 34 & 0.018\\
% 	0.5 & 0.0030 & 656& 33 & 0.016\\
% 	0.7 & 0.0031 & 602& 29 & 0.014\\
% 	0.9 & 0.0029 & 554& 24 & 0.014\\
% }\tsvd




% \pgfplotstableread[row sep=\\,col sep=&]{
% 	length & WTTC & DOTTT \\
% 	0.1  & 135077 & 446289\\
% 	0.3  & 129966 & 489479\\
% 	0.5  & 131001 & 456316\\
% 	0.7  & 143781 & 472154\\
% 	0.9  & 144079 & 477172\\
% }\stackoverflowonline

% \pgfplotstableread[row sep=\\,col sep=&]{
% 	length & WTTC & DOTTT \\
% 	0.1  & 885420 & 2235810\\
% 	0.3  & 998820 & 2189250\\
% 	0.5  & 887647 & 2107530\\
% 	0.7  & 902199 & 2236160\\
% 	0.9  & 990356 & 2149010\\
% }\graphfiveonline

% \pgfplotstableread[row sep=\\,col sep=&]{
% 	length & WTTC & DOTTT \\
% 	0.1  & 11314 & 44509\\
% 	0.3  & 11479 & 43830\\
% 	0.5  & 11713 & 43723\\
% 	0.7  & 11376 & 43584\\
% 	0.9  & 11391 & 42510\\
% }\wikionline

% \pgfplotstableread[row sep=\\,col sep=&]{
% 	length & WTTC & DOTTT \\
% 	0.1  & 238 & 3898\\
% 	0.3  & 211 & 3781\\
% 	0.5  & 203 & 3615\\
% 	0.7  & 174 & 3607\\
% 	0.9  & 158 & 3259\\
% }\emailonline

% \pgfplotstableread[row sep=\\,col sep=&]{
% 	length & WTTC & DOTTT \\
% 	0.1  & 37 & 742\\
% 	0.3  & 34 & 671\\
% 	0.5  & 33 & 656\\
% 	0.7  & 29 & 602\\
% 	0.9  & 24 & 554\\
% }\tsvonline

% \pgfplotstableread[row sep=\\,col sep=&]{
% 	length & LSC &wavelet tree \\
% 	0.01 & 577425 & 609146 \\
% 	0.05 & 798674 & 808090 \\
% 	0.1 & 918819 & 1154172 \\
% 	0.2 & 1288712 & 1476684 \\
% 	0.3 & 1973168 & 1983985 \\
% }\stackoverflowf

% \pgfplotstableread[row sep=\\,col sep=&]{
% 	length & LSC &wavelet tree \\
% 	0.01 & 2038564 & 2172252 \\
% 	0.05 & 2189398 &  2123597\\
% 	0.1 & 2099664 &  2288220 \\
% 	0.2 & 2208948 &  2259616\\
% 	0.3 & 2288616 & 2471462 \\
% }\graphfivef

% \pgfplotstableread[row sep=\\,col sep=&]{
% 	length & LSC &wavelet tree \\
% 	0.01 & 67333 & 74486 \\
% 	0.05 & 104348 & 129932 \\
% 	0.1 & 134255 & 163451 \\
% 	0.2 & 237004 & 261851 \\
% 	0.3 & 288011 & 353856 \\
% }\wikif

% \pgfplotstableread[row sep=\\,col sep=&]{
% 	length & LSC &wavelet tree \\
% 	0.01 & 8393 & 9800 \\
% 	0.05 & 18320 & 22861 \\
% 	0.1 & 30235 & 39094 \\
% 	0.2 & 50959 & 68203 \\
% 	0.3 & 66685 & 94540 \\
% }\emailf

% \pgfplotstableread[row sep=\\,col sep=&]{
% 	length & LSC &wavelet tree \\
% 	0.01 & 637 & 702 \\
% 	0.05 & 1109 & 1316 \\
% 	0.1 & 1687 & 2014 \\
% 	0.2 & 2775 & 3319 \\
% 	0.3 &  3676 & 4454 \\
% }\tsvf



% \pgfplotstableread[row sep=\\,col sep=&]{
% 	length & LSC &wavelet tree \\
% 	0.01 & 0.053 & 0.0134 \\
% 	0.05 & 0.0647 & 0.0154 \\
% 	0.1 & 0.0748 & 0.016 \\
% 	0.2 & 0.0856 & 0.0168 \\
% 	0.3 & 0.1066 & 0.0245 \\
% }\stackoverflowfq

% \pgfplotstableread[row sep=\\,col sep=&]{
% 	length & LSC &wavelet tree \\
% 	0.01 & 0.021 & 0.0113 \\
% 	0.05 & 0.0361 &  0.0118\\
% 	0.1 & 0.0314 &  0.0117 \\
% 	0.2 & 0.0385 & 0.0121 \\
% 	0.3 & 0.0404 & 0.0132 \\
% }\graphfivefq

% \pgfplotstableread[row sep=\\,col sep=&]{
% 	length & LSC &wavelet tree& kd-tree \\
% 	0.01 & 0.0224 & 0.0091& 1 \\
% 	0.05 & 0.0331 & 0.0124& 5\\
% 	0.1 & 0.0422 & 0.0125&  12 \\
% 	0.2 & 0.0485 & 0.0136& 24 \\
% 	0.3 & 0.051 & 0.0141&37 \\
% }\wikifq

% \pgfplotstableread[row sep=\\,col sep=&]{
% 	length & LSC &wavelet tree&kd-tree \\
% 	0.01 & 0.0109 & 0.006& 0.07 \\
% 	0.05 & 0.0157 & 0.0073& 0.4 \\
% 	0.1 & 0.018 & 0.0077& 0.7 \\
% 	0.2 & 0.021 & 0.0084&1.4 \\
% 	0.3 & 0.0218 & 0.0092& 2 \\
% }\emailfq

% \pgfplotstableread[row sep=\\,col sep=&]{
% 	length & LSC &wavelet tree \\
% 	0.01 & 0.0042 & 0.0038 \\
% 	0.05 & 0.0055 & 0.0045 \\
% 	0.1 & 0.0059 & 0.0046 \\
% 	0.2 & 0.0064 & 0.0048 \\
% 	0.3 &  0.0069 & 0.005 \\
% }\tsvfq


% \pgfplotstableread[row sep=\\,col sep=&]{
% 	datasets &fetching& LSC &wavelet tree \\
% 	1 & 2064077& 11557 & 19410 \\
% 	2 & 524830& 40454 & 82720 \\
% 	3 & 47308&7936 & 15302 \\
% 	4 & 6043& 2806 & 4239 \\
% 	5 &  530& 112 & 182 \\
% }\construction

% \pgfplotstableread[row sep=\\,col sep=&]{
% 	datasets & LSC &wavelet tree & TTC \\
% 	1 & 16179 & 18022 & 12083 \\
% 	2 & 9318 & 51200 & 6451\\
% 	3 & 1638 & 7680 & 1228\\
% 	4 & 557 & 789 & 25\\
% 	5 &  44 & 58 &6\\
% }\constructionspace


% \pgfplotstableread[row sep=\\, col sep = &]{
%     length & LSC & TTC & kd-tree\\
%     0.2 & 0.0398 & 14870& 1.2947\\
%     0.4 &0.0395 & 33858& 2.8918\\
%     0.6 &0.041 & 59278& 4.5552\\
%     0.8 &0.0426 & 81808& 7.0367\\
%     1 & 0.0181& 105739&10.5806\\
% }\stackoverflowdirectcircle


% \pgfplotstableread[row sep=\\, col sep = &]{
%     length & LSC & TTC & kd-tree\\
%     0.2 & 0.0255 & 966& 0.0825\\
%     0.4 &0.0288 & 2089& 0.1846\\
%     0.6 &0.0246 & 3131& 0.2971\\
%     0.8 &0.0238 & 4568& 0.4229\\
%     1 & 0.0143& 6551&0.5386\\
% }\wikidirectcircle

% \pgfplotstableread[row sep=\\, col sep = &]{
% datasets & LSC & TTC & kd-tree & DOTTT\\
% 1 & 0.005 & 885420& 146 & 2235810\\
% 2 & 0.0123 & 135077 & 26 &466289\\
% 3 & 0.007 & 11915 & 1.2 & 44509 \\
% 4 & 0.004 & 238 & 0.06 & 3898\\
% 5 & 0.003 & 33 & 0.03 & 742\\
% }\baseline

% \begin{figure*}[ht]
%         \centering
% 	\ref{named0}\\
%         \vspace{-5pt}
% 	\subfigure[GR]{
% 	\begin{tikzpicture}[scale = 0.37]
% 	   \begin{axis}[
% 			    legend style = {
% 				    legend columns=-1,
% 				    font=\footnotesize,
%                         draw=none,
% 				},
% 				legend to name=named0,
%                     legend image post style={scale=0.8, ultra thick},
%                     width=.5\textwidth,
%                     height=0.4\textwidth,
%                     xtick={0,0.1,0.3,0.5,0.7,0.9},
% 				xticklabels={,$10\%$,$30\%$,$50\%$,$70\%$,$90\%$},
% 				xmin=0,xmax=1,
% 				ymin=0,ymax=10000000,
% 				ymode = log,
% 				mark size=4pt,
%                     line width=2.5pt,
% 				ylabel={\huge \bf Response time ($\mu$s)},
% 				ylabel style={yshift=-2pt},
% 				xlabel={\huge \bf $\delta$ length ratio},
% 				ticklabel style={font=\huge},
% 				every axis plot/.append style={ultra thick},
% 				every axis/.append style={ultra thick},
% 				]
% 				\addplot [mark=x,color=c4] table[x=length,y=LSC]{\graphfived};
				
% 				\addplot [mark=o,color=c6] table[x=length,y=TTC]{\graphfived};		
%     \addplot [mark=star,color=c8] table[x=length,y=kd-tree]{\graphfived};
% 			\addplot [mark=o,color=c7] table[x=length,y=DOTTT]{\graphfived};
% 				\legend{{\small \LSC},{\small \EDTTC}, {\small kd-tree}, {\small \DOTTT}}
% 			\end{axis}
% 	\end{tikzpicture}
% 	}
%  \quad
%  \subfigure[ST]{
% 	\begin{tikzpicture}[scale = 0.37]
% 	   \begin{axis}[
%                     width=.5\textwidth,
%                     height=0.4\textwidth,
% 				xtick={0,0.1,0.3,0.5,0.7,0.9},
% 				xticklabels={,$10\%$,$30\%$,$50\%$,$70\%$,$90\%$},
% 				xmin=0,xmax=1,
% 				ymin=0,ymax=10000000,
% 				ymode = log,
% 				mark size=4pt,
%                     line width=2.5pt,
% 				% ylabel={\huge \bf Running time (ms)},
% 				% ylabel style={yshift=-5pt},
% 				xlabel={\huge \bf $\delta$ length ratio},
% 				ticklabel style={font=\huge},
% 				every axis plot/.append style={ultra thick},
% 				every axis/.append style={ultra thick},
% 				]
% 				\addplot [mark=x,color=c4] table[x=length,y=LSC]{\stackoverflowd};
				
% 				\addplot [mark=o,color=c6] table[x=length,y=TTC]{\stackoverflowd};		
%     \addplot [mark=star,color=c8] table[x=length,y=kd-tree]{\stackoverflowd};
% 			\addplot [mark=o,color=c7] table[x=length,y=DOTTT]{\stackoverflowd};
% 			\end{axis}
% 	\end{tikzpicture}
% 	}
%  \quad
%  \subfigure[WK]{
% 	\begin{tikzpicture}[scale = 0.37]
% 	   \begin{axis}[
%                     width=.5\textwidth,
%                     height=0.4\textwidth,
% 				xtick={0,0.1,0.3,0.5,0.7,0.9},
% 				xticklabels={,$10\%$,$30\%$,$50\%$,$70\%$,$90\%$},
% 				xmin=0,xmax=1,
% 				ymin=0,ymax=100000,
% 				ymode = log,
% 				mark size=4pt,
%                     line width=2.5pt,
% 				% ylabel={\huge \bf Running time (ms)},
% 				% ylabel style={yshift=-5pt},
% 				xlabel={\huge \bf $\delta$ length ratio},
% 				ticklabel style={font=\huge},
% 				every axis plot/.append style={ultra thick},
% 				every axis/.append style={ultra thick},
% 				]
% 				\addplot [mark=x,color=c4] table[x=length,y=LSC]{\wikid};
				
% 				\addplot [mark=o,color=c6] table[x=length,y=TTC]{\wikid};		
%     \addplot [mark=star,color=c8] table[x=length,y=kd-tree]{\wikid};
% 			\addplot [mark=o,color=c7] table[x=length,y=DOTTT]{\wikid};
% 			\end{axis}
% 	\end{tikzpicture}
% 	}
%     \subfigure[EM]{
% 	\begin{tikzpicture}[scale = 0.37]
% 	   \begin{axis}[
%                     width=.5\textwidth,
%                     height=0.4\textwidth,
% 				xtick={0,0.1,0.3,0.5,0.7,0.9},
% 				xticklabels={,$10\%$,$30\%$,$50\%$,$70\%$,$90\%$},
% 				xmin=0,xmax=1,
% 				ymin=0,ymax=100000,
% 				ymode = log,
% 				mark size=4pt,
%                     line width=2.5pt,
% 				% ylabel={\huge \bf Running time (ms)},
% 				% ylabel style={yshift=-5pt},
% 				xlabel={\huge \bf $\delta$ length ratio},
% 				ticklabel style={font=\huge},
% 				every axis plot/.append style={ultra thick},
% 				every axis/.append style={ultra thick},
% 				]
% 				\addplot [mark=x,color=c4] table[x=length,y=LSC]{\emaild};
				
% 				\addplot [mark=o,color=c6] table[x=length,y=TTC]{\emaild};		
%     \addplot [mark=star,color=c8] table[x=length,y=kd-tree]{\emaild};
% 			\addplot [mark=o,color=c7] table[x=length,y=DOTTT]{\emaild};
% 			\end{axis}
% 	\end{tikzpicture}
% 	}
%  \subfigure[CT]{
% 	\begin{tikzpicture}[scale = 0.37]
% 	   \begin{axis}[
%                     width=.5\textwidth,
%                     height=0.4\textwidth,
% 				xtick={0,0.1,0.3,0.5,0.7,0.9},
% 				xticklabels={,$10\%$,$30\%$,$50\%$,$70\%$,$90\%$},
% 				xmin=0,xmax=1,
% 				ymin=0,ymax=10000,
% 				ymode = log,
% 				mark size=4pt,
%                     line width=2.5pt,
% 				% ylabel={\huge \bf Running time (ms)},
% 				% ylabel style={yshift=-5pt},
% 				xlabel={\huge \bf $\delta$ length ratio},
% 				ticklabel style={font=\huge},
% 				every axis plot/.append style={ultra thick},
% 				every axis/.append style={ultra thick},
% 				]
% 				\addplot [mark=x,color=c4] table[x=length,y=LSC]{\tsvd};
				
% 				\addplot [mark=o,color=c6] table[x=length,y=TTC]{\tsvd};		
%     \addplot [mark=star,color=c8] table[x=length,y=kd-tree]{\tsvd};
% 			\addplot [mark=o,color=c7] table[x=length,y=DOTTT]{\tsvd};
% 			\end{axis}
% 	\end{tikzpicture}
% 	}
% \caption{Effect of the ratio of query $\delta$.}
% \label{fig:query-delta}
%  \end{figure*}



% \begin{figure*}[ht]
%         \centering
% 	\ref{named1}\\
%         \vspace{-5pt}
% 	\subfigure[GR]{
% 	   \begin{tikzpicture}[scale=0.38]
% 	   \begin{axis}[
% 			    legend style = {
% 				    legend columns=-1,
% 				    font=\footnotesize,
%                         draw=none,
% 				},
% 				legend to name=named1,
%                     legend image post style={scale=0.8, ultra thick},
%                     width=.5\textwidth,
%                     height=0.4\textwidth,
%                     xtick={0,0.2,0.4,0.6,0.8,1},
% 				xticklabels={,$20\%$,$40\%$,$60\%$,$80\%$,$100\%$},
% 				xmin=0.1,xmax=1.1,
% 				ymin=0,ymax=10000000,
% 				ymode = log,
% 				mark size=4pt,
%                     line width=2.5pt,
% 				ylabel={\huge \bf Response time ($\mu$s)},
% 				ylabel style={yshift=-5pt},
% 				xlabel={\huge \bf Interval length ratio},
% 				ticklabel style={font=\huge},
% 				every axis plot/.append style={ultra thick},
% 				every axis/.append style={ultra thick},
% 				]
% 				\addplot [mark=x,color=c4] table[x=length,y=LSC]{\graphfive};
				
% 				\addplot [mark=o,color=c6] table[x=length,y=online]{\graphfive};
%     \addplot [mark=star,color=c8] table[x=length,y=kd-tree]{\graphfive};
    
% 				\legend{{\small \LSC}, {\small  \EDTTC},{\small kd-tree} }
% 			\end{axis}
% 	\end{tikzpicture}
% 	}
%         \subfigure[ST]{
% 	   \begin{tikzpicture}[scale=0.38]
% 	   \begin{axis}[
%                     width=.5\textwidth,
%                     height=0.4\textwidth,
% 				xtick={0,0.2,0.4,0.6,0.8,1},
% 				xticklabels={$0\%$,$20\%$,$40\%$,$60\%$,$80\%$,$100\%$},
% 				xmin=0.1,xmax=1.1,
% 				ymin=0,ymax=10000000,
% 				ymode = log,
% 				mark size=4pt,
%                     line width=2.5pt,
% 				% ylabel={\huge \bf Running time (ms)},
% 				% ylabel style={yshift=-5pt},
% 				xlabel={\huge \bf Interval length ratio},
% 				ticklabel style={font=\huge},
% 				every axis plot/.append style={ultra thick},
% 				every axis/.append style={ultra thick},
% 				]
%                     \addplot [mark=x,color=c4] table[x=length,y=LSC]{\stackoverflow};
				
% 				\addplot [mark=o,color=c6] table[x=length,y=online]{\stackoverflow};		
%     \addplot [mark=star,color=c8] table[x=length,y=kd-tree]{\stackoverflow};
% 			\end{axis}
% 	\end{tikzpicture}
% 	}
%         \subfigure[WK]{
% 	   \begin{tikzpicture}[scale=0.38]
% 	   \begin{axis}[
%                     width=.5\textwidth,
%                     height=0.4\textwidth,
% 				xtick={0,0.2,0.4,0.6,0.8,1},
% 				xticklabels={,$20\%$,$40\%$,$60\%$,$80\%$,$100\%$},
% 				xmin=0.1,xmax=1.1,
% 				ymin=0,ymax=100000,
% 				ymode = log,
% 				mark size=4pt,
%                     line width=2.5pt,
% 				% ylabel={\huge \bf Running time (ms)},
% 				% ylabel style={yshift=-5pt},
% 				xlabel={\huge \bf Interval length ratio},
% 				ticklabel style={font=\huge},
% 				every axis plot/.append style={ultra thick},
% 				every axis/.append style={ultra thick},
% 				]
% 				\addplot [mark=x,color=c4] table[x=length,y=LSC]{\wiki};
				
% 				\addplot [mark=o,color=c6] table[x=length,y=online]{\wiki};
%     \addplot [mark=star,color=c8] table[x=length,y=kd-tree]{\wiki};
% 			\end{axis}
% 	\end{tikzpicture}
% 	}
%         \subfigure[EM]{
% 	   \begin{tikzpicture}[scale=0.38]
% 	   \begin{axis}[
%                     width=.5\textwidth,
%                     height=0.4\textwidth,
% 				xtick={0,0.2,0.4,0.6,0.8,1},
% 				xticklabels={,$20\%$,$40\%$,$60\%$,$80\%$,$100\%$},
% 				xmin=0.1,xmax=1.1,
% 				ymin=0,ymax=100000,
% 				ymode = log,
% 				mark size=4pt,
%                     line width=2.5pt,
% 				% ylabel={\huge \bf Running time (ms)},
% 				% ylabel style={yshift=-5pt},
% 				xlabel={\huge \bf Interval length ratio},
% 				ticklabel style={font=\huge},
% 				every axis plot/.append style={ultra thick},
% 				every axis/.append style={ultra thick},
% 				]
% 				\addplot [mark=x,color=c4] table[x=length,y=LSC]{\email};
				
% 				\addplot [mark=o,color=c6] table[x=length,y=online]{\email};
%     \addplot [mark=star,color=c8] table[x=length,y=kd-tree]{\email};
% 			\end{axis}
% 	\end{tikzpicture}
% 	}
%         \subfigure[CT]{
% 	   \begin{tikzpicture}[scale=0.38]
% 	   \begin{axis}[
%                     width=.5\textwidth,
%                     height=0.4\textwidth,
% 				xtick={0,0.2,0.4,0.6,0.8,1},
% 				xticklabels={,$20\%$,$40\%$,$60\%$,$80\%$,$100\%$},
% 				xmin=0.1,xmax=1.1,
% 				ymin=0,ymax=100000,
% 				ymode = log,
% 				mark size=4pt,
%                     line width=2.5pt,
% 				% ylabel={\huge \bf Running time (ms)},
% 				% ylabel style={yshift=-5pt},
% 				xlabel={\huge \bf Interval length ratio},
% 				ticklabel style={font=\huge},
% 				every axis plot/.append style={ultra thick},
% 				every axis/.append style={ultra thick},
% 				]
% 				\addplot [mark=x,color=c4] table[x=length,y=LSC]{\tsv};
				
% 				\addplot [mark=o,color=c6] table[x=length,y=online]{\tsv};
%     \addplot [mark=star,color=c8] table[x=length,y=kd-tree]{\tsv};
% 			\end{axis}
% 	\end{tikzpicture}
% 	}
% \caption{Effect of the length of query interval.}
% \label{fig:length}
% \end{figure*}



% \subsection{Overall result}
% \label{sec:experiment-overall}
% We will provide the average response time under default settings as the overall experiment results. For each dataset, we generate 1000 queries with the default setting and record the average response time of each algorithm. Note that although weighted counting and binary counting are different problems, the online algorithms and query format of these two problems are the same. So we put the experiment results of these two problems into the same figure, Figure \ref{fig:overall}. From Figure \ref{fig:overall}, we can conclude that our online algorithm \EDTTC runs faster than \DOTTT in every dataset. Besides, the running time of \LSC remains extremely low in every dataset no matter the dataset size. The response time of kd-tree increases with the dataset size, but it remains very efficient.

% \begin{figure}[ht]	
% 	\centering

% 	\begin{tikzpicture}[scale=0.7]
%     		\begin{axis}[
%     			ybar,
%     			bar width=0.2cm,
%     			width=.5\textwidth,
%                     height=0.4\textwidth,
%     			xtick={1,2,3,4,5},	xticklabels={GR,ST,WK,EM,CT},
%     			x tick label style={rotate=0},
%     			legend style = {
%                         legend columns=-1,
%             		font=\footnotesize,
%                         draw=none,
%                     },
%     			legend entries={DOTTT, EDTTC, kd-tree, LSC},
%     			xmin=0,xmax = 6,
%     			ymin=0,ymax=100000000,
%     			ymode =log,
%             log origin=infty,
%     			ylabel style={yshift=-5pt},
%     			ylabel={\Large Average response time ($\mu$s)},
%     			ticklabel style={font=\large},
% 				every axis plot/.append style={ultra thick},
% 				every axis/.append style={ultra thick},
%     			]
%     			\addplot[pattern=north west lines, pattern color=orange] table[x=datasets,y=DOTTT]{\baseline};
%     			\addplot[pattern = grid, pattern color=blue] table[x=datasets,y=TTC]{\baseline};
%     			\addplot[pattern = crosshatch dots,pattern color=green] table[x=datasets,y=kd-tree]{\baseline};
%        \addplot[pattern = crosshatch dots,pattern color=red] table[x=datasets,y=LSC]{\baseline};
%     		\end{axis}
%     \end{tikzpicture}
%     	\caption{Overall result under default settings on different datasets.}
% 	\label{fig:overall}
% \end{figure}

% \subsection{Effect of different query settings}
% \label{sec:experiment-query}
% \subsubsection{Effect of the ratio of query duration $\delta$}

% We will first evaluate the effect of query $\delta$. For each graph, we fix the query interval length and sampling factor as defaulted. We consider five types of $\delta$, i.e., $10\%$, $30\%$, $50\%$, $70\%$, $90\%$ of the query interval length respectively. For each $\delta$, we generate 1000 queries where $t_s$ is selected randomly. Figure \ref{fig:query-delta} shows the testing result. Note that only kd-tree's time complexity is affected by the duration $\delta$. So only the response time of kd-tree varies with the change of $\delta$, while the other three algorithms' response times don't change too much. 

 


% \subsubsection{Effect of the length of query intewrval}


% To evaluate the effect of the length of query intervals, for each graph, we fix the query $\delta$ and sampling factor as defaulted and consider five query time interval lengths, i.e., $20\%$, $40\%$, $60\%$, $80\%$, $100\%$ of $t_{max}$ respectively. For each length, we generate 10000 queries where $t_s$ is selected randomly (For the \EDTTC algorithm on ST and GR, we only run 1000 queries due to the large time cost). Figure \ref{fig:length} reports the average time cost of answering one query on all five graphs as efficiency results. Note that since \DOTTT is designed only for different duration $\delta$ and it is not as efficient as our \online algorithm, we only use \online as our online algorithm. When the time interval becomes larger, \online and kd-tree take more time to respond, 
% while \LSC's cost does not change much, even taking less time. Because when the query interval is $[0,t_{max}]$, the $t_s, t_e$ indexes only cost $O(1)$ time complexity to respond.

% \subsection{Effect of different construction settings}
% \label{sec:experiment-construction-settings}
% \subsubsection{Weighted counting}
% %
% We analyze the query runtime variation caused by different sampling factors as well. For each graph, we choose five different values of sampling factor $k$, i.e., $1\%$, $5\%$, $10\%$, $20\%$, and $30\%$. We generate 10000 queries with default attributes and select $t_s$ randomly. We answer the queries with indexes with different sampling factors and compute the average responding time. Note the increasing rate of responding time is much slower than the increasing rate of sampling factor.


% \begin{figure}[ht]
%         \centering
% 	\ref{named4}\\
% 	\subfigure[WK]{
% 	\begin{tikzpicture}[scale=0.38]
% 	   \begin{axis}[
% 			    legend style = {
% 				    legend columns=-1,
% 				    font=\footnotesize,
%                         draw=none,
% 				},
% 				legend to name=named4,
%                 legend image post style={scale=0.8, ultra thick},
%                  width=.5\textwidth,
%                 height=0.4\textwidth,
%                 xtick={0.01,0.05,0.1,0.2,0.3},
% 				xticklabels={$1\%$,$5\%$,$10\%$,$20\%$,$30\%$},
% 				xmin=0,xmax=0.3,
% 				ymin=0,ymax=0.1,
% 				ymode = log,
% 				mark size=4pt,
% 				ylabel={\huge \bf Running time ($\mu$s)},
% 				ylabel style={yshift=-5pt},
% 				xlabel={\huge \bf sampling factor},
% 				ticklabel style={font=\huge},
% 				every axis plot/.append style={ultra thick},
% 				every axis/.append style={ultra thick},
% 				]
% 				\addplot [mark=x,color=c4] table[x=length,y=LSC]{\wikifq};
				
				
				
% 				\legend{{\small \LSC}, {\small kd-tree}, {\small  \EDTTC}}
% 			\end{axis}
% 	\end{tikzpicture}
% 	}
%  \subfigure[EM]{
% 	\begin{tikzpicture}[scale=0.38]
% 	   \begin{axis}[
%     width=.5\textwidth,
%                 height=0.4\textwidth,
% 				xtick={0.01,0.05,0.1,0.2,0.3},
% 				xticklabels={$1\%$,$5\%$,$10\%$,$20\%$,$30\%$},
% 				xmin=0,xmax=0.3,
% 				ymin=0,ymax=0.1,
% 				ymode = log,
% 				mark size=4pt,
% 				xlabel={\huge \bf sampling factor},
% 				ticklabel style={font=\huge},
% 				every axis plot/.append style={ultra thick},
% 				every axis/.append style={ultra thick},
% 				]
% 				\addplot [mark=x,color=c4] table[x=length,y=LSC]{\emailfq};
				
				
% 			\end{axis}
% 	\end{tikzpicture}
% 	}
%  \caption{Effect of the sampling factor for querying.}
% \label{fig:factor}
% \end{figure}


\begin{comment}
     \subfigure[WK]{
	\begin{tikzpicture}[scale=0.28]
	   \begin{axis}[
				xtick={0.01,0.05,0.1,0.2,0.3},
				xticklabels={$1\%$,$5\%$,$10\%$,$20\%$,$30\%$},
				xmin=0,xmax=0.3,
				ymin=0,ymax=1,
				ymode = log,
				mark size=4pt,
				ylabel={\huge \bf Running time (ms)},
				ylabel style={yshift=-5pt},
				xlabel={\huge \bf sampling factor},
				ticklabel style={font=\huge},
				every axis plot/.append style={ultra thick},
				every axis/.append style={ultra thick},
				]
				\addplot [mark=x,color=c4] table[x=length,y=LSC]{\wikifq};
				\addplot [mark=square,color=c5] table[x=length,y=wavelet tree]{\wikifq};
				
			\end{axis}
	\end{tikzpicture}
	}
 \subfigure[EM]{
	\begin{tikzpicture}[scale=0.28]
	   \begin{axis}[
				xtick={0.01,0.05,0.1,0.2,0.3},
				xticklabels={$1\%$,$5\%$,$10\%$,$20\%$,$30\%$},
				xmin=0,xmax=0.3,
				ymin=0,ymax=1,
				ymode = log,
				mark size=4pt,
				ylabel={\huge \bf Running time (ms)},
				ylabel style={yshift=-5pt},
				xlabel={\huge \bf sampling factor},
				ticklabel style={font=\huge},
				every axis plot/.append style={ultra thick},
				every axis/.append style={ultra thick},
				]
				\addplot [mark=x,color=c4] table[x=length,y=LSC]{\emailfq};
				\addplot [mark=square,color=c5] table[x=length,y=wavelet tree]{\emailfq};
				
			\end{axis}
	\end{tikzpicture}
	}
 \subfigure[CT]{
	\begin{tikzpicture}[scale=0.28]
	   \begin{axis}[
				xtick={0.01,0.05,0.1,0.2,0.3},
				xticklabels={$1\%$,$5\%$,$10\%$,$20\%$,$30\%$},
				xmin=0,xmax=0.3,
				ymin=0,ymax=1,
				ymode = log,
				mark size=4pt,
				ylabel={\huge \bf Running time (ms)},
				ylabel style={yshift=-5pt},
				xlabel={\huge \bf sampling factor},
				ticklabel style={font=\huge},
				every axis plot/.append style={ultra thick},
				every axis/.append style={ultra thick},
				]
				\addplot [mark=x,color=c4] table[x=length,y=LSC]{\tsvfq};
				\addplot [mark=square,color=c5] table[x=length,y=wavelet tree]{\tsvfq};
				
			\end{axis}
	\end{tikzpicture}
	}

  


Last, we will show the affection of the sampling factor to the entire processing time, including index construction and query solving. We choose five different values of sampling factor $k$, i.e., $1\%$, $5\%$, $10\%$, $20\%$, and $30\%$. Figure 16 reports the processing results. Note that  Additionally, even the construction time complexity of LSC and wavelet tree are the same. The construction of LSC is faster than that of the wavelet tree, possibly because it does not need any recursion, which reduces the constant factor.


\begin{figure}[ht]
        \centering
	\ref{named3}\\
	\subfigure[WK]{
	\begin{tikzpicture}[scale=0.38]
	   \begin{axis}[
			    legend style = {
				    legend columns=-1,
				    font=\footnotesize,
                        draw=none,
				},
				legend to name=named3,
                legend image post style={scale=0.8, ultra thick},
                width=.5\textwidth,
                height=0.4\textwidth,
                xtick={0.01,0.05,0.1,0.2,0.3},
				xticklabels={$1\%$,$5\%$,$10\%$,$20\%$,$30\%$},
				xmin=0,xmax=0.3,
				ymin=0,ymax=400000,
				ymode = normal,
				mark size=4pt,
				ylabel={\huge \bf Processing time (ms)},
				ylabel style={yshift=-5pt},
				xlabel={\huge \bf sampling factor},
				ticklabel style={font=\huge},
				every axis plot/.append style={ultra thick},
				every axis/.append style={ultra thick},
				]
				\addplot [mark=x,color=c4] table[x=length,y=LSC]{\wikif};
				\addplot [mark=square,color=c5] table[x=length,y=wavelet tree]{\wikif};
				
				
				\legend{{\small LSC}, {\small Wavelet Tree}, {\small  DOTTT}}
			\end{axis}
	\end{tikzpicture}
	}
 \subfigure[EM]{
	\begin{tikzpicture}[scale=0.38]
	   \begin{axis}[
                width=.5\textwidth,
                height=0.4\textwidth,
				xtick={0.01,0.05,0.1,0.2,0.3},
				xticklabels={$1\%$,$5\%$,$10\%$,$20\%$,$30\%$},
				xmin=0,xmax=0.3,
				ymin=0,ymax=100000,
				ymode = normal,
				mark size=4pt,
				xlabel={\huge \bf sampling factor},
				ticklabel style={font=\huge},
				every axis plot/.append style={ultra thick},
				every axis/.append style={ultra thick},
				]
				\addplot [mark=x,color=c4] table[x=length,y=LSC]{\emailf};
				\addplot [mark=square,color=c5] table[x=length,y=wavelet tree]{\emailf};
				
			\end{axis}
	\end{tikzpicture}
	}

     \subfigure[GR]{
	\begin{tikzpicture}[scale=0.28]
	   \begin{axis}[
				xtick={0.01,0.05,0.1,0.2,0.3},
				xticklabels={$1\%$,$5\%$,$10\%$,$20\%$,$30\%$},
				xmin=0,xmax=0.3,
				ymin=0,ymax=5000000,
				ymode = normal,
				mark size=4pt,
				ylabel={\huge \bf Construction time (ms)},
				ylabel style={yshift=-5pt},
				xlabel={\huge \bf sampling factor},
				ticklabel style={font=\huge},
				every axis plot/.append style={ultra thick},
				every axis/.append style={ultra thick},
				]
				\addplot [mark=x,color=c4] table[x=length,y=LSC]{\graphfivef};
				\addplot [mark=square,color=c5] table[x=length,y=wavelet tree]{\graphfivef};
				
			\end{axis}
	\end{tikzpicture}
	}
 
  \subfigure[WK]{
	\begin{tikzpicture}[scale=0.28]
	   \begin{axis}[
				xtick={0.01,0.05,0.1,0.2,0.3},
				xticklabels={$1\%$,$5\%$,$10\%$,$20\%$,$30\%$},
				xmin=0,xmax=0.3,
				ymin=0,ymax=500000,
				ymode = normal,
				mark size=4pt,
				ylabel={\huge \bf Construction time (ms)},
				ylabel style={yshift=-5pt},
				xlabel={\huge \bf sampling factor},
				ticklabel style={font=\huge},
				every axis plot/.append style={ultra thick},
				every axis/.append style={ultra thick},
				]
				\addplot [mark=x,color=c4] table[x=length,y=LSC]{\wikif};
				\addplot [mark=square,color=c5] table[x=length,y=wavelet tree]{\wikif};
				
			\end{axis}
	\end{tikzpicture}
	}
 
 \subfigure[CT]{
	\begin{tikzpicture}[scale=0.28]
	   \begin{axis}[
				xtick={0.01,0.05,0.1,0.2,0.3},
				xticklabels={$1\%$,$5\%$,$10\%$,$20\%$,$30\%$},
				xmin=0,xmax=0.3,
				ymin=0,ymax=10000,
				ymode = normal,
				mark size=4pt,
				ylabel={\huge \bf Construction time (ms)},
				ylabel style={yshift=-5pt},
				xlabel={\huge \bf sampling factor},
				ticklabel style={font=\huge},
				every axis plot/.append style={ultra thick},
				every axis/.append style={ultra thick},
				]
				\addplot [mark=x,color=c4] table[x=length,y=LSC]{\tsvf};
				\addplot [mark=square,color=c5] table[x=length,y=wavelet tree]{\tsvf};
				
			\end{axis}
	\end{tikzpicture}
	}

 
\caption{Effect of the sampling factor for the entire process.}
\label{fig:factor}
\end{figure}

\end{comment}

% Last, we will report the accuracy of the sampling algorithm with different sampling factors. For each graph, we choose five different values of sampling factor $k$, i.e., $1\%$, $5\%$, $10\%$, $20\%$, and $30\%$. We generate 10000 queries with default attributes and select $t_s$ randomly. Suppose the exact answer is $a$ and the sampling answer is $b$, we will measure the accuracy by using the relative error, $\frac{|b-a|}{a}$. As is shown in table \ref{tab:accuracy}, storing only $1\%$ of the C-points can keep the relative error less than one thousandth. 
% And if we store $20\%$ of the points, we can keep the relative error less than one ten-thousandth. Thus, the sampling algorithm performs well on large graphs.
% \begin{table}[ht]
% \centering
%     \scalebox{0.8}{
%     \begin{tabular}{c|c|c|c|c|c}
%     \hline
%      \diagbox{Datasets}{$k$}    &  $1\%$ & $5\%$ & $10\%$ & $20\%$ & $30\%$ \\
%     \hline\hline
%          GR& $0.0666\%$ & $0.0361\%$ & $0.0208\%$ & $0.0071\%$& $0.0039\%$
% \\
%     \hline
%      ST & $0.0166\%$ & $0.0098\%$ & $0.0076\%$ & $0.0095\%$ & $0.0073\%$\\
%      \hline
%     \end{tabular}
%     }
%     \caption{Sampling accuracy measured by relative error for weighted counting.}
%     \label{tab:accuracy}
% \end{table}
%

% \subsubsection{Binary counting}

% We will first evaluate the effect of the length of the query intervals. Similar to what we did in the weighted part, for each graph, we fix the time intervals, i.e., $20\%$, $40\%$, $60\%$, $80\%$, $100\%$ of $t_{max}$ respectively. For each length, we generate 10000 queries where $t_s$ is selected randomly. Figure \ref{fig:length} reports the average time cost of answering one query on all five graphs as efficiency results. Note that although binary counting is a 3D point counting problem and harder than weighted counting, our index-based solution still runs up to $10^4\times$ faster than the online algorithm.

\begin{comment}
Then we will evaluate the influence of query $\delta$. For each graph, we fix the query interval length and sampling factor as defaulted. We consider five types of $\delta$, i.e., $10\%$, $30\%$, $50\%$, $70\%$, $90\%$ of the query interval length respectively. For each $\delta$, we generate 10000 queries where $t_s$ is selected randomly. Figure \ref{fig:deltaB} shows the testing result. Note that for kd-tree, $\delta$ also affects the running time, which is different from the other algorithms. This is because the converted binary counting is a 3D counting problem and $\delta$ is one of the dimensions. 

\begin{figure}[ht]
        \centering
	\ref{named6}\\
	\subfigure[GR]{
	\begin{tikzpicture}[scale=0.38]
	   \begin{axis}[
			    legend style = {
				    legend columns=-1,
				    font=\footnotesize,
                        draw=none,
				},
				legend to name=named6,
                legend image post style={scale=0.8, ultra thick},
                width=.5\textwidth,
                height=0.4\textwidth,
                xtick={0,0.1,0.3,0.5,0.7,0.9},
				xticklabels={$0\%$,$10\%$,$30\%$,$50\%$,$70\%$,$90\%$},
				xmin=0,xmax=1,
				ymin=0,ymax=200,
				ymode = normal,
				mark size=4pt,
				ylabel={\huge \bf Running time ($\mu$s)},
				ylabel style={yshift=-5pt},
				xlabel={\huge \bf $\delta$ length ratio},
				ticklabel style={font=\huge},
				every axis plot/.append style={ultra thick},
				every axis/.append style={ultra thick},
				]
				\addplot [mark=star,color=c8] 
				table[x=length,y=kd-tree]{\graphfived};
				\legend{{\small  kd-tree}}
			\end{axis}
	\end{tikzpicture}
	}
 \subfigure[ST]{
	\begin{tikzpicture}[scale=0.38]
	   \begin{axis}[
                width=.5\textwidth,
                height=0.4\textwidth,
				xtick={0,0.1,0.3,0.5,0.7,0.9},
				xticklabels={$0\%$,$10\%$,$30\%$,$50\%$,$70\%$,$90\%$},
				xmin=0,xmax=1,
				ymin=0,ymax=20,
				ymode = normal,
				mark size=4pt,
				xlabel={\huge \bf $\delta$ length ratio},
				ticklabel style={font=\huge},
				every axis plot/.append style={ultra thick},
				every axis/.append style={ultra thick},
				]
				\addplot [mark=star,color=c8] 
				table[x=length,y=kd-tree]{\stackoverflowd};
			\end{axis}
	\end{tikzpicture}
	}
 \caption{Effect of the ratio of query $\delta$.}
\label{fig:deltaB}
\end{figure}

\end{comment}

% We analyze the query runtime variation caused by different sampling factors as well. The method of evaluation is the same as the one for weighted counting. As is shown in Figure \ref{fig:factorB}, the increase rate of kd-tree's response time is linear to the sampling factor's increase rate.


% \begin{figure}[ht]
%         \centering
% 	\ref{named6}\\
% 	\subfigure[WK]{
% 	\begin{tikzpicture}[scale=0.38]
% 	   \begin{axis}[
% 			    legend style = {
% 				    legend columns=-1,
% 				    font=\footnotesize,
%                         draw=none,
% 				},
% 				legend to name=named6,
%                 legend image post style={scale=0.8, ultra thick},
%                  width=.5\textwidth,
%                 height=0.4\textwidth,
%                 xtick={0.01,0.05,0.1,0.2,0.3},
% 				xticklabels={$1\%$,$5\%$,$10\%$,$20\%$,$30\%$},
% 				xmin=0,xmax=0.3,
% 				ymin=0,ymax=40,
% 				ymode = normal,
% 				mark size=4pt,
% 				ylabel={\huge \bf Running time ($\mu$s)},
% 				ylabel style={yshift=-5pt},
% 				xlabel={\huge \bf sampling factor},
% 				ticklabel style={font=\huge},
% 				every axis plot/.append style={ultra thick},
% 				every axis/.append style={ultra thick},
% 				]
% 				\addplot [mark=star,color=c8] table[x=length,y=kd-tree]{\wikifq};
				
				
% 				\legend{{\small  kd-tree}}
% 			\end{axis}
% 	\end{tikzpicture}
% 	}
%  \subfigure[EM]{
% 	\begin{tikzpicture}[scale=0.38]
% 	   \begin{axis}[
%     width=.5\textwidth,
%                 height=0.4\textwidth,
% 				xtick={0.01,0.05,0.1,0.2,0.3},
% 				xticklabels={$1\%$,$5\%$,$10\%$,$20\%$,$30\%$},
% 				xmin=0,xmax=0.3,
% 				ymin=0,ymax=5,
% 				ymode = normal,
% 				mark size=4pt,
% 				xlabel={\huge \bf sampling factor},
% 				ticklabel style={font=\huge},
% 				every axis plot/.append style={ultra thick},
% 				every axis/.append style={ultra thick},
% 				]
% 				\addplot [mark=star,color=c8] table[x=length,y=kd-tree]{\emailfq};
				
				
% 			\end{axis}
% 	\end{tikzpicture}
% 	}
%  \caption{Effect of the sampling factor for querying.}
% \label{fig:factorB}
% \end{figure}

% Last, we will report the accuracy of the sampling algorithm with different sampling factors. The setup and measurement are the same as what we do in the weighted counting part.
% As is shown in table \ref{tab:accuracyB}, storing only $1\%$ of the C-points can keep the relative error around one thousandth. 
% And if we store $30\%$ of the points, we can keep the relative error less than three ten-thousandth. Thus, the sampling algorithm also performs well on large graphs for binary counting.
% \begin{table}[ht]
% \centering
%     \scalebox{0.8}{
%     \begin{tabular}{c|c|c|c|c|c}
%     \hline
%      \diagbox{Datasets}{$k$}    &  $1\%$ & $5\%$ & $10\%$ & $20\%$ & $30\%$ \\
%     \hline\hline
%          GR& $0.0323\%$ & $0.0664\%$ & $0.0226\%$ & $0.0212\%$& $0.0264\%$
% \\
%     \hline
%      ST & $0.1675\%$ & $0.1388\%$ & $0.0787\%$ & $0.0456\%$ & $0.0284\%$\\
%      \hline
%     \end{tabular}
%     }
%     \caption{Sampling accuracy measured by relative error for binary counting.}
%     \label{tab:accuracyB}
% \end{table}
%


% \subsection{Index construction}
% \label{sec:experiment-index}
% \subsubsection{\LSC index}

% We report the time and space cost of index construction on all graphs with the default sampling factor in Table \ref{tab:LSC-construction}. Note that, the space costs of the \LSC index match the sizes of the large graphs. For the small graph, the ratio of $\frac{m}{n}$ is large so that the number of triangles will be far larger than the number of edges of the small graph. For instance, to address the weighted counting problem in dataset WK, we need to store 6599388 C-points in \LSC, while for EM, 3634450 C-points will be stored. Although WK is 23 times larger than EM, its C-point number is just two times that of EM. So the space costs of the index are relatively large in the small graphs.


% \begin{table}[ht]
%     \centering
%     \scalebox{0.88}{
%     \begin{tabular}{c|c|c|c|c|c}
%     \hline
%     \diagbox{costs}{Datasets}     & GR&ST&WK&EM&CT \\
%     \hline\hline
%      time($\mu$s)    & 5,636,728 & 8,491,875 & 723,897& 292,519& 11,488\\
%      \hline
%      space(mb) & 77,107 & 339,456 & 81,612 & 26,215 & 1,946\\
%      \hline
%     \end{tabular}
%     }
%     \caption{Construction costs for weighted counting.}
%     \label{tab:LSC-construction}
% \end{table}

% \begin{figure}[h]	
% 	\centering
% 	\begin{tikzpicture}[scale=0.37]
%     		\begin{axis}[
%     			ybar,
%     			bar width=0.8cm,
%     			width=.65\textwidth,
%                     height=0.3\textwidth,
%     			xtick={1,2,3,4,5},	
%                     xticklabels={GR,ST,WK,EM,CT},
%     			legend style = {
%                         legend columns=-1,
%             		font=\large,
%                         draw=none,
%                         at={(0.8,1)/}
%                     },
%     			legend entries={{\tt LSC}},
%     			xmin=0,xmax = 6,
%     			ymin=0,ymax=100000000,
%     			ymode =log,
%                     log origin=infty,
%     			ylabel style={yshift=-4pt},
%     			ylabel={\LARGE Time ($\mu$s)},
%     			ticklabel style={font=\LARGE},
% 				every axis plot/.append style={ultra thick},
% 				every axis/.append style={ultra thick},
%     		]
%     		% \addplot[pattern=north west lines, pattern color=orange] table[x=datasets,y=DOTTT]{\baseline};
%     		% \addplot[pattern = grid, pattern color=blue] table[x=datasets,y=TTC]{\baseline};
%     		% % \addplot[pattern = crosshatch dots,pattern color=green] table[x=datasets,y=kd-tree]{\baseline};
%                 \addplot[pattern = crosshatch dots,pattern color=red] table[x=datasets,y=LSC]{\constructiontime};
%         \end{axis}
%         \end{tikzpicture}

%         \begin{tikzpicture}[scale=0.37]
%     		\begin{axis}[
%     			ybar,
%     			bar width=0.8cm,
%     			width=.65\textwidth,
%                     height=0.3\textwidth,
%     			xtick={1,2,3,4,5},	
%                     xticklabels={GR,ST,WK,EM,CT},
%     			legend style = {
%                         legend columns=-1,
%             		font=\large,
%                         draw=none,
%                         at={(0.8,1)/}
%                     },
%     			legend entries={{\tt LSC}},
%     			xmin=0,xmax = 6,
%     			ymin=0,ymax=1000000,
%     			ymode =log,
%                     log origin=infty,
%     			ylabel style={yshift=-4pt},
%     			ylabel={\LARGE Space (mb)},
%     			ticklabel style={font=\LARGE},
% 				every axis plot/.append style={ultra thick},
% 				every axis/.append style={ultra thick},
%     		]
%     		% \addplot[pattern=north west lines, pattern color=orange] table[x=datasets,y=DOTTT]{\baseline};
%     		% \addplot[pattern = grid, pattern color=blue] table[x=datasets,y=TTC]{\baseline};
%     		% % \addplot[pattern = crosshatch dots,pattern color=green] table[x=datasets,y=kd-tree]{\baseline};
%                 \addplot[pattern = crosshatch dots,pattern color=blue] table[x=datasets,y=LSC]{\constructionspace};
%         \end{axis}
%         \end{tikzpicture}
%         \caption{Index construction time and space.}
% 	\label{fig:overall}
% \end{figure}




\begin{comment}
\begin{figure}[h]	
	\centering
 \subfigure[time costs]{
	\begin{tikzpicture}[scale=0.38]
    		\begin{axis}[
    			ybar,
    			bar width=0.2cm,
    			width=.5\textwidth,
                    height=0.4\textwidth,
    			xtick=data,	xticklabels={GR,ST,WK,EM,CT},
    			x tick label style={rotate=-15,anchor=west},
    			legend style = {
                        legend columns=-1,
            		font=\footnotesize,
                        draw=none,
                    },
    			legend entries={TTM, LSC, wavelet tree},
    			xmin=0,xmax = 6,
    			ymin=0,ymax=20000000,
    			ymode =log,
    			ylabel style={yshift=-5pt},
    			ylabel={\Large Constrution time ($\mu$s)},
    			ticklabel style={font=\large},
				every axis plot/.append style={ultra thick},
				every axis/.append style={ultra thick},
    			]
    			\addplot[pattern=north west lines, pattern color=orange] table[x=datasets,y=fetching]{\construction};
    			\addplot[pattern = grid, pattern color=blue] table[x=datasets,y=LSC]{\construction};
    			\addplot[pattern = crosshatch dots,pattern color=green] table[x=datasets,y=wavelet tree]{\construction};
    		\end{axis}
    \end{tikzpicture}
    }
    \subfigure[space costs]{
	\begin{tikzpicture}[scale=0.38]
    		\begin{axis}[
    			ybar,
    			bar width=0.2cm,
    			width=.5\textwidth,
                    height=0.4\textwidth,
    			xtick=data,	xticklabels={GR,ST,WK,EM,CT},
    			x tick label style={rotate=-15,anchor=west},
                    legend style = {
                        legend columns=-1,
            		font=\footnotesize,
                        draw=none,
                    },
    			legend entries={LSC, wavelet tree},
    			xmin=0,xmax = 6,
    			ymin=0,ymax=200000,
    			ymode =log,
    			ylabel style={yshift=-5pt},
    			ylabel={\Large Space cost (mb)},
    			ticklabel style={font=\large},
				every axis plot/.append style={ultra thick},
				every axis/.append style={ultra thick},
    			]
    			\addplot[pattern=north west lines, pattern color=orange] table[x=datasets,y=TTC]{\constructionspace};
    			\addplot[pattern = grid, pattern color=blue] table[x=datasets,y=LSC]{\constructionspace};
    			\addplot[pattern = crosshatch dots,pattern color=green] table[x=datasets,y=wavelet tree]{\constructionspace};
    		\end{axis}
    \end{tikzpicture}
    }
	\caption{Construction on different data sets ($k=0.01$).}
	\label{fig:construction}
\end{figure}

\end{comment}
% Last, for each graph, we represent the minimum number of full-length queries needed to make the index algorithm based on \LSC faster than the online algorithm based on \EDTTC in Table \ref{tab:amortize}. These numbers are very small compared with the number of timestamps, especially for the large graph. So the index-based algorithm performs better than the online algorithm even if we only need a small number of queries.

% \begin{table}[htbp]
%   \centering
  
  
%   \small
%   \begin{tabular}{c|c|c|c|c|c}
%     \hline
%      Datasets & GR & ST & WK & EM & CT \\
%     \hline\hline
%     Numbers & 7 & 63 & 61 & 1230 & 311\\
%     \hline
%   \end{tabular}
%   \caption{Number of queries with default setting required to amortize the index construction cost.}
%   \label{tab:amortize}
% \end{table}

% \subsubsection{kd-tree}

%
% For binary counting, we will report the time costs and the space costs of all graphs in one table, with the default sampling factor.
% \begin{table}[ht]
%     \centering
%     \scalebox{0.88}{
%     \begin{tabular}{c|c|c|c|c|c}
%     \hline
%     Datasets     & GR&ST&WK&EM&CT \\
%     \hline\hline
%      time($\mu$s)    & 13,001,082 & 1,280,309 & 50,188& 752 & 154\\
%      \hline
%      space(mb) & 137,830 & 17,203 & 1,638 & 45 & 13\\
%      \hline
%     \end{tabular}
%     }
%     \caption{Construction costs for binary counting.}
%     \label{tab:my_label}
% \end{table}
%
% For binary counting, we will report the time costs and the space costs of all graphs in one table, with the default sampling factor.
% We will also report the minimum number of full-length queries needed to make the index algorithm faster than the \EDTTC algorithm.
% \begin{table}[htbp]
%   \centering
  
  
%   \small
%   \begin{tabular}{c|c|c|c|c|c}
%     \hline
%      Datasets & GR & ST & WK & EM & CT \\
%     \hline\hline
%     Numbers & 12 & 10 & 5 & 4 & 5\\
%     \hline
%   \end{tabular}
%   \caption{Number of queries with default setting required to amortize the index construction cost for binary counting.}
%   \label{tab:amortize}
% \end{table}
% For binary counting, we will report the time costs and the space costs of all graphs in one table, with the default sampling factor.
% We will also report the minimum number of full-length queries needed to make the index algorithm faster than the \EDTTC algorithm.
% Note that for all datasets, the number needed is small, so for binary counting, our index-based algorithm performs better than the online algorithm even if there are only a few queries needed.
%
% \subsection{Experiments for extensions}
% \label{sec:experiment-extension}
% We will first report the experiments in directed graphs by evaluating the effect of the query interval length on both weighted query and binary query. We fix the query $\delta$ and sampling factor as defaulted and consider five different query time interval lengths, i.e.,  $20\%$, $40\%$, $60\%$, $80\%$, $100\%$ of $t_{max}$ respectively. For each length, we generate 10000 queries where $t_s$ is selected randomly (For the \EDTTC algorithm on ST, we only generate 1000 queries due to the large time cost). Figure \ref{fig:direct} reports the average time cost of answering one query as efficiency results. Note that the experiment results of directed graphs are very similar to the results of undirected graphs, indicating that our algorithms can be applied to directed graphs as well.

% \begin{figure}[ht]
%         \centering
% 	\ref{named19}\\
% 	\subfigure[ST]{
% 	   \begin{tikzpicture}[scale=0.38]
% 	   \begin{axis}[
% 			    legend style = {
% 				    legend columns=-1,
% 				    font=\footnotesize,
%                         draw=none,
% 				},
% 				legend to name=named19,
%                     legend image post style={scale=0.8, ultra thick},
%                     width=.5\textwidth,
%                     height=0.4\textwidth,
%                     xtick={0,0.2,0.4,0.6,0.8,1},
% 				xticklabels={,$20\%$,$40\%$,$60\%$,$80\%$,$100\%$},
% 				xmin=0.1,xmax=1.1,
% 				ymin=0,ymax=10000000,
% 				ymode = log,
% 				mark size=4pt,
%                     line width=2.5pt,
% 				ylabel={\huge \bf Running time ($\mu$s)},
% 				ylabel style={yshift=-5pt},
% 				xlabel={\huge \bf Interval length ratio},
% 				ticklabel style={font=\huge},
% 				every axis plot/.append style={ultra thick},
% 				every axis/.append style={ultra thick},
% 				]
% 				\addplot [mark=x,color=c4] table[x=length,y=LSC]{\stackoverflowdirectcircle};
% 				\addplot [mark=o,color=c6] table[x=length,y=TTC]{\stackoverflowdirectcircle};
%     \addplot [mark=star,color=c8] table[x=length,y=kd-tree]{\stackoverflowdirectcircle};
% 				\legend{{\small LSC}, {\small  TTC},{\small kd-tree}}
% 			\end{axis}
% 	\end{tikzpicture}
% 	}
%         \subfigure[WK]{
% 	   \begin{tikzpicture}[scale=0.38]
% 	   \begin{axis}[
%                     width=.5\textwidth,
%                     height=0.4\textwidth,
% 				xtick={0,0.2,0.4,0.6,0.8,1},
% 				xticklabels={,$20\%$,$40\%$,$60\%$,$80\%$,$100\%$},
% 				xmin=0.1,xmax=1.1,
% 				ymin=0,ymax=100000,
% 				ymode = log,
% 				mark size=4pt,
%                     line width=2.5pt,
% 				% ylabel={\huge \bf Running time (ms)},
% 				% ylabel style={yshift=-5pt},
% 				xlabel={\huge \bf Interval length ratio},
% 				ticklabel style={font=\huge},
% 				every axis plot/.append style={ultra thick},
% 				every axis/.append style={ultra thick},
% 				]
% 				\addplot [mark=x,color=c4] table[x=length,y=LSC]{\wikidirectcircle};
% 				\addplot [mark=o,color=c6] table[x=length,y=TTC]{\wikidirectcircle};
%     \addplot [mark=star,color=c8] table[x=length,y=kd-tree]{\wikidirectcircle};
% 			\end{axis}
% 	\end{tikzpicture}
% 	}
        
%  \caption{Effect of the length of query interval for directed graph.}
% \label{fig:direct}

% \end{figure}
% Then we will evaluate the efficiency of index maintenance. We will first build the LSC index with edges in $[0,0.8t_{max}]$. Then, we update the index with the remaining edges in $[0.8t_{max}+1, t_{max}]$. Table \ref{tab:update} shows the average time for updating one edge in the \LSC index of different datasets. It is obvious that updating the index is faster than rebuilding the index from scratch.


% \subsection{Case study for dataset WK}

% \label{sec:experiment-case-study}
% We have settled a series of queries for tracking the trend of counting results in dataset WK.
% Formally speaking, the query settings will be $t_e -t_s = 0.2t_{max}$, $\delta = 0.1(t_e-t_s)$, and sampling factor $k = 1$ (storing every C-point to keep the accuracy). Then we generate 1000 queries and the $t_s$ of $ith$ query is $\frac{0.8t_{max}}{1000}\times i$.

% Now we can track the trend of counting results, and the result is shown in Figure \ref{fig:case_study}. Note that for dataset WK, the sharp decline of the number of $\delta$-temporal triangles at the beginning is obvious, which means something unusual happened during that time in this dataset.

% \begin{figure}[h]
%     \centering
%     \centering %图片居中
%     {
%         \begin{tikzpicture}[scale=0.9] %tikz图片
%         \begin{axis}[
%             xlabel= {\footnotesize \bf Query id}, %横坐标名
%             ylabel= {\footnotesize \bf Number of $\delta$-temporal triangles}, %纵坐标名
%             % ylabel style={yshift=-12pt},
%             xlabel style={yshift=5pt},
%             ymode = normal,
%             xmode = normal,
%             xmin=-10,xmax=1010,
%             ymin=1,ymax=3e8,
%             mark size=0.0pt,
%             width=0.515\textwidth,
%             height=0.24\textwidth,
%             ticklabel style={font=\footnotesize},
%             every axis plot/.append style={line width= 1.1pt},
%             %ytick = {1e3, 1e5, 1e8},
%             ytick = {5e7,1e8,1.5e8,2e8,2.5e8},
%             xtick = {1, 250, 500, 750, 1000},
%             every axis/.append style={line width= 0.8pt},
%             legend style = {
%                 at={(0.92,1)},
%     		legend columns=5,
%                 font=\footnotesize,
%                 draw=none,
%     	},
%          ]
        
%         % \addplot[smooth,mark=*,c3] table {figure/trend/res_CT.txt};
%         % \addlegendentry{CT}
%         \addplot[smooth,mark=*,c1] table {figure/trend/rw_WK.txt};
%         \addlegendentry{Number of $\delta$-temporal triangles in WK}
%      \end{axis}
%      \end{tikzpicture}
%  }
 
%  \setlength{\abovecaptionskip}{-0.001cm}
%  \setlength{\belowcaptionskip}{-5pt}
% \caption{Trends of counting result in dataset WK.}
% \label{fig:case_study}
% \end{figure}
