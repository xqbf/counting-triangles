\vspace{-0.1in}
\section{Related Work}
\label{sec:related}

In this section, we review the related works of triangle counting in static and temporal graphs.

$\bullet$ {\bf Triangle counting in static graphs.}
%
As an essential problem in network analysis, triangle counting has garnered significant research attention.
%
Most existing works have concentrated on static undirected graphs.
%
The exact counting solution, relying on enumeration, is bounded by $O(m\kappa)$~\cite{chiba1985arboricity}. 
%
Alternatively, counting triangles through matrix multiplication improves time complexity~\cite{al2018triangle}.
%
Recent advancements have pushed the time complexity bottleneck of matrix multiplication to $O(n^{2.371552})$~\cite{williams2024new}.

Additionally, some works focus on approximating the number of triangles either through local or global counting methods.
%
For example, Kolountzakis and Miller~\cite{kolountzakis2012efficient} presented a method for approximating the local number of triangles, while several studies \cite{tsourakakis2009approximate,eden2017approximately,chen2018mining} focused on estimating global triangle counting.
%
Despite this positive research progress, these methods cannot be directly applied to triangle counting in temporal graphs since they do not consider temporal information.

$\bullet$ {\bf Triangle counting in temporal graphs.}
%
Several recent works have studied triangle counting in temporal graphs.
\cite{paranjape2017motifs} defined the $\delta$-temporal triangle and provided an algorithm of counting $\delta$-temporal triangles. \cite{pashanasangi2021faster} further defined $(\delta_{1,3},\delta_{1,2},\delta_{2,3})$-temporal triangles and provided an algorithm based on degeneracy order to count the $(\delta_{1,3},\delta_{1,2},\delta_{2,3})$-temporal triangles.
%
There are several papers \cite{jha2015counting,gou2021sliding} that counted the triangles within a sliding time window.
%
\cite{buriol2006counting,lee2020temporal} proposed efficient sampling methods for updating the number of triangles in continuously updating temporal graphs.
%
To the best of our knowledge, our work is the first to study counting $\delta$-triangles within arbitrary time windows in temporal graphs.

In addition, many works study the counting of motifs, which are more general than triangles \cite{boekhout2019efficiently,paranjape2017motifs,liu2018sampling, zhu2021leveraging, zhu2019scalable}.
%
For instance, \cite{paranjape2017motifs} proposed the concept of $\delta$-temporal motif and provided a counting method for it; \cite{zhu2021leveraging} developed a temporal subgraph reporting method which could be used for solving our $\delta$-temporal triangle counting problem.
%
Some works focus on counting special motifs like clique \cite{himmel2016enumerating} and butterfly \cite{sanei2018butterfly,cai2023efficient}.
%
Besides, counting and enumerating motifs, especially the cliques, are commonly used in data mining \cite{wang2024approach,wang2023clique,wang2022preference}.

\vspace{-0.1in}