\documentclass[acmsmall]{acmart}
\usepackage{graphicx} % Required for inserting images
%
\usepackage[linesnumbered,vlined,boxed,commentsnumbered, ruled, english]{algorithm2e}
\usepackage{tikz}
\usepackage{diagbox}
\usepackage{hyperref}
\usepackage{listings}
\usepackage{enumitem}
\usepackage{subfigure}
\usepackage{color}
\usepackage{pgfplots}
\usepackage{multirow}
\usetikzlibrary{patterns}
\usepackage{balance}
\usepackage{subfiles}
\usepackage{tabularx}
\usepackage{amsmath}
\usepackage{tikz-qtree}
\usepackage[english]{babel}
\usetikzlibrary{plotmarks}
\usepackage{verbatim}
\usepackage{xcolor}
\usepackage{algpseudocode}
\usepackage{array}
% \usepackage{extpfeil}

\usepackage{threeparttable}


\renewcommand{\multirowsetup}{\centering}
\newtheorem{definition}{Definition}
\newtheorem{example}{Example} 
\newtheorem{problem}{Problem} 
\newtheorem{theorem}{Theorem} 
\newtheorem{lemma}{Lemma}

\lstset{
     % columns = fixed,       
     basicstyle = \linespread{0.7} \sffamily,             % 设置行距,字体
     numbers = left,                                      % 在左侧显示行号
     numberstyle = \tiny \color{gray},                    % 设定行号格式
     keywordstyle = \bfseries \color[RGB]{40,40,255},     % 设定关键字颜色
     numberstyle = \footnotesize \color{darkgray},           
     commentstyle = \color[RGB]{0,96,96},                 % 设置代码注释的格式
     stringstyle = \color[RGB]{128,0,0},                  % 设置字符串格式
     frame = single,                                      % 不显示背景边框
     backgroundcolor = \color[RGB]{245,245,244},          % 设定背景颜色
     showstringspaces = false,                            % 不显示字符串中的空格
     language=python                                          % 设置语言
}

\pgfplotsset{compat=1.11,
	/pgfplots/ybar legend/.style={
		/pgfplots/legend image code/.code={
			\draw [#1] (0cm,-0.12cm) rectangle (0.6cm,0.15cm);},
	},
}


\definecolor{c1}{RGB}{42,99,172} % Basic3D
\definecolor{c2}{RGB}{255,88,93}
\definecolor{c3}{RGB}{255,181,73}
\definecolor{c4}{RGB}{119,71,64} % Fast3D
\definecolor{c5}{RGB}{228,123,121} % Basic2D
\definecolor{c6}{RGB}{208,167,39} % Graph size
\definecolor{c7}{RGB}{0,51,153}
\definecolor{c8}{RGB}{56,140,139} % Fast2D
\definecolor{c9}{RGB}{0,0,0} % black
\definecolor{c10}{RGB}{140,138,185} % purple

\pgfplotstableread[row sep=\\,col sep=&]{
time&AI0& AI1 & AI2 & AI3 &DB0 &DB1 &DB2&DB3 \\
6 & 5.451 & 13.196 & 17.584 & 19.212 & 2.702 & 5.694 &7.353 & 7.46\\
5 & 1.638& 2.689 & 3.261 & 3.564 & 1.156 & 1.923 &2.397& 2.666\\
4 & 0.742& 1.183 & 1.418 & 1.519 & 1.145 & 1.598 & 2.199 & 2.264\\
3 & 0.241& 0.329 & 0.48 & 0.498& 0.328 & 0.649 & 0.752 & 0.795\\
2 & 0.092& 0.133& 0.141 & 0.145 & 0.172 & 0.262 & 0.28 & 0.284\\
1 & 0.005& 0.006& 0.006 & 0.006& 0.013& 0.018 & 0.018 & 0.018\\
}\weighteddensity


\newcommand{\TTM} {{\tt TTM}\xspace}
\newcommand{\OTTC} {{\tt OTTC}\xspace}
\newcommand{\BTTC} {{\tt BTTC}\xspace}
\newcommand{\DOTTT}{{\tt DOTTT}\xspace}
\newcommand{\wavelet}{\xspace{wavelet tree}\xspace}

\newcommand{\tabincell}[2]{\begin{tabular}{@{}#1@{}}#2\end{tabular}}

% \acmConference[SIGMOD ’25]{International Conference on Management of Data}{June 9 - June 15, 2025}{Berlin, Germany}
% \acmPrice{15.00}

\AtBeginDocument{%
  \providecommand\BibTeX{{%
    \normalfont B\kern-0.5em{\scshape i\kern-0.25em b}\kern-0.8em\TeX}}}
% \acmConference
% \acmBooktitle
\setcopyright{acmlicensed}
% See completed rightsreview form for your code
\acmJournal{PACMMOD}
\acmYear{2025} \acmVolume{3} \acmNumber{1 (SIGMOD)}
\acmArticle{38} \acmMonth{2} \acmPrice{15.00}
\acmDOI{10.1145/3709688}

\textfloatsep 1mm plus 1mm \intextsep 1mm plus 1mm
%\renewcommand{\baselinestretch}{0.995}

\title{Efficiently Counting Triangles in Large Temporal Graphs}

% \author{Yuyang Xia}
% \date{August 2023}

\begin{document}


\author{Yuyang Xia}
\email{yuyangxia@link.cuhk.edu.cn}
\affiliation{%
  \institution{The Chinese University of Hong Kong, Shenzhen}
  \state{Guangdong}
  \country{China}
}

\author{Yixiang Fang}
\authornote{Corresponding author.}
\email{fangyixiang@cuhk.edu.cn}
\affiliation{%
  \institution{The Chinese University of Hong Kong, Shenzhen}
  \state{Guangdong}
  \country{China}
}

\author{Wensheng Luo}
\email{luowensheng@cuhk.edu.cn}
\affiliation{%
  \institution{The Chinese University of Hong Kong, Shenzhen}
  \state{Guangdong}
  \country{China}
}

\begin{CCSXML}
<ccs2012>
<concept>
<concept_id>10002950.10003624.10003633.10010917</concept_id>
<concept_desc>Mathematics of computing~Graph algorithms</concept_desc>
<concept_significance>500</concept_significance>
</concept>
<concept>
<concept_id>10003752.10003809.10003635.10010038</concept_id>
<concept_desc>Theory of computation~Dynamic graph algorithms</concept_desc>
<concept_significance>500</concept_significance>
</concept>
<concept>
<concept_id>10003752.10003809.10010031</concept_id>
<concept_desc>Theory of computation~Data structures design and analysis</concept_desc>
<concept_significance>500</concept_significance>
</concept>
</ccs2012>
\end{CCSXML}



\ccsdesc[500]{Mathematics of computing~Graph algorithms}
\ccsdesc[500]{Theory of computation~Dynamic graph algorithms}
\ccsdesc[500]{Theory of computation~Data structures design and analysis}
\renewcommand{\shortauthors}{Yuyang Xia, Yixiang Fang, and Wensheng Luo}
\input{0-abstract}
% \settopmatter{printfolios=true}

\maketitle



\section{Introduction}
\label{sec:intro}

In many real-world applications (e.g., email networks,  social networks, and phone call networks), the relationships between entities can be modeled as a temporal graph \cite{zhao2019t,han2014chronos,erlebach2021temporal,kovanen2013temporal}, in which each edge is associated with a timestamp representing the interaction time.
%
Figure \ref{fig:temporal_graph}(a) depicts an example temporal graph, where the numbers on the edges represent their occurring timestamps.
%
For example, the two temporal edges between vertices $v_1$ and $v_4$ indicate that they have two interactions at timestamps $0$ and $2$.

\begin{figure}[ht]
    \small
    \centering
    \subfigure[An example temporal graph $G$]{
    \includegraphics[width = .27\linewidth]{figure/temporal graph example.pdf}
    }
    \text{  }
    \subfigure[Two 1-temporal  triangles]{
    \includegraphics[width = .3\linewidth]{figure/weighted triangle example.pdf}
    }
    \caption{An example of counting $\delta$-temporal triangles.}
    \label{fig:temporal_graph}
\end{figure}
\begin{comment}
    Triangle counting is an essential task in network analysis and has been extensively researched, as indicated by many studies \cite{tsourakakis2011spectral}. This problem has a wide range of applications including spam detection\cite{becchetti2010efficient}, online link recommendation\cite{tsourakakis2011spectral}, and so on. 

There are several works aiming for temporal triangle counting. 
\end{comment}

As a fundamental task in temporal graph analysis, triangle counting has received a great deal of attention \cite{lee2020temporal,paranjape2017motifs,pashanasangi2021faster}. Many analytical parameters contain the result of triangle counting, for instance, transitivity \cite{chu2011triangle} and clustering coefficient \cite{soffer2005network}. Besides, triangle counting plays an important role in many applications, such as spam detection \cite{becchetti2008efficient}, social network analysis \cite{pfeiffer2012fast},  community detection \cite{klymko2014using,gleich2012vertex}, and so on \cite{huang2018triangle, khan2011neighborhood}.
%
Different from triangles in static graphs, the triangles in temporal graphs often take the timestamps into consideration and several temporal triangle models have been developed, {\color{black}including $\delta$-temporal triangle \cite{paranjape2017motifs}, sliding-window triangle \cite{gou2021sliding}, $(\delta_{1,3}, \delta_{1,2}, \delta_{2,3})$-temporal triangle \cite{pashanasangi2021faster}, 
and temporal-triangle subgraph \cite{zhu2019scalable}}. 
%
Among these models, the $\delta$-temporal triangle, proposed by Paranjape, Benson, and Leskovec \cite{paranjape2017motifs}, has been demonstrated effective in many real applications.
%
Conceptually, a $\delta$-temporal triangle is a triangle formed by three temporal edges, such that the gap between the timestamps of any two temporal edges is bounded by a threshold $\delta$.
%
Consider the temporal graph in Figure \ref{fig:temporal_graph}(a) and let $\delta$=1. Then there are two 1-temporal triangles as shown in Figure \ref{fig:temporal_graph}(b).

In this paper, we study the problem of efficiently counting $\delta$-temporal triangles in large temporal graphs.
%
Given a temporal graph $G$, a duration $\delta$, and a time window $[t_s, t_e]$, the goal is to count all the $\delta$-temporal triangles within $[t_s, t_e]$.
%
{\color{black}For example, in Figure \ref{fig:temporal_graph}(a), assuming that $\delta$=1 and $[t_s,t_e]$=[0,3], then there are two 1-temporal triangles within the time window $[0,3]$ as depicted in Figure \ref{fig:temporal_graph}(b).}
%
As shown in the literature \cite{paranjape2017motifs,pashanasangi2021faster,bouritsas2022improving}, counting $\delta$-temporal triangles in temporal graphs has been demonstrated effective in many real applications, to name a few:
%
\begin{itemize}
    \item \textbf{Graph cohesiveness analysis.} A triangle represents the strong and stable relationship among three vertices \cite{huang2014querying}. As a result, $\delta$-temporal triangle counting can be used to measure the cohesiveness of temporal graphs.
    %
    For example, the triangle density of a graph is defined as the total number of triangles over the number of vertices \cite{tsourakakis2015k,samusevich2016local}, and it can be used to identify graph communities \cite{fang2019efficient}.

    \item \textbf{Graph transitivity.} Given a graph, its transitivity \cite{chu2011triangle} evaluates how strong the vertices are aggregated, and is defined as three times the number of triangles over the number of wedges, where a wedge is a path of two connected edges. The transitivity can be easily extended for temporal graphs by using $\delta$-temporal triangles.

    \item \textbf{Higher-order network analysis.} As a typical higher-order structure, triangles are one building block of many networks \cite{benson2016higher}, and $\delta$-temporal triangle counting is useful in the analysis of users' behaviors (e.g., blocking communication and bitcoin transfer) in temporal networks \cite{paranjape2017motifs}.
\end{itemize}

{\color{black} %(R3.A1-A2) 
Recently, we have also found that the $\delta$-temporal triangle counting can be used for analyzing the evolution of research topics in academic areas.
%
%To further demonstrate the effectiveness of $\delta$-temporal triangle counting, we present a case study in our experiments (Section \ref{sec:experiment-case-study}), analyzing the evolution of research topics within the KDD community from 2000 to 2023.
%
Specifically, we have tried to analyze the evolution of the research community of the KDD conference from 2000 to 2023.
%
We first construct two temporal co-authorship networks for the DB and AI communities respectively, and divide the entire time window [2000, 2023] into six disjoint intervals, each spanning four years.
%
We then adapt the triangle density \cite{samusevich2016local,tsourakakis2015k}, which is defined as the number of $\delta$-temporal triangles over the number of vertices, to analyze collaboration trends in the DB and AI communities within KDD.
%
Our finding of triangle density, depicted in Figure \ref{fig:case-study-weighted-intro}, shows that after 2016, the triangle density of the AI community surpassed that of the DB community.
%
This implies that the AI community has become more active than the traditional DB community since 2016.
%
Besides, the number of $\delta$-temporal triangles with $\delta$=1 is significantly higher than that with $\delta$=0, while the difference between $\delta$=2 and $\delta$=3  is minimal, suggesting that authors tend to collaborate more frequently over short time intervals.}

\begin{figure}[t]
        \subfigure[$\delta \in \{0,1\}$]{
	\begin{tikzpicture}[scale = 0.5]
	   \begin{axis}[
                    width=0.7\textwidth,
                    height=0.42\textwidth,
				xtick={0,1,2,3,4,5,6},
				xticklabels={,2000,2004,2008,2012,2016,2020},
				xmin=0.5,xmax=6.5,
				ymin=0,ymax=20,
                    legend style = {
                        legend columns=2,
                        draw=none,
                        at={(0.9,1)/}
				},
				% ymode = log,
				mark size=5pt,
                    line width=2.5pt,
				% xlabel={\huge \bf $\delta$ length ratio},
    ylabel={\bf $\delta$-temporal triangle density},
				every axis plot/.append style={ultra thick},
				every axis/.append style={ultra thick},
				]
				
    \addplot [mark=x,color=c8,line width=2.5pt] table[x=time,y=AI0]{\weighteddensity};
    \addplot [mark=o,color=c2,line width=2.5pt] table[x=time,y=DB0]{\weighteddensity};
    \addplot [mark=x,color=c7,line width=2.5pt] table[x=time,y=AI1]{\weighteddensity};
    \addplot [mark=o,color=c4,line width=2.5pt] table[x=time,y=DB1]{\weighteddensity};
    \legend{{ { AI ($\delta = 0$)}}, {  DB ($\delta = 0$)}, { { AI ($\delta = 1$)}}, { { DB ($\delta = 1$)}}}
    \end{axis}
    \end{tikzpicture}
    }
        \subfigure[$\delta\in \{2,3\}$]{
	\begin{tikzpicture}[scale = 0.5]
	   \begin{axis}[
                    width=0.7\textwidth,
                    height=0.42\textwidth,
				xtick={0,1,2,3,4,5,6},
				xticklabels={,2000,2004,2008,2012,2016,2020},
				xmin=0.5,xmax=6.5,
				ymin=0.001,ymax=30,
                    legend style = {
                        legend columns=2,
                        draw=none,
                        at={(0.9,1)/}
				},
				mark size=5pt,
                    line width=2.5pt,
                    ylabel={\bf $\delta$-temporal triangle density},
				every axis plot/.append style={ultra thick},
				every axis/.append style={ultra thick},
				]
					
                \addplot [mark=x,color=c8,line width=2.5pt] table[x=time,y=AI2]{\weighteddensity};
			\addplot [mark=o,color=c2,line width=2.5pt] table[x=time,y=DB2]{\weighteddensity};
                \addplot [mark=x,color=c7,line width=2.5pt] table[x=time,y=AI3]{\weighteddensity};
			\addplot [mark=o,color=c4,line width=2.5pt] table[x=time,y=DB3]{\weighteddensity};
                \legend{{ { AI($\delta = 2$)}}, {  DB($\delta = 2$)}, { { AI($\delta = 3$)}}, { { DB($\delta = 3$)}}}
			\end{axis}
	\end{tikzpicture}
	}   
    \caption{{\color{black}$\delta$-temporal triangle density.}}
    \label{fig:case-study-weighted-intro}
    \vspace{-0.05in}
\end{figure}


%The detailed studies of the above two applications are provided in Section \ref{sec:experiment-case-study}.

%counting on static graphs, triangle counting on temporal graphs can take the temporal ordering of the edges into concern to get more information. Thus, Paranjape, Benson, and Leskovec introduced $\delta$-temporal triangles that all edges of the triangle have to occur within $\delta$ time stamps \cite{paranjape2017motifs}. Then Noujan and Seshadhri improved the time complexity of $\delta$-temporal triangle counting and generalized it into $(\delta_{1,3}, \delta_{1,2}, \delta_{2,3})$-temporal triangle counting \cite{pashanasangi2021faster}, which is the state-of-art algorithm for $\delta$-temporal triangle counting so far. However, the definition of $(\delta_{1,3}, \delta_{1,2}, \delta_{2,3})$-temporal triangle is too strict for applications where the specific relationships inside the triangle are irrelevant. Nowadays, many temporal graph applications analyze the graph problems in different time windows \cite{mertzios2019sliding,akrida2020temporal,gou2021sliding,xie2023querying}. Thus, we want to discuss counting $\delta$-temporal triangles in arbitrary time windows.

Despite its wide usefulness, counting $\delta$-temporal triangles in large temporal graphs is a very challenging task, due to  the following two reasons:
1) Different from counting triangles in static graphs, counting $\delta$-temporal triangles is more complicated since it needs to consider the timestamp of each edge;
and 2) the number of $\delta$-temporal triangles is huge, especially when $\delta$ is large and the query time window covers a long duration.

{\bf Prior works.} To tackle the above challenges, Paranjape et al.~\cite{paranjape2017motifs} developed a $\delta$-temporal triangle counting algorithm with $O(m\sqrt{\Gamma})$ time, where $m$ is the number of edges in the graph $G$ and $\Gamma\leq O(m^{1.5})$ is the number of static triangles in $G$, so it achieves a time complexity of $O(m^{1.75})$.
%
Noujan and Seshadhri proposed the state-of-the-art (SOTA) algorithm for $\delta$-temporal triangle counting, \DOTTT, with $O(m\kappa\log(m))$ time cost \cite{pashanasangi2021faster}, where $\kappa$ is the degeneracy of the graph and is up to $O(m^{0.5})$.
%
{\color{black} Additionally, Zhu et al. \cite{zhu2021leveraging} designed a temporal subgraph reporting method and it could be adapted for counting $\delta$-temporal triangles.}
%
Nevertheless, these algorithms are still inefficient for processing large temporal graphs. 
% %
% Besides, they are slower than the triangle counting algorithm for static graphs through enumeration, which can achieve $O(m\sqrt{m})$ \cite{al2018triangle} but cannot be applied for counting $\delta$-temporal triangles.
% %
% Note that although triangle counting through matrix multiplication can achieve a lower time complexity than $O(m\sqrt{m})$, the space complexity of matrix multiplication is up to $O(n^2)$, which is too large to be applied to large graphs, so $O(m\sqrt{m})$ is still the best bound for counting triangles on large graphs.
% %
% Hence, it is desirable to develop faster algorithms for counting $\delta$-temporal triangles.

\textbf{Online solutions.} 
To efficiently count $\delta$-temporal triangles, we first propose an efficient online algorithm \OTTC.
%
Specifically, given a duration $\delta$ and a query time window $[t_s,t_e]$, we first derive a projected graph $G_{[t_s,t_e]}$ by extracting edges appearing in $[t_s,t_e]$.
%
Then, we enumerate the temporal edges chronologically to count $\delta$-temporal triangles.
%
We theoretically show that \OTTC completes in $O(m\kappa)$ time, so it is faster than the SOTA algorithm \cite{pashanasangi2021faster} with $O(m\kappa \log(m))$ time cost, as shown in Table \ref{tab:overview}.

\textbf{Index-based solutions.}
Although the above online algorithm outperforms existing algorithms, it may not be efficient in some scenarios because in practice, to accomplish a specific task, users often have to frequently issue $\delta$-temporal triangle counting queries by varying the time window $[t_s,t_e]$ and duration $\delta$ multiple times.
%
In light of this, we propose an elegant index-based solution, which converts the triangle counting problem into a point counting problem.
%
Specifically, for each $\delta$-temporal triangle, we first convert it into a 2-dimensional point $(t_s,t_e)$, where $[t_s,t_e]$ is the minimal time interval that $G_{[t_s,t_e]}$ contains this $\delta$-temporal triangle.
%
Afterward, we employ the wavelet tree index to solve this 2-dimensional point counting problem.
%
{\color{black} 
Besides, in many real-world applications, the temporal graphs often evolve frequently with many new inserted temporal edges as time goes on. 
%
For example, Yahoo! processed over 500M new communications between 33M IP addresses in one day\footnote{\url{https://webscope.sandbox.yahoo.com/catalog.php?datatype=g&guccounter=1}\label{web}}.
%
Given the dynamic nature of temporal graphs, it is crucial to efficiently maintain the index for dynamic temporal graphs.
%
Therefore, we also design a fast wavelet tree index maintenance algorithm to avoid rebuilding the index from scratch for every update.}

Besides, in some real-world scenarios, we may only need to consider the existence of $\delta$-temporal triangles among three vertices, so we introduce the binary $\delta$-temporal triangle counting problem, which counts the number of vertex triplets that form one or more $\delta$-temporal triangles.
%
We also develop efficient online and index-based solutions to solve this variant.

In addition, we have performed extensive experimental evaluations on four real and one synthetic temporal graphs, and the results show that our algorithms are highly efficient and scalable.
%
In particular, our online algorithm is up to 70$\times$ faster than the SOTA online algorithms, and our index-based algorithm runs up to $10^8\times$ faster than the online algorithms.
%
Besides, our algorithms for binary $\delta$-temporal triangle counting are also highly efficient.

In summary, our principal contributions are as follows.
%
\begin{itemize}
    \item We have developed an efficient online algorithm for counting $\delta$-temporal triangles, whose time complexity is lower than that of the SOTA algorithm.
    
    \item We have designed a novel index-based solution by converting the $\delta$-temporal triangles into 2-dimensional points.

    \item We have introduced the binary $\delta$-temporal triangle counting problem and also developed efficient algorithms.

    \item We have performed comprehensive experiments on both real and synthetic large temporal graphs, and the results show that our algorithms are highly effective and efficient.
\end{itemize}

\textbf{Outline.} 
We present the $\delta$-temporal triangle counting problem in Section \ref{sec:preliminaries}.
%
We introduce our online and index-based solutions in Sections \ref{sec:online} and \ref{sec:weighted}, respectively.
%
In Section \ref{sec:binary}, we focus on binary $\delta$-temporal triangles counting.
%
Section \ref{sec:experiment} shows experimental results.
%
We review related works in Section \ref{sec:related} and conclude in Section \ref{sec:conclusion}.

% {\color{black} (R3.A4) 
% To better illustrate our solutions for $\delta$-temporal triangle counting, we provide a flow chart of our proposed solutions in Figure \ref{fig:flow-chart}. We have also provided the bird's-eye views in front of Section \ref{sec:edttc} and \ref{sec:weighted}.

% \begin{figure}[ht]
%     \centering
%     \includegraphics[width=\linewidth]{figure/workflow of δ-temporal triangle counting.pdf}
%     \caption{\color{black} Flow chart of our solutions.}
%     \label{fig:flow-chart}
% \end{figure}
% }

\begin{table}[t]
  \small
  \centering
  \caption{Overview of the complexities of solutions.}
  \label{tab:overview}
  \begin{threeparttable}
  \begin{tabular}{c|c|c|c|c}
    \hline
      \multirow{2}{*}{Algorithm} & \multicolumn{2}{c|}{Indexing complexity}  & \multicolumn{2}{c}{Counting complexity}\\
     
     \cline{2-5}
     {} & time & space & time & space \\
     \hline\hline
     
     \DOTTT \cite{pashanasangi2021faster} & -- & --  & $O(m \kappa \log(m))$ &$O(m)$\\
     \cline{1-5}
     
     \OTTC & -- & --  & $O(m \kappa)$ &$O(m)$\\
     \cline{1-5}
    \color{black} {\tt TSRjoin} \cite{zhu2021leveraging} & 
    \tabincell{c}{\color{black}$O(n^2+$ \color{black}$m^2\log(m))$}
    & \color{black} $O(m^2)$ 
    & \color{black} $O(\max(m, \Delta))$ &
    \color{black} $O(m^2)$ \\
    
    \cline{1-5}
     
     {\tt WT-Index} & 
     
     %$O(m\kappa+ \Delta \times \log(\Delta))$
     \tabincell{c}{\color{black}$O(m\kappa+$ \color{black}$\pi \log(m))$}
     
     &
     
     %$O(m+\Delta \log(\Bar{c}))$ 
     \tabincell{c}{\color{black}$O(\pi)$}
     
     &\color{black}$O(poly\log(m))$ & 
     \color{black}$O(\pi)$\\
     \hline
  \end{tabular}
  {\color{black} \raggedright Note: given a temporal graph $G$, $n$ and $m$ are the numbers of vertices and edges in $G$ respectively, $\kappa$ is the degeneracy of $G$, $\pi$ is the number of C-points of $G$, and $\Delta$ is the number of counted $\delta$-temporal triangles in a query.
  \par}
  \end{threeparttable}
\end{table}

\section{Preliminaries}
\label{sec:preliminaries}

In this section, we formally introduce the concepts of temporal graph and $\delta$-temporal triangles, and our studied counting problems.
%
The frequently used notations are provided in Table \ref{tab:notation}.

\begin{definition}[Temporal graph]
\label{def:temporalG}
A temporal graph is a graph $G=(V, E)$ with a set $V$ of vertices and a set $E$ of edges, such that each edge $e \in E$ is a triplet $(u,v,t)$, where $t$ is a timestamp indicating the interaction time between two vertices $u$ and $v$.
\end{definition}

Given a temporal graph $G$ = ($V$, $E$), we use $n = |V|$ and $m = |E|$ to denote the numbers of vertices and edges in $G$ respectively.
%
For a vertex $u\in G$, the set of its neighbors in the $G$ is denoted as $N(u)$.
% 
We use $E(u,v)$ to denote the list of edges sharing end vertices $u$ and $v$, sorted in ascending order of their timestamps.
%
For simplicity, we assume there is no more than one edge with the same timestamp between any two vertices in the graph.
%
For instance, Figure \ref{fig:temporal_graph}(a) gives a temporal graph comprising five vertices and eight edges, where each edge is labeled with an integer representing the timestamp.
%
W.l.o.g., assume that all the edge timestamps fall within the range of consecutive integers in $[0,t_{max}]$, where $t_{max} \le m$.
{\color{black}% (R1.A1-A2, R3.A5)
Note that in real-world scenarios, for some real timestamps, there may not exist any temporal edge, so the total number of real timestamps could be larger than $m$.
%
However, these real timestamps without temporal edges can be ignored when counting the temporal triangles, so we simply skip them by mapping the real timestamps to a list of consecutive integers in $[0, t_{max}]$, such that there exist temporal edge(s) at each timestamp in $[0, t_{max}]$, so we have $t_{max} \leq m$, and $t_{max}=m$ is achieved when each edge has a distinct timestamp.
%
Additionally, when answering a counting query, the real query time window needs to be mapped as well, by following the mapping mechanism above before running the algorithm.
}


\begin{definition}[$\delta$-temporal triangle \cite{paranjape2017motifs}]
\label{def:delta-trianggle}
Given a temporal graph $G$ and duration $\delta$ ($\delta\geq0$), a $\delta$-temporal triangle is a subgraph of three temporal edges, i.e., $(v_1$, $v_2$, $t_1)$, $(v_2$, $v_3$, $t_2)$, and $(v_3$, $v_1$, $t_3)$, such that the gap between the timestamps of any two temporal edges is at most $\delta$, i.e., $\forall i,j\in [1,3]$, $|{t_i}-{t_j}|\leq\delta$.
\end{definition}

\begin{figure}[ht]
    \small
    \centering
    \subfigure[The projected graph $G_{[0,2]}$]{
    \includegraphics[width = .27\linewidth]{figure/projected graph example.pdf}
    }
    \subfigure[The static graph $G_{[0,2]}^*$]{
    \includegraphics[width = .27\linewidth]{figure/static graph example.pdf}
    }
%
    \caption{Illustrating the projected graph and static graph.}
    \label{fig:tgraph}
\end{figure}


\begin{definition}[Projected graph]
\label{def:pgraph}
Given a temporal graph $G$ and a time window $[t_s,t_e]$, the projected graph of $G$ over $[t_s,t_e]$ is a temporal graph, denoted by $G_{[t_s,t_e]}$, that encompasses all the edges $(u,v,t) \in G$ with timestamps $t\in[t_s,t_e]$.
\end{definition}


\begin{problem}[$\delta$-temporal triangle counting \cite{paranjape2017motifs}]
\label{prob:delta-counting}
Given a temporal graph $G$, a time window $[t_s,t_e]$, and a duration $\delta$ ($\delta\geq0$), return the number of $\delta$-temporal triangles in $G_{[t_s,t_e]}$.
\end{problem}



For example, in the temporal graph of Figure \ref{fig:temporal_graph}(a), let $[t_s,t_e]$=[0,2] and $\delta$=1.
%
We can first obtain the projected graph $G_{[0,2]}$ as depicted in Figure \ref{fig:tgraph}(a), and then find that there is one 1-temporal triangle in $G_{[0,2]}$ as shown in the right part of Figure \ref{fig:temporal_graph}(b).

\begin{table}[t]
    \small
    \caption{Notations and meanings.}
    \label{tab:notation}
    \begin{tabular}{ c | l }
        \hline
        \textbf{Notation(s)} & \textbf{Meaning}\\
        \hline\hline
        $G=(V,E)$ & \tabincell{l}{A temporal graph with vertex set $V$ and edge set $E$}\\
        \hline
        $( u,v,t)$ & \tabincell{l}{A temporal edge between $u$ and $v$ with timestamp $t$}\\
        \hline
        $n,m$ & \tabincell{l}{The number of vertices and edges of $G$ respectively}\\
        \hline
        $t_{max}$ & \tabincell{l}{The maximum timestamp in $G$}\\
        \hline
        $\kappa$ & \tabincell{l}{The degeneracy of $G$}\\
        \hline
        $N(u)$ & \tabincell{l}{The set of neighbors of $u\in V$}\\
        \hline
        $[t_s,t_e]$ & \tabincell{l}{A time window with $t_s\leq t_e$} \\
        \hline
        $E(u,v)$ & \tabincell{l}{A list of edges with ending vertices $u$ and $v$}\\
        \hline
        $\langle (x,y),c\rangle$ & \tabincell{l}{A C-point at $(x,y)$ with count $c$}\\
        \hline
        $\pi$ & \tabincell{l}{The total number of C-points of $G$}\\
        \hline
        $\Delta$ & \tabincell{l}{The number of counted $\delta$-temporal triangles in a query}\\
        \hline
        $\Delta_{u,v,w}$& \tabincell{l}{A static triangle with three vertices set $\{u,v,w\}$}\\
        \hline
    \end{tabular}
\end{table}

\section{Online Algorithms}
%\section{Online algorithms for counting $\delta$-temporal triangles}
\label{sec:online}

We begin with the concept of static graph: the static graph of a temporal graph $G$=$(V,E)$ is a graph, denoted by $G^*$=($V$, $E^*$), that only stores the static edges of $G$, where $E^*=\{(u,v)|(u,v,t)\in E\}$.
%
To count triangles in static graphs, researchers often use the concepts of degeneracy and degeneracy order \cite{chiba1985arboricity,pashanasangi2021faster}:

\begin{definition}[Degeneracy and degeneracy order \cite{pashanasangi2021faster}]
\label{def:degeneracy}
Given a static graph $G^*$=$(V, E^*)$, its degeneracy is the smallest integer $\kappa$, such that there exists an order $P$ of all the vertices, $v_1\prec v_2 \prec \cdots \prec v_n$, a.k.a. degeneracy order, satisfying $d^+(v) \leq \kappa$ for each $v \in V$,
% \begin{equation}
%     
% \end{equation}
where $d^+(v)$ is the size of set $N^+(v)$ of $v$'s neighbors with larger orders in $P$, i.e., $N^+(v)=\{u|v \prec u\}$.
\end{definition}

Multiple degeneracy orders may exist in a static graph. %and one degeneracy order can be efficiently computed by using an algorithm whose time complexity is linear to the number of edges in the graph \cite{matula1983smallest}.
%
For instance, Figure \ref{fig:tgraph}(b) depicts the static graph $G^*_{[0,2]}$ of the projected graph $G_{[0,2]}$.
%
Both $v_1\prec v_3 \prec v_4 \prec v_2$ and $v_3\prec v_1\prec v_4\prec v_2$ can be the possible $P$ of $G^*_{[0,2]}$, and the degeneracy of $G^*_{[0,2]}$ is two.

%In the following, we first present the state-of-the-art (SOTA) algorithm \DOTTT \cite{pashanasangi2021faster} and then introduce our proposed algorithm.

\subsection{The SOTA Algorithm \DOTTT}
\label{sec:sota}

To solve Problem \ref{prob:delta-counting}, a natural idea is to first extract the projected graph $G_{[t_s,t_e]}$, and then employ the algorithm of counting $\delta$-temporal triangles in $G_{[t_s,t_e]}$.
%
To the best of our knowledge, the SOTA algorithm is \DOTTT \cite{pashanasangi2021faster}, which has three steps:
%
\begin{enumerate}
    \item Derive the degeneracy order $P$ of all vertices in $G^*_{[t_s,t_e]}$;
    
    \item Enumerate all the (static) triangles $\Delta_{u,v,w}$ in ${G^*_{[t_s,t_e]}}$;
    
    \item For each triangle in ${G^*_{[t_s,t_e]}}$, derive the number $R$ of $\delta$-temporal triangles in $G_{[t_s,t_e]}$ which share its three vertices.
\end{enumerate}

In step (1), by continually removing the vertices with the smallest degrees from the graph, we can get the degeneracy order as the deletion order. This step completes in $O(n+m)$ time \cite{matula1983smallest}.

In step (2), we can use a classic triangle counting algorithm \cite{chiba1985arboricity} which costs $O(m\kappa)$ time by using the degeneracy order.
%
Specifically, for each edge $(u,v)$ in ${G^*_{[t_s,t_e]}}$, we first obtain $N^+(u)$ and $N^+(v)$, and then get their intersection to enumerate the triangles.

In step (3), for each static triangle $\Delta_{u,v,w}$, \DOTTT counts all the $\delta$-temporal triangles containing vertices $\{u,v,w\}$, within $O( (|E(u,v)|+|E( u,w)|) \log (|E(v,w)|))$ time.
%
Since the details of step (3) are very complicated, we omit the introduction here.

As a result, by considering all the static triangles, \DOTTT achieves a time complexity of 
$O(\sum_{\{u,v,w\}}  ($ $|E(u,v)|+|E(u,w)|) \log (|E(v,w)|))$.
%
It is proved in \cite{pashanasangi2021faster} that $O(\sum_{\{u,v,w\}}  (|E( u,v)|+|E(u,w)|))= O(m\kappa)$, where $u$ has the smallest degeneracy order among $\{u,v,w\}$.
%
Hence, \DOTTT costs $O(m\kappa\log (m))$ time in total.

\subsection{Our Online Algorithm \OTTC}
\label{sec:edttc}

In this section, we propose a faster online algorithm \OTTC with $O(m\kappa)$ time cost, 
%
{\color{black} %(R3.A4)
which counts the number of $\delta$-temporal triangles by enumerating the temporal edges chronologically and tracking the number of temporal edges and wedges.
%
We further illustrate the steps of \OTTC by a flowchart in Figure \ref{fig:workflow-of-online}.

\begin{figure}[H]
    \centering
    \includegraphics[width=0.7\linewidth]{figure/workflow of online.pdf}
    \caption{\color{black} The flowchart of \OTTC.}
    \label{fig:workflow-of-online}
\end{figure}
}


{\color{black}
Similar to \DOTTT, \OTTC also runs in three steps:
%
\begin{enumerate}
    \item Derives the degeneracy order $P$ of $G^*_{[t_s,t_e]}$;
    
    \item Sequentially enumerates the timestamps $t_i\in [t_s,t_e]$, and dynamically maintains the numbers of temporal edges and wedges for each vertex pair in a sliding window of size $\delta$;
    
    \item Derive the counting result by exploiting the numbers of temporal edges and wedges maintained above.
\end{enumerate}
}
%
Here, a {\it temporal wedge} is formed by two temporal edges $(v_1,v_2,t)$ and $(v_1,v_3,t')$ satisfying $v_1\prec v_2$, $v_1\prec v_3$, and $|t-t'|\leq\delta$.
%
In Figure \ref{fig:tgraph}(a), for instance, assuming $\delta$=1, then $(v_2,v_1,1)$ and $(v_1,v_4,2)$ form a temporal wedge where the center is $v_1$.

The step (1) above is the same as the step (1) of \DOTTT.
%
In step (2), we dynamically maintain the set $I(u,v)$ of temporal edges and the set $\Lambda(u,v)$ of temporal wedges.
%
Specifically, assume $I(u,v)$ has stored the number of temporal edges between $u$ and $v$, and $\Lambda(u,v)$ has stored the number of temporal wedges that take $u$ and $v$ as end vertices, in a time window $[{t_i}-\delta -1,t_i -1]$.
%
Then, when sliding to the next timestamp $t_{i}$, we update $I(u,v)$ and $\Lambda(u,v)$ by throwing historical edges and taking new edges (always assume $u\prec v$):
%
\begin{itemize}
    \item {\bf Throw historical edges:} For each historical edge $(u,v,t_i-\delta-1)$, decrease $I(u,v)$ by 1 since this edge is not in the new time window $[t_i-\delta,t_i]$.
    %
    Meanwhile, decrease $\Lambda(v,w)$ by $I(u,w)$ for each vertex $w\in N^+(u)$, since removing this edge will eliminate temporal wedges that use $u$ as the centers.

    \item {\bf Take new edges:} For each new edge $(u,v,t_i)$, increase $I(u,v)$ by 1 since this edge is in a new time window $[t_i-\delta,t_i]$.
    %
    Meanwhile, increase $\Lambda(v,w)$ by $I(u,w)$ for each vertex $w\in N^+(u)$, since adding this edge will generate temporal wedges that use $u$ as the centers.
\end{itemize}

In step (3), with the updated $I(u,v)$ and $\Lambda(u,v)$, we can count the number of temporal triangles with vertices $\{u,v,w\}$ that include the temporal edge $e=(u,v,t_i)$, by considering two different cases:
%
\begin{enumerate}
    \item \textbf{$w$ has the smallest order ($w\prec u \wedge w\prec v$)}: The number of $\delta$-temporal triangles containing $e$ is $\Lambda(u,v)$, since each of them includes a temporal wedge between $u$ and $v$.

    \item \textbf{$u$ has the smallest order ($u \prec w \wedge u\prec v$)} : The number of $\delta$-temporal triangles containing $e$ is $\sum_{w\in N^+(u)}I(u,w)\times I(v,w)$, since any temporal edge between $u$ and $w$ forms a triangle with any temporal edge between $v$ and $w$, and $e$.
\end{enumerate}

Note that since we assume $u\prec v$ and $v$ cannot have the smallest order, there are only two cases above.
%
Algorithm \ref{alg:ottc} shows \OTTC.
%
We first obtain $G^*_{[t_s,t_e]}$ and $P$, and initialize $I(u,v)$ and $\Lambda(u,v)$ (lines 1-2).
%
Then, we sequentially enumerate the timestamps (lines 4-15), during which we dynamically maintain $I(u,v)$ and $\Lambda(u,v)$ in a sliding window of duration $\delta$.
%
For each timestamp $t_i$, we first throw historical edges (lines 5-8), then consider each new edge, and count the number of temporal triangles containing it (lines 9-15).
%
Finally, we obtain the result $\Delta$ (line 16).

\begin{algorithm}[ht]
\small
\caption{\OTTC}
\label{alg:ottc}

\KwIn{A temporal graph $G$, a time window $[t_s,t_e]$, a threshold $\delta$.}

\KwOut{The number of $\delta$-temporal triangles in $G_{[t_s,t_e]}$.}

Extract the static graph $G^*_{[t_s,t_e]}$ and derive its degeneracy order $P$\;

\textbf{foreach} {\it $(u,v)\in G^*_{[t_s,t_e]}$} \textbf{do} $I(u,v)\gets0$, $\Lambda(u,v)\gets0$\;

$\Delta \gets 0$\;

\ForEach{$t_i\in [t_s,t_e]$}{
    
    \ForEach{$(u,v,t)\in G_{[t_s,t_e]}$ with $t  = t_i-\delta-1$}{
        W.l.o.g., assume $u \prec v$ under $P$\;

        \textbf{foreach} {$w\in N^+(u)$} \textbf{do} $\Lambda(v,w) \gets \Lambda(v,w)-I(u,w)$\;

        $I(u,v) \gets I(u,v) - 1$\;
    }
    \ForEach{$(u,v,t_i) \in G_{[t_s,t_e]}$}{
        W.l.o.g., assume $u \prec v$ under $P$\;
        \ForEach{$w\in N^+(u)$}{
            $\Lambda(v,w) \gets \Lambda(v,w) + I(u,w)$\;
            
            $\Delta \gets \Delta + I(u,w) \times I(v,w)$\;
        }
        $I(u,v) \gets I(u,v) + 1$\;
        $\Delta \gets \Delta + \Lambda(u,v)$\;
    }
    
}

\Return $\Delta$\;

\end{algorithm}

Example \ref{eg:OTTC} further illustrates our algorithm \OTTC.

\begin{example}
\label{eg:OTTC}
Consider the temporal graph in Figure \ref{fig:temporal_graph}(a) and let $[t_s,t_e]$ = [0,2] and $\delta$ = 1.
%
We first obtain the static graph $G^*_{[0,2]}$ shown in Figure \ref{fig:tgraph}(b) and derive its degeneracy order $P: v_1 \prec v_3 \prec v_4 \prec v_2$.
%
Then, we sequentially enumerate the timestamps in [0,2], during which we dynamically maintain $I(u,v)$ and $\Lambda(u,v)$, with values in Table \ref{tab:array}, where each cell is presented in the form $(I(u,v),\Lambda(u,v))$.

Let's take the vertex pair $(v_2,v_4)$ as an example to show how to update $I(v_2,v_4)$ and $\Lambda(v_2,v_4)$.
%
We sequentially enumerate all the edges:
%
(1) $(v_1,v_2,1)$: as there is a temporal wedge between $v_2$ and $v_4$, we increase $\Lambda(v_2,v_4)$ to 1.
%
(2) $(v_1,v_4,2)$: the temporal wedge of $(v_1,v_4,0)$ and $(v_1,v_2,1)$ expires since $(v_1,v_4,0)$ is a historical edge, but there is another temporal wedge between $v_2$ and $v_4$, so $\Lambda(v_1,v_2)$ is still 1.
%
(3) $(v_2,v_4,2)$: we increase $I(v_2,v_4)$ to 1.
\end{example}

\begin{table}[ht]
    \small
    \centering
    \caption{Dynamically maintain $I(u,v)$ and $\Lambda(u,v)$.}
    \label{tab:array}
    \resizebox{\textwidth}{!}{
     % Adjust the font size
    \setlength{\tabcolsep}{3pt} % Adjust horizontal padding
    \renewcommand{\arraystretch}{1.1} % Adjust vertical padding
    \begin{tabularx}{\textwidth}{c*{6}{|>{\centering\arraybackslash}X}}
    \hline
    
    Static edges & \multicolumn{6}{c}{Temporal edges}\\
    \hline 
   -- &  $(v_1,v_4,0)$ & {$(v_3,v_4,0)$} & {$(v_1,v_2,1)$} & {$(v_1,v_4,2)$} & {$(v_2,v_3,2)$}  & {$(v_2,v_4,2)$} \\
    \hline\hline
    
       $(v_1,v_4)$  & (1 , 0) & (1 , 0)  & (1 , 0) & (1 , 0) & (1 , 0) & (1 , 0) \\
    \hline
    
        $(v_1,v_2)$ & (0 , 0) & (0 , 0)  & (1 , 0) & (1 , 0) & (1 , 0) & (1 , 0)\\
    \hline
    
        $(v_2,v_4)$ & (0 , 0) & (0 , 0)  & (0 , 1) & (0 , 1) & (0 , 1) & (1 , 1)\\
    \hline
    
        $(v_2,v_3)$ & (0 , 0) & (0 , 0)  & (0 , 0) & (0 , 0) & (1 , 0) & (1 , 0)\\
    \hline
    
        $(v_3,v_4)$ & (0 , 0) & (1 , 0)  & (1 , 0) & (0 , 0) & (0 , 0) & (0 , 0)\\
    \hline
    \end{tabularx}
    }
\end{table}







\begin{lemma}
\label{lemma:OTTCTime}
Given a temporal graph $G$, a query time window $[t_s,t_e]$, and a threshold $\delta$, \OTTC completes in $O(m\kappa)$ time.
\end{lemma}

\begin{proof}
In \OTTC, for each temporal edge $(u,v,t)$, we need to access each vertex in $N^+(u)$ constant times.
%
Since $|N^+(u)| = d^+(u) \le \kappa$, the total time complexity of \OTTC is $O(m\kappa)$.
\end{proof}



\section{An Index-Based Solution}
%\section{Index-based solutions for counting $\delta$-temporal triangles}
\label{sec:weighted}

%In practice, to accomplish a specific task, users often have to issue $\delta$-temporal triangle counting queries by varying the query time window $[t_s,t_e]$ and duration $\delta$ multiple times.
%
%In such scenarios, the online solution may incur significant overhead due to frequent access to the projected graph for each query.

To enable frequent $\delta$-temporal triangle counting queries, in this section, we propose an index-based solution that offers accelerated response times while keeping acceptable preprocessing time and space costs.
%
One direct index-based solution is to run the online algorithm for all possible time windows $[t_s,t_e]$ and durations $\delta$, and then store the counting results.
%
This, however, incurs an index space cost of $O(t_{max}^3)$ and requires $O(m\kappa\times t_{max}^3)$ time to build the index, making it impractical for large $t_{max}$.

To alleviate this issue, we propose a novel index-based solution using the \wavelet, which significantly reduces the indexing time and space costs.
%
Moreover, the index-based counting algorithm is much faster than \OTTC.
%
In the following, we first present the main idea of our index, and then show the detailed solution.
%
{\color{black} %(R3.A4)
We also illustrate the index-based solution in a flowchart below.

\begin{figure}[ht]
    \centering
    \includegraphics[width=0.6\linewidth]{figure/workflow of index.pdf}
    \caption{\color{black} The flowchart of our index-based solution.}
    \label{fig:workflow-of-index}
\end{figure}
}

\subsection{Main Idea}
\label{sec:index-overview}
{\color{black}

%(R3.A4)
%
% We first observe a necessary condition for a $\delta$-temporal triangle to be considered in a query: the time interval formed by the minimum and maximum timestamps of its three edges must be completely encompassed by the query interval. 
%
% This observation inspires us to construct a 2-dimensional space, where one axis represents the minimum timestamps of $\delta$-temporal triangles and the other represents the maximum timestamps, providing a framework to facilitate answering queries.
% 
We begin with a necessary condition for a $\delta$-temporal triangle to be considered in a query: the minimum and maximum timestamps of its three edges must be fully covered by the query interval $[t_s,t_e]$.
%
This condition motivates us to treat each $\delta$-temporal triangle as a 2-dimensional point, whose two dimension values are the minimum and maximum timestamps of its three edges respectively, and further indexing the 2-dimensional points of all $\delta$-temporal triangles for answering the counting queries.
%
%This condition motivates us to construct an index for points in a 2-dimensional space, with one axis representing the minimum timestamps and the other the maximum timestamps of $\delta$-temporal triangles, providing a structured approach for answering queries.
%
Therefore, our core idea is to convert $\delta$-temporal triangles to points in this 2-dimensional space, referred to as {\it counting-points} or {\it C-points}, and demonstrate that the $\delta$-temporal triangle counting problem can be reduced to counting C-points.
}
% Our main idea is to convert the $\delta$-temporal triangles to some points, called {\it counting-points} or {\it C-points}, in the 2-dimensional space, and then show that the $\delta$-temporal triangle counting problem can be solved by counting C-points.
%
We first formally define C-point as follows.

\begin{definition}[C-point]
\label{def:c-point}
Given a temporal graph $G$, a C-point, denoted by $\langle (x,y),c \rangle$, means that the number of $\delta$-temporal triangles in $G$ satisfying: 1) $\delta$=$\infty$ and 2) the minimum and maximum timestamps of each triangle are exactly $x$ and $y$ respectively, is $c$.
%Given a temporal graph $G$, a C-point is a 2-dimensional point $(x, y)$ associated with count $c$, denoted by $\langle (x,y),c \rangle$, meaning that there are $c$ $\delta$-temporal triangles in $G$ such that for each of them, the minimum and maximum timestamps of its three edges are exactly $x$ and $y$, respectively.
\end{definition}

In the above definition, we set $\delta$=$\infty$, which means that any three temporal edges that form a triangle structure in the time window $[x,y]$ will be considered a $\delta$-temporal triangle.
%
Assuming that $G$ has $t_{max}$ timestamps, there are{ \color{black}$O(t_{max}^2) = O(m^2)$ }possible different time windows, so the total number of C-points, denoted as $\pi$, is bounded by {\color{black} $O(t_{max}^2) = O(m^2)$}.
%
Meanwhile, the number of C-points is also bounded by the number of all the possible $\delta$-temporal triangles in $G$.
%
All these C-points can be placed on a 2-dimensional space of $x$ and $y$, or more precisely in the area above the curve $y=x$ since $x\leq y$.
%
Example \ref{eg:cp} further illustrates the concept of C-point.

\begin{figure}[htbp]
\centering
\includegraphics[width=.5\linewidth]{figure/weighted C-point.pdf}
    \caption{C-points of the temporal graph in Figure \ref{fig:temporal_graph}.}
    \label{fig:C-point}
\end{figure}

\begin{example}
\label{eg:cp}
In Figure \ref{fig:temporal_graph}, there are 5 $\delta$-temporal triangles in total ($\delta$=$\infty$), but there are 4 C-points, as depicted in the 2-dimensional space of Figure \ref{fig:C-point}. 
%
For instance, the C-point $\langle (1,2),1\rangle$ indicates that in the time window $[1,2]$, there is one $\delta$-temporal triangle whose minimum and maximum timestamps are exactly 1 and 2 respectively, since its three edges are $(v_1,v_2,1)$, $(v_1,v_4,2)$, and $(v_2,v_4,2)$.
\end{example}

After converting all the $\delta$-temporal triangles in the temporal graph to C-points, we can directly count the number of $\delta$-temporal triangles for any arbitrary query parameters $\delta$ and  $[t_s,t_e]$, by only using these C-points without accessing the original graph.
%
Specifically, we can first collect all the C-points $\langle (x, y),c\rangle$ satisfying $x\geq t_s$, $y \leq t_e$, and $y-x\leq \delta$.
%
Afterward, we just need to add their $c$ values together to get the final counting result.

With careful observation, we find that all these C-points are actually covered by a trapezoid area and a rectangle area :
%
\begin{itemize}
    \item {\bf Trapezoid area:} $y-\delta \leq x\leq t_e$ and $t_s + \delta < y \leq t_e$.

    \item {\bf Rectangle area:} $t_s\leq x\leq t_e$ and $t_s\leq y\leq t_s+\delta$.
\end{itemize}

\begin{example}
\label{eg:areas}
Reconsider Example \ref{eg:cp} and let $[t_s,t_e]$ = $[0,3]$ and $\delta$=2.
%
To answer this counting, we need to collect all the C-points (marked in blue) covered by a square area ($0\leq x\leq 3$ and $0\leq y\leq 2$) and a trapezoid area ($y-2 \leq x\leq3$ and $2 < y\leq 3$) which are marked in two different colored areas.
%
The counting result is the sum of all the $c$ values of these C-points, i.e.,  2 + 1 + 1 = 4.
\end{example}

%Next, we would like to discuss the algorithms of how to efficiently convert $\delta$-temporal triangles to C-points and how to effectively index C-points to support $\delta$-temporal triangle counting.

Next, we introduce efficient algorithms for generating C-points and indexing C-points to support $\delta$-temporal triangle counting.

\subsubsection{Converting $\delta$-temporal Triangles to C-points}
\label{sec:convert-triangle-to-C-point}
%
To generate C-points, we sequentially enumerate the edges in $G$, and for each edge $(u,v,t)$, we count the number of $\delta$-temporal triangles $(\delta = \infty)$ that include $u$ and $v$ and take $t$ as the maximum timestamp in three steps:
%
\begin{enumerate}
    \item Find the vertex set $W = N(u) \cap N(v)$;

    \item For each vertex $w \in W$, we enumerate each edge $(w,v,t')$ in $E(w,v)$, then count the number $c$ of $\delta$-temporal triangles that take $t'$ and $t$ as the minimum and maximum timestamps respectively which can be completed efficiently by using binary search, and finally obtain a C-point $\langle (t',t), c\rangle$.

    \item For each vertex $w \in W$, we also enumerate each edge in $E(u,w)$ and obtain a list of C-points in a similar manner.
\end{enumerate}

\begin{algorithm}[ht]
\small
\caption{Convert $\delta$-temporal triangles to C-points}
\label{alg:ttm}

\KwIn{A temporal graph $G=(V,E)$}

\KwOut{A list of C-points $L$}

$L$ $\leftarrow$ $\emptyset$\;
\textbf{foreach} {\it $(u,v,t)\in G$} \textbf{do} $E(u,v)\gets \emptyset$\;    
    
\ForEach{$( u,v,t)\in G$ with timestamps in ascending order}{
append $(u,v,t)$ to $E(u,v)$\;
    \ForEach{$w \in N(u) \cap N(v)$}{
        \ForEach{$( w,v,t') \in E( w,v)$ in ascending order of $t'$}{
            $c \gets |\{(u,w,t'') | (u,w,t'') \in G \wedge t'' \in [t',t]\}$|\;
        
            append $\langle (t',t),c \rangle$ to $L$\;
        }
    
        \ForEach{$( u,w,t') \in E(u,w)$ in ascending order of $t'$}{
            $c$ $\gets$ |$\{(w,v,t'') | ( w,v,t'') \in G \wedge t'' \in (t',t]\}$|\;
    
            append $\langle (t',t),c \rangle$ to $L$\;
        }
    }
}

\Return $L$\;
\end{algorithm}

Algorithm \ref{alg:ttm} shows the details.
%
We first initialize a list $L$ and some edge lists (lines 1-2).
%
Then, we generate C-points by enumerating the edges chronologically (lines 3-11).
%
Specifically, we first update $E(u,v)$ (line 4), then find a vertex set $W = N(u) \cap N(v)$ (line 5), and count the number of $\delta$-temporal triangles $(\delta$=$\infty)$ that include vertices $\{u,v,w\}$ and take $t$ as the maximum timestamp to obtain the C-points (lines 6-11).
%
Finally, we get all the C-points (line 12).

\begin{lemma}
\label{lemma:alg:ttm-time}
Given a temporal graph $G$, Algorithm \ref{alg:ttm} completes in {\color{black} $O(m\kappa + \pi \log(m))$ time.}
\end{lemma}

\begin{proof}
For lines 3-5, the time cost of finding $N(u) \cap N(v)$ is the same as that of finding static triangles in $G^*$, which is $O(m\kappa)$ \cite{al2018triangle}.
%
For lines 6-11, we can finish counting in $O(|E( w,v)| \log(|E(u,w)|) + |E( u,w)| \log(|E(w,v)|))$ time.
%
Note that $O(|E(w,v)|+|E(u,w)|)$ equals the time complexity of the for-loops at lines 6-9, where each iteration produces a C-point, so the time cost of lines 6-9 is {\color{black}$O(\pi)$}.
%
The total time cost of lines 6-11 is bounded by {\color{black}$O(\pi\log(m))$}.
%
Hence, Algorithm \ref{alg:ttm} costs {\color{black} $O(m\kappa + \pi\log(m))$} time.
\end{proof}

\subsubsection{Indexing C-points for Counting $\delta$-temporal Triangles} After obtaining all the C-points, a natural method to count the $\delta$-temporal triangles is to collect all the C-points $\langle (x, y),c\rangle$ satisfying $x\geq t_s$, $y \leq t_e$, and $y-x\leq \delta$, and summarize their $c$ values together.
%
This method, however, is very costly since it takes $O(\pi)$ time, and may be slower than our online algorithm \OTTC in some cases.

We notice that in the literature, many effective index structures, such as wavelet tree \cite{grossi2003high}, kd-tree \cite{10.1145/361002.361007}, Fenwick tree \cite{fenwick1994new}, segment tree \cite{de2000computational}, etc., have been developed for supporting range search, which aim to efficiently count the number of points in an arbitrary rectangle area in the 2-dimensional space.
%
As aforementioned, the C-points for counting the $\delta$-temporal triangles in $[t_s,t_e]$ are actually covered by a rectangle area and a trapezoid area.
%
For the C-points in the rectangle area, we can efficiently count them by building these index structures for all the C-points of the whole graph.

However, for the C-points in the trapezoid area, they cannot be counted directly by employing these index structures since they focus on rectangle areas.
%
To resolve this issue, we propose to change the 2-dimensional location of each C-point, so that the set of C-points in any original trapezoid area can be covered by a rectangle area, making the above existing index structures applicable.
%
Recall that for any C-point $\langle (x,y),c \rangle$ in the trapezoid area, we have
%
\begin{equation}
\begin{cases}
y-\delta \leq x\leq t_e\\
x \le y
\end{cases}
\stackrel{}{\Longrightarrow }
0 \le y - x\leq \delta
\end{equation}

Since $x \le y$ and $y-x\leq\delta$, by setting $z=y-x$, we further obtain 
%
\begin{equation}
\begin{cases}
0\leq y-x\leq \delta\\
t_{s} +\delta < y\leq t_{e}
\end{cases}
%
\stackrel{z=y-x}{\Longrightarrow}
%
\begin{cases}
0\leq z\leq \delta\\
t_{s} +\delta < y\leq t_{e}
\end{cases}
\end{equation}

Hence, for any C-point $\langle (x,y),c \rangle$ in the 2-dimensional space of $x$ and $y$, by changing $x$ to $z=y-x$, we can obtain another kind of C-point $\langle (z,y),c \rangle$, which is in the 2-dimensional space of $z$ and $y$.
%
To distinguish it from the original C-point, we call it a $\widehat {\text{C}}$-point.
%
{\color{black} Clearly, the number of $\widehat {\text{C}}$-points is also bounded by $\pi$.}

A nice feature of all the $\widehat {\text{C}}$-points is that they must be in a rectangle area.
%
Thus, by making such changes, all the C-points in the original trapezoid area can be covered by a rectangle area, so the existing index structures for range search can also be used.
%
%Example \ref{eg:trapezoid-area} further illustrates the $\widehat {\text{C}}$-points.

\begin{figure}[ht]
    \centering
    \includegraphics[width=.4\linewidth]{figure/converted-area.pdf}
    \caption{Changing C-points to $\widehat {\text{C}}$-points.}
    \label{fig:hat-C-points}
\end{figure}

\begin{example}
\label{eg:trapezoid-area}
Consider all the C-points in the 2-dimensional space of $x$ and $y$ in Figure \ref{fig:C-point}.
%
By changing $x$ to $z=y-x$ for each C-point, we get all the $\widehat {\text{C}}$-points in another 2-dimensional space of $z$ and $y$ in Figure \ref{fig:hat-C-points}.
%
For instance, a C-point $\langle (1,3),1\rangle$ is changed to $\langle(2,3),1\rangle$.
%
Besides, for all the C-points in the trapezoid area of Figure \ref{fig:C-point}, their corresponding $\widehat {\text{C}}$-points are in a rectangle area marked in grey.
\end{example}

As shown in the literature, the Chazelle's structure \cite{chazelle1988functional}, a.k.a. wavelet tree \cite{grossi2003high}, achieves a better balance between time and space than others for counting points in the 2-dimensional space \cite{deng2023space}.
%
Thus, in this paper we employ it to index all the C-points and $\widehat {\text{C}}$-points respectively, and call the wavelet tree-based index {\tt WT-Index}.

\subsection{Our {\tt WT-Index}-Based Solution}
\label{sec:wavelet}

We first introduce the {\tt WT-Index} for all the C-points, then discuss the index-based counting algorithm, and finally show the index maintenance algorithm for dynamic temporal graphs.
%
The {\tt WT-Index} for $\widehat {\text{C}}$-points can be built similarly, so we omit the details.

\subsubsection{Overview of {\tt WT-Index}}
\label{sec:overview-wavelet}

Given a C-point list $L$, a {\tt WT-Index} is a binary tree, where each node\footnote{To avoid ambiguity, we use ``node'' to mean a node of the tree index, and use ``vertex'' to denote a graph vertex in this paper.} has a time interval and stores the information of C-points whose $x$ value falls in this interval.
%
To ease the illustration, we declare that a tree node \textit{includes} a C-point if its time interval covers the C-point's $x$ value.
%
Specifically, each node of {\tt WT-Index} has three key components:
%
\begin{enumerate}
    \item {\bf A time interval [$l$, $r$]:} It indicates that the node stores the information of all the C-points $\langle (x,y),c\rangle$ with $x\in [l,r]$.
    %
    The root node has the largest interval $[0,T]$ with $T\textless 2t_{max}$.
    %
    For each node, if $l \neq r$, then it has two children whose time intervals are $\left[l,\lfloor \frac{l+r}{2} \rfloor\right]$ and $\left[\lfloor \frac{l+r}{2} \rfloor +1, r\right]$ respectively; otherwise, it is a leaf node with $l=r$.

    \item {\bf An integer array $C$[ ]:} It stores the $c$ values of all the C-points included by the node, where all the C-points are assumed to be sorted in ascending order of their $y$ values.

    \item {\bf A boolean array $Pos$[ ]:} It records which child nodes include the C-points that the current node includes.
    %
    Given that all the C-points included in the current node are sorted in ascending order of their $y$ values if the $i$-th C-point is also included by the left node, then $Pos[i]$=0; otherwise, $Pos[i]$=1.
    %
    For leaf nodes, their $Pos$[ ] = $\emptyset$.
\end{enumerate}

In addition, to enable efficient retrieval, we add a prefix sum array for $C$[ ] and $Pos$[ ] in each node, respectively.

\begin{figure}[ht]
    \centering
    \includegraphics[width=.37\linewidth]{figure/wavelet tree.pdf}
    \caption{{\tt WT-Index} for the C-points in Figure \ref{fig:C-point}.}
    \label{fig:wavelet-tree}
\end{figure}

We further illustrate the {\tt WT-Index} via Example \ref{eg:wavelet}.

\begin{example}
\label{eg:wavelet}
Figure \ref{fig:wavelet-tree} depicts the {\tt WT-Index} built for all the C-points in Figure \ref{fig:C-point}.
%
For instance, the internal node with time interval $[0,1]$, it includes three C-points, i.e., $\langle(0,2),2\rangle$,  $\langle(0,3),1\rangle$and $\langle(1,2),1\rangle$, so $C[1]=2$, $C[2]=1$, and $C[3]=1$.
%
Since these three C-points are also included by its left and right child nodes respectively, we have $Pos[1]=0$, $Pos[2]=1$, and $Pos[3]=0$.
\end{example}

As shown in \cite{chazelle1988functional}, given a list of 2-dimensional points, we can build the  {\tt WT-Index} efficiently in a bottom-up manner: we first initialize all the leaf nodes, and then iteratively build their parent nodes by merging child nodes' information until the root is built, which is similar to the process of MergeSort.
%
The indexing time and space costs \cite{chazelle1988functional} of building the {\tt WT-Index} are {\color{black} $O(\pi\log(m))$ and $O(\pi)$ respectively}, and the space cost of {\tt WT-Index} is {\color{black} $O(\pi)$}.

\subsubsection{{\tt WT-Index}-Based Counting Algorithm}
\label{sec:indexCounting}

Recall that the number of $\delta$-temporal triangles in $[t_s,t_e]$ equals the summarized value of the $c$ values of all the C-points in a rectangle area and a trapezoid area.
%
Since the trapezoid area can be transferred to a rectangle area by changing C-points to $\widehat{\text{C}}$-points.
%
Therefore, the key question is that given a rectangle area: $t_s \le x \le t_e$ and $t_s \le y \le t_s +\delta$, how to obtain the summarized value of the $c$ values of all the C-points it covers.
%
%Recall that the number of $\delta$-temporal triangles in $[t_s,t_e]$ equals to the summarized value of the $c$ values of all the C-points in the rectangle area: $t_s \le x \le t_e -\delta$ and $t_s \le y \le t_s +\delta$.
%
Since this rectangle is the difference of the other two rectangle areas: (1) $t_s \le x \le t_e$ and $0 \le y \le t_s +\delta$ and (2) $t_s \le x \le t_e$ and $0\leq y \le t_s-1$, we can derive the summarized values from them respectively and subtract them.
%
As these two rectangle areas can be processed by the same procedure, we design an efficient algorithm to obtain the summarized value of the $c$ values of all the C-points in a rectangle area with $t_s \le x \le t_e $ and $0 \le y \le \theta$ where $\theta\in\{t_s+\delta,t_s-1\}$.

Specifically, we start from the root of the {\tt WT-Index} and recursively find all the nodes whose time intervals are covered by $[t_s,t_e]$ level by level in a top-down manner, during which we add the $c$ values of all the C-points whose $y \le \theta$ in these nodes.
%
Algorithm \ref{alg:summarize-cvalue} shows the details.
%
We first get the root and obtain the number of C-points whose $y\le \theta$ (lines 1-2).
%
Then, we use a function {\tt sum} to sum the $c$ values recursively (lines 3-10).
%
For each node, if its time interval is covered by $[t_s,t_e]$, we sum up the $c$ values of the C-points whose $y\le \theta$ and included by the node (lines 5-6);
if its time interval is not intersected with $[t_s,t_e]$, the count is 0; otherwise, we need to traverse its left and right child nodes and get the summarized result recursively (lines 7-10).

\begin{algorithm}[ht]
\small
\caption{Sum the $c$ values of C-points in a rectangle}
\label{alg:summarize-cvalue}
\KwIn{{\tt WT-Index}, a rectangle area $t_s \le x \le t_e $ and $0 \le y \le \theta$}

\KwOut{The summarized $c$ values of C-points in the rectangle}

$node\gets$root of the {\tt WT-Index}\;

$p \gets \#$  of C-points whose $y \leq \theta$\;

\SetKwFunction{query}{{\tt sum}}
    \SetKwProg{Fn}{Function}{:}{}
    \Fn{\query {$node, p$}} { 
        $[l,r] \gets$the time interval of $node$\;

        \If{$[l,r]\subseteq[t_s,t_e]$}{
            \Return $\sum_{i=1}^{p} node.C[i]$\;
        }
        
        \Else{
            $o \gets \sum_{i=1}^{p} node.Pos[i]$\;

            {\bf if} $[l,r]\cap[t_s,t_e]=\emptyset$ {\bf then} {\bf return} 0\;

            {\bf else return }{\tt sum}($node.lChild$, $p$--$o$)+{\tt sum}($node.rChild$, $o$)\;
       }
    }

\end{algorithm}

According to \cite{chazelle1988functional}, {\color{black} by storing $\sum_{i=1}^p node.Pos[i]$ and $\sum_{i=1}^p node.C[i]$ for $p \% \log(m) = 0$} to accelerate the calculation, Algorithm \ref{alg:summarize-cvalue}  can complete in {\color{black} $O(poly\log(m))$} time.
%
Thus, the overall time cost of the index-based counting algorithm is also {\color{black} $O(poly\log(m))$}.

\subsubsection{Index Maintenance}
\label{sec:WT-index-maintain}

In real-world applications, the temporal graphs are often  continually updating, due to the generation of new temporal edges,
which requires the {\tt WT-Index} to be updated as well.
%
A simple method of updating the index is to rebuild the {\tt WT-Index} from scratch whenever a new temporal edge is inserted, but it is extremely costly when the new edges are frequently inserted.
%
A few works have studied the maintenance of wavelet trees, but they are often limited to very specific scenarios.
%
For instance, \cite{da2017online} presents an algorithm for maintaining the wavelet tree, but it aims to count the frequency of some texts, which actually counts points in a 1-dimensional space, so it cannot process our C-points.

In this paper, we develop a novel efficient maintenance algorithm to update the {\tt WT-Index} for the C-points after inserting a new temporal edge ($u$, $v$, $t_{max}+1$) into the graph.
%
The {\tt WT-Index} for the $\widehat {\text{C}}$-points can be maintained similarly, so we omit its details.
%
Our key idea is that after an edge insertion, we first identify all the newly formed $\delta$-temporal triangles ($\delta$=$\infty$) and convert them into some C-points, which can be easily implemented by following Algorithm \ref{alg:ttm} since it generates C-points by sequentially processing the temporal edges one by one.
%
Afterward, we update the {\tt WT-Index} to incorporate the information of the new C-points.
%
Specifically, for each new C-point $\langle(x,y),c\rangle$, we consider two cases:
\begin{enumerate}
    \item[1)] {\bf $\langle(x,y),c\rangle$ is included by the root:} we can start from the root of the {\tt WT-Index} and recursively update the information of all the nodes including the new C-point level by level in a top-down manner.

    \item[2)] {\bf $\langle(x,y),c\rangle$ is not included by the root:} Let the time interval of the root be $[0, T]$. We first create a new root with a time interval $[0, 2T]$ and initialize its arrays using the information in the original root, then continue the generating process until the new root includes $\langle (x,y),c\rangle$, and finally use the idea of case 1) to update the remaining nodes.
\end{enumerate}

Algorithm \ref{alg:update} shows our maintenance algorithm.
%
We first generate new C-points by Algorithm \ref{alg:ttm} (line 1).
%
Then, for each new C-point $\langle (x,y),c \rangle$, we check whether it is included by the root of {\tt WT-Index} (lines 2-4).
%
If not, we generate new roots until the new root includes it (lines 5-6).
%
Next, we use a recursive function to update the nodes' information from the root node (lines 7-16).
%
In {\tt update}, we append the $c$ value to $node.C$[ ] of the current node (line 9), then if the current node is a leaf node, we terminate the recursion (line 10); otherwise, we check which child node includes the new C-point to update $node.Pos$[ ] and continue the recursion (lines 11-16).

\begin{algorithm}[ht]
\small

\caption{{\tt WT-Index} maintenance algorithm}
\label{alg:update}

\KwIn{$G$, {\tt WT-Index} for $G$, and a new edge ($u$, $v$, $t_{max}$+1)}

\KwOut{The updated {\tt WT-Index}}

run lines 3-11 of Algorithm \ref{alg:ttm} to generate a new list $L'$ of C-points\;

\ForEach{C-point $\langle (x,y),c\rangle \in L'$}{

    $node \gets$ root of the {\tt WT-Index} with time interval $[0,T]$\;

    \While{$node$ does not include $\langle (x,y),c\rangle$}{
        generate a new root with time interval $[0,2T]$ storing the information of the original C-points\;
        
        $node \gets$ new root of the {\tt WT-Index}\;
    }

    {\tt update}($node$, $\langle (x,y),c \rangle$)\;
}
\SetKwFunction{update}{{\tt update}}
\SetKwProg{Fn}{Function}{:}{}
\Fn{\update {$node, \langle(x,y),c\rangle$}}{
    
    append $c$ to the end of $node.C$[ ]\;
    
    {\bf if} {\it $node$ is a leaf node} {\bf then} return\;
    
    \If{ $node.lChild$ includes $\langle (x,y),c \rangle$ }{
        append 0 to the end of $node.Pos$[ ]\;
        {\tt update}($node.lChild$,$\langle(x,y),c\rangle$)\;
    }
    \Else{
        append 1 to the end of $node.Pos$[ ]\;
        {\tt update}($node. rChild$,$\langle(x,y),c\rangle$)\;
    }
}

\end{algorithm}

%We illustrate the above maintenance algorithm by Example \ref{eg:maintain}.

\begin{figure}[ht]
    \centering
    \includegraphics[width=.65\linewidth]{figure/wavelet update2.pdf}
    \caption{An example of updating {\tt WT-Index}.}
    \label{fig:wavelet-update2}
\end{figure}

\begin{example}
\label{eg:maintain}
Figure \ref{fig:wavelet-update2}(a) shows the {\tt WT-Index} for C-points of the projected graph $G_{[0,2]}$ in Figure \ref{fig:tgraph}(a).
%
After inserting edges with timestamp 3, we get a new C-point $\langle (2,3),1\rangle$.
%
By Algorithm \ref{alg:update}, since 2 $\notin [0,1]$, we first generate a new root node with time interval $[0,2]$.
%
We then use {\tt update} function to update {\tt WT-Index}. 
%
Specifically, we first append 1 to $C$[ ] and $Pos$[ ] of the root node, then go to the right child of the root since it includes the new C-point, and finally append 1 to $C$[ ] of the leaf node with time interval $[2,2]$.
\end{example}


\begin{lemma}
\label{lemma:maintain-time}
Given a temporal graph $G$ and its {\tt WT-Index}, Algorithm \ref{alg:update} costs {\color{black} $O(\pi \log(m))$} time to process an edge insertion.
\end{lemma}

\begin{proof}
Assume the root node of the {\tt WT-Index} has a time interval $[0, T]$ where $T \textless 2t_{max}$.
%
Then, the tree height of {\tt WT-Index} is {\color{black} $O(\log (t_{max})) = O(\log(m))$}.
%
To generate a new root node, we have to copy all the information from the original root, which takes {\color{black} $O(\pi)$} time, and the number of new root nodes is bounded by {\color{black} $O(\log(m))$}, so generating the root nodes costs at most {\color{black} $O(\pi \log(m))$} time.
%
Besides, the time complexity of running {\tt update} is linear to the height of the tree, so it costs {\color{black}$O(\log(m))$} time.
%
Since there are $|L'|$ C-points, the overall cost of running {\tt update} is {\color{black} $O(|L'|\log(m))$} time.
%
Hence, the total time cost of Algorithm \ref{alg:update} is  {\color{black} $O(\pi \log(m))$}.
\end{proof}

{\color{black}
The time cost of generating new root nodes is high, but there will be at most $O(\log(m))$ new root nodes, so the total time cost of Algorithm \ref{alg:update} is faster than rebuilding {\tt WT-Index} from scratch.
%
We will experimentally evaluate Algorithm \ref{alg:update} in Section \ref{sec:experiment-weighted}.

% The total time cost of invoking Algorithm \ref{alg:update} to process multiple edge insertions remains $O(m^2 \log(m))$. Because multiple edge insertions create only more new C-points instead of more new root nodes. As a result, the time complexity of generating root nodes remains $O(m^2 \log(m))$, and the time complexity of updating new C-points is $O(|L'|\log(m))$. Since $|L'| = O(\pi)  = O(m^2)$, the total time cost of invoking Algorithm \ref{alg:update} to process multiple edge insertions is still $O(m^2 \log(m))$, which is much faster than rebuilding {\tt WT-Index} from scratch multiple times. The experiment results in Section \ref{sec:experiment-weighted} also depict that our method is efficient.
}


{\color{black}$\bullet$ \textbf{Extension for counting other patterns.}
%(R1.A5)
Our index-based solution above can be easily extended for counting other patterns, like rectangles \cite{zhu2018fast}, cliques (of constant size) \cite{jain2020power}, and stars \cite{finocchi2024stars}. 
%
Given a target pattern, we can first enumerate the matched instances from the graph, and then for each instance, we fetch the minimum and maximum timestamps, denoted by $x$ and $y$ respectively.
%
As a result, each instance can be converted to a C-point $\langle (x,y):1 \rangle$.
%
Finally, we can apply our index-based solution by using these C-points.
}

\begin{comment}
\subsubsection{Approximate $\delta$-temporal triangle counting}
\label{sec:index-approximate}

Recall that the {\tt WT-Index} costs $O(\Delta \log(\Bar{c}))$ space.
%
When the number of C-points is extremely large, its space cost may not be affordable.
%
To tackle this issue, we propose a sampling-based indexing approach with theoretical error guarantee.
%
The main idea is that we first sample a small proportion of the C-points and then build a smaller {\tt WT-Index} for them.
%
Specifically, given a list of original C-points $L$ and a sampling factor $\alpha \in (0,1]$, we sample a new list $\widehat{L}$ of C-points uniformly at random by selecting $\alpha \cdot |L|$ C-points from $L$. 
%
Then, for each C-point $\langle (x,y),c \rangle$ in $\widehat{L}$, we enlarge its $c$ value to $\frac{c}{\alpha}$.
%
Finally, we construct the {\tt WT-Index} based on $\widehat{L}$ rather than $L$.
%
When answering the counting problem, our index-based counting algorithm in Section \ref{sec:indexCounting} can be applied directly.

Now we analyze the error caused by sampling.
%
Given the $c$ values of original C-points are $c_1,c_2,\cdots, c_{|L|}$, assume that their expectation and variance are $\mu$ and $\sigma^2$ respectively.
%
Then, for the $c$ values of C-points in $\widehat{L}$, their expectation and variance should be $\frac{\mu}{\alpha}$ and $\frac{\sigma^2}{\alpha^2}$, respectively.

Given a time window $[t_s,t_e]$ and a duration $\delta$, to count the $\delta$-temporal triangles, let the expected number of C-points that are involved in the counting process from $L$ be $g$.
%
Then, the expected number of C-points that are involved in the counting process from $\widehat{L}$ should be $\alpha\cdot g$.
%
As a result, the expected numbers of $\delta$-triangles by using $L$ and $\widehat{L}$ are $g \cdot \mu$, but their variances are $g\times \sigma^2$ and $g\times \frac{\sigma^2}{\alpha}$ respectively.
%
By the central limit theorem \cite{kwak2017central}, if the C-point counting query is randomly issued, we can model the distributions of the counting results on $L$ and $\widehat{L}$ with the Gaussian distribution, and further derive the following error bound:

\begin{lemma}
\label{lem:error}
Given two lists of C-points $L$ and $\widehat{L}$, let $R$ and $\widehat{R}$ denote the counting results using $L$ and $\widehat{L}$ respectively.
%
Then, the error can be bounded by
%
\begin{equation}
     \Pr\left[|R-\widehat{R}|\geq \epsilon \cdot g \cdot \mu\right] \leq \frac{\sigma^2 \cdot (1+\alpha)}{g\cdot \mu^2 \cdot \alpha\cdot \epsilon^2},
\end{equation}
where $\epsilon > 0$ is a user-specific parameter.
\end{lemma}

\begin{proof}
Since the distributions of the counting results on $L$ and $\widehat{L}$ with the Gaussian distribution, we can conclude $E[R- \widehat{R}] = 0$  and $Var[R - \widehat{R}] =g \cdot \sigma^2+ g\cdot \frac{\sigma^2}{\alpha} =\frac{g \cdot \sigma^2 \cdot (1+\alpha)}{\alpha}$.

By using Chebyshev's inequality \cite{alsmeyer2011chebyshev}, we know that the probability that the error exceeds $\epsilon \cdot g \cdot \mu$ with a sampling factor $\alpha$ is bounded by:
%
\begin{equation}
    \Pr\left[|R-\widehat{R}|\geq \epsilon\cdot g\cdot \mu\right] \le \frac{g\times \sigma^2 \cdot (1+\alpha)}{g^2\cdot \mu^2\cdot \alpha \cdot \epsilon^2} = \frac{\sigma^2 \cdot (1+\alpha)}{g\cdot \mu^2 \cdot \alpha \cdot \epsilon^2}.
\end{equation}
\end{proof}

Clearly, as the number of sampled C-points increases, the difference between the counting results using $L$ and $\widehat{L}$ becomes small.
%
For instance, on the StackOverflow dataset ($|V|$ = 2.6M, $|E|$ = 63M, $t_{max}$ = 41M) \footnote{https://snap.stanford.edu/data/sx-stackoverflow.html}, we have $\mu$=5.6074, $\sigma^2$=293.453, and the expected counting result $g \cdot \mu = 2.7\times 10^9$.
%
If we set $\alpha=0.01$, then the probability of the relative error exceeding $0.01$ is less than $0.03$ by Lemma \ref{lem:error}.
%
In other words, by only using $1\%$ of the C-points, we can achieve a relative error of 0.01 with a probability of $0.97$.
\end{comment}

\section{Binary $\delta$-Temporal Triangle Counting}
\label{sec:binary}

As shown in Definition \ref{def:delta-trianggle}, multiple $\delta$-temporal triangles may share the same three vertices, due to the existence of multiple temporal edges between two vertices.
%
In many real-world scenarios, we may only need to consider the existence of $\delta$-temporal triangle among three vertices \cite{jung2019furl,gou2021sliding}, so we introduce the binary $\delta$-temporal triangle counting problem:

\begin{problem}[Binary $\delta$-temporal triangle counting]
\label{prob:binary-delta-counting}
Given a temporal graph $G$, a time window $[t_s,t_e]$, and a duration $\delta$ ($\delta\geq0$), return the number of sets of vertices, each of which contains three vertices and forms at least one $\delta$-temporal triangle within $G_{[t_s,t_e]}$, where each such set is called a binary $\delta$-temporal triangle.
\end{problem}

We illustrate Problem \ref{prob:binary-delta-counting} via Example \ref{eg:deltaCounting}.

\begin{example}
\label{eg:deltaCounting}
In the temporal graph of Figure \ref{fig:temporal_graph}(a), let $[t_s,t_e]$=[0,2] and $\delta$=2.
%
The projected graph $G_{[0,2]}$ is depicted in Figure \ref{fig:tgraph}(a).
%
Clearly, there are two binary 2-temporal triangles in $G_{[0,2]}$, i.e., $\{v_1,v_2,v_4\}$ and $\{v_2,v_3,v_4\}$.
%
Note that there are three 2-temporal triangles since two 2-temporal triangles share a vertex set $\{v_1,v_2,v_4\}$.
\end{example}

To solve Problem \ref{prob:binary-delta-counting}, we propose an efficient online algorithm and an index-based solution in Sections \ref{sec:bttc-algo} and \ref{sec:binary-index}, respectively.

\subsection{Our Online Algorithm \BTTC}
\label{sec:bttc-algo}

In Problem \ref{prob:binary-delta-counting}, when multiple $\delta$-temporal triangles share the same set of three vertices, they will be counted only once, thereby making both  \OTTC and \DOTTT inapplicable.
%
Nevertheless, we can adapt \DOTTT to solve Problem \ref{prob:binary-delta-counting} by slightly changing step (3) of \DOTTT; that is, after deriving the value $R$, if $R\textgreater 1$, we lower it to 1.
%
We denote the adapted algorithm by {\tt B}-\DOTTT.

However, {\tt B}-\DOTTT is inefficient due to the complex process of computing $R$.
%
We further develop a novel faster algorithm, denoted as \BTTC, which follows the three steps of \DOTTT, but changes its step (3) to check the existence of a $\delta$-temporal triangle.
%
Specifically, given a static triangle $\Delta_{u,v,w}$ of $G^*_{[t_s,t_e]}$, we enumerate each edge $(u,v,t)$ in $E(u,v)$, and find edges $e_1 \in E(u,w), e_2 \in E(v,w)$ such that both of them appear in a time interval either $[t,t+\delta]$ or $[t-\delta,t]$.
%
If such two edges exist, we identify a $\delta$-temporal triangle with $(u,v,t)$ as the first edge or last edge.
%
The edge $(u,v,t)$ may also serve as the middle edge in the $\delta$-temporal triangle, and to find such triangles, we find $\delta$-temporal triangles with $(u,w,t')$ as the first edge or last edge through a similar process.

Algorithm \ref{alg:bttc} presents \BTTC.
%
We first extract $G^*_{[t_s,t_e]}$ and $P$, and initialize $R$ (lines 1-2).
%
Then, we enumerate static triangles in $G^*_{[t_s,t_e]}$, and check the existence of $\delta$-temporal triangle sharing the vertices in each static triangle (lines 3-18).
%
We initialize the flag of existence and find $u$ (line 5).
%
For each edge $(u,v,t)$, check the existence of $\delta$-temporal triangle with $(u,v,t)$ as the first edge or the last edge (lines 6-11).
%
We also enumerate edges $\in E(u,w)$ to check the existence of $\delta$-temporal triangle with $(u,w,t')$ as the first edge or the last edge (lines 12-17).
%
Finally, we get the result (lines 18-19).

\begin{algorithm}[t]
    \small
    \caption{\BTTC}
    \label{alg:bttc}
    
    \KwIn{A temporal graph $G$, a time window $[t_s,t_e]$, a threshold $\delta$}
    \KwOut{The number of binary $\delta$-temporal triangles in $G_{[t_s,t_e]}$}

    Extract the static graph $G^*_{[t_s,t_e]}$ and derive its degeneracy order $P$\;

    $R \gets 0$\;
    
    Enumerate the static triangles in $G^*_{[t_s,t_e]}$\;
    
    \ForEach{static triangle $\Delta_{u,v,w} \in G^*_{[t_s,t_e]}$}{
        $f \gets 0$, $u \gets$ vertex with the minimum degeneracy order\;
        \ForEach{$(u,v,t) \in E(u,v)$}{
            $S_1 \gets \{(u,w,t')|(u,w,t')\in E(u,w), t' \in [t,t+\delta]\}$\;
            $S_2 \gets \{(v,w,t'')|(v,w,t'')\in E(v,w), t'' \in [t,t+\delta]\}$\;
            $S_3 \gets \{(u,w,t')|(u,w,t')\in E(u,w), t' \in [t-\delta,t]\}$\;
            $S_4 \gets \{(v,w,t'')|(v,w,t'')\in E(v,w), t'' \in [t-\delta,t]\}$\;
            {\bf if }{($|S_1|$ and $|S_2| >0$) or ($|S_3|$ and $|S_4| >0$)} {\bf then}{
                $f \gets 1$\;
            }
        }

        \ForEach{$(u,w,t') \in E(u,w)$}{
            $S_1 \gets \{(u,v,t)|(u,v,t)\in E(u,v), t \in [t',t'+\delta]\}$\;
            $S_2 \gets \{(v,w,t'')|(v,w,t'')\in E(v,w), t'' \in [t',t'+\delta]\}$;
            
            $S_3 \gets \{(u,v,t)|(u,v,t)\in E(u,v), t \in [t'-\delta,t']\}$\;
            $S_4 \gets \{(v,w,t'')|(v,w,t'')\in E(v,w), t'' \in [t'-\delta,t']\}$\;
            {\bf if }{($|S_1|$ and $|S_2| >0$) or ($|S_3|$ and $|S_4| >0$)} {\bf then}{
                $f \gets 1$\;
            }
        }
        $R \gets R+f$\;
    }
    \Return $R$\;
\end{algorithm}

%Example \ref{eg:bttc} further illustrates the \BTTC.

\begin{example}
\label{eg:bttc}
Consider the temporal graph in Figure \ref{fig:temporal_graph}(a) and let $[t_s,t_e] = [0,2] $and $\delta = 2$.
%
We first get the static graph $G^*_{[0,2]}$ with 2 static triangles $\Delta_{v_1,v_2,v_4}$ and $\Delta_{v_2,v_3,v_4}$, shown in Figure \ref{fig:tgraph} and derive its degeneracy order $P: v_1 \prec v_3 \prec v_4 \prec v_2$.
%
Then, for $\Delta_{v_1,v_2,v_4}$, when enumerating edge $(v_1,v_2,1)$, we find $(v_1,v_4,2)$ and $(v_2,v_4,2)$, so there is a binary 2-temporal triangle $\{v_1,v_2,v_4\}$.
%
For $\Delta_{v_2,v_3,v_4}$, we can also find a binary 2-temporal triangle.
%
Hence, the counting result is 2.
\end{example}


\begin{lemma}
Given a temporal graph $G$, a query time window $[t_s,t_e]$, and a duration $\delta$, \BTTC completes in $O(m\kappa \log(m))$ time.
\end{lemma}

\begin{proof}
As analyzed in Section \ref{sec:sota}, the time cost of steps (1) and (2) is $O(m\kappa)$.
%
Step (3) (lines 4-18 in Algorithm \ref{alg:bttc}) costs $O(\sum_{\Delta_{u,v,w}}$ $(|E(u,v)|+|E(u,w)|) \log(m))$  = $O(m\kappa \log(m))$ time.
%
Hence, the lemma holds.
\end{proof}

\subsection{An Index-Based Solution}
\label{sec:binary-index}

%In this section, we present an index-based solution to support efficient frequent binary $\delta$-temporal triangle counting queries.
%
%In the following, we first present the main idea of our index, and then show the detailed index-based solution.

\subsubsection{Main Idea}
\label{sec:bc-points}
Similar to {\tt WT-Index}, we propose to convert the binary $\delta$-temporal triangle counting problem into a point counting problem.
%
However, the idea of converting $\delta$-temporal triangle to C-points cannot be applied to binary $\delta$-temporal triangle as multiple $\delta$-temporal triangles may be regarded as a single binary $\delta$-temporal triangle.
%
%For instance, in Figure \ref{fig:temporal_graph}(a), there are two $\delta$-temporal triangles sharing the vertices $\{v_1,v_2,v_4\}$, corresponding to two C-points $\langle (0,2),1 \rangle$ and $\langle (1,2),1 \rangle$, but there is only one binary $\delta$-temporal triangle.

To resolve the above issue, we proposed to store the binary $\delta$-temporal triangle counting results, rather than the binary $\delta$-temporal triangles.
%
A naive idea is to run \BTTC for every possible time window $[x,y]$ and duration $z$, and then keep a 3-dimensional point $\langle(x,y,z),c_{(x,y,z)}\rangle$ where $c_{(x,y,z)}$ denotes the number of binary $z$-temporal triangles in $[x,y]$.
%
However, it costs $O(t_{max}^3)$ space and $O(m\kappa\log(m)\times t_{max}^3)$ time to build the index, thereby being impractical for large $t_{max}$.
%
To alleviate this issue, we propose to compress these 3-dimensional points into some BC-points, defined as follows, by only storing the differences between counting results.

\begin{definition}[BC-point]
Given a temporal graph $G$, a time window $[x,y]$, and a duration $z$, a BC-point, denoted by $\langle (x,y,z),d\rangle$, is a 3-dimensional point with 
$$
{\begin{split}
        d = c_{(x,y,z)}-c_{(x,y,z-1)}-c_{(x,y-1,z)}-c_{(x+1,y,z)}+ c_{(x+1,y-1,z)}\\+
    c_{(x+1,y,z-1)}+c_{(x,y-1,z-1)}-c_{(x+1,y-1,z-1)}.
\end{split}}
$$
\end{definition}

Note that when $d=0$, we omit the BC-point due to the zero difference.
%
Clearly, given all the BC-points, the number of binary $\delta$-temporal triangles in the time window $[t_s,t_e]$ equals the summarized $d$ values of BC-points in the cube $[t_s,t_e]\times [t_s,t_e]\times [0,\delta]$.
%We further illustrate this by Example \ref{eg:bcp}.

\begin{figure}[ht]
    \small
    \centering
    \subfigure[A temporal graph $G_2$]{\includegraphics[width=.2\linewidth]{figure/B-temporal-graph-example.pdf}}
    \subfigure[3-dimensional points]{\includegraphics[width=.21\linewidth]{figure/binary-c-value.pdf}}
    \subfigure[All the BC-points]{\includegraphics[width=.21\linewidth]{figure/BC-point.pdf}}
    \caption{A temporal graph and its 3-dimensional points before and after compression (we omit points with $c=d=0$).}
    \setlength{\abovecaptionskip}{-0.2cm}
 \setlength{\belowcaptionskip}{-2pt}
    \label{fig:bcp}
\end{figure}

\begin{example}
\label{eg:bcp}
In Figure \ref{fig:bcp}, we present a temporal graph $G_2$, and its 3-dimensional points before and after compression, where points with $c$ and $d$ being 0 are omitted.
%
For instance, there is a BC-point $\langle (0,2,1),-1\rangle$ because $c_{(0,2,1)}-0-c_{(0,1,1)}-c_{(1,2,1)} + 0 + 0 + 0 - 0 = -1$.
\end{example}

To obtain BC-points, a naive method is to enumerate all the possible time windows and durations and compress the 3-dimensional points, requiring $O(m\kappa \log(m)\times t_{max}^3)$ time, which is very costly.
%
To speed up this process, we generate BC-points from each static triangle.
%
Specifically, for each static triangle $\Delta_{u,v,w}$, we first fetch all the timestamps of edges in $E(u,v)$, $E(u,w)$, and $E(w,v)$, denoted by a set $\Phi$.
%
Then, we only need to obtain the $c_{(x,y,z)}$ for $\Delta_{u,v,w}$ with every time window $[x,y]$ and duration $z$ where $x,y,z \in \Phi$.
%
Finally, we get all the BC-points.

As aforementioned, to answer a counting query, we need to summarize the $d$ values of BC-points in the cube $[t_s,t_e]\times [t_s,t_e]\times[0,\delta]$, which actually is a 3-dimensional point counting problem.
%
Many index structures have been developed to solve this problem, such as KD-tree \cite{10.1145/361002.361007}, wavelet tree \cite{grossi2003high}, and segment tree \cite{de2000computational}.
%
Given {\color{black} $O(\pi)$} BC-points, since KD-tree costs {\color{black} $O(\pi)$ space}, while the other two indexes cost {\color{black} $O(\pi \log(m))$ space}, we employ the KD-tree in this paper and denote the KD-tree-based index by {\tt KDT-Index}.

\subsubsection{\tt KDT-Index}
\label{sec:kdt-index}

Given a BC-point list $L$, a {\tt KDT-Index} is a binary tree, where each node stores a BC-point and the information of BC-points in its subtrees, with four components in total:
\begin{enumerate}
    \item \textbf{A BC-point} $\langle (x,y,z),d \rangle$.
    
    \item \textbf{A selected dimension} $D$: It indicates the construct rule of the KD-tree. For any BC-point, if its coordinate in $D$-th dimension is smaller than or equal to the $D$-th dimension of $\langle (x,y,z),d \rangle$, it will be in the left child of the node; otherwise, it will be in the right child of the node.
    
    \item \textbf{A cube} $A$: The smallest cube containing all BC-points in the subtree of the node, denoted by $[x_1,x_2]\times [y_1,y_2]\times [z_1,z_2]$.
    
    \item \textbf{An integer} $S$: It records the summarized $d$ value of all BC-points in the subtree of the node.
\end{enumerate}
%
We further illustrate the {\tt KDT-Index} via Example \ref{eg:kdt}.

\begin{figure}[ht]
    \centering
    \includegraphics[width=.35\linewidth]{figure/kd-tree.pdf}
    \caption{The {\tt KDT-Index} for the BC-points in Figure \ref{fig:bcp}.}
    \label{fig:kd-tree}
    \vspace{-5pt}
\end{figure}

\begin{example}
\label{eg:kdt}
Figure \ref{fig:kd-tree} depicts the {\tt KDT-Index} built for all BC-points in Figure \ref{fig:bcp}.
%
Let us take the root node as an example: we pick a BC-point $\langle (0,2,1),-1\rangle$ and select the $x$ dimension ($D=x$).
%
The smallest cube $A$ to contain all the BC-points is $[0,1]\times[1,2]\times[1,1]$.
%
Since the summarized $d$ value of all BC-points is 2, 
$S=2$.
\end{example}

The result of a counting query equals the summarized $S$ values of nodes in the {\tt KDT-Index} whose cubes are contained by the cube $[t_s,t_e]\times[t_s,t_e]\times[0,\delta]$.
%
We apply a recursive method starting from the root to find the counting result.
%
For lack of space, we show the counting algorithm, {\tt KDT-Index} construction and maintenance algorithms \cite{10.1007/BF00263763,10.1145/361002.361007,de2008orthogonal} in the appendix of our technical report \cite{github}.

{\color{black}
$\bullet$ \textbf{Extension for counting other patterns.}
%(R1.A5)
Our index-based solution above can be easily extended for counting other patterns.
%
Given a target pattern, we first fetch all the static instances from the static graph.
%
Then, we can generate the BC-points from each static instance following the approach described in Section \ref{sec:bc-points}, and apply our {\tt KDT-Index} to manage the BC-points.}

\section{Experiments}
\label{sec:experiment}

We now present the experimental results.
%
We begin with the setup in Section \ref{sec:experiment-setup}, then show the efficiency results in Sections \ref{sec:experiment-weighted} and \ref{sec:experiment-binary}, and finally present case study results in Section \ref{sec:experiment-case-study}.


\subsection{Setup}
\label{sec:experiment-setup}

\begin{table}[h]
    \small
    \caption{Datasets used in our experiments.}
    \label{tab:datasets}
    \centering
    \setlength{\tabcolsep}{0.7mm}{
    \begin{tabular}{c|c|r|r|r|r|r}
    \hline
    Datasets & Abbr. & \multicolumn{1}{c|}{$n$} & \multicolumn{1}{c|}{$m$} & \multicolumn{1}{c|}{$t_{max}$} & \multicolumn{1}{c|}{$\pi$} & \multicolumn{1}{c}{$\kappa$} \\
    \hline\hline
    contact & CT & 274 & 28,244 & 15,661 & 15M & 39 \\
    \hline
    email-eu & EM & 986 & 332,334 & 207,879 & 211M & 34 \\
    \hline
    wiki-talk & WK & 1,140,149 & 7,833,140 & 5,799,205 & 651M & 119 \\
    \hline
    stackoverflow & ST & 2,601,977 & 63,497,050 & 41,484,768 & 2.7B &198 \\
    \hline
    graph500-23 & GR & 4,610,222 & 129,333,677 & 6,807,835 & 
 3.2B&1,222 \\
    \hline
    \end{tabular}
    }
\end{table}

{\bf Datasets.}
%
We use four real-world temporal graphs sourced from SNAP \cite{leskovec2016snap} and KONECT \cite{konect}, along with a synthetic temporal graph generated from LDBC \cite{erling2015ldbc}.
%
Table \ref{tab:datasets} summarizes the statistics of each graph, including the number of vertices ($n$) and edges ($m$), the maximum timestamp ($t_{max}$), the number of C-points ($\pi$), and the degeneracy ($\kappa$).
%
The last graph does not have timestamps, so we randomly generate timestamps for it.


{\bf Algorithms.} We mainly evaluate the following algorithms.

\noindent $\bullet$ {\tt DOTTT}~\cite{pashanasangi2021faster}: SOTA online $\delta$-temporal triangle counting algorithm;
    
\noindent $\bullet$ {\tt OTTC}: our online $\delta$-temporal triangle counting algorithm;
    
%\item {\tt WT-Index}: the {\tt WT-Index} construction algorithm as illustrated in Section \ref{sec:overview-wavelet};

{\color{black}
\noindent $\bullet$ {\tt TSRjoin}~\cite{zhu2021leveraging}: an index-based $\delta$-temporal triangle counting algorithm, originally designed for counting temporal-clique subgraphs;}

\noindent $\bullet$ {\tt WT-Index-query}: the {\tt WT-Index}-based $\delta$-temporal triangle counting algorithm;

\noindent $\bullet$ {\tt B-DOTTT}: adapted \DOTTT for binary $\delta$-temporal triangle counting;

\noindent $\bullet$ {\tt BTTC}: our online binary $\delta$-temporal triangle counting algorithm;

%\noindent $\bullet$ {\tt KDT-Index}: the {\tt KDT-Index} construction algorithm to construct a {\tt KDT-Index}~\cite{10.1145/361002.361007, 10.1007/BF00263763} for all BC-points;

\noindent $\bullet$ {\tt KDT-Index-query}: the {\tt KDT-Index}-based binary $\delta$-triangle counting algorithm.

{\color{black}
We have implemented all the algorithms above and placed the codes in a GitHub repository\footnote{\url{https://github.com/xqbf/counting-triangles}}.
%
Note that {\tt TSRjoin} was designed for counting temporal-clique subgraphs, but it can be adapted for counting $\delta$-temporal triangle as follows:
%
Given a temporal graph $G$ and a duration $\delta$, we first convert each temporal edge $(u,v,t)$ to an edge $(u,v)$ with a time interval $[t,t+\delta]$, then convert the query time interval $[t_s,t_e]$ to $[t_s+\delta, t_e]$, and finally apply the algorithm in \cite{zhu2021leveraging} to get the result.
%
Its time and space costs are analyzed as follows:
%
{\it (1) Index construction}: For a specific $\delta$, the index construction time for the {\tt TSRjoin} is $O(m \log(m) + n^2)$, with a space complexity $O(m)$.
%
To accommodate all possible values of $\delta$, the solution needs to convert one temporal edge into $O(m)$ edges with different time intervals and then build the index with total $O(m^2)$ converted edges.
%
Consequently, the overall time and space complexities become $O(m^2 \log(m) + n^2)$ and $O(m^2)$, respectively.
%
{\it (2) Query processing}: Given a specific $\delta$, {\tt TSRjoin} needs to enumerate the graph and all $\delta$-temporal triangles, costing $O(\max(m, \Delta))$ time.}

To evaluate the efficiency, for each dataset, we consider 5 different interval lengths $|t_e-t_s|=t_{max}\cdot x$ with $x\in\{20\%, 40\%, 60\%, 80\%,$ $100\%\}$, and then for each interval length, we randomly generate 1,000 queries by varying $\delta = |t_e-t_s| \cdot y$ with $y\in \{10\%, 30\%, 50\%, 70\%,$ $90\%\}$, where the default interval length and $\delta$ are set to $100\% \times t_{max}$ and $10\%|t_e-t_s|$ respectively.
%
%Subsequently, we execute these queries sequentially and compute the average query time.
%
All algorithms are implemented in C++ and compiled with the g++ compiler at the -O3 optimization level.
%
The experiments are conducted on a Linux machine equipped with an Intel Xeon 2.90GHz CPU and 512GB RAM.
%
%Memory requirements are measured by the maximum RES memory used during the process.



\pgfplotstableread[row sep=\\,col sep=&]{
	length & WT &OTTC&DOTTT & tsrjoin \\
	0.2 & 0.184 & 24513 & 39485& 29501\\
	0.4 & 0.1664 & 48495 & 94581& 40065\\
	0.6 & 0.1527 & 75381 & 192302& 46778\\
	0.8 & 0.1379 & 102008 & 258482& 52960\\
	1.0 & 0.0036 & 148810 & 466289& 65843\\
}\stackoverflow

\pgfplotstableread[row sep=\\,col sep=&]{
	length & WT &OTTC&DOTTT& tsrjoin \\
	0.2 & 0.197 & 83142 & 268044& 914462\\
	0.4 & 0.1934 & 218599 & 947287& 918252\\
	0.6 & 0.1611 & 423764 & 2179308& 1005424\\
	0.8 & 0.1283 & 620516 & 3825394& 1111297\\
	1.0 & 0.0034 & 878521 & 6784310& 1190203\\
}\graphfive

\pgfplotstableread[row sep=\\,col sep=&]{
	length & WT & OTTC & DOTTT & tsrjoin \\
	0.2 & 0.0302 & 1599 & 3339& 1795\\
	0.4 & 0.0287 & 2796 & 10779& 2583\\
	0.6 & 0.0306 & 4116 & 19781& 2862\\
	0.8 & 0.0259 & 5430 & 30516& 3755\\
	1.0 & 0.0032 & 7118 & 44509& 4842\\
}\wiki

\pgfplotstableread[row sep=\\,col sep=&]{
	length & WT & OTTC & DOTTT & tsrjoin \\
	0.2 & 0.0259 & 15 & 302& 53\\
	0.4 & 0.0265 & 34 & 854 & 123\\
	0.6 & 0.0256 & 57 & 1547 & 178\\
	0.8 & 0.0272 & 81 & 2319 & 253\\
	1.0 & 0.0042 & 106 & 3292 & 336\\
}\email

\pgfplotstableread[row sep=\\,col sep=&]{
	length & WT & OTTC & DOTTT &tsrjoin \\
	0.2 & 0.0087 & 2 & 73& 14\\
	0.4 & 0.0077 & 4 & 157& 26\\
	0.6 & 0.0077 & 6 & 269& 34\\
	0.8 & 0.007 & 8 & 408& 49\\
	1.0 & 0.0024 & 10 & 742& 73\\
}\tsv

\pgfplotstableread[row sep=\\,col sep=&]{
	length & KD &BTTC&DOTTT \\
	0.2 & 356 & 19510 & 39485\\
	0.4 & 770 & 41176 & 94581\\
	0.6 & 1218 & 67995 & 192302\\
	0.8 & 1651 & 103166 & 258482\\
	1.0 & 2196 & 135077 & 466289\\
}\stackoverflowbinary

\pgfplotstableread[row sep=\\,col sep=&]{
	length & KD &BTTC&DOTTT& kd-tree \\
	0.2 & 3961 & 138338 & 268044& 15\\
	0.4 & 10737 & 609170 & 947287& 42\\
	0.6 & 19264 & 1243640 & 2179308& 77\\
	0.8 & 28765 & 2361215 & 3825394& 114\\
	1.0 & 37011 & 4222026 & 6784310& 146\\
}\graphfivebinary

\pgfplotstableread[row sep=\\,col sep=&]{
	length & KD & BTTC & DOTTT & kd-tree \\
	0.2 & 27 & 1506 & 3339& 0.3561\\
	0.4 & 60 & 3490 & 10779& 0.8596\\
	0.6 & 96 & 5718 & 19781& 1.553\\
	0.8 & 134 & 8277 & 30516& 2.5349\\
	1.0 & 171 & 11314 & 44509& 3.4783\\
}\wikibinary

\pgfplotstableread[row sep=\\,col sep=&]{
	length & KD & BTTC & DOTTT & kd-tree \\
	0.2 & 3.5 & 69 & 302& 0.02\\
	0.4 & 6.9 & 175 & 854 & 0.04\\
	0.6 & 9.6 & 296 & 1547 & 0.06\\
	0.8 & 11.3 & 430 & 2319 & 0.07\\
	1.0 & 11.08 & 574 & 3292 & 0.06\\
}\emailbinary

\pgfplotstableread[row sep=\\,col sep=&]{
	length & KD & BTTC & DOTTT &kd-tree \\
	0.2 & 0.8 & 10 & 73& 0.008\\
	0.4 & 1.6 & 24 & 157& 0.02\\
	0.6 & 2.2 & 40 & 269& 0.02\\
	0.8 & 2.5 & 57 & 408& 0.02\\
	1.0 & 2.38 & 80 & 742& 0.03\\
}\tsvbinary




\pgfplotstableread[row sep=\\,col sep=&]{
	length & WT &OTTC& DOTTT& tsrjoin \\
	0.1 & 0.0036 & 148810 & 466289& 65843\\
	0.3 & 0.0033 & 154328 & 489479& 62627\\
	0.5 & 0.0015 & 151089 & 456316& 57711\\
	0.7 & 0.0034 & 124471 & 472154& 54273\\
	0.9 & 0.0036 & 125371 & 477172& 52533\\
}\stackoverflowd

\pgfplotstableread[row sep=\\,col sep=&]{
	length & WT &OTTC & DOTTT& tsrjoin \\
	0.1 & 0.0034 & 878521 & 6784310& 1190203\\
	0.3 & 0.0037 & 891262 & 6554200& 1182613\\
	0.5 & 0.0013 & 954383 & 6363630& 1108821\\
	0.7 & 0.0032 & 1039122 & 6531770&1074491\\
	0.9 & 0.0032 & 932256 & 6333730& 1050953\\
}\graphfived

\pgfplotstableread[row sep=\\,col sep=&]{
	length & WT &DOTTT & OTTC & tsrjoin \\
	0.1 & 0.0032 & 44509 & 7188 & 4842\\
	0.3 & 0.0033 & 43830 & 7412 & 4646\\
	0.5 & 0.0013 & 43723 & 7257 &  4325\\
	0.7 & 0.003 & 43584 & 6641 & 4051\\
	0.9 & 0.003 & 42510 & 6058 & 3887\\
}\wikid

\pgfplotstableread[row sep=\\,col sep=&]{
	length & WT & DOTTT & OTTC & tsrjoin \\
	0.1 & 0.0042 & 3292 & 106 & 336\\
	0.3 & 0.0052 & 3581 & 101 & 363\\
	0.5 & 0.0027 & 3615 & 88 & 339\\
	0.7 & 0.0043 & 3607 & 75 & 319\\
	0.9 & 0.0044 & 3259 & 67 & 292\\
}\emaild

\pgfplotstableread[row sep=\\,col sep=&]{
	length & WT & DOTTT& OTTC& tsrjoin \\
	0.1 & 0.0024 & 742& 10 & 73 \\
	0.3 & 0.0021 & 671& 8.6 & 77\\
	0.5 & 0.0011 & 656& 7.8 & 73\\
	0.7 & 0.0035 & 602& 6.9 & 66\\
	0.9 & 0.0021 & 554& 6.2 & 61\\
}\tsvd

\pgfplotstableread[row sep=\\,col sep=&]{
	length & KD &BTTC& DOTTT& kd-tree \\
	0.1 & 2196 & 135077 & 466289& 26\\
	0.3 & 1659 & 129966 & 489479& 24\\
	0.5 & 1070 & 131001 & 456316& 19\\
	0.7 & 520 & 143781 & 472154& 15\\
	0.9 & 97 & 144079 & 477172& 12\\
}\stackoverflowdbinary

\pgfplotstableread[row sep=\\,col sep=&]{
	length & KD &BTTC & DOTTT& kd-tree \\
	0.1 & 37011 & 4222026 & 6784310& 146\\
	0.3 & 47151 & 4035798 & 6554200& 193\\
	0.5 & 33729 & 4322033 & 6363630& 197\\
	0.7 & 23585 & 4079659 & 6531770&180\\
	0.9 & 5618 & 4091265 & 6333730& 161\\
}\graphfivedbinary

\pgfplotstableread[row sep=\\,col sep=&]{
	length & KD &DOTTT & BTTC & kd-tree \\
	0.1 & 171 & 44509 & 11314 & 1.2\\
	0.3 & 142 & 43830 & 11479 & 6\\
	0.5 & 95 & 43723 & 11713 &  14.4\\
	0.7 & 47 & 43584 & 11376 & 28.8\\
	0.9 & 8 & 42510 & 11391 & 44.4\\
}\wikidbinary

\pgfplotstableread[row sep=\\,col sep=&]{
	length & KD & DOTTT & BTTC & kd-tree \\
	0.1 & 11.08 & 3292 & 238 & 31\\
	0.3 & 5.0016 & 3581 & 211 & 20\\
	0.5 & 1.6591 & 3615 & 203 & 16\\
	0.7 & 0.577 & 3607 & 174 & 14.6\\
	0.9 & 0.0461 & 3259 & 158 & 13\\
}\emaildbinary

\pgfplotstableread[row sep=\\,col sep=&]{
	length & KD & DOTTT& BTTC& kd-tree \\
	0.1 & 2.38 & 742& 37 & 0.03 \\
	0.3 & 1.5 & 671& 34 & 0.018\\
	0.5 & 1.2 & 656& 33 & 0.016\\
	0.7 & 1.2 & 602& 29 & 0.014\\
	0.9 & 1.1 & 554& 24 & 0.014\\
}\tsvdbinary


\pgfplotstableread[row sep=\\,col sep=&]{
	length & LSC &wavelet tree \\
	0.01 & 577425 & 609146 \\
	0.05 & 798674 & 808090 \\
	0.1 & 918819 & 1154172 \\
	0.2 & 1288712 & 1476684 \\
	0.3 & 1973168 & 1983985 \\
}\stackoverflowf

\pgfplotstableread[row sep=\\,col sep=&]{
	length & LSC &wavelet tree \\
	0.01 & 2038564 & 2172252 \\
	0.05 & 2189398 &  2123597\\
	0.1 & 2099664 &  2288220 \\
	0.2 & 2208948 &  2259616\\
	0.3 & 2288616 & 2471462 \\
}\graphfivef

\pgfplotstableread[row sep=\\,col sep=&]{
	length & LSC &wavelet tree \\
	0.01 & 67333 & 74486 \\
	0.05 & 104348 & 129932 \\
	0.1 & 134255 & 163451 \\
	0.2 & 237004 & 261851 \\
	0.3 & 288011 & 353856 \\
}\wikif

\pgfplotstableread[row sep=\\,col sep=&]{
	length & LSC &wavelet tree \\
	0.01 & 8393 & 9800 \\
	0.05 & 18320 & 22861 \\
	0.1 & 30235 & 39094 \\
	0.2 & 50959 & 68203 \\
	0.3 & 66685 & 94540 \\
}\emailf

\pgfplotstableread[row sep=\\,col sep=&]{
	length & LSC &wavelet tree \\
	0.01 & 637 & 702 \\
	0.05 & 1109 & 1316 \\
	0.1 & 1687 & 2014 \\
	0.2 & 2775 & 3319 \\
	0.3 &  3676 & 4454 \\
}\tsvf



\pgfplotstableread[row sep=\\,col sep=&]{
	length & LSC &wavelet tree \\
	0.01 & 0.053 & 0.0134 \\
	0.05 & 0.0647 & 0.0154 \\
	0.1 & 0.0748 & 0.016 \\
	0.2 & 0.0856 & 0.0168 \\
	0.3 & 0.1066 & 0.0245 \\
}\stackoverflowfq

\pgfplotstableread[row sep=\\,col sep=&]{
	length & LSC &wavelet tree \\
	0.01 & 0.021 & 0.0113 \\
	0.05 & 0.0361 &  0.0118\\
	0.1 & 0.0314 &  0.0117 \\
	0.2 & 0.0385 & 0.0121 \\
	0.3 & 0.0404 & 0.0132 \\
}\graphfivefq

\pgfplotstableread[row sep=\\,col sep=&]{
	length & LSC &wavelet tree& kd-tree \\
	0.01 & 0.0224 & 0.0091& 1 \\
	0.05 & 0.0331 & 0.0124& 5\\
	0.1 & 0.0422 & 0.0125&  12 \\
	0.2 & 0.0485 & 0.0136& 24 \\
	0.3 & 0.051 & 0.0141&37 \\
}\wikifq

\pgfplotstableread[row sep=\\,col sep=&]{
	length & LSC &wavelet tree&kd-tree \\
	0.01 & 0.0109 & 0.006& 0.07 \\
	0.05 & 0.0157 & 0.0073& 0.4 \\
	0.1 & 0.018 & 0.0077& 0.7 \\
	0.2 & 0.021 & 0.0084&1.4 \\
	0.3 & 0.0218 & 0.0092& 2 \\
}\emailfq

\pgfplotstableread[row sep=\\,col sep=&]{
	length & LSC &wavelet tree \\
	0.01 & 0.0042 & 0.0038 \\
	0.05 & 0.0055 & 0.0045 \\
	0.1 & 0.0059 & 0.0046 \\
	0.2 & 0.0064 & 0.0048 \\
	0.3 &  0.0069 & 0.005 \\
}\tsvfq





\pgfplotstableread[row sep=\\, col sep = &]{
    length & LSC & TTC & kd-tree\\
    0.2 & 0.0398 & 14870& 1.2947\\
    0.4 &0.0395 & 33858& 2.8918\\
    0.6 &0.041 & 59278& 4.5552\\
    0.8 &0.0426 & 81808& 7.0367\\
    1 & 0.0181& 105739&10.5806\\
}\stackoverflowdirectcircle


\pgfplotstableread[row sep=\\, col sep = &]{
    length & LSC & TTC & kd-tree\\
    0.2 & 0.0903 & 966& 0.3561\\
    0.4 &0.0974 & 2089& 0.8596\\
    0.6 &0.0998 & 3131& 1.553\\
    0.8 &0.1085 & 4568& 2.5349\\
    1 & 0.0201& 6551& 3.4783\\
}\wikidirectcircle

\pgfplotstableread[row sep=\\, col sep = &]{
datasets & WT & OTTC & tsrjoin & DOTTT\\
1 & 0.0034 & 878521& 1190203 & 6784310\\
2 & 0.0036 & 148810 & 65843 &466289\\
3 & 0.0032 & 7188 & 4842 & 44509 \\
4 & 0.0042 & 106 & 336 & 3292\\
5 & 0.0018 & 10 & 73 & 742\\
}\baseline


\pgfplotstableread[row sep=\\, col sep = &]{
datasets & KD & BTTC  & DOTTT\\
1 & 37011 & 4222026 & 6784310\\
2 & 2196 & 135077  &466289\\
3 & 171 & 11915  & 44509 \\
4 & 11 & 238  & 3292\\
5 & 5 & 33  & 742\\
}\binarybaseline


\pgfplotstableread[row sep=\\,col sep=&]{
	datasets &WT & TSR \\
	5 &  9537886 & 7550215\\
	4 &  5986268 & 367765 \\
	3 &  1232820 & 36895 \\
	2 & 283948 & 106 \\
	1 & 12896 & 10\\
}\constructiontime

\pgfplotstableread[row sep=\\,col sep=&]{
	datasets & WT & TSR & origin \\
	5 & 387379& 72704 & 3002\\
	4 & 393830& 8397 & 1119\\
	3 & 82227& 1536 & 116\\
	2 & 21709& 31 & 5\\
	1 & 1946& 6 & 1 \\
}\constructionspace

\pgfplotstableread[row sep=\\,col sep=&]{
	length & CT & EM & WK & ST & GR \\
	1 & 704 & 5228 & 128050 & 926023 & 423414 \\
	2 & 2446 & 31705 & 305446 & 2100846 & 1692602\\
	3 & 5073 & 80703 & 527970 & 3330921 & 3160997\\
	4 & 9383 & 169046 & 764345 & 4910566 & 6949984\\
	5 & 12896 &283948 & 1232820 & 5986268 & 9537886\\
}\sctime

\pgfplotstableread[row sep=\\,col sep=&]{
	length & CT & EM & WK & ST & GR \\
	1 & 71 & 587 & 11059 & 68813 & 17306\\
	2 & 268 & 2662 & 24371 & 151347 & 63078\\
	3 & 550 & 6554 & 42701 & 234598 & 129946\\
	4 & 911 & 12390 & 60314 &315392 & 244326\\
	5 & 1946 &21709 & 82227& 393830 & 387379\\
}\scspace



\begin{figure*}
        \centering
	\ref{named1}\\
        \vspace{-20pt}
	\subfigure[CT]{
	   \begin{tikzpicture}[scale=0.49]
	   \begin{axis}[
                    width=.39\textwidth,
                    height=0.3\textwidth,
				xticklabels={,,$20\%$,$40\%$,$60\%$,$80\%$,100\%},
				xmin=0.1,xmax=1.1,
				ymin=0,ymax=100000,
				ymode = log,
				mark size=5pt,
                    line width=2.5pt,
				ylabel={\LARGE \bf Response time (ms)},
                    ylabel style={xshift=12pt,yshift=-4pt},
				ticklabel style={font=\Large},
				every axis plot/.append style={ultra thick},
				every axis/.append style={ultra thick},
			]
			\addplot [mark=x,color=c4,line width=2.5pt] table[x=length,y=WT]{\tsv};
			\addplot [mark=o,color=c6,line width=2.5pt] table[x=length,y=OTTC]{\tsv};
            \addplot [mark=star,color=c7,line width=2.5pt] table[x=length,y=DOTTT]{\tsv};
            \addplot [mark=o,color=c8,line width=2.5pt] table[x=length,y=tsrjoin]{\tsv};
		\end{axis}
	\end{tikzpicture}
	}
        \subfigure[EM]{
	   \begin{tikzpicture}[scale=0.49]
	   \begin{axis}[
                    width=.39\textwidth,
                    height=0.3\textwidth,
				xticklabels={,,$20\%$,$40\%$,$60\%$,$80\%$,$100\%$},
				xmin=0.1,xmax=1.1,
				ymin=0,ymax=100000,
				ymode = log,
				mark size=5pt,
                    line width=2.5pt,
				ticklabel style={font=\Large},
				every axis plot/.append style={ultra thick},
				every axis/.append style={ultra thick},
				]
                    \addplot [mark=x,color=c4,line width=2.5pt] table[x=length,y=WT]{\email};	
                    \addplot [mark=o,color=c6,line width=2.5pt] table[x=length,y=OTTC]{\email};
                    \addplot [mark=star,color=c7,line width=2.5pt] table[x=length,y=DOTTT]{\email};
                    \addplot [mark=o,color=c8,line width=2.5pt] table[x=length,y=tsrjoin]{\email};
			\end{axis}
	\end{tikzpicture}
	}
        \subfigure[WK]{
	   \begin{tikzpicture}[scale=0.49]
	   \begin{axis}[
                    width=.39\textwidth,
                    height=0.3\textwidth,
				xticklabels={,,$20\%$,$40\%$,$60\%$,$80\%$,$100\%$},
				xmin=0.1,xmax=1.1,
				ymin=0,ymax=100000,
				ymode = log,
				mark size=5pt,
                    line width=2.5pt,
				ticklabel style={font=\Large},
				every axis plot/.append style={ultra thick},
				every axis/.append style={ultra thick},
				]
				\addplot [mark=x,color=c4,line width=2.5pt] table[x=length,y=WT]{\wiki};
				\addplot [mark=o,color=c6,line width=2.5pt] table[x=length,y=OTTC]{\wiki};
                \addplot [mark=star,color=c7,line width=2.5pt] table[x=length,y=DOTTT]{\wiki};
                \addplot [mark=o,color=c8,line width=2.5pt] table[x=length,y=tsrjoin]{\wiki};
			\end{axis}
	\end{tikzpicture}
	}
        \subfigure[ST]{
	   \begin{tikzpicture}[scale=0.49]
	   \begin{axis}[
                    width=.39\textwidth,
                    height=0.3\textwidth,
				xticklabels={,,$20\%$,$40\%$,$60\%$,$80\%$,$100\%$},
				xmin=0.1,xmax=1.1,
				ymin=0,ymax=10000000,
				ymode = log,
				mark size=5pt,
                    line width=2.5pt,
				ticklabel style={font=\Large},
				every axis plot/.append style={ultra thick},
				every axis/.append style={ultra thick},
				]
                    \addplot [mark=x,color=c4,line width=2.5pt] table[x=length,y=WT]{\stackoverflow};
				\addplot [mark=o,color=c6,line width=2.5pt] table[x=length,y=OTTC]{\stackoverflow};
                \addplot [mark=star,color=c7,line width=2.5pt] table[x=length,y=DOTTT]{\stackoverflow};
                \addplot [mark=o,color=c8,line width=2.5pt] table[x=length,y=tsrjoin]{\stackoverflow};
		\end{axis}
	\end{tikzpicture}
        }
        \subfigure[GR]{
	   \begin{tikzpicture}[scale=0.49]
	   \begin{axis}[
			legend style = {
                    legend columns=-1,
                    draw=none,
			},
                legend to name=named1,
                legend image post style={scale=0.7, ultra thick},
                width=.39\textwidth,
                height=0.3\textwidth,
			xticklabels={,,$20\%$,$40\%$,$60\%$,$80\%$,$100\%$},
			xmin=0.1,xmax=1.1,
			% ymin=0,ymax=10000000,
			ymode = log,
			mark size=5pt,
                line width=2.5pt,
			ticklabel style={font=\Large},
			every axis plot/.append style={ultra thick},
			every axis/.append style={ultra thick},
			]
			\addplot [mark=x,color=c4,line width=2.5pt] table[x=length,y=WT]{\graphfive};
			\addplot [mark=o,color=c6,line width=2.5pt] table[x=length,y=OTTC]{\graphfive};
            \addplot [mark=star,color=c7,line width=2.5pt] table[x=length,y=DOTTT]{\graphfive};
            \addplot [mark=o,color=c8,line width=2.5pt] table[x=length,y=tsrjoin]{\graphfive};
			\legend{{\footnotesize {\tt WT-Index-query}}, {\footnotesize  \OTTC}, {\footnotesize \DOTTT}, {\footnotesize {\tt TSRjoin}}}
			\end{axis}
	\end{tikzpicture}
    }
    \setlength{\abovecaptionskip}{0.05cm}
    \caption{Effect of $(t_e-t_s)$ (which varies from
    20\% to 40\%, 60\%, 80\%, and 100\% of $t_{max}$).}
    \label{fig:length}
\end{figure*}

\begin{figure*}
        \centering
	\ref{edelta}\\
        \vspace{-20pt}
        \subfigure[CT]{
	\begin{tikzpicture}[scale = 0.49]
	   \begin{axis}[
                    width=0.39\textwidth,
                    height=0.3\textwidth,
				xtick={0,0.1,0.3,0.5,0.7,0.9},
				xticklabels={,$10\%$,$30\%$,$50\%$,$70\%$,$90\%$},
				xmin=0,xmax=1,
				ymin=0,ymax=10000,
				ymode = log,
				mark size=5pt,
                    line width=2.5pt,
				% xlabel={\huge \bf $\delta$ length ratio},
                    ylabel={\LARGE \bf Response time (ms)},
				ylabel style={xshift=12pt,yshift=-3pt},
				ticklabel style={font=\Large},
				every axis plot/.append style={ultra thick},
				every axis/.append style={ultra thick},
				]
				\addplot [mark=x,color=c4,line width=2.5pt] table[x=length,y=WT]{\tsvd};
				\addplot [mark=o,color=c6,line width=2.5pt] table[x=length,y=OTTC]{\tsvd};		
                    % \addplot [mark=star,color=c8,line width=2.5pt] table[x=length,y=kd-tree]{\tsvd};
                    \addplot [mark=star,color=c7,line width=2.5pt] table[x=length,y=DOTTT]{\tsvd};
                    \addplot [mark=o,color=c8,line width=2.5pt] table[x=length,y=tsrjoin]{\tsvd};
			\end{axis}
	\end{tikzpicture}
	}
        \subfigure[EM]{
	\begin{tikzpicture}[scale = 0.49]
	   \begin{axis}[
                    width=.39\textwidth,
                    height=0.3\textwidth,
				xtick={0,0.1,0.3,0.5,0.7,0.9},
				xticklabels={,$10\%$,$30\%$,$50\%$,$70\%$,$90\%$},
				xmin=0,xmax=1,
				ymin=0,ymax=100000,
				ymode = log,
				mark size=5pt,
                    line width=2.5pt,
				% ylabel={\huge \bf Running time (ms)},
				% ylabel style={yshift=-5pt},
				% xlabel={\huge \bf $\delta$ length ratio},
				ticklabel style={font=\Large},
				every axis plot/.append style={ultra thick},
				every axis/.append style={ultra thick},
				]
				\addplot [mark=x,color=c4,line width=2.5pt] table[x=length,y=WT]{\emaild};
				\addplot [mark=o,color=c6,line width=2.5pt] table[x=length,y=OTTC]{\emaild};		
                    % \addplot [mark=star,color=c8,line width=2.5pt] table[x=length,y=kd-tree]{\emaild};
                    \addplot [mark=star,color=c7,line width=2.5pt] table[x=length,y=DOTTT]{\emaild};
                    \addplot [mark=o,color=c8,line width=2.5pt] table[x=length,y=tsrjoin]{\emaild};
			\end{axis}
	\end{tikzpicture}
	}
        \subfigure[WK]{
	\begin{tikzpicture}[scale = 0.49]
	   \begin{axis}[
                    width=.39\textwidth,
                    height=0.3\textwidth,
				xtick={0,0.1,0.3,0.5,0.7,0.9},
				xticklabels={,$10\%$,$30\%$,$50\%$,$70\%$,$90\%$},
				xmin=0,xmax=1,
				ymin=0,ymax=200000,
				ymode = log,
				mark size=5pt,
                    line width=2.5pt,
				% ylabel={\huge \bf Running time (ms)},
				% ylabel style={yshift=-5pt},
				% xlabel={\huge \bf $\delta$ length ratio},
				ticklabel style={font=\Large},
				every axis plot/.append style={ultra thick},
				every axis/.append style={ultra thick},
				]
				\addplot [mark=x,color=c4,line width=2.5pt] table[x=length,y=WT]{\wikid};
				\addplot [mark=o,color=c6,line width=2.5pt] table[x=length,y=OTTC]{\wikid};		
                    \addplot [mark=star,color=c7,line width=2.5pt] table[x=length,y=DOTTT]{\wikid};
                    \addplot [mark=o,color=c8,line width=2.5pt] table[x=length,y=tsrjoin]{\wikid};
			\end{axis}
	\end{tikzpicture}
	}
        \subfigure[ST]{
	\begin{tikzpicture}[scale = 0.49]
	   \begin{axis}[
                    width=.39\textwidth,
                    height=0.3\textwidth,
				xtick={0,0.1,0.3,0.5,0.7,0.9},
				xticklabels={,$10\%$,$30\%$,$50\%$,$70\%$,$90\%$},
				xmin=0,xmax=1,
				ymin=0,ymax=10000000,
				ymode = log,
				mark size=5pt,
                    line width=2.5pt,
				% ylabel={\huge \bf Running time (ms)},
				% ylabel style={yshift=-5pt},
				% xlabel={\huge \bf $\delta$ length ratio},
				ticklabel style={font=\Large},
				every axis plot/.append style={ultra thick},
				every axis/.append style={ultra thick},
				]
				\addplot [mark=x,color=c4,line width=2.5pt] table[x=length,y=WT]{\stackoverflowd};
				\addplot [mark=o,color=c6,line width=2.5pt] table[x=length,y=OTTC]{\stackoverflowd};		
                    % \addplot [mark=star,color=c8] table[x=length,y=kd-tree]{\stackoverflowd};
                    \addplot [mark=star,color=c7,line width=2.5pt] table[x=length,y=DOTTT]{\stackoverflowd};
                    \addplot [mark=o,color=c8,line width=2.5pt] table[x=length,y=tsrjoin]{\stackoverflowd};
			\end{axis}
	\end{tikzpicture}
	}
        \subfigure[GR]{
	\begin{tikzpicture}[scale = 0.49]
	   \begin{axis}[
			    legend style = {
			          legend columns=-1,
                        draw=none,
				},
				legend to name=edelta,
                    legend image post style={scale=0.8, ultra thick},
                    width=.39\textwidth,
                    height=0.3\textwidth,
                    xtick={0,0.1,0.3,0.5,0.7,0.9},
				xticklabels={,$10\%$,$30\%$,$50\%$,$70\%$,$90\%$},
				xmin=0,xmax=1,
				ymin=0,ymax=100000000,
				ymode = log,
				mark size=5pt,
                    line width=2.5pt,
				ticklabel style={font=\Large},
				every axis plot/.append style={ultra thick},
				every axis/.append style={ultra thick},
				]
				\addplot [mark=x,color=c4,line width=2.5pt] table[x=length,y=WT]{\graphfived};
				\addplot [mark=o,color=c6,line width=2.5pt]     table[x=length,y=OTTC]{\graphfived};	
                    \addplot [mark=star,color=c7,line width=2.5pt] table[x=length,y=DOTTT]{\graphfived};
                    \addplot [mark=o,color=c8,line width=2.5pt] table[x=length,y=tsrjoin]{\graphfived};
				\legend{{\footnotesize {\tt WT-Index-query} },{\footnotesize  \OTTC}, {\footnotesize \DOTTT}, {\footnotesize {\tt TSRjoin}}}
			\end{axis}
	   \end{tikzpicture}
	}
   \setlength{\abovecaptionskip}{0.05cm}
    \caption{\color{black}Effect of $\delta$ (which varies from 10\% to 30\%, 50\%, 70\%, and 90\% of $t_{max}$).}
\label{fig:query-delta}
 \end{figure*}


 \begin{figure*}	
    \begin{minipage}[t]{0.47\textwidth}
    \vspace{0pt}
        \begin{tikzpicture}
            \begin{axis}[
        	ybar,
        	bar width=0.16cm,
        	width=\textwidth,
                height=0.5\textwidth,
        	xtick={1,2,3,4,5},	
                xticklabels={GR,ST,WK,EM,CT},
        	legend style = {
                    legend columns=-1,
                    font=\footnotesize,
                    draw=none,
                    at={(1.1,1.35)/}
                },
                ylabel style={yshift=-5pt},
        	legend entries={{\tt DOTTT}, {\tt OTTC},{\tt TSRjoin},{\tt WT-Index-query}},
                legend image post style={scale=0.5, ultra thick},
    		xmin=0.5,xmax = 5.5,
    		ymin=0,ymax=100000000,
    		ymode =log,
                log origin=infty,
    		ylabel={\scriptsize \bf Time (ms)},
    		ticklabel style={font=\scriptsize},
			every axis plot/.append style={ultra thick},
			every axis/.append style={ultra thick},
                x dir=reverse,
    	]
    	\addplot[pattern=north west lines, pattern color=c1] table[x=datasets,y=DOTTT]{\baseline};
    	\addplot[pattern = grid, pattern color=c2] table[x=datasets,y=OTTC]{\baseline};
    	\addplot[pattern = crosshatch dots,pattern color=green] table[x=datasets,y=tsrjoin]{\baseline};
            \addplot[pattern = crosshatch,pattern color=c3] table[x=datasets,y=WT]{\baseline};
        \end{axis}
        \end{tikzpicture}
        \caption{Efficiency of $\delta$-temporal triangle counting.}
	\label{fig:overall}
    \end{minipage}
    \begin{minipage}[t]{0.5\textwidth}
    \vspace{0pt}
        \centering
	\ref{tsindex}\\
        \vspace{-6pt}
        \subfigure[Time cost]{
    	\begin{tikzpicture}[scale=0.5]
        		\begin{axis}[
        			ybar,
        			bar width=0.3cm,
        			width=.935\textwidth,
                        height=0.65\textwidth,
        			xtick={1,2,3,4,5},	
                        xticklabels={CT,EM,WK,ST,GR},
        			xmin=0.5,xmax = 5.5,
        			ymax=2000000000,
        			ymode =log,
        			ylabel style={yshift=-4pt},
        			ylabel={\LARGE \bf Time (ms)},
        			ticklabel style={font=\LARGE},
    				every axis plot/.append style={line width=2pt},
			          every axis/.append style={line width=2pt},
        		]
          \addplot[pattern = crosshatch dots,pattern color=green] table[x=datasets,y=TSR]{\constructiontime};
                    \addplot[pattern = crosshatch dots,pattern color=blue] table[x=datasets,y=WT]{\constructiontime};
                    
                \end{axis}
            \end{tikzpicture}
        }
        \subfigure[{\color{black}Space cost}]{
            \begin{tikzpicture}[scale=0.5]
        		\begin{axis}[
        			ybar,
        			bar width=0.2cm,
        			width=.935\textwidth,
                        height=0.65\textwidth,
        			xtick={1,2,3,4,5},	
                        xticklabels={CT,EM,WK,ST,GR},
        			legend style = {
                            legend columns=-1,
                		font=\footnotesize,
                            draw=none,
                        },
        			legend entries={{\tt Graph size},{\tt TSRjoin}, {\tt WT-Index}},
                        legend to name = tsindex,
                        legend image post style={scale=0.6},
        			xmin=0.5,xmax = 5.5,
        			ymax=100000000,
        			ymode =log,
                        log origin=infty,
        			ylabel style={yshift=-4pt},
        			ylabel={\LARGE \bf Space (mb)},
        			ticklabel style={font=\LARGE},
    				every axis plot/.append style={line width=2pt},
    				every axis/.append style={line width=2pt},
        		]
                \addplot[pattern=north west lines, pattern color=c2] table[x=datasets,y=origin]{\constructionspace};
                \addplot[pattern = crosshatch dots,pattern color=green] table[x=datasets,y=TSR]{\constructionspace};
                    \addplot[pattern = crosshatch dots,pattern color=blue] table[x=datasets,y=WT]{\constructionspace};
                    
            \end{axis}
            \end{tikzpicture}
        }
        \setlength{\abovecaptionskip}{0.07cm}
        \caption{Time and space cost of {\tt WT-Index} construction.}
	\label{fig:cons}
    \end{minipage}
\end{figure*}


% \begin{figure}
% 	\centering
% 	\begin{tikzpicture}[scale=0.65]
%     		\begin{axis}[
%     			ybar,
%     			bar width=0.35cm,
%     			width=.7\textwidth,
%                     height=0.26\textwidth,
%     			xtick={1,2,3,4,5},	
%                     xticklabels={GR,ST,WK,EM,CT},
%     			legend style = {
%                         legend columns=-1,
%             		font=\large,
%                         draw=none,
%                         at={(0.81,1)/}
%                     },
%     			legend entries={{\tt DOTTT}, {\tt OTTC},{\tt TSRjoin},{\tt WT-Index-query}},
%     			xmin=0.5,xmax = 5.5,
%     			ymin=0,ymax=100000000,
%     			ymode =log,
%                     log origin=infty,
%     			% ylabel style={yshift=-4pt},
%     			ylabel={\LARGE \bf Time (ms)},
%     			ticklabel style={font=\LARGE},
% 				every axis plot/.append style={ultra thick},
% 				every axis/.append style={ultra thick},
%                     x dir=reverse,
%     		]
%     		\addplot[pattern=north west lines, pattern color=c1] table[x=datasets,y=DOTTT]{\baseline};
%     		\addplot[pattern = grid, pattern color=c2] table[x=datasets,y=OTTC]{\baseline};
%     		\addplot[pattern = crosshatch dots,pattern color=green] table[x=datasets,y=tsrjoin]{\baseline};
%                 \addplot[pattern = crosshatch,pattern color=c3] table[x=datasets,y=WT]{\baseline};
%         \end{axis}
%         \end{tikzpicture}
%         \caption{{\color{black}Efficiency of $\delta$-temporal triangle counting.}}
% 	\label{fig:overall}
% \end{figure}

% \begin{figure}[h]	
% 	\centering
% 	\ref{named2}\\
%         \vspace{-5pt}
%         \subfigure[{\color{black}Time cost}]{
%     	\begin{tikzpicture}[scale=0.5]
%         		\begin{axis}[
%         			ybar,
%         			bar width=0.4cm,
%         			width=0.45\textwidth,
%                         height=0.3\textwidth,
%         			xtick={1,2,3,4,5},	
%                         xticklabels={CT,EM,WK,ST,GR},
%         			% legend style = {
%            %                  legend columns=-1,
%            %      		font=\huge,
%            %                  draw=none,
%            %                  at={(1,1)/}
%            %              },
%         			% legend entries={{\tt TSRjoin},{\tt WT-Index}},
%         			xmin=0.5,xmax = 5.5,
%         			ymax=2000000000,
%         			ymode =log,
%         			ylabel style={yshift=-4pt},
%         			ylabel={\huge \bf Time (ms)},
%         			ticklabel style={font=\huge},
%     				every axis plot/.append style={line width=2pt},
%     				every axis/.append style={line width=2pt},
%         		]
%           \addplot[pattern = crosshatch dots,pattern color=green] table[x=datasets,y=TSR]{\constructiontime};
%                     \addplot[pattern = crosshatch dots,pattern color=blue] table[x=datasets,y=WT]{\constructiontime};
                    
%                 \end{axis}
%             \end{tikzpicture}
%         }
%         \subfigure[{\color{black}Space cost}]{
%             \begin{tikzpicture}[scale=0.5]
%         		\begin{axis}[
%         			ybar,
%         			bar width=0.25cm,
%         			width=0.45\textwidth,
%                         height=0.3\textwidth,
%         			xtick={1,2,3,4,5},	
%                         xticklabels={CT,EM,WK,ST,GR},
%         			legend style = {
%                             legend columns=-1,
%                 		font=\footnotesize,
%                             draw=none,
%                             %at={(1,1)/}
%                         },
%         			legend entries={{\tt graph size},{\tt TSRjoin}, {\tt WT-Index}},
%                         legend to name = named2,
%                         legend image post style={scale=0.8},
%         			xmin=0.5,xmax = 5.5,
%         			ymax=100000000,
%         			ymode =log,
%                         log origin=infty,
%         			ylabel style={yshift=-4pt},
%         			ylabel={\huge \bf Space (mb)},
%         			ticklabel style={font=\huge},
%     				every axis plot/.append style={line width=2pt},
%     				every axis/.append style={line width=2pt},
%         		]
%                 \addplot[pattern=north west lines, pattern color=c2] table[x=datasets,y=origin]{\constructionspace};
%                 \addplot[pattern = crosshatch dots,pattern color=green] table[x=datasets,y=TSR]{\constructionspace};
%                     \addplot[pattern = crosshatch dots,pattern color=blue] table[x=datasets,y=WT]{\constructionspace};
                    
%             \end{axis}
%             \end{tikzpicture}
%         }
%         \caption{Time and space cost of {\tt WT-Index} construction.}
% 	\label{fig:cons}
% \end{figure}

 \begin{figure*}	
    \begin{minipage}[t]{0.48\textwidth}
    \vspace{0pt}
    \centering
        \ref{sctest}\\
        \vspace{-10pt}
        \subfigure[Time cost]{
    	\begin{tikzpicture}[scale=0.54]
        		\begin{axis}[
        			width=.9\textwidth,
                        height=0.65\textwidth,
                        xtick={1,2,3,4,5},
                        xticklabels={20\%,40\%,60\%,80\%,100\%},
        			legend style = {
                            legend columns=-1,
                            draw=none,
                        },
                        mark size=5pt,
                        line width=2.5pt,
                        legend to name=sctest,
                        legend image post style={scale=0.8, ultra thick},
        			xmin=0.5,xmax = 5.5,
        			ymin=0,ymax=100000000,
        			ymode =log,
        			ylabel={\large \bf Time (ms)},
                        ylabel style={yshift=-4pt},
        			ticklabel style={font=\large},
    				every axis plot/.append style={ultra thick},
    				every axis/.append style={ultra thick},
        		]
                    \addplot [mark=x,color=c3,line width=2.5pt] table[x=length,y=CT]{\sctime};
				\addplot [mark=o,color=c4,line width=2.5pt] table[x=length,y=EM]{\sctime};
                    \addplot [mark=square,color=c5,line width=2.5pt] table[x=length,y=WK]{\sctime};
                    \addplot [mark=triangle,color=c6,line width=2.5pt] table[x=length,y=ST]{\sctime};
                    \addplot [mark=diamond,color=c7,line width=2.5pt] table[x=length,y=GR]{\sctime};
				\legend{{\footnotesize \tt CT},{\footnotesize \tt EM}, {\footnotesize \tt WK}, {\footnotesize \tt ST}, {\footnotesize \tt GR}}
                \end{axis}
            \end{tikzpicture}
        }
        \subfigure[Space cost]{
            \begin{tikzpicture}[scale=0.54]
        		\begin{axis}[
        			width=.9\textwidth,
                        height=0.65\textwidth,
    				every axis plot/.append style={ultra thick},
                        every axis/.append style={ultra thick},	
                        mark size=5pt,
                        line width=2.5pt,
                        xtick={1,2,3,4,5},
                        xticklabels={20\%,40\%,60\%,80\%,100\%},
        			xmin=0.5,xmax = 5.5,
        			ymin=0,ymax=1000000,
        			ymode =log,
        			ylabel style={yshift=-4pt},
        			ylabel={\large \bf Space (mb)},
        			ticklabel style={font=\large},
        		]
                    \addplot [mark=x,color=c3,line width=2.5pt] table[x=length,y=CT]{\scspace};
				\addplot [mark=o,color=c4,line width=2.5pt] table[x=length,y=EM]{\scspace};
                    \addplot [mark=square,color=c5,line width=2.5pt] table[x=length,y=WK]{\scspace};
                    \addplot [mark=triangle,color=c6,line width=2.5pt] table[x=length,y=ST]{\scspace};
                    \addplot [mark=diamond,color=c7,line width=2.5pt] table[x=length,y=GR]{\scspace};
            \end{axis}
            \end{tikzpicture}
        }
        \caption{Scalability test of indexing time and space.}
	\label{fig:scalable}
    \end{minipage}
    \begin{minipage}[t]{0.48\textwidth}
    \vspace{0pt}
    \centering
	\begin{tikzpicture}[scale=0.49]
    		\begin{axis}[
    			ybar,
    			bar width=0.35cm,
    			width=2\textwidth,
                    height=0.8\textwidth,
    			xtick={1,2,3,4,5},	
                    xticklabels={GR,ST,WK,EM,CT},
    			legend style = {
                        legend columns=-1,
            		font=\LARGE,
                        draw=none,
                        at={(0.9,1.2)/}
                    },
    			legend entries={{\tt B-DOTTT}, {\tt BTTC},{\tt KDT-Index-query}},
    			xmin=0.5,xmax = 5.5,
    			ymin=0,ymax=20000000,
    			ymode =log,
                    log origin=infty,
    			ylabel={\LARGE \bf Time (ms)},
    			ticklabel style={font=\LARGE},
				every axis plot/.append style={ultra thick},
				every axis/.append style={ultra thick},
                    x dir=reverse,
    		]
    		\addplot[pattern=north west lines, pattern color=c1] table[x=datasets,y=DOTTT]{\binarybaseline};
    		\addplot[pattern = grid, pattern color=c2] table[x=datasets,y=BTTC]{\binarybaseline};
                \addplot[pattern = crosshatch,pattern color=c3] table[x=datasets,y=KD]{\binarybaseline};
        \end{axis}
        \end{tikzpicture}
        \caption{Efficiency of binary $\delta$-temporal triangle counting.}
	\label{fig:overall-binary}
    \end{minipage}
\end{figure*}

% \begin{figure}[h]	
% 	\centering
%         \ref{sctest}\\
%         \vspace{-10pt}
%         \subfigure[Time cost]{
%     	\begin{tikzpicture}[scale=0.75]
%         		\begin{axis}[
%         			width=.5\textwidth,
%                         height=0.3\textwidth,	
%                         xticklabels={,,20\%,40\%,60\%,80\%,100\%},
%         			legend style = {
%                             legend columns=-1,
%                 		font=\Large,
%                             draw=none,
%                         },
%                         mark size=5pt,
%                         line width=2.5pt,
%                         legend to name=sctest,
%                         legend image post style={scale=0.8, ultra thick},
%         			xmin=0.5,xmax = 5.5,
%         			ymin=0,ymax=100000000,
%         			ymode =log,
%         			% ylabel style={yshift=-4pt},
%         			ylabel={\normalsize \bf Time (ms)},
%         			ticklabel style={font=\large},
%     				every axis plot/.append style={ultra thick},
%     				every axis/.append style={ultra thick},
%         		]
%                     \addplot [mark=x,color=c3,line width=2.5pt] table[x=length,y=CT]{\sctime};
% 				\addplot [mark=o,color=c4,line width=2.5pt] table[x=length,y=EM]{\sctime};
%                     \addplot [mark=square,color=c5,line width=2.5pt] table[x=length,y=WK]{\sctime};
%                     \addplot [mark=triangle,color=c6,line width=2.5pt] table[x=length,y=ST]{\sctime};
%                     \addplot [mark=diamond,color=c7,line width=2.5pt] table[x=length,y=GR]{\sctime};
% 				\legend{{\small \tt CT},{\small \tt EM}, {\small \tt WK}, {\small \tt ST}, {\small \tt GR}}
%                 \end{axis}
%             \end{tikzpicture}
%         }
%         \subfigure[Space cost]{
%             \begin{tikzpicture}[scale=0.75]
%         		\begin{axis}[
%         			width=.5\textwidth,
%                         height=0.3\textwidth,
%     				every axis plot/.append style={ultra thick},
%                         every axis/.append style={ultra thick},	
%                         mark size=5pt,
%                         line width=2.5pt,
%                         xticklabels={,,20\%,40\%,60\%,80\%,100\%},
%         			xmin=0.5,xmax = 5.5,
%         			ymin=0,ymax=1000000,
%         			ymode =log,
%         			% ylabel style={yshift=-4pt},
%         			ylabel={\normalsize \bf Space (mb)},
%         			ticklabel style={font=\large},
%         		]
%                     \addplot [mark=x,color=c3,line width=2.5pt] table[x=length,y=CT]{\scspace};
% 				\addplot [mark=o,color=c4,line width=2.5pt] table[x=length,y=EM]{\scspace};
%                     \addplot [mark=square,color=c5,line width=2.5pt] table[x=length,y=WK]{\scspace};
%                     \addplot [mark=triangle,color=c6,line width=2.5pt] table[x=length,y=ST]{\scspace};
%                     \addplot [mark=diamond,color=c7,line width=2.5pt] table[x=length,y=GR]{\scspace};
%             \end{axis}
%             \end{tikzpicture}
%         }
%         \caption{Scalability test of indexing time and space.}
% 	\label{fig:scalable}
% \end{figure}



%  

% \begin{figure}[h]	
% 	\centering
% 	\begin{tikzpicture}[scale=0.7]
%     		\begin{axis}[
%     			ybar,
%     			bar width=0.35cm,
%     			width=.65\textwidth,
%                     height=0.3\textwidth,
%     			xtick={1,2,3,4,5},	
%                     xticklabels={GR,ST,WK,EM,CT},
%     			legend style = {
%                         legend columns=-1,
%             		font=\large,
%                         draw=none,
%                         at={(0.8,1)/}
%                     },
%     			legend entries={{\tt DOTTT}, {\tt EDTTC},{\tt LSC-query}},
%     			xmin=0,xmax = 6,
%     			ymin=0,ymax=100000000,
%     			ymode =log,
%                     log origin=infty,
%     			ylabel style={yshift=-4pt},
%     			ylabel={\LARGE Time (ms)},
%     			ticklabel style={font=\LARGE},
% 				every axis plot/.append style={ultra thick},
% 				every axis/.append style={ultra thick},
%     		]
%     		\addplot[pattern=north west lines, pattern color=orange] table[x=datasets,y=DOTTT]{\baseline};
%     		\addplot[pattern = grid, pattern color=blue] table[x=datasets,y=TTC]{\baseline};
%     		% \addplot[pattern = crosshatch dots,pattern color=green] table[x=datasets,y=kd-tree]{\baseline};
%                 \addplot[pattern = crosshatch dots,pattern color=red] table[x=datasets,y=LSC]{\baseline};
%         \end{axis}
%         \end{tikzpicture}
%         \caption{Efficiency of $\delta$-triangle counting on directed graphs.}
% 	\label{fig:directed}
% \end{figure}


% \begin{figure}[h]	
% 	\centering
%         \subfigure[Time cost]{
%     	\begin{tikzpicture}[scale=0.5]
%         		\begin{axis}[
%         			ybar,
%         			bar width=0.5cm,
%         			width=.45\textwidth,
%                         height=0.33\textwidth,
%         			xtick={1,2,3,4,5},	
%                         xticklabels={CT,EM,WK,ST,GR},
%         			legend style = {
%                             legend columns=-1,
%                 		font=\huge,
%                             draw=none,
%                             at={(0.55,1)/}
%                         },
%         			legend entries={{\tt KDT-Index}},
%         			xmin=0,xmax = 6,
%         			ymax=100000000,
%         			ymode =log,
%         			ylabel style={yshift=-4pt},
%         			ylabel={\huge \bf Time (ms)},
%         			ticklabel style={font=\huge},
%     				every axis plot/.append style={line width=2pt},
%     				every axis/.append style={line width=2pt},
%         		]
%                     \addplot[pattern = crosshatch dots,pattern color=red] table[x=datasets,y=KD]{\constructiontime};
%                 \end{axis}
%             \end{tikzpicture}
%         }
%         \subfigure[Space cost]{
%             \begin{tikzpicture}[scale=0.5]
%         		\begin{axis}[
%         			ybar,
%         			bar width=0.5cm,
%         			width=.45\textwidth,
%                         height=0.33\textwidth,
%         			xtick={1,2,3,4,5},	
%                         xticklabels={CT,EM,WK,ST,GR},
%         			legend style = {
%                             legend columns=-1,
%                 		font=\huge,
%                             draw=none,
%                             at={(0.55,1)/}
%                         },
%         			legend entries={{\tt KDT-Index}},
%         			xmin=0,xmax = 6,
%         			ymax=1000000, ymin=10,
%         			ymode =log,
%         			ylabel style={yshift=-4pt},
%         			ylabel={\huge \bf Space (mb)},
%         			ticklabel style={font=\huge},
%     				every axis plot/.append style={line width=2pt},
%     				every axis/.append style={line width=2pt},
%         		]
%                     \addplot[pattern = crosshatch dots,pattern color=blue] table[x=datasets,y=KD]{\constructionspace};
%             \end{axis}
%             \end{tikzpicture}
%         }
%         \caption{Time and space cost of {\tt KDT-Index} construction.}
% 	\label{fig:cons-binary}
% \end{figure}

%==========case study=============

\pgfplotstableread[row sep=\\,col sep=&]{
time&AI0& AI1 & AI2 & AI3 &DB0 &DB1 &DB2&DB3 \\
6 & 9811 & 23753 & 31651 & 34581 & 4863 & 10249 &13235 & 13428 \\
5 & 2948& 4841 & 5870 & 6415 & 2081 & 3462 &4314& 4798\\
4 & 1336& 2130 & 2553 & 2734 & 2061 & 2877 & 3959 & 4075\\
3 & 434& 593 & 864 & 896& 590 & 1168 & 1353 & 1431\\
2 & 165& 239& 254 & 261 & 309 & 471 & 504 & 511\\
1 & 9& 11& 11 & 11& 23& 32 & 32 & 32\\
}\weighted


\pgfplotstableread[row sep=\\,col sep=&]{
time&AI0& AI1 & AI2 & AI3 &DB0 &DB1 &DB2&DB3 \\
6 & 0.592 & 0.458 & 0.415 & 0.398 & 0.745 & 0.592 &0.533 & 0.512 \\
5 & 0.614& 0.469 & 0.411 & 0.388 & 0.732 & 0.586 &0.514& 0.497\\
4 & 0.76& 0.63 & 0.563 & 0.548 & 0.779 & 0.629 & 0.557& 0.523\\
3 & 0.758& 0.587 & 0.526 & 0.5& 0.787 & 0.647 & 0.584& 0.552\\
2 & 0.784& 0.617& 0.542 & 0.505 & 0.718 & 0.585 & 0.544 & 0.501\\
1 & 1& 0.73& 0.711 & 0.711& 0.563& 0.467 & 0.429 & 0.414\\
}\binary

\pgfplotstableread[row sep=\\,col sep=&]{
time&AI0& AI1 & AI2 & AI3 &DB0 &DB1 &DB2&DB3 \\
6 & 5.451 & 13.196 & 17.584 & 19.212 & 2.702 & 5.694 &7.353 & 7.46\\
5 & 1.638& 2.689 & 3.261 & 3.564 & 1.156 & 1.923 &2.397& 2.666\\
4 & 0.742& 1.183 & 1.418 & 1.519 & 1.145 & 1.598 & 2.199 & 2.264\\
3 & 0.241& 0.329 & 0.48 & 0.498& 0.328 & 0.649 & 0.752 & 0.795\\
2 & 0.092& 0.133& 0.141 & 0.145 & 0.172 & 0.262 & 0.28 & 0.284\\
1 & 0.005& 0.006& 0.006 & 0.006& 0.013& 0.018 & 0.018 & 0.018\\
}\weighteddensity



\begin{figure*}[h]
        \centering
	\ref{tbc}\\
        \vspace{-20pt}
	\subfigure[CT]{
	   \begin{tikzpicture}[scale=0.49]
	   \begin{axis}[
                    width=.39\textwidth,
                    height=0.3\textwidth,
				xticklabels={,,$20\%$,$40\%$,$60\%$,$80\%$,100\%},
				xmin=0.1,xmax=1.1,
				ymin=0,ymax=100000,
				ymode = log,
				mark size=5pt,
                    line width=2.5pt,
				ylabel={\LARGE \bf Response time (ms)},
                    ylabel style={xshift=12pt,yshift=-2pt},
				ticklabel style={font=\Large},
				every axis plot/.append style={ultra thick},
				every axis/.append style={ultra thick},
			]
			\addplot [mark=x,color=c4,line width=2.5pt] table[x=length,y=KD]{\tsvbinary};
			\addplot [mark=o,color=c6,line width=2.5pt] table[x=length,y=BTTC]{\tsvbinary};
            \addplot [mark=star,color=c7,line width=2.5pt] table[x=length,y=DOTTT]{\tsvbinary};
		\end{axis}
	\end{tikzpicture}
	}
        \subfigure[EM]{
	   \begin{tikzpicture}[scale=0.49]
	   \begin{axis}[
                    width=.39\textwidth,
                    height=0.3\textwidth,
				xticklabels={,,$20\%$,$40\%$,$60\%$,$80\%$,$100\%$},
				xmin=0.1,xmax=1.1,
				ymin=0,ymax=100000,
				ymode = log,
				mark size=5pt,
                    line width=2.5pt,
				ticklabel style={font=\Large},
				every axis plot/.append style={ultra thick},
				every axis/.append style={ultra thick},
				]
                    \addplot [mark=x,color=c4,line width=2.5pt] table[x=length,y=KD]{\emailbinary};	
                    \addplot [mark=o,color=c6,line width=2.5pt] table[x=length,y=BTTC]{\emailbinary};
                    \addplot [mark=star,color=c7,line width=2.5pt] table[x=length,y=DOTTT]{\emailbinary};
			\end{axis}
	\end{tikzpicture}
	}
        \subfigure[WK]{
	   \begin{tikzpicture}[scale=0.49]
	   \begin{axis}[
                    width=.39\textwidth,
                    height=0.3\textwidth,
				xticklabels={,,$20\%$,$40\%$,$60\%$,$80\%$,$100\%$},
				xmin=0.1,xmax=1.1,
				ymin=0,ymax=100000,
				ymode = log,
				mark size=5pt,
                    line width=2.5pt,
				ticklabel style={font=\Large},
				every axis plot/.append style={ultra thick},
				every axis/.append style={ultra thick},
				]
				\addplot [mark=x,color=c4,line width=2.5pt] table[x=length,y=KD]{\wikibinary};
				\addplot [mark=o,color=c6,line width=2.5pt] table[x=length,y=BTTC]{\wikibinary};
                \addplot [mark=star,color=c7,line width=2.5pt] table[x=length,y=DOTTT]{\wikibinary};
			\end{axis}
	\end{tikzpicture}
	}
        \subfigure[ST]{
	   \begin{tikzpicture}[scale=0.49]
	   \begin{axis}[
                    width=.39\textwidth,
                    height=0.3\textwidth,
				xticklabels={,,$20\%$,$40\%$,$60\%$,$80\%$,$100\%$},
				xmin=0.1,xmax=1.1,
				ymin=0,ymax=10000000,
				ymode = log,
				mark size=5pt,
                    line width=2.5pt,
				ticklabel style={font=\Large},
				every axis plot/.append style={ultra thick},
				every axis/.append style={ultra thick},
				]
                    \addplot [mark=x,color=c4,line width=2.5pt] table[x=length,y=KD]{\stackoverflowbinary};
				\addplot [mark=o,color=c6,line width=2.5pt] table[x=length,y=BTTC]{\stackoverflowbinary};
                \addplot [mark=star,color=c7,line width=2.5pt] table[x=length,y=DOTTT]{\stackoverflowbinary};
		\end{axis}
	\end{tikzpicture}
        }
        \subfigure[GR]{
	   \begin{tikzpicture}[scale=0.49]
	   \begin{axis}[
			legend style = {
                    legend columns=-1,
                    draw=none,
			},
                legend to name=tbc,
                legend image post style={scale=0.8, ultra thick},
                width=.39\textwidth,
                height=0.3\textwidth,
			xticklabels={,,$20\%$,$40\%$,$60\%$,$80\%$,$100\%$},
			xmin=0.1,xmax=1.1,
			% ymin=0,ymax=10000000,
			ymode = log,
			mark size=5pt,
                line width=2.5pt,
			ticklabel style={font=\Large},
			every axis plot/.append style={ultra thick},
			every axis/.append style={ultra thick},
			]
			\addplot [mark=x,color=c4,line width=2.5pt] table[x=length,y=KD]{\graphfivebinary};
			\addplot [mark=o,color=c6,line width=2.5pt] table[x=length,y=BTTC]{\graphfivebinary};
            \addplot [mark=star,color=c7,line width=2.5pt] table[x=length,y=DOTTT]{\graphfivebinary};
			\legend{{\footnotesize {\tt KDT-Index-query}}, {\footnotesize  \BTTC}, {\footnotesize {\tt B-DOTTT}}}
			\end{axis}
	\end{tikzpicture}
    }
    \setlength{\abovecaptionskip}{0.05cm}
    \caption{Effect of $(t_e-t_s)$ (which varies from
    20\% to 40\%, 60\%, 80\%, and 100\% of $t_{max}$) for binary counting.}
    \label{fig:length-binary}
\end{figure*}



\begin{figure*}[h]
        \centering
	\ref{dbc}\\
        \vspace{-20pt}
        \subfigure[CT]{
	\begin{tikzpicture}[scale = 0.49]
	   \begin{axis}[
                    width=0.39\textwidth,
                    height=0.3\textwidth,
				xtick={0,0.1,0.3,0.5,0.7,0.9},
				xticklabels={,$10\%$,$30\%$,$50\%$,$70\%$,$90\%$},
				xmin=0,xmax=1,
				ymin=0,ymax=10000,
				ymode = log,
				mark size=5pt,
                    line width=2.5pt,
                    ylabel={\LARGE \bf Response time (ms)},
				ylabel style={xshift=12pt,yshift=-2pt},
				% xlabel={\huge \bf $\delta$ length ratio},
				ticklabel style={font=\Large},
				every axis plot/.append style={ultra thick},
				every axis/.append style={ultra thick},
				]
				\addplot [mark=x,color=c4,line width=2.5pt] table[x=length,y=KD]{\tsvdbinary};
				\addplot [mark=o,color=c6,line width=2.5pt] table[x=length,y=BTTC]{\tsvdbinary};		
                    % \addplot [mark=star,color=c8,line width=2.5pt] table[x=length,y=kd-tree]{\tsvd};
                    \addplot [mark=star,color=c7,line width=2.5pt] table[x=length,y=DOTTT]{\tsvdbinary};
			\end{axis}
	\end{tikzpicture}
	}
        \subfigure[EM]{
	\begin{tikzpicture}[scale = 0.49]
	   \begin{axis}[
                    width=.39\textwidth,
                    height=0.3\textwidth,
				xtick={0,0.1,0.3,0.5,0.7,0.9},
				xticklabels={,$10\%$,$30\%$,$50\%$,$70\%$,$90\%$},
				xmin=0,xmax=1,
				ymin=0,ymax=100000,
				ymode = log,
				mark size=5pt,
                    line width=2.5pt,
				% ylabel={\huge \bf Running time (ms)},
				% ylabel style={yshift=-5pt},
				% xlabel={\huge \bf $\delta$ length ratio},
				ticklabel style={font=\Large},
				every axis plot/.append style={ultra thick},
				every axis/.append style={ultra thick},
				]
				\addplot [mark=x,color=c4,line width=2.5pt] table[x=length,y=KD]{\emaildbinary};
				\addplot [mark=o,color=c6,line width=2.5pt] table[x=length,y=BTTC]{\emaildbinary};		
                    % \addplot [mark=star,color=c8,line width=2.5pt] table[x=length,y=kd-tree]{\emaild};
                    \addplot [mark=star,color=c7,line width=2.5pt] table[x=length,y=DOTTT]{\emaildbinary};
			\end{axis}
	\end{tikzpicture}
	}
        \subfigure[WK]{
	\begin{tikzpicture}[scale = 0.49]
	   \begin{axis}[
                    width=.39\textwidth,
                    height=0.3\textwidth,
				xtick={0,0.1,0.3,0.5,0.7,0.9},
				xticklabels={,$10\%$,$30\%$,$50\%$,$70\%$,$90\%$},
				xmin=0,xmax=1,
				ymin=0,ymax=200000,
				ymode = log,
				mark size=5pt,
                    line width=2.5pt,
				% ylabel={\huge \bf Running time (ms)},
				% ylabel style={yshift=-5pt},
				% xlabel={\huge \bf $\delta$ length ratio},
				ticklabel style={font=\Large},
				every axis plot/.append style={ultra thick},
				every axis/.append style={ultra thick},
				]
				\addplot [mark=x,color=c4,line width=2.5pt] table[x=length,y=KD]{\wikidbinary};
				\addplot [mark=o,color=c6,line width=2.5pt] table[x=length,y=BTTC]{\wikidbinary};		
                    % \addplot [mark=star,color=c8,line width=2.5pt] table[x=length,y=kd-tree]{\wikid};
                    \addplot [mark=star,color=c7,line width=2.5pt] table[x=length,y=DOTTT]{\wikidbinary};
			\end{axis}
	\end{tikzpicture}
	}
        \subfigure[ST]{
	\begin{tikzpicture}[scale = 0.49]
	   \begin{axis}[
                    width=.39\textwidth,
                    height=0.3\textwidth,
				xtick={0,0.1,0.3,0.5,0.7,0.9},
				xticklabels={,$10\%$,$30\%$,$50\%$,$70\%$,$90\%$},
				xmin=0,xmax=1,
				ymin=0,ymax=10000000,
				ymode = log,
				mark size=5pt,
                    line width=2.5pt,
				% ylabel={\huge \bf Running time (ms)},
				% ylabel style={yshift=-5pt},
				% xlabel={\huge \bf $\delta$ length ratio},
				ticklabel style={font=\Large},
				every axis plot/.append style={ultra thick},
				every axis/.append style={ultra thick},
				]
				\addplot [mark=x,color=c4,line width=2.5pt] table[x=length,y=KD]{\stackoverflowdbinary};
				\addplot [mark=o,color=c6,line width=2.5pt] table[x=length,y=BTTC]{\stackoverflowdbinary};		
                    % \addplot [mark=star,color=c8] table[x=length,y=kd-tree]{\stackoverflowd};
                    \addplot [mark=star,color=c7,line width=2.5pt] table[x=length,y=DOTTT]{\stackoverflowdbinary};
			\end{axis}
	\end{tikzpicture}
	}
        \subfigure[GR]{
	\begin{tikzpicture}[scale = 0.49]
	   \begin{axis}[
			    legend style = {
				      legend columns=-1,
                        draw=none,
				},
				legend to name=dbc,
                    legend image post style={scale=0.8, ultra thick},
                    width=.39\textwidth,
                    height=0.3\textwidth,
                    xtick={0,0.1,0.3,0.5,0.7,0.9},
				xticklabels={,$10\%$,$30\%$,$50\%$,$70\%$,$90\%$},
				xmin=0,xmax=1,
				ymin=0,ymax=100000000,
				ymode = log,
				mark size=5pt,
                    line width=2.5pt,
				ticklabel style={font=\Large},
				every axis plot/.append style={ultra thick},
				every axis/.append style={ultra thick},
				]
				\addplot [mark=x,color=c4,line width=2.5pt] table[x=length,y=KD]{\graphfivedbinary};
				\addplot [mark=o,color=c6,line width=2.5pt] table[x=length,y=BTTC]{\graphfivedbinary};		
                    \addplot [mark=star,color=c7,line width=2.5pt] table[x=length,y=DOTTT]{\graphfivedbinary};
				\legend{{\footnotesize {\tt KDT-Index-query}}, {\footnotesize  \BTTC}, {\footnotesize {\tt B-DOTTT}}}
			\end{axis}
	   \end{tikzpicture}
	}
 \setlength{\abovecaptionskip}{0.05cm}
\caption{Effect of $\delta$ (which varies from 10\% to 30\%, 50\%, 70\%, and 90\% of $t_{max}$) for binary counting.}
\label{fig:query-delta-binary}
% \vspace{-0.1in}
 \end{figure*}
\subsection{Efficiency of $\delta$-Temporal Triangle Counting}
\label{sec:experiment-weighted}

$\bullet$ {\bf Overall results.}
%
For each dataset, we report the average response time of each algorithm in Figure \ref{fig:overall}. 
%
Clearly, our online algorithm \OTTC consistently outperforms \DOTTT across all datasets.
%
For example, on the CT dataset, \OTTC is 70 $\times$ faster than \DOTTT.
%
{\color{black}
The baseline index solution, {\tt TSRjoin}, performs similarly to our \OTTC.
%
This is because {\tt TSRjoin} needs to enumerate the graph and all $\delta$-temporal triangles, which is time-consuming. }
%
{\color{black}Moreover, {\tt WT-Index-query} demonstrates superior performance since it is up to eight orders of magnitude faster than \OTTC, {\tt TSRjoin}, and \DOTTT}.

% Notably, our online algorithm \OTTC outperforms \DOTTT on all datasets. For instance, on the CT dataset, \OTTC is 70$\times$ faster than \DOTTT. Additionally, {\tt WT-Index-query} achieves the best performance among all algorithms.
% Specifically, {\tt WT-Index-query} is up to eight orders of magnitude faster than \OTTC and \DOTTT.
% demonstrating at most $10^8\times$ faster execution than \OTTC. 

$\bullet$ {\bf Effect of $(t_e-t_s)$.}
%
In this experiment, we consider five different interval lengths: 20\%, 40\%, 60\%, 80\%, and 100\% of $t_{max}$.
%
For each interval length, we conduct 1,000 counting queries with $t_s$ randomly selected, where $\delta$ is consistently set to 10\% of each interval length, and depict the efficiency results in Figure \ref{fig:length}.
%
As the interval length increases, \OTTC, { \color{black}{\tt TSRjoin}}, and \DOTTT experience longer response times, since the number of $\delta$-temporal triangles increases.
%
In contrast, the response time of {\tt WT-Index} remains relatively stable, exhibiting a slight decrease. 
%
This is because when the query interval spans the entire duration $[0, t_{max}]$, the {\tt WT-Index} responses with $O(1)$ time complexity.


% To evaluate the impact of query interval lengths, we explore five different lengths for query time intervals: $20\%$, $40\%$, $60\%$, $80\%$, and $100\%$ of $t_{max}$ respectively.
% %
% For each length, we execute 1000 queries with $t_s$ randomly selected. 
% %
% The query parameter $\delta$ is set to 10\% of each interval length.
% %
% The results, depicted in Figure \ref{fig:length}, illustrate the average query response times across all five graphs.
% %
% As the time interval widens, \OTTC and \DOTTT experience increased response times. Conversely, the response time of {\tt WT-Index} remains relatively stable, exhibiting a slight decrease. This stability arises from the fact that when the query interval spans the entire duration $[0,t_{max}]$, the {\tt WT-Index} entails only $O(1)$ time complexity for the response.

% To evaluate the effect of the length of query intervals, for each graph, we fix the query $\delta$ and sampling factor as defaulted and consider five query time interval lengths, i.e., $20\%$, $40\%$, $60\%$, $80\%$, $100\%$ of $t_{max}$ respectively. For each length, we generate 10000 queries where $t_s$ is selected randomly (For the \EETTC algorithm on ST and GR, we only run 1000 queries due to the large time cost). Figure \ref{fig:length} reports the average time cost of answering one query on all five graphs as efficiency results. Note that since \DOTTT is designed only for different duration $\delta$ and it is not as efficient as our \online algorithm, we only use \online as our online algorithm. When the time interval becomes larger, \online and kd-tree take more time to respond, 
% while \LSC's cost does not change much, even taking less time. Because when the query interval is $[0,t_{max}]$, the $t_s, t_e$ indexes only cost $O(1)$ time complexity to respond.

$\bullet$ {\bf Effect of $\delta$.}
%
In this experiment, we set $[t_s,t_e]= [0, t_{max}]$, and vary $\delta = t_{max}\cdot y$ with $y\in \{10\%, 30\%, 50\%, 70\%, 90\%\}$.
%
For each $\delta$, we conduct 1,000 queries and record the average response time. The findings are summarized in Figure \ref{fig:query-delta}.
%
Clearly, the choice of $\delta$ demonstrates little impact on counting time cost because the time complexity of all {\color{black} four} algorithms is independent of $\delta$. 
%
Again, the {\tt WT-Index-query} is much faster than all other algorithms.


% In this experiment, we investigate how varying the value of $\delta$ influences query efficiency. We maintain a constant time interval length $[0,t_{max}]$ and explore different ratios of $\delta$ to the length of time interval, selecting from the set $\{10\%, 30\%, 50\%, 70\%, 90\%\}$.
% %
% For each $\delta$ value, we conduct 1,000 queries and record the average response time. The findings are summarized in Figure \ref{fig:query-delta}.
% %
% Notably, the choice of $\delta$ demonstrates minimal impact on query time. Moreover, the index-based algorithm exhibits a significant performance advantage, with response speeds up to at most eight orders of magnitude faster compared to online approaches.



$\bullet$ {\bf Time and space costs of index construction.}
%
In this experiment, we report the time and space costs of index construction for all graphs in Figure \ref{fig:cons}.
%
{\color{black} %(R2.A1)
Since building the {\tt TSRjoin} index with all the converted $O(m^2)$ edges caused an out-of-memory error even for the second smallest dataset (i.e., EM), we only use the converted edges whose length equals one of the nine different $\delta$ values used in experiments for each dataset to build the index. 
%
As a result, the time and space costs of constructing the {\tt TSRjoin} index with only $O(m)$ edges are small.
%
Clearly, the time and space costs of the {\tt WT-Index} and {\tt TSRjoin} index increase as the graphs become larger.
%
Note that the space cost of {\tt WT-Index} on ST is larger than that of GR, mainly because the $t_{max}$ of ST is larger than that of GR, leading to a larger tree height for the {\tt WT-Index} of ST.
%
We also compare the space cost of our index with the graph size in Figure \ref{fig:cons}(b).
%
Note that in the CT dataset, the space cost of {\tt WT-Index} is 5000$\times$ larger than the graph size. This is due to the graph's high density; despite having only 28,244 edges and 274 nodes, it contains 15M C-points.
%
Since the number of C-points $\pi$ is bounded by $O(m^2)$, the space cost of our {\tt WT-Index} is larger than the graph size, but it is still affordable.}

In the above experiments, the efficiency of the index-based counting algorithms is measured without considering index construction time, so it may be unfair when the indexing time should be considered. 
%
To make a fair comparison, we amortize the index construction time across the index-based queries and compare it with the online query algorithm \OTTC. 
%
Table \ref{tab:amortize} reports the number of counting queries required for our index method to surpass the online algorithm for each dataset, where the queries span the full-length time interval with randomly selected $\delta$.
%
These values are remarkably low compared to the total number of timestamps, especially for larger graphs.
%
Hence, even with a modest number of queries, the index-based algorithm consistently outperforms the online algorithm.


% {\color{red} Why do we need to repeatedly query the number of triangles in the entire graph? The $\delta$ of each query will be different.}

\begin{table}[htbp]
  \small
  \setlength{\abovecaptionskip}{0.15cm}
  \caption{Number of counting queries to offset indexing time.}
  \label{tab:amortize}
  \centering
  \begin{tabular}{c|c|c|c|c|c}
    \hline
     Dataset & CT & EM & WK & ST & GR \\
    \hline\hline
    Number of counting queries & 1.6K & 2.8K & 172 &41 & 11\\
    \hline
  \end{tabular}
\end{table}




$\bullet$ \textbf{Scalability test.}
% 
To evaluate the scalability of our index construction algorithm, we randomly select 20\%, 40\%, 60\%, 80\%, and 100\% of the edges from each graph, thereby obtaining five induced subgraphs from these edges.
%
We then build indices on these subgraphs of all datasets.
%
As shown in Figure \ref{fig:scalable}, the indexing time and space costs of our index construction algorithm increase linearly with the graph size, thereby demonstrating good scalability.

$\bullet$ {\bf Index maintenance.}
%
In this experiment, we first build the {\tt WT-Index} by using edges in the range $[0,0.8t_{max}]$, and then update {\tt WT-Index} by sequentially considering the remaining edges from $[0.8t_{max}+1, t_{max}]$.
%
{\color{black} Table \ref{tab:update} presents the average time cost of updating {\tt WT-Index} for each new edge across various datasets.
%
Our results demonstrate that the proposed index maintenance algorithm is significantly faster than rebuilding {\tt WT-Index} from scratch.}

\begin{table}[ht]
  \small
  \centering
  \setlength{\abovecaptionskip}{0.15cm}
  \caption{\color{black}Average time cost of updating {\tt WT-Index} 
 for each edge.}
  \label{tab:update}
  % \small
  \begin{tabular}{c|c|c|c|c|c}
    \hline
     Dataset & CT & EM & WK & ST & GR \\
    \hline\hline
    time costs $(ms)$ & 2.03 & 6.39 & 2.13 & 0.62 & 0.43\\
    \hline
  \end{tabular}
\end{table}
\subsection{Efficiency of Binary $\delta$-Temporal Triangle Counting}
\label{sec:experiment-binary}

$\bullet$ {\bf Overall results.}
%
For each dataset, we report the average response
time of each algorithm in Figure \ref{fig:overall-binary}. 
%
Our online algorithm \BTTC consistently outperforms {\tt B-DOTTT} across all datasets.
%
For example, on the CT dataset, \BTTC is 23$\times$ faster than {\tt B-DOTTT}. 
%
Moreover, {\tt KDT-Index-query} achieves the best performance, as it is up to four orders of magnitude faster than \BTTC.

% \begin{figure}[h]	
% 	\centering
% 	\begin{tikzpicture}[scale=0.65]
%     		\begin{axis}[
%     			ybar,
%     			bar width=0.35cm,
%     			width=.73\textwidth,
%                     height=0.26\textwidth,
%     			xtick={1,2,3,4,5},	
%                     xticklabels={GR,ST,WK,EM,CT},
%     			legend style = {
%                         legend columns=-1,
%             		font=\large,
%                         draw=none,
%                         at={(0.64,1)/}
%                     },
%     			legend entries={{\tt B-DOTTT}, {\tt BTTC},{\tt KDT-Index-query}},
%     			xmin=0.5,xmax = 5.5,
%     			ymin=0,ymax=20000000,
%     			ymode =log,
%                     log origin=infty,
%     			% ylabel style={yshift=-4pt},
%     			ylabel={\LARGE \bf Time (ms)},
%     			ticklabel style={font=\LARGE},
% 				every axis plot/.append style={ultra thick},
% 				every axis/.append style={ultra thick},
%                     x dir=reverse,
%     		]
%     		\addplot[pattern=north west lines, pattern color=c1] table[x=datasets,y=DOTTT]{\binarybaseline};
%     		\addplot[pattern = grid, pattern color=c2] table[x=datasets,y=BTTC]{\binarybaseline};
%     		% \addplot[pattern = crosshatch dots,pattern color=green] table[x=datasets,y=kd-tree]{\baseline};
%                 \addplot[pattern = crosshatch,pattern color=c3] table[x=datasets,y=KD]{\binarybaseline};
%         \end{axis}
%         \end{tikzpicture}
%         \caption{Efficiency of binary $\delta$-temporal triangle counting.}
% 	\label{fig:overall-binary}
% \end{figure}
 
$\bullet$ {\bf Effect of $(t_e-t_s)$.}
%
In this experiment, we test five different lengths: $20\%$, $40\%$, $60\%$, $80\%$, and $100\%$ of $t_{max}$, with $\delta$ set to the default value.
%
For each length, we execute 1,000 counting queries with randomly selected $t_s$ and report the average response time in Figure \ref{fig:length-binary}.
%
Again, the response time increases as the interval length increases since more binary $\delta$-temporal triangles are involved.

$\bullet$ {\bf Effect of $\delta$.}
%
In this experiment, we set $[t_s,t_e]= [0, t_{max}]$, and vary $\delta = t_{max}\cdot y$ with $y\in \{10\%, 30\%, 50\%, 70\%, 90\%\}$.
%
For each $\delta$, we conduct 1,000 queries and record the average response time.
%
The findings are summarized in Figure  \ref{fig:query-delta-binary}.
%
We observe that $\delta$ has little effect on the efficiency on \BTTC and {\tt B-DOTTT}. But it affects the efficiency of {\tt KDT-Index-query} a lot because when $\delta$ goes larger, the response time complexity of {\tt KDT-Index} approaches $O(1)$.


% $\bullet$ {\bf Time and space costs of index construction.}
%
% {\color{red}In this experiment, we detail the time and space costs of index construction for all graphs in Figure \ref{fig:cons}. 
% %
% The time and space requirements of the {\tt WT-Index} scale with the graph sizes. Notably, the space cost of ST is larger than GR, possibly because the tree height of {\tt WT-Index} of ST is larger than that of GR since the $t_{max}$ of ST is larger than that of GR.
% }
%
% In this experiment, we evaluate the time and space costs of index construction for all graphs. As shown in Figure \ref{fig:cons-binary}, the time and space requirements of the {\tt KDT-Index} scale with graph sizes.


\subsection{Case Study}
\label{sec:experiment-case-study}

We consider two temporal co-authorship graphs of papers published in database (DB) and artificial intelligence (AI) areas from 2000 to 2023, respectively.
%
Specifically, we first identify the top-50 most frequent keywords in titles of papers in SIGMOD, VLDB, and ICDE as representative DB keywords, and the top-50 most frequent keywords in titles of papers in NIPS, ICML, and ICLR as representative AI keywords (stopwords are omitted).
%
Then, we classify each paper into DB, or AI, or none of them, if the corresponding area has more representative keywords in its title.
%
Afterward, we build two temporal graphs, $G_{AI} = (V, E_{AI})$ and $G_{DB} = (V, E_{DB})$, where $V$ consists of authors who have published at least three papers at KDD, an edge $(u,v,t) \in E_{AI}$ indicates that authors $u$ and $v$ collaborate on an AI paper published in year $t$, and an edge in $E_{DB}$ indicates similar collaborations on DB papers.
%
{\color{black}
We find that $|V| = 2024$, $|E_{AI}| = 9,049$, and $|E_{DB}| = 7,093$, indicating the number of collaborations in the AI community is more than that in the DB community.}
%
Finally, we divide the whole time interval $[2000,2023]$ into six disjoint intervals, each having a 4-year length, and analyze the AI and DB communities by counting $\delta$-temporal triangles.

\begin{figure}[h]
    \centering
    \centering %图片居中
    \subfigure[$\delta \in \{0,1\}$]{
        \begin{tikzpicture} %tikz图片
        \begin{axis}[
            xlabel= {\scriptsize \bf Number of $\delta$-temporal triangles}, %横坐标名
            ylabel= {\scriptsize \bf Number of authors}, %纵坐标名
            ylabel style={yshift=-5pt},
            xlabel style={yshift=5pt},
            ymode = normal,
            xmode = log,
            xmin=0,xmax=5000,
            ymin=0,ymax=250,
            mark size=0.0pt,
            width=0.4\textwidth,
            height=0.25\textwidth,
            ticklabel style={font=\footnotesize},
            every axis plot/.append style={line width= 1.1pt},
            ytick = {10, 50, 100,200},
            xtick = {1, 10, 1e2,1e3,1e4},
            every axis/.append style={line width= 0.8pt},
            legend style = {
                at={(0.97,1)},
    		legend columns=2,
                draw=none,
                font=\Huge,
                nodes={scale=0.35, transform shape}
    	},
            legend image post style={scale=0.45, ultra thick},
         ]
        \addplot[smooth,mark=*,color=c8] table {figure/trend/ai0.txt};
        \addplot[smooth,mark=*,color = c2] table {figure/trend/db0.txt};
        %\addlegendentry{AI ($\delta$=0)}
        \addplot[smooth,mark=*,color =c7] table {figure/trend/ai1.txt};
        %\addlegendentry{AI ($\delta$=1)}
        
        %\addlegendentry{DB ($\delta$=0)}
        \addplot[smooth,mark=*,color = c4] table {figure/trend/db1.txt};
        %\addlegendentry{DB ($\delta$=1)}
        \legend{{ {AI ($\delta$=0)}}, { DB ($\delta$=0)}, { { AI ($\delta$=1)}}, { { DB ($\delta$=1)}}}
        % \addplot[smooth,mark=*,cc5] table {pic/k_0/lj.dat};
        % \addlegendentry{LJ}
        % \addplot[smooth,mark=*,cc6] table {pic/k_0/ew.dat};
        % \addlegendentry{EW}
        % \addplot[smooth,mark=*,cc7] table {pic/k_0/hw.dat};
        % \addlegendentry{HW}
        % \addplot[smooth,mark=*,cc8] table {pic/k_0/wb.dat};
        % \addlegendentry{WB}
        % \addplot[smooth,mark=*,cc9] table {pic/k_0/it.dat};
        % \addlegendentry{IT}
        % \addplot[smooth,mark=*,cc10] table {pic/k_0/uk.dat};
        % \addlegendentry{UK}
     \end{axis}
     \end{tikzpicture}
 }
 \subfigure[$\delta \in \{2,3\}$]{
        \begin{tikzpicture} %tikz图片
        \begin{axis}[
            xlabel= {\scriptsize \bf Number of $\delta$-temporal triangles}, %横坐标名
            %ylabel= {\footnotesize \bf Number of authors}, %纵坐标名
            % ylabel style={yshift=-12pt},
            xlabel style={yshift=5pt},
            ymode = normal,
            xmode = log,
            xmin=0,xmax=5000,
            ymin=0,ymax=250,
            mark size=0.0pt,
            width=0.4\textwidth,
            height=0.25\textwidth,
            ticklabel style={font=\scriptsize},
            every axis plot/.append style={line width= 1.1pt},
            ytick = {10, 50, 100,200},
            xtick = {1, 10, 1e2,1e3,1e4},
            every axis/.append style={line width= 0.8pt},
            legend style = {
                at={(0.97,1)},
    		legend columns=2,
                %font=\footnotesize,
                draw=none,
                font=\Huge,
                nodes={scale=0.35, transform shape}
    	},
            legend image post style={scale=0.45, ultra thick},
         ]
        \addplot[smooth,mark=*,color=c8] table {figure/trend/ai2.txt};
        \addplot[smooth,mark=*,color = c2] table {figure/trend/db2.txt};
        %\addlegendentry{AI ($\delta$=0)}
        \addplot[smooth,mark=*,color =c7] table {figure/trend/ai3.txt};
        %\addlegendentry{AI ($\delta$=1)}
        
        %\addlegendentry{DB ($\delta$=0)}
        \addplot[smooth,mark=*,color = c4] table {figure/trend/db3.txt};
        %\addlegendentry{DB ($\delta$=1)}
        \legend{{ {AI ($\delta$=2)}}, { DB ($\delta$=2)}, { { AI ($\delta$=3)}}, { { DB ($\delta$=3)}}}
        % \addplot[smooth,mark=*,cc5] table {pic/k_0/lj.dat};
        % \addlegendentry{LJ}
        % \addplot[smooth,mark=*,cc6] table {pic/k_0/ew.dat};
        % \addlegendentry{EW}
        % \addplot[smooth,mark=*,cc7] table {pic/k_0/hw.dat};
        % \addlegendentry{HW}
        % \addplot[smooth,mark=*,cc8] table {pic/k_0/wb.dat};
        % \addlegendentry{WB}
        % \addplot[smooth,mark=*,cc9] table {pic/k_0/it.dat};
        % \addlegendentry{IT}
        % \addplot[smooth,mark=*,cc10] table {pic/k_0/uk.dat};
        % \addlegendentry{UK}
     \end{axis}
     \end{tikzpicture}
 }
 \setlength{\abovecaptionskip}{-0.1cm}
 \setlength{\belowcaptionskip}{-2pt}
\caption{\color{black}Distribution of $\delta$-temporal triangles.}
\label{fig:triangle-distribution}
\end{figure}

$\bullet$ {\bf Collaboration density trends of DB and AI communities.} 
%
As a well-known metric of measuring the subgraph cohesiveness \cite{samusevich2016local,tsourakakis2015k}, the triangle density of a graph is defined as the number of $\delta$-temporal triangles over the number of vertices.
%
Figure \ref{fig:case-study-weighted-intro} shows the $\delta$-temporal triangle densities for the DB and AI communities across all time intervals with varying $\delta$ values.
%
We observe that after 2016, the $\delta$-temporal triangle density of the AI community surpasses that of the DB community, indicating AI's rising prominence post-2016.
%
Besides, the number of $\delta$-temporal triangles with $\delta$=1 is significantly higher than that with $\delta$=0, while the difference between $\delta$=2 and $\delta$=3  is minimal.
%
{\color{black} For instance, during the time interval [2020, 2023], the numbers of $\delta$-temporal triangles in the AI community are 9,811, 23,753, 31,651, and 34,581, when $\delta$ is set to 0, 1, 2, and 3, respectively.}
%
This suggests that authors prefer to continue collaborating over short periods.

{\color{black}
Besides, we count the number of $\delta$-temporal triangles that each author is involved in, and report the distribution in Figure \ref{fig:triangle-distribution}, where each point $(x,y)$ means that there are $y$ authors with each participating $x$ $\delta$-temporal triangles.
%
The distribution roughly follows the long-tail distribution \cite{kordumova2016exploring}, indicating that most authors engage with only a few $\delta$-temporal triangles.
%
% Similar to the triangle density, the difference in distribution is more significant between $\delta$=0 and $\delta$=1 compared to $\delta$=2 and $\delta$=3.}

\begin{figure}[h]
        \subfigure[$\delta\in \{0,1\}$]{
	\begin{tikzpicture}[scale = 0.5]
	   \begin{axis}[
                    width=0.7\textwidth,
                    height=0.42\textwidth,
				xtick={0,1,2,3,4,5,6},
				xticklabels={,2000,2004,2008,2012,2016,2020},
				xmin=0.5,xmax=6.5,
				ymin=0.2,ymax=1.4,
                    legend style = {
                        legend columns=2,
                        draw=none,
                        at={(0.9,1)/}
				},
				mark size=5pt,
                    line width=2.5pt,
				% xlabel={\huge \bf $\delta$ length ratio},
                ylabel={\huge \bf $\delta$-transitivity},
				ticklabel style={font=\huge},
				every axis plot/.append style={ultra thick},
				every axis/.append style={ultra thick},
				]
				
            \addplot [mark=x,color=c8,line width=2.5pt] table[x=time,y=AI0]{\binary};
				\addplot [mark=o,color=c2,line width=2.5pt] table[x=time,y=DB0]{\binary};
                    \addplot [mark=x,color=c7,line width=2.5pt] table[x=time,y=AI1]{\binary};
				\addplot [mark=o,color=c4,line width=2.5pt] table[x=time,y=DB1]{\binary};			\legend{{ {\LARGE AI ($\delta$=0)}}, { \LARGE DB ($\delta$=0)}, { {\LARGE AI ($\delta$=1)}}, { {\LARGE DB ($\delta$=1)}}}
                    
			\end{axis}
	\end{tikzpicture}
	}
        \subfigure[$\delta\in \{2,3\}$]{
	\begin{tikzpicture}[scale = 0.5]
	   \begin{axis}[
                    width=0.7\textwidth,
                    height=0.42\textwidth,
				xtick={0,1,2,3,4,5,6},
				xticklabels={,2000,2004,2008,2012,2016,2020},
				xmin=0.5,xmax=6.5,
				ymin=0.2,ymax=1.4,
                    legend style = {
                        legend columns=2,
                        draw=none,
                        at={(0.9,1)/}
				},
				%ymode = log,
				mark size=5pt,
                    line width=2.5pt,
				ticklabel style={font=\huge},
				every axis plot/.append style={ultra thick},
				every axis/.append style={ultra thick},
				]
					
                    \addplot [mark=x,color=c8,line width=2.5pt] table[x=time,y=AI2]{\binary};
				\addplot [mark=o,color=c2,line width=2.5pt] table[x=time,y=DB2]{\binary};
                    \addplot [mark=x,color=c7,line width=2.5pt] table[x=time,y=AI3]{\binary};
				\addplot [mark=o,color=c4,line width=2.5pt] table[x=time,y=DB3]{\binary};	
                    \legend{{ {\LARGE AI ($\delta$=2)}}, { \LARGE DB ($\delta$=2)}, { {\LARGE AI ($\delta$=3)}}, { {\LARGE DB ($\delta$=3)}}}
			\end{axis}
	\end{tikzpicture}
	}
    \setlength{\abovecaptionskip}{-0.1cm}
 \setlength{\belowcaptionskip}{-2pt}
    \caption{$\delta$-transitivity.}
    \label{fig:case-study-binary}
\end{figure}

$\bullet$ {\bf Transitivity trends of DB and AI communities.}
% 
Transitivity is a widely used metric for measuring graph sparsity \cite{chu2011triangle}.
%
We extend the $\delta$-transitivity as three times the number of binary $\delta$-temporal triangles divided by the number of binary $\delta$-temporal wedges, where a binary $\delta$-temporal wedge is a path of three vertices $u$-$v$-$w$ with the timestamp gap of the two edges not exceeding $\delta$.
%
Figure \ref{fig:case-study-binary} shows the $\delta$-transitivity of the DB and AI communities in all the six time intervals with varying $\delta$ values.
%
We observe a continuous decline in the $\delta$-transitivity of the AI community, indicating that it has become sparser as more researchers join it.

% \pgfplotstableread[row sep=\\,col sep=&]{
% 	length & LSC &wavelet tree&online & kd-tree \\
% 	0.2 & 0.05 & 0.02 & 47393& 4\\
% 	0.4 & 0.07 & 0.02 & 134577& 8\\
% 	0.6 & 0.07 & 0.02 & 225871& 13\\
% 	0.8 & 0.05 & 0.01 & 376713& 19\\
% 	1.0 & 0.01 &0.002 & 491971& 26\\
% }\stackoverflow

% \pgfplotstableread[row sep=\\,col sep=&]{
% 	length & LSC &wavelet tree&online& kd-tree \\
% 	0.2 & 0.02 & 0.01 & 64280& 15\\
% 	0.4 & 0.03 & 0.01 & 166448& 42\\
% 	0.6 & 0.02 & 0.01 & 336011& 77\\
% 	0.8 & 0.02 & 0.01 & 676620& 114\\
% 	1.0 & 0.005 &0.002 & 1133938& 146\\
% }\graphfive

% \pgfplotstableread[row sep=\\,col sep=&]{
% 	length & LSC &wavelet tree&online& kd-tree \\
% 	0.2 & 0.021 & 0.01 & 1489& 0.2\\
% 	0.4 & 0.023 & 0.01 & 3490& 0.4\\
% 	0.6 & 0.024 & 0.012 & 5718& 0.7\\
% 	0.8 & 0.022 &0.009 & 8277& 1\\
% 	1.0 & 0.007 &0.002 & 11915& 1.2\\
% }\wiki

% \pgfplotstableread[row sep=\\,col sep=&]{
% 	length & LSC &wavelet tree&online& kd-tree \\
% 	0.2 & 0.011 & 0.007 & 30& 0.02\\
% 	0.4 & 0.012 & 0.007 & 73& 0.04\\
% 	0.6 & 0.012 & 0.008 & 124& 0.06\\
% 	0.8 & 0.011 &0.006 & 181& 0.07\\
% 	1.0 & 0.004 &0.002 & 238& 0.06\\
% }\email

% \pgfplotstableread[row sep=\\,col sep=&]{
% 	length & LSC &wavelet tree&online&kd-tree \\
% 	0.2 & 0.0052 & 0.0041 & 5& 0.008\\
% 	0.4 & 0.0046 & 0.004 & 12& 0.02\\
% 	0.6 & 0.005 & 0.004 & 18& 0.02\\
% 	0.8 & 0.004 &0.004 & 25& 0.02\\
% 	1.0 & 0.003 &0.002 & 33& 0.03\\
% }\tsv



% \pgfplotstableread[row sep=\\,col sep=&]{
% 	length & LSC &DOTTT&TTC& kd-tree \\
% 	0.1 & 0.0123 & 466289 & 135077& 26\\
% 	0.3 & 0.0145 & 489479 & 129966& 24\\
% 	0.5 & 0.0122 & 456316 & 131001& 19\\
% 	0.7 & 0.0112 & 472154 & 143781& 15\\
% 	0.9 & 0.0107 & 477172 & 144079& 12\\
% }\stackoverflowd

% \pgfplotstableread[row sep=\\,col sep=&]{
% 	length & LSC &DOTTT & TTC& kd-tree \\
% 	0.1 & 0.0056 & 2235810 & 885420& 146\\
% 	0.3 & 0.0057 & 2189250 & 998820& 193\\
% 	0.5 & 0.0063 & 2107530 & 887647& 197\\
% 	0.7 & 0.0073 & 2236160 & 902199&180\\
% 	0.9 & 0.0056 &3149010 & 990356& 161\\
% }\graphfived

% \pgfplotstableread[row sep=\\,col sep=&]{
% 	length & LSC &DOTTT & TTC & kd-tree \\
% 	0.1 & 0.0073 & 44509 & 11314 & 1.2\\
% 	0.3 & 0.0081 & 43830 & 11479 & 6\\
% 	0.5 & 0.0077 & 43723 & 11713 &  14.4\\
% 	0.7 & 0.0078 & 43584 & 11376 & 28.8\\
% 	0.9 & 0.0081 & 42510 & 11391 & 44.4\\
% }\wikid

% \pgfplotstableread[row sep=\\,col sep=&]{
% 	length & LSC &DOTTT & TTC & kd-tree \\
% 	0.1 & 0.0042 & 3898 & 238 & 31\\
% 	0.3 & 0.0049 & 3781 & 211 & 20\\
% 	0.5 & 0.0052 & 3615 & 203 & 16\\
% 	0.7 & 0.0056 & 3607 & 174 & 14.6\\
% 	0.9 & 0.0054 & 3259 & 158 & 13\\
% }\emaild

% \pgfplotstableread[row sep=\\,col sep=&]{
% 	length & LSC &DOTTT&TTC& kd-tree \\
% 	0.1 & 0.0026 & 742& 37 & 0.03 \\
% 	0.3 & 0.0029 & 671& 34 & 0.018\\
% 	0.5 & 0.0030 & 656& 33 & 0.016\\
% 	0.7 & 0.0031 & 602& 29 & 0.014\\
% 	0.9 & 0.0029 & 554& 24 & 0.014\\
% }\tsvd




% \pgfplotstableread[row sep=\\,col sep=&]{
% 	length & WTTC & DOTTT \\
% 	0.1  & 135077 & 446289\\
% 	0.3  & 129966 & 489479\\
% 	0.5  & 131001 & 456316\\
% 	0.7  & 143781 & 472154\\
% 	0.9  & 144079 & 477172\\
% }\stackoverflowonline

% \pgfplotstableread[row sep=\\,col sep=&]{
% 	length & WTTC & DOTTT \\
% 	0.1  & 885420 & 2235810\\
% 	0.3  & 998820 & 2189250\\
% 	0.5  & 887647 & 2107530\\
% 	0.7  & 902199 & 2236160\\
% 	0.9  & 990356 & 2149010\\
% }\graphfiveonline

% \pgfplotstableread[row sep=\\,col sep=&]{
% 	length & WTTC & DOTTT \\
% 	0.1  & 11314 & 44509\\
% 	0.3  & 11479 & 43830\\
% 	0.5  & 11713 & 43723\\
% 	0.7  & 11376 & 43584\\
% 	0.9  & 11391 & 42510\\
% }\wikionline

% \pgfplotstableread[row sep=\\,col sep=&]{
% 	length & WTTC & DOTTT \\
% 	0.1  & 238 & 3898\\
% 	0.3  & 211 & 3781\\
% 	0.5  & 203 & 3615\\
% 	0.7  & 174 & 3607\\
% 	0.9  & 158 & 3259\\
% }\emailonline

% \pgfplotstableread[row sep=\\,col sep=&]{
% 	length & WTTC & DOTTT \\
% 	0.1  & 37 & 742\\
% 	0.3  & 34 & 671\\
% 	0.5  & 33 & 656\\
% 	0.7  & 29 & 602\\
% 	0.9  & 24 & 554\\
% }\tsvonline

% \pgfplotstableread[row sep=\\,col sep=&]{
% 	length & LSC &wavelet tree \\
% 	0.01 & 577425 & 609146 \\
% 	0.05 & 798674 & 808090 \\
% 	0.1 & 918819 & 1154172 \\
% 	0.2 & 1288712 & 1476684 \\
% 	0.3 & 1973168 & 1983985 \\
% }\stackoverflowf

% \pgfplotstableread[row sep=\\,col sep=&]{
% 	length & LSC &wavelet tree \\
% 	0.01 & 2038564 & 2172252 \\
% 	0.05 & 2189398 &  2123597\\
% 	0.1 & 2099664 &  2288220 \\
% 	0.2 & 2208948 &  2259616\\
% 	0.3 & 2288616 & 2471462 \\
% }\graphfivef

% \pgfplotstableread[row sep=\\,col sep=&]{
% 	length & LSC &wavelet tree \\
% 	0.01 & 67333 & 74486 \\
% 	0.05 & 104348 & 129932 \\
% 	0.1 & 134255 & 163451 \\
% 	0.2 & 237004 & 261851 \\
% 	0.3 & 288011 & 353856 \\
% }\wikif

% \pgfplotstableread[row sep=\\,col sep=&]{
% 	length & LSC &wavelet tree \\
% 	0.01 & 8393 & 9800 \\
% 	0.05 & 18320 & 22861 \\
% 	0.1 & 30235 & 39094 \\
% 	0.2 & 50959 & 68203 \\
% 	0.3 & 66685 & 94540 \\
% }\emailf

% \pgfplotstableread[row sep=\\,col sep=&]{
% 	length & LSC &wavelet tree \\
% 	0.01 & 637 & 702 \\
% 	0.05 & 1109 & 1316 \\
% 	0.1 & 1687 & 2014 \\
% 	0.2 & 2775 & 3319 \\
% 	0.3 &  3676 & 4454 \\
% }\tsvf



% \pgfplotstableread[row sep=\\,col sep=&]{
% 	length & LSC &wavelet tree \\
% 	0.01 & 0.053 & 0.0134 \\
% 	0.05 & 0.0647 & 0.0154 \\
% 	0.1 & 0.0748 & 0.016 \\
% 	0.2 & 0.0856 & 0.0168 \\
% 	0.3 & 0.1066 & 0.0245 \\
% }\stackoverflowfq

% \pgfplotstableread[row sep=\\,col sep=&]{
% 	length & LSC &wavelet tree \\
% 	0.01 & 0.021 & 0.0113 \\
% 	0.05 & 0.0361 &  0.0118\\
% 	0.1 & 0.0314 &  0.0117 \\
% 	0.2 & 0.0385 & 0.0121 \\
% 	0.3 & 0.0404 & 0.0132 \\
% }\graphfivefq

% \pgfplotstableread[row sep=\\,col sep=&]{
% 	length & LSC &wavelet tree& kd-tree \\
% 	0.01 & 0.0224 & 0.0091& 1 \\
% 	0.05 & 0.0331 & 0.0124& 5\\
% 	0.1 & 0.0422 & 0.0125&  12 \\
% 	0.2 & 0.0485 & 0.0136& 24 \\
% 	0.3 & 0.051 & 0.0141&37 \\
% }\wikifq

% \pgfplotstableread[row sep=\\,col sep=&]{
% 	length & LSC &wavelet tree&kd-tree \\
% 	0.01 & 0.0109 & 0.006& 0.07 \\
% 	0.05 & 0.0157 & 0.0073& 0.4 \\
% 	0.1 & 0.018 & 0.0077& 0.7 \\
% 	0.2 & 0.021 & 0.0084&1.4 \\
% 	0.3 & 0.0218 & 0.0092& 2 \\
% }\emailfq

% \pgfplotstableread[row sep=\\,col sep=&]{
% 	length & LSC &wavelet tree \\
% 	0.01 & 0.0042 & 0.0038 \\
% 	0.05 & 0.0055 & 0.0045 \\
% 	0.1 & 0.0059 & 0.0046 \\
% 	0.2 & 0.0064 & 0.0048 \\
% 	0.3 &  0.0069 & 0.005 \\
% }\tsvfq


% \pgfplotstableread[row sep=\\,col sep=&]{
% 	datasets &fetching& LSC &wavelet tree \\
% 	1 & 2064077& 11557 & 19410 \\
% 	2 & 524830& 40454 & 82720 \\
% 	3 & 47308&7936 & 15302 \\
% 	4 & 6043& 2806 & 4239 \\
% 	5 &  530& 112 & 182 \\
% }\construction

% \pgfplotstableread[row sep=\\,col sep=&]{
% 	datasets & LSC &wavelet tree & TTC \\
% 	1 & 16179 & 18022 & 12083 \\
% 	2 & 9318 & 51200 & 6451\\
% 	3 & 1638 & 7680 & 1228\\
% 	4 & 557 & 789 & 25\\
% 	5 &  44 & 58 &6\\
% }\constructionspace


% \pgfplotstableread[row sep=\\, col sep = &]{
%     length & LSC & TTC & kd-tree\\
%     0.2 & 0.0398 & 14870& 1.2947\\
%     0.4 &0.0395 & 33858& 2.8918\\
%     0.6 &0.041 & 59278& 4.5552\\
%     0.8 &0.0426 & 81808& 7.0367\\
%     1 & 0.0181& 105739&10.5806\\
% }\stackoverflowdirectcircle


% \pgfplotstableread[row sep=\\, col sep = &]{
%     length & LSC & TTC & kd-tree\\
%     0.2 & 0.0255 & 966& 0.0825\\
%     0.4 &0.0288 & 2089& 0.1846\\
%     0.6 &0.0246 & 3131& 0.2971\\
%     0.8 &0.0238 & 4568& 0.4229\\
%     1 & 0.0143& 6551&0.5386\\
% }\wikidirectcircle

% \pgfplotstableread[row sep=\\, col sep = &]{
% datasets & LSC & TTC & kd-tree & DOTTT\\
% 1 & 0.005 & 885420& 146 & 2235810\\
% 2 & 0.0123 & 135077 & 26 &466289\\
% 3 & 0.007 & 11915 & 1.2 & 44509 \\
% 4 & 0.004 & 238 & 0.06 & 3898\\
% 5 & 0.003 & 33 & 0.03 & 742\\
% }\baseline

% \begin{figure*}[ht]
%         \centering
% 	\ref{named0}\\
%         \vspace{-5pt}
% 	\subfigure[GR]{
% 	\begin{tikzpicture}[scale = 0.37]
% 	   \begin{axis}[
% 			    legend style = {
% 				    legend columns=-1,
% 				    font=\footnotesize,
%                         draw=none,
% 				},
% 				legend to name=named0,
%                     legend image post style={scale=0.8, ultra thick},
%                     width=.5\textwidth,
%                     height=0.4\textwidth,
%                     xtick={0,0.1,0.3,0.5,0.7,0.9},
% 				xticklabels={,$10\%$,$30\%$,$50\%$,$70\%$,$90\%$},
% 				xmin=0,xmax=1,
% 				ymin=0,ymax=10000000,
% 				ymode = log,
% 				mark size=4pt,
%                     line width=2.5pt,
% 				ylabel={\huge \bf Response time ($\mu$s)},
% 				ylabel style={yshift=-2pt},
% 				xlabel={\huge \bf $\delta$ length ratio},
% 				ticklabel style={font=\huge},
% 				every axis plot/.append style={ultra thick},
% 				every axis/.append style={ultra thick},
% 				]
% 				\addplot [mark=x,color=c4] table[x=length,y=LSC]{\graphfived};
				
% 				\addplot [mark=o,color=c6] table[x=length,y=TTC]{\graphfived};		
%     \addplot [mark=star,color=c8] table[x=length,y=kd-tree]{\graphfived};
% 			\addplot [mark=o,color=c7] table[x=length,y=DOTTT]{\graphfived};
% 				\legend{{\small \LSC},{\small \EDTTC}, {\small kd-tree}, {\small \DOTTT}}
% 			\end{axis}
% 	\end{tikzpicture}
% 	}
%  \quad
%  \subfigure[ST]{
% 	\begin{tikzpicture}[scale = 0.37]
% 	   \begin{axis}[
%                     width=.5\textwidth,
%                     height=0.4\textwidth,
% 				xtick={0,0.1,0.3,0.5,0.7,0.9},
% 				xticklabels={,$10\%$,$30\%$,$50\%$,$70\%$,$90\%$},
% 				xmin=0,xmax=1,
% 				ymin=0,ymax=10000000,
% 				ymode = log,
% 				mark size=4pt,
%                     line width=2.5pt,
% 				% ylabel={\huge \bf Running time (ms)},
% 				% ylabel style={yshift=-5pt},
% 				xlabel={\huge \bf $\delta$ length ratio},
% 				ticklabel style={font=\huge},
% 				every axis plot/.append style={ultra thick},
% 				every axis/.append style={ultra thick},
% 				]
% 				\addplot [mark=x,color=c4] table[x=length,y=LSC]{\stackoverflowd};
				
% 				\addplot [mark=o,color=c6] table[x=length,y=TTC]{\stackoverflowd};		
%     \addplot [mark=star,color=c8] table[x=length,y=kd-tree]{\stackoverflowd};
% 			\addplot [mark=o,color=c7] table[x=length,y=DOTTT]{\stackoverflowd};
% 			\end{axis}
% 	\end{tikzpicture}
% 	}
%  \quad
%  \subfigure[WK]{
% 	\begin{tikzpicture}[scale = 0.37]
% 	   \begin{axis}[
%                     width=.5\textwidth,
%                     height=0.4\textwidth,
% 				xtick={0,0.1,0.3,0.5,0.7,0.9},
% 				xticklabels={,$10\%$,$30\%$,$50\%$,$70\%$,$90\%$},
% 				xmin=0,xmax=1,
% 				ymin=0,ymax=100000,
% 				ymode = log,
% 				mark size=4pt,
%                     line width=2.5pt,
% 				% ylabel={\huge \bf Running time (ms)},
% 				% ylabel style={yshift=-5pt},
% 				xlabel={\huge \bf $\delta$ length ratio},
% 				ticklabel style={font=\huge},
% 				every axis plot/.append style={ultra thick},
% 				every axis/.append style={ultra thick},
% 				]
% 				\addplot [mark=x,color=c4] table[x=length,y=LSC]{\wikid};
				
% 				\addplot [mark=o,color=c6] table[x=length,y=TTC]{\wikid};		
%     \addplot [mark=star,color=c8] table[x=length,y=kd-tree]{\wikid};
% 			\addplot [mark=o,color=c7] table[x=length,y=DOTTT]{\wikid};
% 			\end{axis}
% 	\end{tikzpicture}
% 	}
%     \subfigure[EM]{
% 	\begin{tikzpicture}[scale = 0.37]
% 	   \begin{axis}[
%                     width=.5\textwidth,
%                     height=0.4\textwidth,
% 				xtick={0,0.1,0.3,0.5,0.7,0.9},
% 				xticklabels={,$10\%$,$30\%$,$50\%$,$70\%$,$90\%$},
% 				xmin=0,xmax=1,
% 				ymin=0,ymax=100000,
% 				ymode = log,
% 				mark size=4pt,
%                     line width=2.5pt,
% 				% ylabel={\huge \bf Running time (ms)},
% 				% ylabel style={yshift=-5pt},
% 				xlabel={\huge \bf $\delta$ length ratio},
% 				ticklabel style={font=\huge},
% 				every axis plot/.append style={ultra thick},
% 				every axis/.append style={ultra thick},
% 				]
% 				\addplot [mark=x,color=c4] table[x=length,y=LSC]{\emaild};
				
% 				\addplot [mark=o,color=c6] table[x=length,y=TTC]{\emaild};		
%     \addplot [mark=star,color=c8] table[x=length,y=kd-tree]{\emaild};
% 			\addplot [mark=o,color=c7] table[x=length,y=DOTTT]{\emaild};
% 			\end{axis}
% 	\end{tikzpicture}
% 	}
%  \subfigure[CT]{
% 	\begin{tikzpicture}[scale = 0.37]
% 	   \begin{axis}[
%                     width=.5\textwidth,
%                     height=0.4\textwidth,
% 				xtick={0,0.1,0.3,0.5,0.7,0.9},
% 				xticklabels={,$10\%$,$30\%$,$50\%$,$70\%$,$90\%$},
% 				xmin=0,xmax=1,
% 				ymin=0,ymax=10000,
% 				ymode = log,
% 				mark size=4pt,
%                     line width=2.5pt,
% 				% ylabel={\huge \bf Running time (ms)},
% 				% ylabel style={yshift=-5pt},
% 				xlabel={\huge \bf $\delta$ length ratio},
% 				ticklabel style={font=\huge},
% 				every axis plot/.append style={ultra thick},
% 				every axis/.append style={ultra thick},
% 				]
% 				\addplot [mark=x,color=c4] table[x=length,y=LSC]{\tsvd};
				
% 				\addplot [mark=o,color=c6] table[x=length,y=TTC]{\tsvd};		
%     \addplot [mark=star,color=c8] table[x=length,y=kd-tree]{\tsvd};
% 			\addplot [mark=o,color=c7] table[x=length,y=DOTTT]{\tsvd};
% 			\end{axis}
% 	\end{tikzpicture}
% 	}
% \caption{Effect of the ratio of query $\delta$.}
% \label{fig:query-delta}
%  \end{figure*}



% \begin{figure*}[ht]
%         \centering
% 	\ref{named1}\\
%         \vspace{-5pt}
% 	\subfigure[GR]{
% 	   \begin{tikzpicture}[scale=0.38]
% 	   \begin{axis}[
% 			    legend style = {
% 				    legend columns=-1,
% 				    font=\footnotesize,
%                         draw=none,
% 				},
% 				legend to name=named1,
%                     legend image post style={scale=0.8, ultra thick},
%                     width=.5\textwidth,
%                     height=0.4\textwidth,
%                     xtick={0,0.2,0.4,0.6,0.8,1},
% 				xticklabels={,$20\%$,$40\%$,$60\%$,$80\%$,$100\%$},
% 				xmin=0.1,xmax=1.1,
% 				ymin=0,ymax=10000000,
% 				ymode = log,
% 				mark size=4pt,
%                     line width=2.5pt,
% 				ylabel={\huge \bf Response time ($\mu$s)},
% 				ylabel style={yshift=-5pt},
% 				xlabel={\huge \bf Interval length ratio},
% 				ticklabel style={font=\huge},
% 				every axis plot/.append style={ultra thick},
% 				every axis/.append style={ultra thick},
% 				]
% 				\addplot [mark=x,color=c4] table[x=length,y=LSC]{\graphfive};
				
% 				\addplot [mark=o,color=c6] table[x=length,y=online]{\graphfive};
%     \addplot [mark=star,color=c8] table[x=length,y=kd-tree]{\graphfive};
    
% 				\legend{{\small \LSC}, {\small  \EDTTC},{\small kd-tree} }
% 			\end{axis}
% 	\end{tikzpicture}
% 	}
%         \subfigure[ST]{
% 	   \begin{tikzpicture}[scale=0.38]
% 	   \begin{axis}[
%                     width=.5\textwidth,
%                     height=0.4\textwidth,
% 				xtick={0,0.2,0.4,0.6,0.8,1},
% 				xticklabels={$0\%$,$20\%$,$40\%$,$60\%$,$80\%$,$100\%$},
% 				xmin=0.1,xmax=1.1,
% 				ymin=0,ymax=10000000,
% 				ymode = log,
% 				mark size=4pt,
%                     line width=2.5pt,
% 				% ylabel={\huge \bf Running time (ms)},
% 				% ylabel style={yshift=-5pt},
% 				xlabel={\huge \bf Interval length ratio},
% 				ticklabel style={font=\huge},
% 				every axis plot/.append style={ultra thick},
% 				every axis/.append style={ultra thick},
% 				]
%                     \addplot [mark=x,color=c4] table[x=length,y=LSC]{\stackoverflow};
				
% 				\addplot [mark=o,color=c6] table[x=length,y=online]{\stackoverflow};		
%     \addplot [mark=star,color=c8] table[x=length,y=kd-tree]{\stackoverflow};
% 			\end{axis}
% 	\end{tikzpicture}
% 	}
%         \subfigure[WK]{
% 	   \begin{tikzpicture}[scale=0.38]
% 	   \begin{axis}[
%                     width=.5\textwidth,
%                     height=0.4\textwidth,
% 				xtick={0,0.2,0.4,0.6,0.8,1},
% 				xticklabels={,$20\%$,$40\%$,$60\%$,$80\%$,$100\%$},
% 				xmin=0.1,xmax=1.1,
% 				ymin=0,ymax=100000,
% 				ymode = log,
% 				mark size=4pt,
%                     line width=2.5pt,
% 				% ylabel={\huge \bf Running time (ms)},
% 				% ylabel style={yshift=-5pt},
% 				xlabel={\huge \bf Interval length ratio},
% 				ticklabel style={font=\huge},
% 				every axis plot/.append style={ultra thick},
% 				every axis/.append style={ultra thick},
% 				]
% 				\addplot [mark=x,color=c4] table[x=length,y=LSC]{\wiki};
				
% 				\addplot [mark=o,color=c6] table[x=length,y=online]{\wiki};
%     \addplot [mark=star,color=c8] table[x=length,y=kd-tree]{\wiki};
% 			\end{axis}
% 	\end{tikzpicture}
% 	}
%         \subfigure[EM]{
% 	   \begin{tikzpicture}[scale=0.38]
% 	   \begin{axis}[
%                     width=.5\textwidth,
%                     height=0.4\textwidth,
% 				xtick={0,0.2,0.4,0.6,0.8,1},
% 				xticklabels={,$20\%$,$40\%$,$60\%$,$80\%$,$100\%$},
% 				xmin=0.1,xmax=1.1,
% 				ymin=0,ymax=100000,
% 				ymode = log,
% 				mark size=4pt,
%                     line width=2.5pt,
% 				% ylabel={\huge \bf Running time (ms)},
% 				% ylabel style={yshift=-5pt},
% 				xlabel={\huge \bf Interval length ratio},
% 				ticklabel style={font=\huge},
% 				every axis plot/.append style={ultra thick},
% 				every axis/.append style={ultra thick},
% 				]
% 				\addplot [mark=x,color=c4] table[x=length,y=LSC]{\email};
				
% 				\addplot [mark=o,color=c6] table[x=length,y=online]{\email};
%     \addplot [mark=star,color=c8] table[x=length,y=kd-tree]{\email};
% 			\end{axis}
% 	\end{tikzpicture}
% 	}
%         \subfigure[CT]{
% 	   \begin{tikzpicture}[scale=0.38]
% 	   \begin{axis}[
%                     width=.5\textwidth,
%                     height=0.4\textwidth,
% 				xtick={0,0.2,0.4,0.6,0.8,1},
% 				xticklabels={,$20\%$,$40\%$,$60\%$,$80\%$,$100\%$},
% 				xmin=0.1,xmax=1.1,
% 				ymin=0,ymax=100000,
% 				ymode = log,
% 				mark size=4pt,
%                     line width=2.5pt,
% 				% ylabel={\huge \bf Running time (ms)},
% 				% ylabel style={yshift=-5pt},
% 				xlabel={\huge \bf Interval length ratio},
% 				ticklabel style={font=\huge},
% 				every axis plot/.append style={ultra thick},
% 				every axis/.append style={ultra thick},
% 				]
% 				\addplot [mark=x,color=c4] table[x=length,y=LSC]{\tsv};
				
% 				\addplot [mark=o,color=c6] table[x=length,y=online]{\tsv};
%     \addplot [mark=star,color=c8] table[x=length,y=kd-tree]{\tsv};
% 			\end{axis}
% 	\end{tikzpicture}
% 	}
% \caption{Effect of the length of query interval.}
% \label{fig:length}
% \end{figure*}



% \subsection{Overall result}
% \label{sec:experiment-overall}
% We will provide the average response time under default settings as the overall experiment results. For each dataset, we generate 1000 queries with the default setting and record the average response time of each algorithm. Note that although weighted counting and binary counting are different problems, the online algorithms and query format of these two problems are the same. So we put the experiment results of these two problems into the same figure, Figure \ref{fig:overall}. From Figure \ref{fig:overall}, we can conclude that our online algorithm \EDTTC runs faster than \DOTTT in every dataset. Besides, the running time of \LSC remains extremely low in every dataset no matter the dataset size. The response time of kd-tree increases with the dataset size, but it remains very efficient.

% \begin{figure}[ht]	
% 	\centering

% 	\begin{tikzpicture}[scale=0.7]
%     		\begin{axis}[
%     			ybar,
%     			bar width=0.2cm,
%     			width=.5\textwidth,
%                     height=0.4\textwidth,
%     			xtick={1,2,3,4,5},	xticklabels={GR,ST,WK,EM,CT},
%     			x tick label style={rotate=0},
%     			legend style = {
%                         legend columns=-1,
%             		font=\footnotesize,
%                         draw=none,
%                     },
%     			legend entries={DOTTT, EDTTC, kd-tree, LSC},
%     			xmin=0,xmax = 6,
%     			ymin=0,ymax=100000000,
%     			ymode =log,
%             log origin=infty,
%     			ylabel style={yshift=-5pt},
%     			ylabel={\Large Average response time ($\mu$s)},
%     			ticklabel style={font=\large},
% 				every axis plot/.append style={ultra thick},
% 				every axis/.append style={ultra thick},
%     			]
%     			\addplot[pattern=north west lines, pattern color=orange] table[x=datasets,y=DOTTT]{\baseline};
%     			\addplot[pattern = grid, pattern color=blue] table[x=datasets,y=TTC]{\baseline};
%     			\addplot[pattern = crosshatch dots,pattern color=green] table[x=datasets,y=kd-tree]{\baseline};
%        \addplot[pattern = crosshatch dots,pattern color=red] table[x=datasets,y=LSC]{\baseline};
%     		\end{axis}
%     \end{tikzpicture}
%     	\caption{Overall result under default settings on different datasets.}
% 	\label{fig:overall}
% \end{figure}

% \subsection{Effect of different query settings}
% \label{sec:experiment-query}
% \subsubsection{Effect of the ratio of query duration $\delta$}

% We will first evaluate the effect of query $\delta$. For each graph, we fix the query interval length and sampling factor as defaulted. We consider five types of $\delta$, i.e., $10\%$, $30\%$, $50\%$, $70\%$, $90\%$ of the query interval length respectively. For each $\delta$, we generate 1000 queries where $t_s$ is selected randomly. Figure \ref{fig:query-delta} shows the testing result. Note that only kd-tree's time complexity is affected by the duration $\delta$. So only the response time of kd-tree varies with the change of $\delta$, while the other three algorithms' response times don't change too much. 

 


% \subsubsection{Effect of the length of query intewrval}


% To evaluate the effect of the length of query intervals, for each graph, we fix the query $\delta$ and sampling factor as defaulted and consider five query time interval lengths, i.e., $20\%$, $40\%$, $60\%$, $80\%$, $100\%$ of $t_{max}$ respectively. For each length, we generate 10000 queries where $t_s$ is selected randomly (For the \EDTTC algorithm on ST and GR, we only run 1000 queries due to the large time cost). Figure \ref{fig:length} reports the average time cost of answering one query on all five graphs as efficiency results. Note that since \DOTTT is designed only for different duration $\delta$ and it is not as efficient as our \online algorithm, we only use \online as our online algorithm. When the time interval becomes larger, \online and kd-tree take more time to respond, 
% while \LSC's cost does not change much, even taking less time. Because when the query interval is $[0,t_{max}]$, the $t_s, t_e$ indexes only cost $O(1)$ time complexity to respond.

% \subsection{Effect of different construction settings}
% \label{sec:experiment-construction-settings}
% \subsubsection{Weighted counting}
% %
% We analyze the query runtime variation caused by different sampling factors as well. For each graph, we choose five different values of sampling factor $k$, i.e., $1\%$, $5\%$, $10\%$, $20\%$, and $30\%$. We generate 10000 queries with default attributes and select $t_s$ randomly. We answer the queries with indexes with different sampling factors and compute the average responding time. Note the increasing rate of responding time is much slower than the increasing rate of sampling factor.


% \begin{figure}[ht]
%         \centering
% 	\ref{named4}\\
% 	\subfigure[WK]{
% 	\begin{tikzpicture}[scale=0.38]
% 	   \begin{axis}[
% 			    legend style = {
% 				    legend columns=-1,
% 				    font=\footnotesize,
%                         draw=none,
% 				},
% 				legend to name=named4,
%                 legend image post style={scale=0.8, ultra thick},
%                  width=.5\textwidth,
%                 height=0.4\textwidth,
%                 xtick={0.01,0.05,0.1,0.2,0.3},
% 				xticklabels={$1\%$,$5\%$,$10\%$,$20\%$,$30\%$},
% 				xmin=0,xmax=0.3,
% 				ymin=0,ymax=0.1,
% 				ymode = log,
% 				mark size=4pt,
% 				ylabel={\huge \bf Running time ($\mu$s)},
% 				ylabel style={yshift=-5pt},
% 				xlabel={\huge \bf sampling factor},
% 				ticklabel style={font=\huge},
% 				every axis plot/.append style={ultra thick},
% 				every axis/.append style={ultra thick},
% 				]
% 				\addplot [mark=x,color=c4] table[x=length,y=LSC]{\wikifq};
				
				
				
% 				\legend{{\small \LSC}, {\small kd-tree}, {\small  \EDTTC}}
% 			\end{axis}
% 	\end{tikzpicture}
% 	}
%  \subfigure[EM]{
% 	\begin{tikzpicture}[scale=0.38]
% 	   \begin{axis}[
%     width=.5\textwidth,
%                 height=0.4\textwidth,
% 				xtick={0.01,0.05,0.1,0.2,0.3},
% 				xticklabels={$1\%$,$5\%$,$10\%$,$20\%$,$30\%$},
% 				xmin=0,xmax=0.3,
% 				ymin=0,ymax=0.1,
% 				ymode = log,
% 				mark size=4pt,
% 				xlabel={\huge \bf sampling factor},
% 				ticklabel style={font=\huge},
% 				every axis plot/.append style={ultra thick},
% 				every axis/.append style={ultra thick},
% 				]
% 				\addplot [mark=x,color=c4] table[x=length,y=LSC]{\emailfq};
				
				
% 			\end{axis}
% 	\end{tikzpicture}
% 	}
%  \caption{Effect of the sampling factor for querying.}
% \label{fig:factor}
% \end{figure}


\begin{comment}
     \subfigure[WK]{
	\begin{tikzpicture}[scale=0.28]
	   \begin{axis}[
				xtick={0.01,0.05,0.1,0.2,0.3},
				xticklabels={$1\%$,$5\%$,$10\%$,$20\%$,$30\%$},
				xmin=0,xmax=0.3,
				ymin=0,ymax=1,
				ymode = log,
				mark size=4pt,
				ylabel={\huge \bf Running time (ms)},
				ylabel style={yshift=-5pt},
				xlabel={\huge \bf sampling factor},
				ticklabel style={font=\huge},
				every axis plot/.append style={ultra thick},
				every axis/.append style={ultra thick},
				]
				\addplot [mark=x,color=c4] table[x=length,y=LSC]{\wikifq};
				\addplot [mark=square,color=c5] table[x=length,y=wavelet tree]{\wikifq};
				
			\end{axis}
	\end{tikzpicture}
	}
 \subfigure[EM]{
	\begin{tikzpicture}[scale=0.28]
	   \begin{axis}[
				xtick={0.01,0.05,0.1,0.2,0.3},
				xticklabels={$1\%$,$5\%$,$10\%$,$20\%$,$30\%$},
				xmin=0,xmax=0.3,
				ymin=0,ymax=1,
				ymode = log,
				mark size=4pt,
				ylabel={\huge \bf Running time (ms)},
				ylabel style={yshift=-5pt},
				xlabel={\huge \bf sampling factor},
				ticklabel style={font=\huge},
				every axis plot/.append style={ultra thick},
				every axis/.append style={ultra thick},
				]
				\addplot [mark=x,color=c4] table[x=length,y=LSC]{\emailfq};
				\addplot [mark=square,color=c5] table[x=length,y=wavelet tree]{\emailfq};
				
			\end{axis}
	\end{tikzpicture}
	}
 \subfigure[CT]{
	\begin{tikzpicture}[scale=0.28]
	   \begin{axis}[
				xtick={0.01,0.05,0.1,0.2,0.3},
				xticklabels={$1\%$,$5\%$,$10\%$,$20\%$,$30\%$},
				xmin=0,xmax=0.3,
				ymin=0,ymax=1,
				ymode = log,
				mark size=4pt,
				ylabel={\huge \bf Running time (ms)},
				ylabel style={yshift=-5pt},
				xlabel={\huge \bf sampling factor},
				ticklabel style={font=\huge},
				every axis plot/.append style={ultra thick},
				every axis/.append style={ultra thick},
				]
				\addplot [mark=x,color=c4] table[x=length,y=LSC]{\tsvfq};
				\addplot [mark=square,color=c5] table[x=length,y=wavelet tree]{\tsvfq};
				
			\end{axis}
	\end{tikzpicture}
	}

  


Last, we will show the affection of the sampling factor to the entire processing time, including index construction and query solving. We choose five different values of sampling factor $k$, i.e., $1\%$, $5\%$, $10\%$, $20\%$, and $30\%$. Figure 16 reports the processing results. Note that  Additionally, even the construction time complexity of LSC and wavelet tree are the same. The construction of LSC is faster than that of the wavelet tree, possibly because it does not need any recursion, which reduces the constant factor.


\begin{figure}[ht]
        \centering
	\ref{named3}\\
	\subfigure[WK]{
	\begin{tikzpicture}[scale=0.38]
	   \begin{axis}[
			    legend style = {
				    legend columns=-1,
				    font=\footnotesize,
                        draw=none,
				},
				legend to name=named3,
                legend image post style={scale=0.8, ultra thick},
                width=.5\textwidth,
                height=0.4\textwidth,
                xtick={0.01,0.05,0.1,0.2,0.3},
				xticklabels={$1\%$,$5\%$,$10\%$,$20\%$,$30\%$},
				xmin=0,xmax=0.3,
				ymin=0,ymax=400000,
				ymode = normal,
				mark size=4pt,
				ylabel={\huge \bf Processing time (ms)},
				ylabel style={yshift=-5pt},
				xlabel={\huge \bf sampling factor},
				ticklabel style={font=\huge},
				every axis plot/.append style={ultra thick},
				every axis/.append style={ultra thick},
				]
				\addplot [mark=x,color=c4] table[x=length,y=LSC]{\wikif};
				\addplot [mark=square,color=c5] table[x=length,y=wavelet tree]{\wikif};
				
				
				\legend{{\small LSC}, {\small Wavelet Tree}, {\small  DOTTT}}
			\end{axis}
	\end{tikzpicture}
	}
 \subfigure[EM]{
	\begin{tikzpicture}[scale=0.38]
	   \begin{axis}[
                width=.5\textwidth,
                height=0.4\textwidth,
				xtick={0.01,0.05,0.1,0.2,0.3},
				xticklabels={$1\%$,$5\%$,$10\%$,$20\%$,$30\%$},
				xmin=0,xmax=0.3,
				ymin=0,ymax=100000,
				ymode = normal,
				mark size=4pt,
				xlabel={\huge \bf sampling factor},
				ticklabel style={font=\huge},
				every axis plot/.append style={ultra thick},
				every axis/.append style={ultra thick},
				]
				\addplot [mark=x,color=c4] table[x=length,y=LSC]{\emailf};
				\addplot [mark=square,color=c5] table[x=length,y=wavelet tree]{\emailf};
				
			\end{axis}
	\end{tikzpicture}
	}

     \subfigure[GR]{
	\begin{tikzpicture}[scale=0.28]
	   \begin{axis}[
				xtick={0.01,0.05,0.1,0.2,0.3},
				xticklabels={$1\%$,$5\%$,$10\%$,$20\%$,$30\%$},
				xmin=0,xmax=0.3,
				ymin=0,ymax=5000000,
				ymode = normal,
				mark size=4pt,
				ylabel={\huge \bf Construction time (ms)},
				ylabel style={yshift=-5pt},
				xlabel={\huge \bf sampling factor},
				ticklabel style={font=\huge},
				every axis plot/.append style={ultra thick},
				every axis/.append style={ultra thick},
				]
				\addplot [mark=x,color=c4] table[x=length,y=LSC]{\graphfivef};
				\addplot [mark=square,color=c5] table[x=length,y=wavelet tree]{\graphfivef};
				
			\end{axis}
	\end{tikzpicture}
	}
 
  \subfigure[WK]{
	\begin{tikzpicture}[scale=0.28]
	   \begin{axis}[
				xtick={0.01,0.05,0.1,0.2,0.3},
				xticklabels={$1\%$,$5\%$,$10\%$,$20\%$,$30\%$},
				xmin=0,xmax=0.3,
				ymin=0,ymax=500000,
				ymode = normal,
				mark size=4pt,
				ylabel={\huge \bf Construction time (ms)},
				ylabel style={yshift=-5pt},
				xlabel={\huge \bf sampling factor},
				ticklabel style={font=\huge},
				every axis plot/.append style={ultra thick},
				every axis/.append style={ultra thick},
				]
				\addplot [mark=x,color=c4] table[x=length,y=LSC]{\wikif};
				\addplot [mark=square,color=c5] table[x=length,y=wavelet tree]{\wikif};
				
			\end{axis}
	\end{tikzpicture}
	}
 
 \subfigure[CT]{
	\begin{tikzpicture}[scale=0.28]
	   \begin{axis}[
				xtick={0.01,0.05,0.1,0.2,0.3},
				xticklabels={$1\%$,$5\%$,$10\%$,$20\%$,$30\%$},
				xmin=0,xmax=0.3,
				ymin=0,ymax=10000,
				ymode = normal,
				mark size=4pt,
				ylabel={\huge \bf Construction time (ms)},
				ylabel style={yshift=-5pt},
				xlabel={\huge \bf sampling factor},
				ticklabel style={font=\huge},
				every axis plot/.append style={ultra thick},
				every axis/.append style={ultra thick},
				]
				\addplot [mark=x,color=c4] table[x=length,y=LSC]{\tsvf};
				\addplot [mark=square,color=c5] table[x=length,y=wavelet tree]{\tsvf};
				
			\end{axis}
	\end{tikzpicture}
	}

 
\caption{Effect of the sampling factor for the entire process.}
\label{fig:factor}
\end{figure}

\end{comment}

% Last, we will report the accuracy of the sampling algorithm with different sampling factors. For each graph, we choose five different values of sampling factor $k$, i.e., $1\%$, $5\%$, $10\%$, $20\%$, and $30\%$. We generate 10000 queries with default attributes and select $t_s$ randomly. Suppose the exact answer is $a$ and the sampling answer is $b$, we will measure the accuracy by using the relative error, $\frac{|b-a|}{a}$. As is shown in table \ref{tab:accuracy}, storing only $1\%$ of the C-points can keep the relative error less than one thousandth. 
% And if we store $20\%$ of the points, we can keep the relative error less than one ten-thousandth. Thus, the sampling algorithm performs well on large graphs.
% \begin{table}[ht]
% \centering
%     \scalebox{0.8}{
%     \begin{tabular}{c|c|c|c|c|c}
%     \hline
%      \diagbox{Datasets}{$k$}    &  $1\%$ & $5\%$ & $10\%$ & $20\%$ & $30\%$ \\
%     \hline\hline
%          GR& $0.0666\%$ & $0.0361\%$ & $0.0208\%$ & $0.0071\%$& $0.0039\%$
% \\
%     \hline
%      ST & $0.0166\%$ & $0.0098\%$ & $0.0076\%$ & $0.0095\%$ & $0.0073\%$\\
%      \hline
%     \end{tabular}
%     }
%     \caption{Sampling accuracy measured by relative error for weighted counting.}
%     \label{tab:accuracy}
% \end{table}
%

% \subsubsection{Binary counting}

% We will first evaluate the effect of the length of the query intervals. Similar to what we did in the weighted part, for each graph, we fix the time intervals, i.e., $20\%$, $40\%$, $60\%$, $80\%$, $100\%$ of $t_{max}$ respectively. For each length, we generate 10000 queries where $t_s$ is selected randomly. Figure \ref{fig:length} reports the average time cost of answering one query on all five graphs as efficiency results. Note that although binary counting is a 3D point counting problem and harder than weighted counting, our index-based solution still runs up to $10^4\times$ faster than the online algorithm.

\begin{comment}
Then we will evaluate the influence of query $\delta$. For each graph, we fix the query interval length and sampling factor as defaulted. We consider five types of $\delta$, i.e., $10\%$, $30\%$, $50\%$, $70\%$, $90\%$ of the query interval length respectively. For each $\delta$, we generate 10000 queries where $t_s$ is selected randomly. Figure \ref{fig:deltaB} shows the testing result. Note that for kd-tree, $\delta$ also affects the running time, which is different from the other algorithms. This is because the converted binary counting is a 3D counting problem and $\delta$ is one of the dimensions. 

\begin{figure}[ht]
        \centering
	\ref{named6}\\
	\subfigure[GR]{
	\begin{tikzpicture}[scale=0.38]
	   \begin{axis}[
			    legend style = {
				    legend columns=-1,
				    font=\footnotesize,
                        draw=none,
				},
				legend to name=named6,
                legend image post style={scale=0.8, ultra thick},
                width=.5\textwidth,
                height=0.4\textwidth,
                xtick={0,0.1,0.3,0.5,0.7,0.9},
				xticklabels={$0\%$,$10\%$,$30\%$,$50\%$,$70\%$,$90\%$},
				xmin=0,xmax=1,
				ymin=0,ymax=200,
				ymode = normal,
				mark size=4pt,
				ylabel={\huge \bf Running time ($\mu$s)},
				ylabel style={yshift=-5pt},
				xlabel={\huge \bf $\delta$ length ratio},
				ticklabel style={font=\huge},
				every axis plot/.append style={ultra thick},
				every axis/.append style={ultra thick},
				]
				\addplot [mark=star,color=c8] 
				table[x=length,y=kd-tree]{\graphfived};
				\legend{{\small  kd-tree}}
			\end{axis}
	\end{tikzpicture}
	}
 \subfigure[ST]{
	\begin{tikzpicture}[scale=0.38]
	   \begin{axis}[
                width=.5\textwidth,
                height=0.4\textwidth,
				xtick={0,0.1,0.3,0.5,0.7,0.9},
				xticklabels={$0\%$,$10\%$,$30\%$,$50\%$,$70\%$,$90\%$},
				xmin=0,xmax=1,
				ymin=0,ymax=20,
				ymode = normal,
				mark size=4pt,
				xlabel={\huge \bf $\delta$ length ratio},
				ticklabel style={font=\huge},
				every axis plot/.append style={ultra thick},
				every axis/.append style={ultra thick},
				]
				\addplot [mark=star,color=c8] 
				table[x=length,y=kd-tree]{\stackoverflowd};
			\end{axis}
	\end{tikzpicture}
	}
 \caption{Effect of the ratio of query $\delta$.}
\label{fig:deltaB}
\end{figure}

\end{comment}

% We analyze the query runtime variation caused by different sampling factors as well. The method of evaluation is the same as the one for weighted counting. As is shown in Figure \ref{fig:factorB}, the increase rate of kd-tree's response time is linear to the sampling factor's increase rate.


% \begin{figure}[ht]
%         \centering
% 	\ref{named6}\\
% 	\subfigure[WK]{
% 	\begin{tikzpicture}[scale=0.38]
% 	   \begin{axis}[
% 			    legend style = {
% 				    legend columns=-1,
% 				    font=\footnotesize,
%                         draw=none,
% 				},
% 				legend to name=named6,
%                 legend image post style={scale=0.8, ultra thick},
%                  width=.5\textwidth,
%                 height=0.4\textwidth,
%                 xtick={0.01,0.05,0.1,0.2,0.3},
% 				xticklabels={$1\%$,$5\%$,$10\%$,$20\%$,$30\%$},
% 				xmin=0,xmax=0.3,
% 				ymin=0,ymax=40,
% 				ymode = normal,
% 				mark size=4pt,
% 				ylabel={\huge \bf Running time ($\mu$s)},
% 				ylabel style={yshift=-5pt},
% 				xlabel={\huge \bf sampling factor},
% 				ticklabel style={font=\huge},
% 				every axis plot/.append style={ultra thick},
% 				every axis/.append style={ultra thick},
% 				]
% 				\addplot [mark=star,color=c8] table[x=length,y=kd-tree]{\wikifq};
				
				
% 				\legend{{\small  kd-tree}}
% 			\end{axis}
% 	\end{tikzpicture}
% 	}
%  \subfigure[EM]{
% 	\begin{tikzpicture}[scale=0.38]
% 	   \begin{axis}[
%     width=.5\textwidth,
%                 height=0.4\textwidth,
% 				xtick={0.01,0.05,0.1,0.2,0.3},
% 				xticklabels={$1\%$,$5\%$,$10\%$,$20\%$,$30\%$},
% 				xmin=0,xmax=0.3,
% 				ymin=0,ymax=5,
% 				ymode = normal,
% 				mark size=4pt,
% 				xlabel={\huge \bf sampling factor},
% 				ticklabel style={font=\huge},
% 				every axis plot/.append style={ultra thick},
% 				every axis/.append style={ultra thick},
% 				]
% 				\addplot [mark=star,color=c8] table[x=length,y=kd-tree]{\emailfq};
				
				
% 			\end{axis}
% 	\end{tikzpicture}
% 	}
%  \caption{Effect of the sampling factor for querying.}
% \label{fig:factorB}
% \end{figure}

% Last, we will report the accuracy of the sampling algorithm with different sampling factors. The setup and measurement are the same as what we do in the weighted counting part.
% As is shown in table \ref{tab:accuracyB}, storing only $1\%$ of the C-points can keep the relative error around one thousandth. 
% And if we store $30\%$ of the points, we can keep the relative error less than three ten-thousandth. Thus, the sampling algorithm also performs well on large graphs for binary counting.
% \begin{table}[ht]
% \centering
%     \scalebox{0.8}{
%     \begin{tabular}{c|c|c|c|c|c}
%     \hline
%      \diagbox{Datasets}{$k$}    &  $1\%$ & $5\%$ & $10\%$ & $20\%$ & $30\%$ \\
%     \hline\hline
%          GR& $0.0323\%$ & $0.0664\%$ & $0.0226\%$ & $0.0212\%$& $0.0264\%$
% \\
%     \hline
%      ST & $0.1675\%$ & $0.1388\%$ & $0.0787\%$ & $0.0456\%$ & $0.0284\%$\\
%      \hline
%     \end{tabular}
%     }
%     \caption{Sampling accuracy measured by relative error for binary counting.}
%     \label{tab:accuracyB}
% \end{table}
%


% \subsection{Index construction}
% \label{sec:experiment-index}
% \subsubsection{\LSC index}

% We report the time and space cost of index construction on all graphs with the default sampling factor in Table \ref{tab:LSC-construction}. Note that, the space costs of the \LSC index match the sizes of the large graphs. For the small graph, the ratio of $\frac{m}{n}$ is large so that the number of triangles will be far larger than the number of edges of the small graph. For instance, to address the weighted counting problem in dataset WK, we need to store 6599388 C-points in \LSC, while for EM, 3634450 C-points will be stored. Although WK is 23 times larger than EM, its C-point number is just two times that of EM. So the space costs of the index are relatively large in the small graphs.


% \begin{table}[ht]
%     \centering
%     \scalebox{0.88}{
%     \begin{tabular}{c|c|c|c|c|c}
%     \hline
%     \diagbox{costs}{Datasets}     & GR&ST&WK&EM&CT \\
%     \hline\hline
%      time($\mu$s)    & 5,636,728 & 8,491,875 & 723,897& 292,519& 11,488\\
%      \hline
%      space(mb) & 77,107 & 339,456 & 81,612 & 26,215 & 1,946\\
%      \hline
%     \end{tabular}
%     }
%     \caption{Construction costs for weighted counting.}
%     \label{tab:LSC-construction}
% \end{table}

% \begin{figure}[h]	
% 	\centering
% 	\begin{tikzpicture}[scale=0.37]
%     		\begin{axis}[
%     			ybar,
%     			bar width=0.8cm,
%     			width=.65\textwidth,
%                     height=0.3\textwidth,
%     			xtick={1,2,3,4,5},	
%                     xticklabels={GR,ST,WK,EM,CT},
%     			legend style = {
%                         legend columns=-1,
%             		font=\large,
%                         draw=none,
%                         at={(0.8,1)/}
%                     },
%     			legend entries={{\tt LSC}},
%     			xmin=0,xmax = 6,
%     			ymin=0,ymax=100000000,
%     			ymode =log,
%                     log origin=infty,
%     			ylabel style={yshift=-4pt},
%     			ylabel={\LARGE Time ($\mu$s)},
%     			ticklabel style={font=\LARGE},
% 				every axis plot/.append style={ultra thick},
% 				every axis/.append style={ultra thick},
%     		]
%     		% \addplot[pattern=north west lines, pattern color=orange] table[x=datasets,y=DOTTT]{\baseline};
%     		% \addplot[pattern = grid, pattern color=blue] table[x=datasets,y=TTC]{\baseline};
%     		% % \addplot[pattern = crosshatch dots,pattern color=green] table[x=datasets,y=kd-tree]{\baseline};
%                 \addplot[pattern = crosshatch dots,pattern color=red] table[x=datasets,y=LSC]{\constructiontime};
%         \end{axis}
%         \end{tikzpicture}

%         \begin{tikzpicture}[scale=0.37]
%     		\begin{axis}[
%     			ybar,
%     			bar width=0.8cm,
%     			width=.65\textwidth,
%                     height=0.3\textwidth,
%     			xtick={1,2,3,4,5},	
%                     xticklabels={GR,ST,WK,EM,CT},
%     			legend style = {
%                         legend columns=-1,
%             		font=\large,
%                         draw=none,
%                         at={(0.8,1)/}
%                     },
%     			legend entries={{\tt LSC}},
%     			xmin=0,xmax = 6,
%     			ymin=0,ymax=1000000,
%     			ymode =log,
%                     log origin=infty,
%     			ylabel style={yshift=-4pt},
%     			ylabel={\LARGE Space (mb)},
%     			ticklabel style={font=\LARGE},
% 				every axis plot/.append style={ultra thick},
% 				every axis/.append style={ultra thick},
%     		]
%     		% \addplot[pattern=north west lines, pattern color=orange] table[x=datasets,y=DOTTT]{\baseline};
%     		% \addplot[pattern = grid, pattern color=blue] table[x=datasets,y=TTC]{\baseline};
%     		% % \addplot[pattern = crosshatch dots,pattern color=green] table[x=datasets,y=kd-tree]{\baseline};
%                 \addplot[pattern = crosshatch dots,pattern color=blue] table[x=datasets,y=LSC]{\constructionspace};
%         \end{axis}
%         \end{tikzpicture}
%         \caption{Index construction time and space.}
% 	\label{fig:overall}
% \end{figure}




\begin{comment}
\begin{figure}[h]	
	\centering
 \subfigure[time costs]{
	\begin{tikzpicture}[scale=0.38]
    		\begin{axis}[
    			ybar,
    			bar width=0.2cm,
    			width=.5\textwidth,
                    height=0.4\textwidth,
    			xtick=data,	xticklabels={GR,ST,WK,EM,CT},
    			x tick label style={rotate=-15,anchor=west},
    			legend style = {
                        legend columns=-1,
            		font=\footnotesize,
                        draw=none,
                    },
    			legend entries={TTM, LSC, wavelet tree},
    			xmin=0,xmax = 6,
    			ymin=0,ymax=20000000,
    			ymode =log,
    			ylabel style={yshift=-5pt},
    			ylabel={\Large Constrution time ($\mu$s)},
    			ticklabel style={font=\large},
				every axis plot/.append style={ultra thick},
				every axis/.append style={ultra thick},
    			]
    			\addplot[pattern=north west lines, pattern color=orange] table[x=datasets,y=fetching]{\construction};
    			\addplot[pattern = grid, pattern color=blue] table[x=datasets,y=LSC]{\construction};
    			\addplot[pattern = crosshatch dots,pattern color=green] table[x=datasets,y=wavelet tree]{\construction};
    		\end{axis}
    \end{tikzpicture}
    }
    \subfigure[space costs]{
	\begin{tikzpicture}[scale=0.38]
    		\begin{axis}[
    			ybar,
    			bar width=0.2cm,
    			width=.5\textwidth,
                    height=0.4\textwidth,
    			xtick=data,	xticklabels={GR,ST,WK,EM,CT},
    			x tick label style={rotate=-15,anchor=west},
                    legend style = {
                        legend columns=-1,
            		font=\footnotesize,
                        draw=none,
                    },
    			legend entries={LSC, wavelet tree},
    			xmin=0,xmax = 6,
    			ymin=0,ymax=200000,
    			ymode =log,
    			ylabel style={yshift=-5pt},
    			ylabel={\Large Space cost (mb)},
    			ticklabel style={font=\large},
				every axis plot/.append style={ultra thick},
				every axis/.append style={ultra thick},
    			]
    			\addplot[pattern=north west lines, pattern color=orange] table[x=datasets,y=TTC]{\constructionspace};
    			\addplot[pattern = grid, pattern color=blue] table[x=datasets,y=LSC]{\constructionspace};
    			\addplot[pattern = crosshatch dots,pattern color=green] table[x=datasets,y=wavelet tree]{\constructionspace};
    		\end{axis}
    \end{tikzpicture}
    }
	\caption{Construction on different data sets ($k=0.01$).}
	\label{fig:construction}
\end{figure}

\end{comment}
% Last, for each graph, we represent the minimum number of full-length queries needed to make the index algorithm based on \LSC faster than the online algorithm based on \EDTTC in Table \ref{tab:amortize}. These numbers are very small compared with the number of timestamps, especially for the large graph. So the index-based algorithm performs better than the online algorithm even if we only need a small number of queries.

% \begin{table}[htbp]
%   \centering
  
  
%   \small
%   \begin{tabular}{c|c|c|c|c|c}
%     \hline
%      Datasets & GR & ST & WK & EM & CT \\
%     \hline\hline
%     Numbers & 7 & 63 & 61 & 1230 & 311\\
%     \hline
%   \end{tabular}
%   \caption{Number of queries with default setting required to amortize the index construction cost.}
%   \label{tab:amortize}
% \end{table}

% \subsubsection{kd-tree}

%
% For binary counting, we will report the time costs and the space costs of all graphs in one table, with the default sampling factor.
% \begin{table}[ht]
%     \centering
%     \scalebox{0.88}{
%     \begin{tabular}{c|c|c|c|c|c}
%     \hline
%     Datasets     & GR&ST&WK&EM&CT \\
%     \hline\hline
%      time($\mu$s)    & 13,001,082 & 1,280,309 & 50,188& 752 & 154\\
%      \hline
%      space(mb) & 137,830 & 17,203 & 1,638 & 45 & 13\\
%      \hline
%     \end{tabular}
%     }
%     \caption{Construction costs for binary counting.}
%     \label{tab:my_label}
% \end{table}
%
% For binary counting, we will report the time costs and the space costs of all graphs in one table, with the default sampling factor.
% We will also report the minimum number of full-length queries needed to make the index algorithm faster than the \EDTTC algorithm.
% \begin{table}[htbp]
%   \centering
  
  
%   \small
%   \begin{tabular}{c|c|c|c|c|c}
%     \hline
%      Datasets & GR & ST & WK & EM & CT \\
%     \hline\hline
%     Numbers & 12 & 10 & 5 & 4 & 5\\
%     \hline
%   \end{tabular}
%   \caption{Number of queries with default setting required to amortize the index construction cost for binary counting.}
%   \label{tab:amortize}
% \end{table}
% For binary counting, we will report the time costs and the space costs of all graphs in one table, with the default sampling factor.
% We will also report the minimum number of full-length queries needed to make the index algorithm faster than the \EDTTC algorithm.
% Note that for all datasets, the number needed is small, so for binary counting, our index-based algorithm performs better than the online algorithm even if there are only a few queries needed.
%
% \subsection{Experiments for extensions}
% \label{sec:experiment-extension}
% We will first report the experiments in directed graphs by evaluating the effect of the query interval length on both weighted query and binary query. We fix the query $\delta$ and sampling factor as defaulted and consider five different query time interval lengths, i.e.,  $20\%$, $40\%$, $60\%$, $80\%$, $100\%$ of $t_{max}$ respectively. For each length, we generate 10000 queries where $t_s$ is selected randomly (For the \EDTTC algorithm on ST, we only generate 1000 queries due to the large time cost). Figure \ref{fig:direct} reports the average time cost of answering one query as efficiency results. Note that the experiment results of directed graphs are very similar to the results of undirected graphs, indicating that our algorithms can be applied to directed graphs as well.

% \begin{figure}[ht]
%         \centering
% 	\ref{named19}\\
% 	\subfigure[ST]{
% 	   \begin{tikzpicture}[scale=0.38]
% 	   \begin{axis}[
% 			    legend style = {
% 				    legend columns=-1,
% 				    font=\footnotesize,
%                         draw=none,
% 				},
% 				legend to name=named19,
%                     legend image post style={scale=0.8, ultra thick},
%                     width=.5\textwidth,
%                     height=0.4\textwidth,
%                     xtick={0,0.2,0.4,0.6,0.8,1},
% 				xticklabels={,$20\%$,$40\%$,$60\%$,$80\%$,$100\%$},
% 				xmin=0.1,xmax=1.1,
% 				ymin=0,ymax=10000000,
% 				ymode = log,
% 				mark size=4pt,
%                     line width=2.5pt,
% 				ylabel={\huge \bf Running time ($\mu$s)},
% 				ylabel style={yshift=-5pt},
% 				xlabel={\huge \bf Interval length ratio},
% 				ticklabel style={font=\huge},
% 				every axis plot/.append style={ultra thick},
% 				every axis/.append style={ultra thick},
% 				]
% 				\addplot [mark=x,color=c4] table[x=length,y=LSC]{\stackoverflowdirectcircle};
% 				\addplot [mark=o,color=c6] table[x=length,y=TTC]{\stackoverflowdirectcircle};
%     \addplot [mark=star,color=c8] table[x=length,y=kd-tree]{\stackoverflowdirectcircle};
% 				\legend{{\small LSC}, {\small  TTC},{\small kd-tree}}
% 			\end{axis}
% 	\end{tikzpicture}
% 	}
%         \subfigure[WK]{
% 	   \begin{tikzpicture}[scale=0.38]
% 	   \begin{axis}[
%                     width=.5\textwidth,
%                     height=0.4\textwidth,
% 				xtick={0,0.2,0.4,0.6,0.8,1},
% 				xticklabels={,$20\%$,$40\%$,$60\%$,$80\%$,$100\%$},
% 				xmin=0.1,xmax=1.1,
% 				ymin=0,ymax=100000,
% 				ymode = log,
% 				mark size=4pt,
%                     line width=2.5pt,
% 				% ylabel={\huge \bf Running time (ms)},
% 				% ylabel style={yshift=-5pt},
% 				xlabel={\huge \bf Interval length ratio},
% 				ticklabel style={font=\huge},
% 				every axis plot/.append style={ultra thick},
% 				every axis/.append style={ultra thick},
% 				]
% 				\addplot [mark=x,color=c4] table[x=length,y=LSC]{\wikidirectcircle};
% 				\addplot [mark=o,color=c6] table[x=length,y=TTC]{\wikidirectcircle};
%     \addplot [mark=star,color=c8] table[x=length,y=kd-tree]{\wikidirectcircle};
% 			\end{axis}
% 	\end{tikzpicture}
% 	}
        
%  \caption{Effect of the length of query interval for directed graph.}
% \label{fig:direct}

% \end{figure}
% Then we will evaluate the efficiency of index maintenance. We will first build the LSC index with edges in $[0,0.8t_{max}]$. Then, we update the index with the remaining edges in $[0.8t_{max}+1, t_{max}]$. Table \ref{tab:update} shows the average time for updating one edge in the \LSC index of different datasets. It is obvious that updating the index is faster than rebuilding the index from scratch.


% \subsection{Case study for dataset WK}

% \label{sec:experiment-case-study}
% We have settled a series of queries for tracking the trend of counting results in dataset WK.
% Formally speaking, the query settings will be $t_e -t_s = 0.2t_{max}$, $\delta = 0.1(t_e-t_s)$, and sampling factor $k = 1$ (storing every C-point to keep the accuracy). Then we generate 1000 queries and the $t_s$ of $ith$ query is $\frac{0.8t_{max}}{1000}\times i$.

% Now we can track the trend of counting results, and the result is shown in Figure \ref{fig:case_study}. Note that for dataset WK, the sharp decline of the number of $\delta$-temporal triangles at the beginning is obvious, which means something unusual happened during that time in this dataset.

% \begin{figure}[h]
%     \centering
%     \centering %图片居中
%     {
%         \begin{tikzpicture}[scale=0.9] %tikz图片
%         \begin{axis}[
%             xlabel= {\footnotesize \bf Query id}, %横坐标名
%             ylabel= {\footnotesize \bf Number of $\delta$-temporal triangles}, %纵坐标名
%             % ylabel style={yshift=-12pt},
%             xlabel style={yshift=5pt},
%             ymode = normal,
%             xmode = normal,
%             xmin=-10,xmax=1010,
%             ymin=1,ymax=3e8,
%             mark size=0.0pt,
%             width=0.515\textwidth,
%             height=0.24\textwidth,
%             ticklabel style={font=\footnotesize},
%             every axis plot/.append style={line width= 1.1pt},
%             %ytick = {1e3, 1e5, 1e8},
%             ytick = {5e7,1e8,1.5e8,2e8,2.5e8},
%             xtick = {1, 250, 500, 750, 1000},
%             every axis/.append style={line width= 0.8pt},
%             legend style = {
%                 at={(0.92,1)},
%     		legend columns=5,
%                 font=\footnotesize,
%                 draw=none,
%     	},
%          ]
        
%         % \addplot[smooth,mark=*,c3] table {figure/trend/res_CT.txt};
%         % \addlegendentry{CT}
%         \addplot[smooth,mark=*,c1] table {figure/trend/rw_WK.txt};
%         \addlegendentry{Number of $\delta$-temporal triangles in WK}
%      \end{axis}
%      \end{tikzpicture}
%  }
 
%  \setlength{\abovecaptionskip}{-0.001cm}
%  \setlength{\belowcaptionskip}{-5pt}
% \caption{Trends of counting result in dataset WK.}
% \label{fig:case_study}
% \end{figure}


\vspace{-0.1in}
\section{Related Work}
\label{sec:related}

In this section, we review the related works of triangle counting in static and temporal graphs.

$\bullet$ {\bf Triangle counting in static graphs.}
%
As an essential problem in network analysis, triangle counting has garnered significant research attention.
%
Most existing works have concentrated on static undirected graphs.
%
The exact counting solution, relying on enumeration, is bounded by $O(m\kappa)$~\cite{chiba1985arboricity}. 
%
Alternatively, counting triangles through matrix multiplication improves time complexity~\cite{al2018triangle}.
%
Recent advancements have pushed the time complexity bottleneck of matrix multiplication to $O(n^{2.371552})$~\cite{williams2024new}.

Additionally, some works focus on approximating the number of triangles either through local or global counting methods.
%
For example, Kolountzakis and Miller~\cite{kolountzakis2012efficient} presented a method for approximating the local number of triangles, while several studies \cite{tsourakakis2009approximate,eden2017approximately,chen2018mining} focused on estimating global triangle counting.
%
Despite this positive research progress, these methods cannot be directly applied to triangle counting in temporal graphs since they do not consider temporal information.

$\bullet$ {\bf Triangle counting in temporal graphs.}
%
Several recent works have studied triangle counting in temporal graphs.
\cite{paranjape2017motifs} defined the $\delta$-temporal triangle and provided an algorithm of counting $\delta$-temporal triangles. \cite{pashanasangi2021faster} further defined $(\delta_{1,3},\delta_{1,2},\delta_{2,3})$-temporal triangles and provided an algorithm based on degeneracy order to count the $(\delta_{1,3},\delta_{1,2},\delta_{2,3})$-temporal triangles.
%
There are several papers \cite{jha2015counting,gou2021sliding} that counted the triangles within a sliding time window.
%
\cite{buriol2006counting,lee2020temporal} proposed efficient sampling methods for updating the number of triangles in continuously updating temporal graphs.
%
To the best of our knowledge, our work is the first to study counting $\delta$-triangles within arbitrary time windows in temporal graphs.

In addition, many works study the counting of motifs, which are more general than triangles \cite{boekhout2019efficiently,paranjape2017motifs,liu2018sampling, zhu2021leveraging, zhu2019scalable}.
%
For instance, \cite{paranjape2017motifs} proposed the concept of $\delta$-temporal motif and provided a counting method for it; \cite{zhu2021leveraging} developed a temporal subgraph reporting method which could be used for solving our $\delta$-temporal triangle counting problem.
%
Some works focus on counting special motifs like clique \cite{himmel2016enumerating} and butterfly \cite{sanei2018butterfly,cai2023efficient}.
%
Besides, counting and enumerating motifs, especially the cliques, are commonly used in data mining \cite{wang2024approach,wang2023clique,wang2022preference}.

\vspace{-0.1in}

\section{Conclusion}
\label{sec:conclusion}

In this paper, we study the problem of efficiently counting $\delta$-temporal triangles in large temporal graphs.
%
We first propose an online algorithm \OTTC by counting the $\delta$-temporal triangles sharing the vertices of each triangle in the static graph.
%
Afterward, we develop an efficient index-based solution that maps $\delta$-temporal triangles into 2-dimensional points and compactly organizes them in a tree structure.
%
Besides, we study efficient algorithms for binary $\delta$-temporal triangle counting.
%
Experiments on both real and synthetic large temporal graphs show that \OTTC is up to 70$\times$ faster than the SOTA algorithm, and our index-based algorithm is up to $10^8\times$ faster than online algorithms.
%
In the future, we will study how to apply our algorithms to directed graphs and try to develop fast approximation algorithms through the sampling-based method.

\begin{acks}
This work was supported in part by NSFC under Grants 62102341 and 62202412, Guangdong Talent Program under Grant 2021QN02X826, and Shenzhen Science and Technology Program under Grants JCYJ20220530143602006 and ZDSYS 20211021111415025.
\end{acks}


\bibliographystyle{ACM-Reference-Format}
\balance
\bibliography{reference}


%\input{technical report}

\received{July 2024}
\received[revised]{September 2024}
\received[accepted]{November 2024}

\end{document}
